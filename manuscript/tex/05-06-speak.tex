\chapter{Early Lessons on Learning How to Speak}

Relatively early on, Ajahn Sumedho began to encourage me to start giving
talks. Sometimes he invited me to accompany him to a conference or a
lecture.

One conference I remember we went on together was called, `Mystics and
Scientists'. There was an extensive programme of speakers, and not just
from the UK. A scientist who had come from the US gave a particularly
impressive presentation that was riveting in its content and eloquent in
its delivery. Thinking about it now, however, what has stayed with me is
not the content of the talk that he gave, but what happened at the
dinner table afterwards. During the midday meal, Ajahn Sumedho and I
were sitting opposite this speaker. When one of the servers wheeled her
meal trolley towards our table, this fellow reached out and helped
himself to a plate of whatever it was that was being served --
spaghetti, I think. The server was having none of it and grabbed the
plate back, saying something about how that food was for the table
opposite, and he would have to wait; a brief tussle followed between the
two of them over the plate of spaghetti. The manner in which this
incident occurred, so soon after his well-received discourse on the
interface between science and spirituality, made it all the more
bizarre.

I don't remember now whether Ajahn Sumedho's talk on that day came
before or after the meal, but I do still remember that it was not
particularly eloquent. My distinct impression was, and still is, that
for Ajahn Sumedho, practice is the priority, not trying to impress his
listeners. From the very early days at Wat Pah Pong, he has trained
himself to speak from `this' moment. As far as I could tell, he puts
very little, or no effort at all, into preparing or planning talks that
he gives. This often results in the same thing being said over and over
again, but this is because primarily he is interested in the quality of
attention that he brings to the occasion, not just in the information he
imparts. As I see it, his offering is not so much the content of the
talk, but the sincerity of his commitment to awakening. The reason
people find so much benefit in listening to his teachings, even when
they have heard the words before, is because the words serve as a
conduit for the spirit; the words are the form, and although important,
they are not the essence. The essence, or the message, is that there is
a path and it is worth walking.

I am reminded of what the Buddha said in Dhammapada v.~93,

\clearpage

\begin{quote}
  There are those who are free from all obstructions;\\
  they don't worry about food.\\
  Their focus is the signless state of liberation.\\
  Like birds flying through the air,\\
  trackless they move on their way.
\end{quote}

I'm not suggesting that this is a description of the reality in which
Ajahn Sumedho lives; it is not my place to speculate about such matters.
However, for me, these words from the Buddha point in the direction that
all who are truly committed to the spiritual life need to be going.
Eloquence is not necessarily an indicator of profundity.

Another quote from the Dhammapada, verse 262-263, says,

\begin{quote}
  Those who are envious, stingy and manipulative\\
  remain unappealing despite good looks\\
  and eloquent speech.\\
  But those who have freed themselves\\
  from their faults\\
  and arrived at wisdom are attractive indeed.
\end{quote}

On at least two or more occasions during the 1980s I was invited to
assist Ajahn Sumedho when he attended the annual Buddhist Society Summer
School\cite{summer}.
To be calling it a Summer School felt just about right.
No doubt there would have been some newcomers each year, but many of the
attendees knew each other well; for them it was a much-loved annual
event. The daily schedule included a comfortable programme of morning
meditations, chanting sessions, and large group lectures by well known
teachers from the Theravada, Mahayana and Vajrayana schools, as well as
small group classes for such activities as Tai Chi and \emph{Ikebana}\cite{ikebana}.

As the leader of the Zen branch of the London Buddhist Society, Dr.~Irmgard Schloegl\cite{irmgard}
was one of the main contributors at this event. She
and I had already met at Chithurst one day when she just happened to
visit, unannounced, at a time when I had been left in charge. What a
good fortune! Bhikshu Ham Wol had spoken to me about her back in New
Zealand, and probably it was through him that I became acquainted with
her book, \emph{Gentling The Bull}.
I think I am right in my understanding that the
chapters in that exceptional book are distillations of talks which she
had given at those Summer Schools. It was at those Summer Schools that
our acquaintance deepened, evolving into a valued Dhamma friendship that
lasted until she passed away in 2007.

Another regular speaker at those gatherings was a contemporary of hers
from Japan, where she had spent over ten years, Ven. Soko Morinaga Roshi\cite{soko}.
My favourite speaker at those gatherings was Trevor Leggett\cite{leggett}
who had also lived many years in Japan working for
the BBC, and who was held in exceptionally high regard within the world
of Judo.

The Ven. Soko Roshi one day agreed to see me, along with one of the
siladhara from our community who was also attending the Summer School.
The precious teaching that I took away from that interview was his
response to a question about the process of establishing a tradition of
training for nuns within our family of monasteries. Implied in the
\mbox{question} was the opinion that the process was taking an awfully long
time. The Roshi was attentive and then gently commented that, in nature,
when there is rapid change, it usually comes in the form of a hurricane,
a volcano, or tsunami, and is disruptive and disharmonious; change that
is harmonious tends to emerge in a way that might not even be noticed. I
think he gave the example of an acorn turning into an oak tree. He
encouraged patience and trust in the harmonious kind of change. He had a
quality that reminded me of Tan Ajahn Chah. I can't think of anyone else
about whom I have ever said that.

I didn't question Ajahn Sumedho's motivation when he put me up on the
stage beside all those proficient speakers; in some sense it was a
compliment that he thought I could do it. It was even a gift in as much
as it taught me a lot; but it was a gift that I didn't fully appreciate
at the time. It fell to me to conduct one rather large class in which I
opted to invite questions from the audience. A junior monk who happened
to be with me on that occasion took the opportunity to ask a burning
question that he had on his mind: `What is the difference between
\emph{samsara} and \emph{sankhara}?' I managed to contain the
embarrassment I felt in having one of our monks, already some years in
the training, asking such a basic question, and went on to offer some
explanation. One good thing though that came out of that, was that some
years later when I was in charge of running a monastery, I compiled
various lists, clearly indicating the books that were required reading
at each stage of monastic training.

A situation that I recall not handling quite so well came at the end of
a period of guided morning meditation. Ajahn Sumedho regularly conducted
those meetings in a very beautiful, wood-lined building that would have
previously served as the chapel for the old Manor House, where the
Summer School was being held. On this occasion he had asked me to lead
it, so I took the opportunity to offer guidance in \emph{Metta}
meditation, or perhaps it was \emph{Karuna} meditation. Either way, the
instruction included words to the effect, `May I be well, may I be free
from suffering.' I like to think that when I offer such guidance I am
sincere in what I say and try to avoid the words becoming empty
platitudes. After the session was over, an older woman with a very
confident bearing came up to the front and quietly but firmly pointed
out to me that, `In this country we don't talk like that.' She was
referring to my encouragement to specifically wish ourselves well, to
wish that we be free from suffering. I suspect she had noticed my
antipodean accent and assumed I wasn't aware that in British culture it
is considered vulgar to be so overtly self-concerned. Her attempts to be
helpful triggered something very unrefined within me. I can't be sure
now what it was exactly: probably a combination of rage from having
grown up in a culture that forcefully and harmfully denied wholesome
self-concern, along with not wanting to be criticized. As far as I
remember I managed to remain outwardly courteous, even though that which
had exploded within me was violent. Obviously I still had a lot more
work to do on \emph{metta} and \emph{karuna}.

(For anyone who might be wondering, out of respect I showed Ajahn
Sumedho the paragraphs above which refer to him and asked if he wanted
me to change anything. He said he didn't.)

