\chapter{The Spirit of Spiritual Community}

From June 1976 until November 1979 I considered Wat Pah Nanachat to be
home. I know Buddhist monks are described as `having gone forth from
home to homelessness'; however, that is an ideal. If we cling too much
to that ideal we could forget to cultivate the wholesome source of
support which comes with having \emph{kalyanamitta}. The culture of Wat
Pah Nanachat was imbued with the spirit of spiritual companionship.
Ajahn Sumedho was like a magnet around which iron filings would
configure, or a flowering shrub which attracted lots of bees. His
evident enthusiasm inspired and sustained the many spiritual aspirants
who passed through Wat Pah Nanachat. Some only briefly stopped by, some
stayed for a short while, and others settled in and made a commitment.

Tan Ajahn Chah would pop in from time to time and offer us some of his
wonderful variety of inspiration. We were like a young family and he was
like a Grandfather who, instead of bringing us presents, would bring
wisdom. In a typical talk that he gave one evening he spoke about how
natural it was that not everyone who showed signs of having faith in
Dhamma will follow it through. In that talk which was translated and is
now called `Dhamma Nature' (printed in
`\emph{The Collected Teachings of Ajahn Chah}\cite{collected}', page 480)
he says,

\begin{quotation}
\ldots{} Sometimes when a fruit tree is in bloom, a breeze stirs and
scatters blossoms to the ground. Some buds remain and grow into a small
green fruit. A wind blows and some of them, too, fall! Still others may
become fruit or nearly ripe, or some even fully ripe, before they
fall\ldots{} When reflecting upon people, consider the nature of fruit in
the wind: both are very uncertain. This uncertain nature of things can
also be seen in the monastic life. Some people come to the monastery
intending to ordain but change their minds and leave, some with heads
already shaved. Others are already novices, then they decide to leave.
Some ordain for only one Rains Retreat then disrobe. Just like fruit in
the wind -- all very uncertain! Our minds are also similar. A mental
impression arises, draws and pulls at the mind, then the mind falls --
just like fruit\ldots{}
\end{quotation}

On another occasion he visited at a time when I was having trouble with
a very nasty infection on my ankle which was taking a long time to heal.
It was so bad in fact that I wasn't going out in the morning on
alms-round, something almost unheard of; if you wanted to eat you had to
go out on alms-round. On this occasion, when Tan Ajahn Chah asked Ajahn
Sumedho how everyone was doing, he mentioned that Tan Uppanno was
struggling with a bad infection. Tan Ajahn Chah asked me to come forward
and show him my foot. Whether or not he said anything at the time isn't
the point, it was his caring that I recall. I mention this incident here
because of the impression his kindness left on me. It wasn't just his
wisdom that he shared with us, it was also his overall quality of
attention; call it compassion, empathy, clarity -- but whatever we might
call it, it was a blessing.

The many blessings that emanated from Tan Ajahn Chah were at times
obvious and welcome and at other times less obvious and they might even
be less welcome. He was aware of this and made no apology for it. His
intention was to offer whatever was needed for us to get the message and
learn to let go of clinging. Once, when he was being questioned as to
which system of meditation he taught: was it \emph{samatha}, or
\emph{vipassana}, or \emph{anapanasati}? -- he replied that his system
was `frustration' (lit. Thai: \emph{toramarn}). He knew he appeared
inconsistent but that didn't matter so long as his disciples got the
message. This sounds similar to what Master Hsuan Hua of the City Of Ten
Thousand Buddhas had said about his job being to trick us.

Not everyone did get the message. There was one guest staying who had
had a very difficult time before coming to Wat Pah Nanachat and Tan
Ajahn Chah spoke about the obstructions we create in our minds so long
as we are still following our habits of clinging. In the case of this
person, apparently they had become stuck with something that arose when
they were in a subtle state of mind and hadn't managed to let go of it.
Tan Ajahn Chah explained that when we cling while in meditation, the
consequences can be very difficult to correct. If, for example, we
become afraid when we are in a somewhat subtle state of mind, and we
attach to that fear, then when we come out of meditation, we remain
stuck and the consequences affect other aspects of our life. Part of me
was not pleased to hear about that; it felt too close to the bone.
However, I understood he had only said it \emph{could} be difficult to
correct, not impossible. My commitment to the training was strong.

One of the guests who arrived and stayed for a longer period of time was
a French Jesuit priest known as Por (Father) Pauset. He came to stay at
Wat Pah Nanachat several times over the years and was responsible for
the planting of a great many trees. He and Tan Ajahn Chah knew each
other quite well and were able to comfortably converse in the local
\emph{Isaan} dialect. When Por Pauset had first arrived in North East
Thailand, many years earlier, he could only speak French. He learned to
speak Thai in the local dialect but then caught the mumps and went deaf.
Although he did learn English that was only after he had already gone
deaf, so his grasp was minimal. When he spoke with us, it was mostly in
Thai or French, mixed with lots of hand gestures. He was a beautifully
humble human being, intensely committed to cultivating the spiritual
life. If he wasn't living at Wat Pah Nanachat, wearing white and fitting
in with the other postulants (\emph{pah kaaw}), he resided at a village
not too far away that was predominantly Christian.

Tan Ajahn Chah seemed unphased by paradox. He often spoke about the
doubts that he had had to learn to handle in his own practice. Once when
Tan Varapañño was helping me explain to him the struggle I was having
coming to terms with my mad mind, he spoke about how at an earlier stage
in his practice he had such intense fear of uncertainty that he `thought
his head was going to explode'. He went on to say that, `If something is
uncertain, and you demand that it be certain, you will suffer'. Again,
in my mind, I can visualise his smile and still feel in my heart the
warmth of wholesome human companionship. There was no judgement. Being
in his company was so special and so normal.

Even his getting things wrong was inspiring, at least to me. After one
of his visits to Wat Pah Nanachat he complimented the \emph{Phra farang}
(Western monks) for putting ugly old dead flowers on the central shrine
in the \emph{sala}, the implication being that he thought we were using
this as a form of contemplation on impermanence and decay. In fact, the
`old dead flowers' was a dried flower arrangement that an artist woman
from Bangkok had offered when she had been staying with us. We didn't
see them as ugly or as an object of contemplation on death and decay.
Probably some people would prefer to think of Tan Ajahn Chah as perfect
in all regards because it makes them feel secure to think that way, but
personally I am happy to see he made such mistakes. We are not talking
here about moral or ethical misjudgements, just errors on the level of
culture and conventions. I never had a chance to discuss such matters
with him, however I feel confident that he would not want us to project
that fantasy of perfection onto him.

If Tan Ajahn Chah wasn't visiting us at Wat Pah Nanachat, Ajahn Sumedho
was going over to visit him at Wat Pah Pong. The two monasteries were
within walking distance. On one of those occasions when Ajahn Sumedho
was over at Wat Pah Pong, he mentioned to Tan Ajahn Chah how happy he
was with the harmony and cooperation of everyone at Wat Pah Nanachat.
Tan Ajahn Chah responded saying something like, `Well you won't develop
in conditions like that.' He was highlighting a principle: to be free
from suffering we must investigate suffering, and not just set ourselves
up with agreeable conditions. Of course Tan Ajahn Chah supported
harmonious cooperative community, but he would caution us against
getting attached to the good feelings that arise in such circumstances;
it is insight into suffering that leads to liberation, not being
surrounded by nice conditions.

Ajahn Sumedho was a thoroughly faithful disciple of Tan Ajahn Chah, both
in terms of his attitude towards monastic training and his approach to
developing the inner life. His attitude seemed to be one of adhering to
the tradition, but while making an effort to avoid clinging.. If something outside of the
tradition felt congruent with the \emph{Dhamma-vinaya} that Tan Ajahn
Chah was teaching, then he was comfortable endorsing it. At one stage,
in conversation with Ajahn Sumedho, he recommended I look into the
teachings of the Chinese Ch'an Master Hsu Yun\cite{hsu-yun}.
He suggested I might look at some translations by Charles Luk that we happened to have in our library. I did, and it was uplifting
to discover that these teachings, although from another tradition, were
so in tune with what was being taught by Tan Ajahn Chah. Also it was a
joy to find how Master Hsu Yun encouraged using the \emph{upaya}, or
skilful means in practice, of enquiring into `who'. The validity of the
experience I had of dropping into a deeper level of awareness on my
first meditation retreat when I had asked, `who is aware?', or `who is
knowing?', was in some helpful way affirmed by those translations.
Presumably, as usual, I only read a small portion of the books I had
been recommended, though some years later I did manage to read all of a
biography of Master Hsu Yun, called Empty Cloud\cite{hsu-yun-bio}.
I am very glad I read that book. An image of Master Hsu Yun
these days hangs beside those of Ajahn Thate and Tan Ajahn
Chah in my kuti.

There were times during my stay at Wat Pah Nanachat when I longed to be
able to openly talk over what had happened for me at Wat Hin Maak Peng.
The wound was far from healed, and occasionally I did attempt to address
the topic. But every time I started to get close to it, echoes of the
terror would resurface, threatening me with a sense that I was again
about to be overwhelmed. If anyone might get it, I thought, it could be
Tan Jotiko, so on one occasion I did try to see if he could recognize
what I was talking about. I am sure he listened, but I failed to
communicate in a way that made any difference.

Of course I wanted to be understood, and probably some of the
others were wondering about this Tan Uppanno \mbox{character} who seemed to be
carrying such a burden around with him, but everyone had their stuff to
be dealing with. What mattered was that we were companions on this
demanding but meaningful journey together. That was already amazing.

I did still have to work hard at suspending assumptions such as, `I feel
bad because I am bad'. However, the work served to build the strength of
patience, of mindfulness, and of trust. Fortunately, I found I was
willing and able to simply give myself into the training: to be on time
for morning chanting, walk on alms-round, eat, sweep leaves, pull water
from the well, be on time for evening chanting, and keep trusting.
Without that ordeal I would have been less well-equipped when it came to
handling the many challenges that lay ahead.
