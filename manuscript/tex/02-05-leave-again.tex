\chapter{Ready to Leave, Again}

\enlargethispage{\baselineskip}

For all the fun we were having, there was still something really
important missing. I felt far from contented. At some point Andrew (from
Owen street) and I hatched a plan to escape to Sydney. Whether Andrew
saw it as an escape or not, I am not sure; he might have had other
motivations. In my case, ongoing indulgence in heedlessness was
undermining my confidence, and, perhaps similar to the way I left
Auckland and left Hamilton, I now wanted to leave New Zealand.

Before departing I visited my parents who lived almost at the other end
of the North Island near the Bay of Islands. After my father had a heart
attack he had left his job managing the Ford garage and they went into
semi-retirement running the General Store at the small seaside village
called Opua. Opua was known internationally as a safe deep harbour where
ocean-going yachts could conveniently moor long term. On one of my
earlier visits to my parents, Green Peace's Rainbow Warrior had been
moored there in preparation for one of their trips to Mururoa Atoll to
protest against France's testing of atomic weapons. Nobody knew at that
time that in 1985 the French secret service would bomb and sink the
Rainbow Warrior\cite{rainbow} while it was moored in Auckland Harbour.

\enlargethispage{\baselineskip}

I suspect that by now my parents had started to accept that I was going
to keep behaving in ways they didn't understand. They still loved me,
but there was almost no real communication between us. My visit home was
not because I actually wished to see them; presumably I felt it was the
right thing to do. That trip north also gave me a chance to say goodbye
to a local potter, Peter Yeates, whose company I had enjoyed in the
past. At the time he was collaborating with the artist Friedensreich
Hundertwasser. Originally from Austria, Hundertwasser was well known,
amongst other things, for his posters advertising the
1972 Munich Olympic Games\cite{olympic}.
In that part of New Zealand he was known for designing the extraordinary
public toilets in Kawakawa\cite{toilets}.

Also, while I was up north, I took the opportunity to visit another
commune I had heard about. It was situated not far away on the other
side of the island at Waiotemarama, near the Hokianga Harbour. The
commune had the rather grand name of \emph{The Foundation of Mankind}. I
managed to get a ride over there on the back of a motorbike and was
pleased I had the chance to see it. There was energy and enthusiasm,
though, thinking back about it now, I suspect they too were finding
communal living not quite as heavenly as they had hoped. As Ajahn
Sumedho has said, ideals are like stars, they are useful for getting
your bearings, but it is not wise to expect to actually reach the stars.
Ideals are also a good way of generating energy, but if we hold them too
tightly, we will end up very disappointed.

