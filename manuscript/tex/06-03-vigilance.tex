\chapter{Creative Vigilance}

As a gift on my birthday, I think in 1987, Ajahn Anando generously
arranged for me to be able to visit family and friends in New Zealand.
At the time I was presented with the card, which had the gift inside it,
I was somewhat confused, and maybe wasn't even sure it was for real. I
had not seen that coming. Ajahn Anando was very thoughtful like that.
Exceptional generosity on the part of many lay friends and supporters of
the sangha over the years meant that that was just the first of several
trips back to New Zealand. The same generosity by supporters extended to
financial gifts being given to my parents, since some of the supporters
knew that as monks we aren't able to make such gestures.

It might have been on that first trip back that I met up with an ex-monk
friend, Mark Overton. Some years earlier, after finishing his medical
training, Mark had heard Ajahn Sumedho speak during one of his visits to
New Zealand. This inspired him to travel to Thailand where, eventually,
he took up the monks' Precepts and trained under Ajahn Pasanno at Wat
Pah Nanachat. He then went on to spend a brief period of time training
at Amaravati, before disrobing and once again practising medicine. I
think it was on that trip that Mark and I went hiking together (called
`tramping' in New Zealand) in the stunningly beautiful North West Nelson
Forest Park. On another one of those occasions when he and I were hiking
in New Zealand, I recall how we had climbed a peak in the Southern Alps,
I think it was Mt Sefton; I took a refreshing dip in a glacial lake at
the summit before we walked back down again, and, on the same day, drove
out to one of the many lovely beaches on the coast of the South Island.
I knew the South Island of New Zealand was beautiful, but now I was
seeing it from a different perspective. At one point, as my eyes scanned
the forest that stretched all the way to the mountains in the distance,
I registered how refreshing it was to simply gaze upon the
un-interfered-with. Thank you, Mark, and, again, thank you, New Zealand.

The following year I once more found myself in New Zealand, this time
for a stay of about two months, most of which was to be spent at the
Vihara in Auckland, on Harris Road. Part of my plan was to try and spend
more time with my parents, who had moved to a retirement village near
Orewa, just a few miles north of Auckland.

Early on during this period in New Zealand, I also took the opportunity
to visit with my good friend Jutta in Palmerston North. It was
noticeable on that visit that something had changed for her. Up until
then, I don't remember her ever having shared much about the terrible
suffering she had endured in Dresden, and throughout the Second World
War, but now she spoke more freely. She also shared with me how she had
learnt a particular breathing technique which meant she no longer felt
burdened by so much old pain.

It was inspiring to meet my friend in this new way and I was happy for
her. Whatever spiritual techniques she had learned, or retreats she had
been on, or psychotherapy she had undertaken -- and there had been a lot
-- nothing had led to the integration she was seeking. This breathing
technique seemed to be the medicine she needed. The technique that she
was now working with appeared to be a gentler form of holotropic
breathing as used by the Czech psychiatrist, Stanislav Grof\cite{grof}.
Jutta had earlier tried the Grof approach and found it
too invasive. Later, I believe, she spoke with me about how people
sometimes use this, and similar breathing techniques such as rebirthing,
in a goal-oriented way, and that she was not at all keen on that
approach. Her way was not necessarily looking to relive the birth
experience, or uncover past lives, it was much more here-and-now and, as
far as I could see, more in harmony with dhamma practice. After hearing
about the benefit she derived from this exercise, I was keen to try it
out for myself and she kindly offered to teach~me.

The subjective experience that this form of disciplined breathing
precipitated in me defies description or explanation. Suffice it to say
that this technique, which here I will refer to as `connected
breathing', along with a here-and-now, whole body-mind quality of
awareness, brought an end to the fourteen year long `holding pattern'
that I mentioned earlier began after my first vassa.

There are `how-to' books that have been written on this subject, but I
would very strongly advise against trying it out alone. An enormous
amount of energy can be accessed and flood the body in exquisitely
agreeable ways, but that same energy can put you in touch with pockets
of old pain that you didn't know you had. The technique is designed to
put you in touch with such pain, but if at the point of opening up to
it, you re-enact the resistance which caused the pain to become stuck in
the first place, you risk re-traumatising yourself and, in the process,
making your state of imbalance even worse. On the level of mind, we
might like to think we can handle it; the same as when we are on retreat
cultivating \emph{metta} towards all beings, we might like to think that from
now on we are going to always behave in a kind and caring manner towards
absolutely everyone. But when we actually meet some of those beings,
maybe we find our emotional reactions are not quite so kindly after all.
So long as we are identified with our thinking, we cannot trust our
mind.

This type of breath work can be very effective in putting us in touch
with that which was previously out of reach. Somebody who has worked
extensively with the technique themselves, could recognize, during a
breathing session, signs that point to where and when old pain is ready
to be received yet is still being resisted. Then, hopefully, they will
be able to suggest, at just the right time, in the just right manner, a
change in approach, or perhaps a change in the rhythm of breathing,
which will lead to a deep letting go of that resistance. Once such
resistance is let go of, there is a chance we will have a much clearer sense of what our
teachers mean when they tell us to be practising `in the body'. Also, we
see more clearly the disastrous consequences of having betrayed
ourselves in the first place by abandoning our bodily intelligence and
taking refuge in thinking. I am not saying that everybody betrays
themselves and becomes lost in their heads, but those who do, suffer a
great deal because of it. In earlier times, the degree of dysfunction
that many of us are defined by these days would have been seen as a form
of madness. If we do find freedom from the madness of being disembodied,
there will be much gratitude.

Back in Auckland at the Vihara, I was feeling grateful for the support
of the Theravada Buddhist community, who had set up a rota of drivers
that took turns in taking me out to see my parents and then bringing me
back again. The same group took it upon themselves to make sure someone
was always there each day to offer a meal. There was a tradition already
established within their community whereby a good number of mostly Sri
Lankans, Burmese, Malaysians and a few Kiwis, would meet for chanting
and meditation each Sunday night. When there was a monk staying at the
Vihara, the numbers swelled. The quality of their interest in Dhamma and
the sincerity of their commitment to meditation and Dhamma practice in
general, was truly impressive. These were not Buddhist-by-name only;
they had a love for the Dhamma and genuinely wanted to make the most of
their good \mbox{fortune} in having an opportunity to practise it. Thinking
about them now, I still find it heart-warming, and I am very grateful to
have met them. In my experience, it is rare to find that quality of
commitment. \emph{Anumodana}.

Other than the visits to see my parents, the daily meal and the
once-a-week pujas, my time was free, which meant I persevered with `the
breathing exercises'. I looked forward to my sessions each day, in the
same way I would look forward to food if I had been starving. It was as
if seeds had been planted a long time ago, but had not had sufficient
water or warmth for them to germinate. Now it felt as if many seeds were
beginning to sprout. A new kind of hope began to emerge. Where I had
felt deeply emotionally and energetically obstructed, I now felt there
was great possibility. I didn't know what those possibilities were, but,
with here-and-now, whole body-mind awareness, that didn't really matter.
This kind of hope was not a naive longing, it was about being positively
oriented towards the future in a way that generated energy, which was
then available to investigate whatever was happening here and now.

Somebody set up a meeting for me one day to see a Christian monk who was
living in Auckland. When we met, he struck me as a dedicated person with
a strong sense of integrity. Some years prior, he had been working (as a
nurse I think) in a hospice in Saigon that was part of the Mother
Theresa's community. During our conversation, he spoke about what he had
witnessed in a number of the Vietnamese patients as they approached
death. He told me that some of those who came into the hospice
professing to be Christians, had previously been Buddhists. He said that
when the end came near, it wasn't to Jesus that they were praying; they
reverted to their faith in Buddhism. What he seemed to be telling me was
that, although these days I called myself a Buddhist, when it came to
the crunch, I could expect to revert to Christianity. What was good
about hearing that was that I didn't feel threatened. Maybe I was
mistaken in what I understood him to be saying, and he was in fact
paying a subtle compliment to Buddhism, but I don't think so. Feeling
that my commitment to Buddha, Dhamma, Sangha was being challenged like
that was helpful. What was about to happen, however, was even more
challenging.

One of the Kiwi fellows who attended the Vihara from time to time, asked
if I would be interested in spending a couple of days hiking along the
coastal footpath just north of Auckland. I jumped at the invitation. We
began at Piha\cite{piha} and walked south. I can't remember now, but I assume he had
arranged for someone to pick us up at the exit point. The description
that follows of what happened during that walk, is an edited extract
from \emph{Alert To The Needs Of The Journey}\cite{alert} Chapter Two (p 15),

\begin{quotation}
We had been hiking for several hours along the coastal footpath; the
weather was hot, and since the beach below us was empty, it seemed fine
to cool off in the water. What I didn't notice was that at the point
where I chose to enter the water, the waves were not breaking. Had I
been better informed about such things, I might have known that the
absence of white-water breakers was a sign that there was probably a
hollow area in the sand beneath the surface of the water, creating a
counter-current that would pull anyone that entered there out to sea;
and being pulled out to sea is exactly what happened to me. My hiking
companion was still standing on the shore, witnessing in desperation the
situation that was unfolding. Many drownings result from just such
situations, when a swimmer is unexpectedly caught in a rip current and
reacts by struggling against it until exhaustion eventually takes over.
Initially, I did struggle, trying to get back to the shore and out of
the danger, doing what I was used to doing whenever I felt threatened:
trying to save myself. But I realized quite quickly that no amount of
fighting to overcome the current was going to work; it was far too
powerful. What did work, thankfully, was surrendering; I flipped over
onto my back and floated: no more fighting, but simply allowing the
current to carry~me.

I spontaneously remembered the connected breathing; instead of
struggling, there was deep trusting and a whole-body sense of
surrendering habitual controlling. I found myself drifting out to sea,
floating and breathing. My head was filled with powerful conflicting
thoughts and images: of being eaten by sharks somewhere between Piha and
Sydney; of my parents being upset on hearing that their son had drowned;
of Ajahn Sumedho being annoyed with me for my heedlessness. But at one
point, associated with the effort to keep floating, trusting and
breathing, came the powerful thought, `Let the Buddha take over': my
translation of \emph{Buddhaṃ saranaṃ gacchami} -- `I go for refuge to
the Buddha'. It felt like a battle was going on within me, between, on
the one hand, strong inclinations towards trying to save myself, and on
the other, an impulse towards trusting. The thought that I mustn't give
up the struggle to save myself was fuelled by guilt and distrust, and
when I engaged it, the rhythm of the breathing became interrupted and my
body began to sink. When there was letting go of the contraction of fear
and trusting again, the body felt supported and I returned to floating.
There was no doubt about the intensity of fear coursing through my body;
I definitely did not know that I was going to be OK. At times it really
did look like I might not be. Thankfully, the intimidation of the
impulse to control was outshone by the impulse to surrender into the
breathing, to trusting, to releasing out of the struggle to save myself.

As it happened, the current did drag me out to sea some distance, but
then carried me down the coast and out of the dangerous area, and
eventually the waves brought me safely ashore. Once I was standing on
the beach again I was elated: not just because I was now safe, but
because I felt I had been given the gift of affirmation of practice. In
a modest but significant way, it felt emblematic of what it meant when
the Buddha conquered \emph{Mara}.
\end{quotation}

Back at the Vihara, during the Sunday night Dhamma talk, I chose to
speak about my joy at receiving such an affirmation. I might have even
included some comments about what the Christian monk had suggested would
happen when it came to the crunch. Unfortunately, not everyone picked up
on my sense of gladness, and instead became upset at the thought of
nearly losing their monk. Later, when I considered what had happened, I
realized that talking about that experience in that context was not at
all clever. In fact, swimming in a place that is renowned for rip
currents, was also not at all clever; it was completely foolish. The
good friends and supporters at the Vihara forgave me quite quickly and
for the remainder of my time in New Zealand there were no more such
escapades.

The impact that the connected breathing was having on me was profound.
It did worry me somewhat, since the energy involved was at times so
dramatic. I didn't want to start talking about it with everyone; it was
too important. Also, in monasteries, such bits of news sometimes lead to
ridicule or to becoming the latest fad. It wasn't that I felt precious
about this technique, I just wanted time to see how it would develop.
Also I suspected I would sound evangelical if I began to speak about it
at that stage. This was the most significant aid for integration that I
had come across. I realized, though, that in its power lay its danger.
Perhaps I would lose perspective and go crazy. So I decided to let two
people that I trusted know about it, and then wait one year to see how
it settled. One person I confided in was Ajahn Viradhammo, the Canadian
abbot of Bodhinyanarama Monastery, near Wellington; I~either wrote to
him or spoke with him on the phone. The other person was Tan Kittisaro,
and I waited until I was back in the UK before telling him. Obviously both of them
respected my wish for discretion, even if they couldn't directly relate
to my experience.

It might also have been during this period of staying at the Auckland
Vihara that a somewhat rough and ready Kiwi fellow called Blue came to
see me. He was already familiar with our tradition, and was hoping I
would accept an invitation to lead a meditation retreat on his property
out on Great Barrier Island. He offered to make all the necessary
arrangements, so I agreed. Great Barrier Island is easily reached by
ferry from Auckland, and when I arrived there, Blue was waiting to pick
me up, on his quad bike. That was different. His house was only half
built but the weather was mild and the group who had gathered for the
retreat were friendly and interested. I suspect that already, by that
stage, Blue was intent on taking up monastic training. Either way, it
wasn't long before he joined the sangha at Bodhinyanarama and was given
the name Kusalo Bhikkhu. From 2012 until now, Ajahn Kusalo has been the
abbot of Bodhinyanarama Monastery.

When it came time to depart New Zealand and return to Britain, it was
with even more inspiration and gratitude than before: inspiration born
out of association with the fine group of supporters at the Auckland
Vihara, and gratitude for this new skill to which I had been introduced.
Besides the hope I mentioned above, there was a new quality of
confidence, and an increased ability to trust and to feel without being
quite so defended, also a readiness to aspire. All of those qualities
contributed to what these days I like to think of as a state of creative
vigilance: creative, inasmuch as it is agile and interested in
investigating conditions from different perspectives -- not a fixed
position or approach --and vigilant in the sense that it is a state of
aliveness, alertness, and somewhat more ready to meet what life gives
us. Perhaps in Pali it is akin to \emph{saddha}.

