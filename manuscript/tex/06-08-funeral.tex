\chapter{Tan Ajahn Chah's Funeral and Teachings}

On January 16th, 1992, our teacher, Tan Ajahn Chah, passed away. He had
appeared to be near death for several years, so that when the time came
it wasn't a terrible surprise. There was of course, nevertheless, a
sense of loss. Some of the sangha from down south travelled out to
Thailand to participate in the chanting sessions honouring his life and
his teachings. In keeping with Thai custom, occasions such as the
passing of a great Dhamma teacher calls for a very special event. In the
case of Tan Ajahn Chah, there was going to be an extended period in
which many thousands of disciples, monastic and lay, would gather at his
monastery, concluding with His Majesty The King of Thailand coming to
pay his respects before the actual cremation. All that would require a
great deal of preparation, so the date for Tan Ajahn Chah's funeral was
set for the following year, 16th January 1993.

At Harnham we established a week-long routine of evening sitting
meditation and chanting, ending with a public event, during which I read
out one of his translated talks, Not Sure\cite{collected}; a teaching on
uncertainty. A special shrine was set up
in the meditation hall on the Dhamma Seat. Our teacher had gone; we were
now left with the teachings.

One of the great appeals of Tan Ajahn Chah's manner of teaching was the
way he used similes. He was skilled in turning everyday situations into
Dhamma lessons. When he spent time in Britain he visited Edinburgh, but
that was well before the monastery in Northumberland was established.
Had he spent time here on Harnham Hill in these old stone buildings, he
would very likely have had something to say about the conditions to
which we were having to adapt.

There was a very good reason why the roof tiles on many of the buildings
on Harnham Hill were made out of thick stone slabs; the winds that
sometimes buffeted us were forceful. The walls of the stone buildings
were thick, also for a good reason; particularly when the winds blew in
from the north, they could be very bitter. The nearby city of Newcastle
upon Tyne is on approximately the same latitude as Moscow.

However, the winds from the north were not the only challenge we had to
handle. We certainly also had our share of being blown around by the
eight worldly winds: praise and blame, gain and loss, pleasure and pain,
honour and insignificance. In the Mahamangala Sutta\cite{mahamangala}
(p39), in the third to last stanza, the Buddha
describes how the heart of one who has insight into the Four Noble
Truths reacts when impacted by the worldly winds,

\clearpage

\begin{quote}
  Phutthasa loka-dhammehi,\\
  cittam yassa na kampati,\\
  asokam, virajam, khemam,\\
  etam mangalam-uttamam

  `Though subjected to the worldly dhammas,\\
  the heart (of one who has insight into the\\ Four Noble Truths) will remain unshakeable,\\
  griefless, dustless, secure.\\
  This is the greatest blessing.'
\end{quote}

The land and buildings surrounding Harnham Monastery were owned by Mr
John Wake, or Farmer Wake, as he was known. The main building, which the
sangha occupied, had been in a semi-derelict condition when Farmer Wake
originally let it to us. Eventually, once repairs had been done, he
would often join in with the sangha for the midday meal. From time to
time the monks would help him out with tasks around the hundred-or-so
acres of farmland. He was already in his eighties by then and could use
the help. He seemed to enjoy having the company of the community on the
hill. Often he would talk about the history of the place, including the
period when an earlier owner of the property,
Madam Babington\cite{babington}, had lived in the Hall and was eventually buried in
a crypt not far from the main house. There had also been a small chapel
that she had built near the walled garden. Some say the ghost of Madam
Babington still appears on the hill.

In the early years, Farmer Wake took part in our festival events,
including a circumambulation on Vesakha puja. I~remember on one
occasion, around the time when the building of the Dhamma Hall was newly
completed, he came inside to see the result of all our work, and
commented, `What a pity that your teacher didn't live long enough to see
this'. Thank you, Farmer Wake, for being so broad-minded and big-hearted
as to welcome a bunch of Buddhist monks onto this wonderful Hill.

In those days, we received regular visits from members of the Leeds and
Edinburgh meditation groups. They were generous and energetic in helping
with the ongoing building projects. One of those visitors was a young
student called Timmy, who was studying Russian at Edinburgh University.
His mother was Thai and his father Malaysian; he seemed keen to help out
and fitted in well. These days he is known as Ajahn Siripañño and serves
as leader of the sangha at a remote residence near the border between
Thailand and Myanmar, a place called Dtao Dam, or The Black Turtle
Hermitage.

\enlargethispage{\baselineskip}

In January 1993, nearly all the senior Western sangha disciples of Tan
Ajahn Chah were gathered at Wat Pah Nanachat. The schedule had been
skilfully arranged so that, besides our having time to participate in
the events taking place nearby at Wat Pah Pong, we also had quality time
together, meeting formally and informally. There was at least one large
meeting in the main \emph{sala}, to which everyone was invited -- monks,
novices and nuns, residents and visitors. I was asked to facilitate. The
atmosphere was surprisingly harmonious and the Q\&A session was not
difficult to mediate. I say surprisingly, since having such a large
group, of mainly men, from different backgrounds, and most of us
strong-headed, differences of opinion were inevitable. I would say that
in a large part the concord was a result of the very grounded
approach Tan Ajahn Chah had towards monastic community life in general,
and towards \emph{Vinaya} in particular.

\enlargethispage{\baselineskip}

\emph{Vinaya}, or the monastic code of discipline, in its original form
was an expression of the wisdom and compassion of the Buddha himself.
However, there are a great variety of interpretations of just how to
apply the many rules and procedures included within that code of
discipline. Approximately two thousand six hundred years ago it had been
spoken in a dialect known as \emph{Magadhi} (with some possible other
related dialects) and was recorded in the Pali language within the first
two hundred years after the Buddha passed away. Pali was not a generally
spoken language (the word Pali actually means `text'); some scholars
think it is somewhat similar to esperanto\cite{esperanto}
inasmuch as it was intentionally generated for a specific
purpose, in this case to codify recorded teachings. These days there are
still many scholars who can understand it, and these Pali texts are
available to consult when seeking clarification on particular points.
Tan Ajahn Chah's approach was to show respect for the tradition and for
the theoretical (\emph{pariyatti}) aspect of our monastic training, but
to always remember that the point of these teachings and traditions is
to awaken to the truth that lies beyond our habits of clinging. Hence a
lot of care is required to avoid falling into the trap of seeking
security by clinging to the \emph{Vinaya}. In other words, remember to
keep your feet on the ground.

At the gathering at Wat Pah Nanachat, there were not just the large
group meetings; there were, for example, other meetings that involved
the abbots of formally appointed branch monasteries. These various
meetings, large and small, were the harbinger of the now
well-established tradition of senior sangha members meeting up
approximately every three or four years to discuss shared concerns. It
was the first opportunity we had to see ourselves as this evolving
worldwide community, and to begin to acknowledge how we were going to
have to work to maintain cooperation. As I said, most of us were
strong-headed and had our own views on things; however, again, I think
it was because of Tan Ajahn Chah's example, and the value he placed on
cultivating cooperative community, that we managed to meet in harmony
and express differences without too much difficulty.

One example of an issue that we successfully navigated our way through
to a mutually agreed solution, was to do with traditions around bowing.
In Thailand, tradition dictates that when paying respects to an elder,
first the sangha of bhikkhus bow, and the elder acknowledges them by
holding his hands in \emph{añjali}. This would then be followed by the
novices, nuns and laity all bowing, with the elder having lowered his
hands. There is an explanation within the \emph{Vinaya} for how this
practice might have developed, but we felt that in this case, where
Buddhism was rapidly spreading to many very different cultures, there
was room for another interpretation. We discussed how, outside of
Thailand, it would more likely lead to harmony and mutual benefit if
everybody bowed together. It was encouraging to find that as a group we
were able to listen to each other -- both those of a traditional
persuasion and those keener on adaptation -- consider the variables, and
eventually arrive at an agreement by consensus. As far as I know the
decision taken at that meeting has never caused any disruption.

It is not insignificant that the Buddha established consensus as the
primary principle involved in making formal decisions in the sangha. In
some cases a decision made by majority vote can still stand; however, it
is better to make the effort to reach a consensus. It can take a
considerable amount of time and patience to reach a decision by true
consensus, as it requires that all those involved in making the decision
feel included. It doesn't require that everyone has exactly the same
view -- that would be unanimity. On any matter of substantial
importance, it is likely that not everyone will hold exactly the same
view, but it is possible that people holding a divergence of views can
all agree on a single course of action. I don't know if social
psychologists have ever done studies on this subject, but if they did, I
expect they would find that when a course of action is decided upon by
way of consensus, there is a better chance of everyone respecting that
decision, because they were all involved.

As an aside, I want to deviate briefly and comment on what has been
happening here and now, in July 2020. As mentioned above, that gathering
at Wat Nanachat in January 1993 was the first in a series of such
gatherings. I don't think we had a name for it at the time since the
main purpose for our being there was the funeral of Tan Ajahn Chah. Over
the years, however, a variety of acronyms have been used to describe
these gatherings: WAM for `World Abbots Meeting'; GEM for `Global Elders
Meetings'; IEM for `International Elders Meetings'. Currently, when
referring to the smaller meeting of abbots only, we are using BAM, which
stands for `Branch Abbots Meeting'. The name change factors in the
difference between the formally appointed Branch Monasteries\cite{branches},
of which there are now fifteen, and the more loosely
affiliated Associated Monasteries, of which there are maybe eleven or
twelve. The fifteen abbots who constitute the BAM group are tasked with
discussing and hopefully reaching decisions on matters pertaining to
these monasteries outside of Thailand. (The abbot of Wat Pah Nanachat in
Thailand is also included). As it happens, the stage I am at currently
in writing these reflections, has just coincided with two days of
meetings of these branch abbots. A gathering, probably in Thailand, was
due to take place around now, but the Covid-19 pandemic meant that was
not possible. The meetings of the past two days took place via the
internet and were hence referred to as v-BAMs, `virtual Branch Abbots
Meetings'.

It is noteworthy that, for roughly a third of the participants, this was
their first abbots' meeting; their predecessors, who probably attended
that meeting in January 1993, have all recently relinquished their roles
as abbots. In advance of these meetings I was considering the fact that
there would be a new configuration of participants, but wasn't
especially concerned. Experience over the years has taught me that there
are grounds for trusting in the goodness and competence which results
from right practice. The six hours of these virtual meetings were not a
picnic: they were work. On the practical level alone, accommodating the
different time zones was tricky enough; some participants were up at two
o'clock in the morning. And inevitably, of course, there were issues
with technology: several of us were at school when the very first
computers were being invented, and not all monasteries have a high speed
internet connection. The more challenging aspect, however, was to do
with how we might raise matters of concern with each other, listen,
discuss and agree, or disagree, and at the same time honour our
commitment to harmony. Given the enthusiasm expressed by all
participants for holding more such events, I would say the meetings were
a wonderful success. I continue to marvel at, and feel grateful for, the
skill Tan Ajahn Chah displayed in his way of imparting the training, and
the beauty of the legacy he left behind. What he gave us was a way of
living in spiritual community with an emphasis on the spirit, not merely
on the form; his way was to cultivate a quality of mutual respect which
allowed for individual differences without compromising concord.

Now back to 1993. On the day of the cremation ceremony itself, there
were approximately 500,000 people\cite{cremation}
at Wat Pah Pong, including the Supreme Patriarch, Ven.
Somdet Nyanasamvara of Wat Boworn, and Their Majesties the King and
Queen of Thailand. From what I could tell, the \mbox{overriding} atmosphere
during this phenomenal event was one of reverence and respect, gratitude
and sadness. It is rare that such beings as Tan Ajahn Chah appear in the
world; it is natural that we feel grateful, and understandable that we
feel as if we have lost something precious. When the Buddha was dying
and was asked who would take over leading the sangha once he was gone,
he pointed to the Dhamma, saying that was to be the teacher. I am sure
Tan Ajahn Chah would likewise have pointed to the teachings.

Anyone who has listened to talks that I give\cite{talks}
would probably have noticed how often I refer to
Tan Ajahn Chah. Perhaps, also, they have observed that there are several
teaching stories or situations on which I regularly comment. Reflecting
back on the teachings of Tan Ajahn Chah, I discovered there are about
twelve points which particularly stand out; seven of these I have
written about earlier in these notes, but I will list them all here
again.

The first, is a teaching shared with me by a Western monk (earlier referred to
as Tan Cittapalo) who was visiting when I was still living at Wat Boworn in
Bangkok.
On that occasion I asked him what
Tan Ajahn Chah had to say regarding right view. Tan Cittapalo said that
Tan Ajahn Chah teaches that even the Buddha's instructions on right view
become wrong view when we are clinging to them out of unawareness. This
introduced me to the emphasis Tan Ajahn Chah placed on being mindful of
how we hold the teachings and the training, rather than merely
struggling to get the `right' idea and becoming attached to it.

The second teaching I would mention is that of experiencing Tan Ajahn
Chah's warmth and sensitivity at Wat Pah Nanachat when my foot was
seriously infected. Some teachers, it seems, insist on always presenting
the highest Dhamma and, unfortunately, in the process, tend to forget
the benefits of shared human companionship. On that occasion, where I
was suffering physically, Tan Ajahn Chah didn't tell me to tough it out;
he offered me his warm-heartedness.

Then there was a time when I was suffering intensely, mentally, because
of doubts I was having. Once more, instead of presenting me with the
ideal of how we must develop faith and strive on to overcome all fears,
he just smiled at me and said, \emph{I've been there.} If he had looked
at the floor, or out into space, and spoken about strengthening my
commitment, I would probably have forgotten the incident. As it was, he
looked at me directly and offered empathy; I still feel touched by it.
Having made the human connection, he went on to talk about his own
experience with doubts. At one stage, he said, the doubts were so severe
he thought his head was going to explode. He also helpfully pointed out
what I might change that could make a difference. He commented that,
`If, when we encounter that which is uncertain, and we insist it be
certain, we create suffering.' I trust deeply that he knew what he was
talking about.

The next teaching came in the form of an audio tape that Tan Tiradhammo
sent me when I was staying in Chiang Rai province, in Northern Thailand.
It coincided with a period when my grasp of the Thai language was
sufficient for me to start translating. That talk was called, \emph{Reading
The Natural Mind}, and was eventually printed in \emph{The Collected Teachings}\cite{collected}
(Chapter 22, p 237).
Paying close attention to the words and the meaning of that talk, I considered with
interest what Tan Ajahn Chah was saying about the difference between the
way unawakened beings and awakened beings relate to desire. Desire is
not the problem, despite what many Buddhist might say; it is clinging to
desire that creates suffering.

The fifth teaching is one that took place one morning when I had the
good fortune to be sitting under Tan Ajahn Chah's kuti before
alms-round, when an elderly female guest came to pay her respects and
take leave before she returned to the UK; an American nun, Maechee Kamfah,
was with her. They asked if Tan Ajahn Chah would say a few words into
the tape recorder so it could be taken back as a memento. As it was, she
received a fifteen minute teaching about Buddhist practice in which Tan
Ajahn Chah summarized the essence of the path and liberation. The talk
is now printed in \emph{The Collected Teachings of Ajahn Chah}\cite{collected},
page 233, with the title, \emph{Living With The Cobra}. The central message, as far as I
was concerned, is: don't invest too much in ideas of enlightenment; look
instead into that which is happening right here and now.

\begin{quotation}
\emph{Nibbana} is found in \emph{saṃsara}. Enlightenment and delusion
exist in the same place, just as do hot and cold. It's hot where it was
cold and cold where it was hot. When heat arises, the coolness
disappears, and when there is coolness, there's no more heat.
(\emph{The Collected Teachings of Ajahn Chah}\cite{collected}, p.235)
\end{quotation}

The sixth situation or teaching that stands out for me and has shaped my
life, stems from an incident which took place when Tan Ajahn Chah was in
hospital in Bangkok. I hadn't long before left hospital myself, after
having had surgery on both knees. Things hadn't gone to plan: the
doctors had initially indicated I would be in and out of hospital quite
quickly, but after three sessions under general anaesthetic and lots of
physiotherapy, my knees remained very stiff and painful. I look back now
and see how I embarrassed myself in front of the other disciples who
were visiting Tan Ajahn Chah at the time, by wallowing in self pity. I
said to Luang Por, `It really shouldn't be this way; this is not what
the doctor said I was to expect.' He looked at me with what I recall as
a mixture of puzzlement and kindness and said rather firmly, `What do
you mean it shouldn't be this way? If it shouldn't be this way, it
wouldn't be this way!' In fact there was no problem with the surgery,
the doctors, or with my body. My resistance created an imaginary
problem. Thank you, Luang Por.

There was another significant teaching occasion which I have already
described in this compilation, that took place at Wat Gor Nork, and I
would like to mention it again here. It occurred when Ajahn Jagaro, who
was then the abbot at Wat Pah Nanachat, and several other non-Thai
monks, visited Tan Ajahn Chah; they were trying to pin him down by
asking questions about exactly what is meant by the term `Original Mind'
and what actually is contemplation. A translation of this conversation
is printed on p.475 in \emph{The Collected Teachings of Ajahn Chah}\cite{collected}.

The comment from that Q\&A
session that has stayed with me all these years, is when Tan Ajahn Chah
was responding to a question about just how much \emph{samadhi} is
needed for true contemplation to arise. The questioner was wondering
whether we are supposed to be using thinking in the process of
investigation, or was it something else that was going on. Tan Ajahn
Chah emphasised that the point of the investigation was to come to
recognize that which is inherently still. He suggested that, as we
observe all that which is arising and ceasing, we should be enquiring,
out of `what' is this movement we call `mind' emerging.

\begin{quotation}
You recognize that all thinking is merely the movement of the mind, and
also that knowing is not born and doesn't die. What do you think all
this movement called `mind' comes out of? What we talk about as the mind
-- all the activity -- is just the conventional mind. It's not the real
mind at all. What is real just IS, it's not arising and it's not passing
away.
\end{quotation}

The eighth teaching was a conversation I heard reported took place between Tan
Ajahn Chah and the first Siladhara in our community, Sister Rocana.
I can no longer recall
whether at the time Sister Rocana had already taken up the training
or if she was still Pat Stoll. What matters though is the particularly
useful way Tan Ajahn Chah answered her question. The question she asked
was, `How is it possible to practise samadhi if there is no self to practise it?' He
answered, `When we are developing samadhi we work with a sense of self.
When we are developing vipassana we work with not-self.
When you know what's what, you are beyond both self and non-self.'

I don't know where the ninth teaching that I want to mention came from.
I do know that I used it one year on a page on our Forest Sangha
calendar. It is a particularly quotable quote\cite{centre}
and it is widely commented upon, not just by me. Tan
Ajahn Chah is reported to have said: `Don't be an arahant, don't be a
bodhisattva, don't be anything at all. If you are anything at all, you
will suffer.' In a way that is characteristic of Tan Ajahn Chah, he cuts
through all fixed positions -- all inclinations to become something. It
wasn't that he left the student of Dhamma with nothing, which might be
assumed when you read words such as these. In reality he left us with
the inspiration to give ourselves fully into the practice. From the
disembodied perspective of the written word and the concepts they give
rise to, that part of the message might be missed.

The tenth teaching came, once more, by way of a tape recording. Tan
Ajahn Chah had been staying at Wat Tham Saeng Phet (Temple of the Cave
of Diamond Light) when a Thai layman, Khun Dtoe, who usually lived in
England and was familiar with the developments in the early years at
Chithurst, went to visit him. On that occasion the Thai friend asked Tan
Ajahn Chah if he would like to record a message that could be taken back
to the sangha at Chithurst. Once the message had been made, the tape
recorder was left running and an informal conversation took place. The
portion of that conversation that I recall was when Tan Ajahn Chah was
speaking about the `real practice'. I don't know if this has ever been
translated and published anywhere but what I remember now is how he was
saying that, sometimes Buddhists think that sitting in formal meditation
is the real practice. In fact that is the preparation. The `real
practice' occurs at the point where a mood impacts on the heart (Thai:
\emph{arom gratop jai}). He was emphasising how we need to do the
preparation so that when we are about to be triggered into reactivity,
we are there for it: we don't default to clinging and creating
suffering. That he highlighted in such a way the real point of practice,
and the work that we need to do so as to be prepared, was a gift.

A less appealing, but still profoundly important teaching might have
come from some notes I scribbled down of translations by Tan Varapañño.
Apparently Tan Ajahn Chah commented something along the lines of, `When
practice is going well, it will take you to the point where it feels as
if you are hanging out with your best friends, and the Buddha comes
along and says, break it up.' I really did not want to hear this
teaching. Indeed, it took many years before I was able to see the point.
There was considerable resistance. Thankfully, eventually, I came to
appreciate that what the teacher was telling us was that the things we
feel we hold dearest are, in truth, the very things we are most attached
to, and will really not want to let go of; they are our addictions. Only
after having been a monk for many years, did I come around to even
beginning to admit that I had undermining addictions, and that feeding
them was a way of avoiding looking at deeper issues. It wasn't that I
was hooked on imbibing substances; my coarsest addictions were travel,
sugar, caffeine. And all three of them were expressions of the deeper
addiction to distraction. Definitely I did not want to stop feeding
them. I tried a number of times over the years, but always went back to
them again. Now that I have been clean for a good while, I think I am
safe to say I have a reasonable handle on them. International travel
stopped about ten years ago. I gave up nearly all sugar (and honey and
maple syrup etc.) a bit over two years ago, and caffeine just over one
year ago. These days I can look at my passport (which has only blank
pages in it) without giving rise to painful longings to visit friends in
New Zealand and walk along beautiful beaches. I can see a tub of Manuka
honey and, while I might start salivating, it is not a struggle to leave
it be. And the thought of consuming caffeine holds very little
attraction. Here I won't go into the issues that were driving me to
distraction; suffice to say I am glad that I didn't wait until I was on
my deathbed before beginning to address them.

The final teaching of Tan Ajahn Chah's that I want to mention is the
simplest and most straightforward, `In the end there is just patient
endurance.' We develop tricks and techniques that help us keep moving
forward on this spiritual journey, but a time will come, probably
several times, when nothing works any more. None of our insights or
ideas or strategies free us from the obstruction with which we feel
confronted. If we insist on making progress, we could hurt ourselves.
There are times when we have to surrender and willingly submit ourselves
to bearing the unbearable. It is not that we are not doing anything;
what we are doing is learning to humbly acknowledge our limitations.

\sectionBreak

After that first tsunami of a Rains Retreat at Wat Hin Maak Peng in
1975, when I had been left with a subjective sense that, instead of my
personality having been transcended, it had been shattered and my heart
scarred, I had the thought that somehow I had to find a way to
reconstruct a more functional sense of self; evidently the one I had,
had not been fit for purpose. In pursuit of that hopefully more stable
and functional sense of self, I went to live with Tan Ajahn Chah and
Ajahn Sumedho. From conversations I had with others who had taken up
training within Tan Ajahn Chah's monasteries, and from reading a few
brief transcribed, translated teachings, I had the impression that this
was the most suitable place to be to do the work that needed to be done.
Forty something years later I have huge gratitude to Tan Ajahn Chah,
Ajahn Sumedho and the sangha that has surrounded them. I don't think it
is too much of an exaggeration to say I owe them my life. I love this
life that I am living. For sure, there are periods that I would prefer
were otherwise, but I don't find myself looking with envy at anyone
else's life.

