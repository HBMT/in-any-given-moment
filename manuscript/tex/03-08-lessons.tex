\chapter{Lessons to Learn}

Once a week the abbot of Wat Boworn, Phra Somdet, would lead a class in
English in an air-conditioned room adjacent to his residence. These
meetings were open to members of the sangha and laity alike. They were
lessons in the \emph{pariyatti}, or theoretical aspects of Dhamma, and
were well attended. Sitting and walking meditation are obviously
important aspects of this path, but if we don't have a good grounding in
\emph{pariyatti}, our meditation can be heading off in an altogether
unhelpful direction. Personally I haven't read many Dhamma books in my
life - `\emph{What the Buddha Taught}' by Walpola Rahula,
`\emph{The Word Of The Buddha}' by Nyanatiloka, and \emph{The Dhammapada}\cite{dhammapada},
being three of the most significant -- but that
doesn't mean I don't respect the \emph{pariyatti} aspect of training. People
are different: some people need a lot of theoretical explanation before
their faith is strong and clear enough, others need less. Besides, some
people learn better by listening than reading. It is only relatively
recently in human history that the majority of humans have been able to
read and write. I am one of those who enjoys learning from listening.

At one of those Wednesday evening classes, the Venerable Master Hsuan
Hua, abbot of The City Of Ten Thousand Buddhas in California, was
present. He had with him several of his Western disciples, including,
Heng Sure and Heng Ch'au. Heng Sure and Heng Ch'au had completed their
\emph{Three Steps, One Bow}\cite{steps} pilgrimage up the coast of California and
their group had been visiting Malaysia. I recall the Master having an
interesting response to the many questions that were asked of him: he
just wouldn't answer. This went on for quite a long while with people
asking for his thoughts on such subjects as \emph{arhats} and
\emph{bodhisattvas}, \emph{anapanasati, vipassana}, the Four Foundations
of Mindfulness, and so on. Then the bell rang for evening chanting and
that was the sign the meeting would conclude. At that point the Master
started talking. What he said, and what stayed with me, was that
\emph{as a teacher his job was to trick us}. It is because of all the
games that we play in our minds that we stay lost. His job was to trick
us out of believing in our games. Thank you, Master Hsuan Hua.

One of the regular attendees at these Wednesday meetings was Mrs Josie
Stanton, the wife of the ex-US \mbox{Ambassador} to Thailand. She had
been living in Bangkok for many years and held up Phra Somdet as her
Dhamma teacher. She was also a very generous supporter of quite a number
of the Western monks and novices at Wat Boworn. The house where she
lived when she was in Thailand was in the lush gardens of Her Majesty
The Queen's mother's residence and was within walking distance of Wat
Boworn. She took her study and practice of Dhamma very seriously and I
believed she genuinely appreciated her conversations with her teacher Phra Somdet.
On one occasion, after she had been travelling around Thailand and was
upset about the decrepit state of many of the monasteries that she had
seen, she asked Phra Somdet how it could be that this wonderful and
precious Dhamma teaching could end up looking so unseemly. Phra Somdet's
calm reply was that she shouldn't be too disturbed by the deterioration
of buildings and institutions as they too are subject to the law of
impermanence, \emph{anicca}: the Dhamma is not \emph{anicca}, just the
structures.

As weeks went by, the question of where I was going to stay and whether
I would request \emph{upasampada} (acceptance into the sangha of monks)
became more relevant. In July the annual Rainy Season Retreat (vassa)
would begin and at that point I would be obliged to stay put for three
months. I'm not sure which came first: the decision to request
\emph{upasampada} at Wat Boworn or my meeting some very friendly monks
from Wat Hin Maak Peng. Wat Hin Maak Peng was Ajahn Thate's monastery
near Nong Khai in North East Thailand. Tan Ajahn Thate was the
meditation master under whom Phra Somdet Nyanasamvara had trained once
he had completed his studies. Another Western monk, Tan Dhammachando,
and I took the decision to travel up-country to Wat Hin Maak Peng for
the vassa. If I recall it correctly, my \emph{upasampada} took place at
Wat Boworn around the time of the full moon of May 1975, which was
Vesakha Puja, the occasion marking the birth, enlightenment and passing
away of the Buddha. Phra Somdet Nyanasamvara was my preceptor, and Mrs
Josie Stanton sponsored and presented my robes. Great gratitude to Phra
Somdet, and a sincere thank you to Josie Stanton.

Wearing the robes of a fully accepted Buddhist monk (\emph{bhikkhu})
brought with it an intensification and many more new lessons. It was
clear that there were stronger expectations to conform to higher
standards of conduct as well as requirements to participate in various
ceremonies. Now, instead of the modest ten precepts that defined the
conduct of a novice (\emph{samanera}) there were 227 that had to be
followed.

Early on after taking up the monk's training, I received a painful
wake-up call regarding the consequences of heedless speech.

One evening, after the group meditation in Phra Somdet's residence, I
made a joke to one of the other participants about how I had been just
about to drop into \emph{jhana} when a mosquito bit me and ruined
things. A short while later, on my way back to my room, I was reflecting
on what I had just said, and was suddenly struck by a truly terrible
upthrust of guilt and anxiety at the thought that I had just made a
false claim to a superhuman state -- an offence of automatic disrobing.
I was so shocked and confused I straight away made my way over to see
Phra Somdet. That he was engaged at the time in a meeting with other
senior sangha members didn't prevent me from approaching him and
explaining how I thought I might have just broken one of the four most
serious monks' rules. He kindly gave me his attention and was quick to
reassure me that, since I had obviously been making a joke, there was
nothing serious to be concerned about.

As it happened, I should have been concerned not because I had committed
a disrobing offence, but that obsessive guilt, a fault line in my
psyche, had been revealed; such a susceptibility wasn't going to
disappear just because of a few words from a kindly abbot.

