\chapter{Being Different}

It must have taken considerable effort on the part of my parents to be
so supportive of my evolving character. New Zealanders back then seemed
to proudly present themselves as a nation defined by Rugby, Racing and
Beer. Despite my father's attempts to direct me towards enjoying the
sport, I think I played rugby only twice during all my school years and
remember on one of those occasions kicking the ball in the wrong
direction. Horse racing was associated with gambling, which in the
Morgan household was seen as evil. The mere smell of beer (at that early
stage of my life) was repulsive to me. Walking past a pub and catching a
waft of the hot, stale air that drifted out -- a combination of tobacco,
crisps, beer and Dettol used for cleaning up the vomit -- was a powerful
disincentive. For the duration of most of my early life, laws in New
Zealand required pubs to close at 6 pm. That meant for many that once
work finished at 5 pm, they would rush to their favourite watering hole
and drink crazily for an hour before loading up with an armful of
bottled beer and driving home. When the law changed, pubs were permitted
to stay open until 10 o'clock, but I think the immoderate drinking
habits of Kiwis took longer to change.

The encouragement I received from those who cared for me was a gift --
not only the support given by my parents but also by my Nana. With
hindsight I see now there were strong undercurrents of fear and
judgement around sensuality and pleasure, but I credit both my parents
for daring to allow me to follow my own creative pursuits. It mustn't
have been easy for them considering their own puritanical upbringing and
the pressures that inevitably come with living in a small country town.
Once, when I went to stay on a farm with some friends of our family, the
children there had their bedroom walls lined with awards they had won on
Calf Club Day. Calf Club Day was the occasion when the sons and
daughters of farming families would bring their pet calf or sheep to
school and parade them around the playground and be judged; I'm not sure
what criteria were used for awarding points. I expect I liked it about
as much as I liked the rugby matches that we watched. (Years later I
developed a keenness for watching rugby, but by that time I was tuning
into the spirit of cooperation, even selflessness).

My bedroom walls, on the other hand, were proudly plastered with awards
I had received at the local Flower Show -- a kind of Morrinsville
version of the Chelsea Flower Show. These shows took place three or four
times a year, coinciding with the seasons. Still I can recall the impact
of a wonderfully intoxicating fragrance that hit me when I walked into
the hall where the show was being held. Many of the prizes given out at
the show were for exceptional blooms or for particularly impressive
pieces of fruit or vegetable; the awards I won were for flower
arranging. I am thankful towards my parents for being so daring.

