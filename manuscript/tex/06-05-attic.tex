\chapter{Ordeal in the Attic}

In part because of the confidence that had arisen from being more in
touch with my body, and because of the sense of hope that the breathing
exercise had given me, when we were preparing for the 1990 Winter
Retreat at Chithurst, I asked Ajahn Anando if I could determine to spend
the two months in solitude. I was interested to see what the increased
intensity would bring up. There were other motivations as well, but
having a chance to meet myself in solitude was appealing. It turned out
that I overreached with that determination.

The room I was living in was in the attic, and the floor space was
roughly three by four metres (there were also two sizable raised
platforms). I covered the windows with tracing paper so light could come
in but I couldn't see out. The idea behind that was to intentionally
frustrate any impulses to distract myself from what was going on inside
of me. Somebody left food outside my door each day, and during the
community's morning chanting period, I would take my slop bucket down
and empty it in the bathroom. Other than that, the only time I left the
room in those two months was to participate in the fortnightly
recitation of the Rule.

Very soon after the retreat began, I was assailed by intense anxiety.
Much of the following two months was spent doing whatever it took to
survive the onslaught of fear and dread. I find it impossible to compare
the horror I endured during those two months with what had occurred at
Wat Hin Maak Peng in my first Rains Retreat, since somehow I was not the
same person. I had accumulated many experiences over the approximately
sixteen years since then, and acquired new skills. None of those skills,
however, protected me from having to go through what turned out to be
another agonizing ordeal.

At one point during the retreat a severe storm struck and, as a result
of my confusion, I was consumed by feelings of fear that I personally
had caused the storm. On this occasion I broke my silence to enquire
whether anyone had been killed in the storm. To say I was consumed, is
partly an exaggeration, since if I had truly been consumed I would not
have survived. I did survive, but the intensity was way more than I
bargained~for.

After the Winter Retreat ended and I again joined the community for
evening puja, Ajahn Anando invited me to give a talk. My memory of that
occasion now is that the act of opening my mouth to speak required such
a huge amount of effort -- so much intensity had built up over the two
months -- that I spoke only one sentence.

It took some time before I could find a sense of balance again. My
confidence hadn't been shattered -- not at all -- but it now had a
companion called modesty. All those hours spent bathing in exquisite
bodily sensations associated with the breathing exercises, hadn't driven
me crazy, but they had led to a degree of delusion. I think the bass had
been turned up a bit too far.

