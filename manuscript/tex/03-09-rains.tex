\chapter{First Rains Retreat}

It was probably in June 1975, not long before the beginning of the Rains
Retreat for that year, that Tan Dhammachando and I travelled to Wat Hin
Maak Peng. We were received in a friendly manner by the monks that we
had met earlier at Wat Boworn and were given our accommodation. I was to
be staying in the meditation hut (kuti) that Phra Somdet had lived in,
which felt like an honour. During the initial period there, we would
bathe each evening in the Mekong River which flowed gently by, just
beneath my kuti. The point where the monastery was situated was one of
the narrowest sections of the river and it was easy to see Laos on the
other bank. Shortly after we arrived, however, the Russians invaded Laos
and it was deemed too dangerous for us to bathe there anymore, so we had
to walk inland a wee way to a stream. There were times when the Russians
could be seen speeding by in their boats and at night tracers could be
seen being fired. On one occasion a farmer who apparently held property
on both sides of the river -- in Thailand and in Laos -- was shot at
whilst attempting to bring his tractor back across into Thailand. I
think his tractor ended up sunk in the river.

Not long after we arrived at Wat Hin Maak Peng, the two of us were
invited to come to Tan Ajahn Thate's kuti. As I recall, the teacher
asked us a little bit about our understanding of practice and then
offered us advice. The instruction and encouragement he gave us
continues to this day to inform my effort in practice. He told us that
\emph{the primary task in practice is to learn to differentiate between
the heart and the activity of the heart, awareness itself and the
activity taking place within awareness}. This was a simple but truly
precious gift of Dhamma.

My grasp of the Thai language was still minimal so I depended on a
translator to understand when the Ajahn gave talks. The very few books
in English that I had with me proved to be a valuable source of
inspiration. I was committed to making progress and felt fortunate to be
in such a supportive environment. The booklet of \emph{Questions and
Answers with Tan Ajahn Chah} that I mentioned earlier was extremely
helpful; it was clear, succinct and relevant. Much of my time was spent
in silent sitting and walking meditation. Every few days, I would attach
to one of the pillars under my kuti a verse from the treatise, \emph{On
Trusting In Mind}, by Master Tsen Tsan; then, as I walked up and down, I
would contemplate it, and in the process, commit the words to memory
(although I don't think I reached beyond the initial few verses).

Tan Ajahn Thate, by that time, was about seventy-three years of age and
had already been diagnosed with leukaemia. His health was a subject
never discussed with me, but I was aware that he was limited in how much
he could take on. It didn't stop him from doing his walking meditation
or being available in the evening for a very vigorous Thai-style
massage. Outwardly his appearance was extremely gentle and his voice was
high-pitched to the point of being almost squeaky. However, when the
Thai monks and novices massaged him, they would dig their elbows deep
into his muscles with a force that even a young person might not be able
to tolerate. Despite outer appearances he was very strong and lived
until he was
\href{https://en.wikipedia.org/wiki/Ajahn_Thate}{\underline{ninety-two
years of age}} {[}36{]}.

Within the sangha of forest monks in Thailand there is a beautiful
tradition of, just around the time of the beginning of the Rains
Retreat, many monks taking the opportunity to visit elders in nearby
monasteries, to pay their respects. For the sangha at Tan Ajahn Thate's
monastery that meant we went to visit Luang Por Kaaw who lived
relatively nearby. Such occasions are highly ritualised, starting with
the visiting sangha offering a tray of candles, flowers and incense, and
then requesting forgiveness from the elder for any wrongdoing. The elder
then reciprocates by asking for forgiveness from the visiting sangha.
More or less the same sequence is performed with each elder; what might
vary would be the degree of conversation that followed the ritual. I
don't remember now, but it might have been that we visited other elders
on that occasion, though Luang Por Kaaw is the only one I recall. The
lasting impression I have of that visit is that I was privileged to have
the opportunity to pay my respects to him. Also I have an impression of
meeting someone with extraordinary dignity and strength. His body was
very frail by this stage. Although he and Tan Ajahn Thate were both
disciples of Tan Ajahn Mun, and would probably have been considered as
contemporaries, Luang Por Kaaw was a lot older. It is difficult to
estimate the degree to which one might be influenced by spending time in
close proximity to highly purified beings, and I am cautious to not
invest too much in speculation; however, I don't dismiss the possibility
that such association can have profound consequences. If one is
susceptible to being influenced, or if kammic affinities are involved,
spending any time in the company of great beings can be a great
blessing.

Back at Wat Hin Maak Peng, even though we were aware of the unwelcome
activity taking place on the other side of the river, life in the
monastery was peaceful and calm. Gradually I felt like I was settling
into a comfortable enough daily routine: morning alms-round, meal, rest,
meditation, bathing, puja -- ending the day with gathering at Tan Ajahn
Thate's kuti for the massage. That was all about to change.

