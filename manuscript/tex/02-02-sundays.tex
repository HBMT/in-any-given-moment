\chapter{Jumping Sundays}

One rare splash of colour that still shows up amidst the memories I have
of that drab period is the `happenings' which were taking place in
Albert Park. This was 1970: London had Hyde Park Corner, San Francisco
had Haight Ashbury, and Auckland had Albert Park. Situated in central
downtown Auckland, near the Auckland
University campus, this handsome stretch of exquisitely manicured
gardens, massive mature trees and a Victorian band rotunda, provided a
great location for weekend get-togethers. I can't recall for sure how I heard
about these \emph{Jumping Sundays} as they were called, possibly through
some of the guys working on the dye floor at Holeproof Mills. Several of
them were members of the PYM, (Progressive Youth Movement)\cite{pym}, a radical
anti-Vietnam-war-and-other-things movement that was upsetting the New
Zealand Establishment. I think I was susceptible in those days to
feeling drawn by anything that looked like it might undermine the
society that I was busy blaming for my misery.

There was music, politics and probably early promoters of the
`back-to-the-earth' movement, also Hare Krishnas, and lots of
colour. Thinking about it now, those events did serve as a hint of
potential aliveness, which was helpful. A dreadful sense of darkness and
confusion was building within me. I had sought solace on a couple of
occasions at local churches -- one on Upper Queens Street and another
at Mt Roskill -- but I walked out of them both feeling more
disillusioned than when I went in. It seems that that is what we tend to
do when we feel pushed to our limits: we try things we perceive as
having worked in the past. This time nothing was working, though those
Jumping Sundays did perhaps quicken the spirit of exploration. To me
they seemed to be in sync with what Professor Jung had to say and what
Marshall McLuhan was about. They may not have provided the missing
ingredient for which I was desperately seeking; however, I would say
they were a harbinger of hope.

The Vice President of America, Mr Spiro Agnew, visited Auckland around
then, very likely trying to muster support for their war in Vietnam. He
was successful in garnering support for street demonstrations and
generated a lot of ill-will. I wasn't convinced though that getting
angry at Agnew or America was the solution.

This was also the time of Peter Fonda's movie Easy Rider which
introduced me to the expression `doing your own thing'. In itself those
few words might not sound like much; however, for me that turn of phrase
somehow symbolized the momentum that was gathering within the
counterculture. On many levels, the ways of doing things that had
previously been considered acceptable had had their day - the old ways were being replaced. Despite all their
might, America never managed to win their war in Vietnam. Man was able
to walk on the moon, but here on earth we weren't managing very well to
get along together.

Earlier I mentioned how I find reading and mathematics challenging. If
that hadn't been the case back then, I might have stayed on at Holeproof
Mills. Now I feel grateful for those difficulties. Of course I wasn't
grateful at the time; I really did want to feel like I fitted in
somewhere.

The despair I felt towards Christianity took me to the point where I
formally took leave from the church. On my Certificate of Confirmation,
issued by the Presbyterian Church of New Zealand, it is written
(presumably by me), `Now on the Monday 6th April 1970 denounce my
membership of the established church of N.Z.' (The word `I' was missing,
and it should have said `renounce'). It seems strange after all these
years that I still have this certificate. I recently found it bundled
with my old school report book and other papers that I believe my mother
gave me on one of my final trips back home. Surely I hadn't given her
that certificate?

For reasons I can't recall I decided to take a trip to visit my uncle,
one of my father's older brothers, who lived about a hundred and fifty
miles south of Auckland, in Taupo\cite{taupo}.
I imagine I was just trying anything to see if something would work. On
the trip back from Taupo to Auckland, something did work. I was
hitchhiking, which was a thoroughly normal way of getting around in
those days, and was picked up by a couple of university students. In the
course of the conversation we were having, I expressed an opinion about
something which prompted one of the guys in the front seat to turn
around and, in a challenging manner, say to me, `Don't you realize that
you have been brainwashed to think that way?' It felt like something hit
me. The next thing that now I can recall was arriving at my brother's
apartment in Hamilton where I was going to stay the night. At some stage
I started writing down my thoughts, and I had a lot to say. They weren't
about aimless confusion or rambling resentments, rather they were
observations. I felt alive again, and there were no drugs involved.

My brother was friendly with the two girls who lived next door, also
university students, and somehow we met up and I shared with them my
mood of inspiration. They suggested instead of using the word
`brainwashed' I could try using `conditioned'. That fitted. Just the
suggestion that thinking was a conditioned process, seemed to release in
me a lot of energy. At least that is how I now perceive what happened. I
wrote and I sang, in particular I sang along with a record of a song
called \emph{Look Through My Window}\cite{window} by The Mamas and the Papas. What
inspired me were the words
about letting go. I seem to remember that I sang my heart out
as I had previously sung hymns alone by the river near Morrinsville. Those words
resonated within in a way that felt relevant. A big thank you
to those university students and to the Mamas and the Papas.

Thank you also to Leonard Cohen. There was a poster on the wall in the
living room of those two girls with the words of a Leonard Cohen song.
What a gift. It puzzles me how some people would find his music
depressing. Yes, melancholic, but so beautiful. Many years later when a
friend who was a personal acquaintance of Leonard Cohen's, gave me a
copy of one of his books, he had signed it and added the note `to Ajahn
Munindo with a deep bow'. There was an embossed circular imprint
on one of the front pages which said `Order of the United Heart'. These days, when I think back, I have the impression that in 1970 on that trip back from Taupo my heart started on the journey towards unity. 
