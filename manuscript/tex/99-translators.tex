\chapter{We Are All Translators}

\emph{Edited and reprinted from \emph{Unexpected Freedom}\cite{unexpected} by Ajahn Munindo}

On this occasion I would like to discuss the effort that we are all
making in our work to translate the practice of Buddhism. Maybe it
hasn't occurred to you that you are a translator. I would like to
suggest that we are all translators, in the sense that the teachings
which we have inherited from our Asian brothers and sisters cannot be
simply uprooted and then replanted in another place on the planet
without due attention to the differing environmental conditions. While
we gladly recognize there are certain universal principles in the
teachings, there are obviously also some aspects that are relative to
culture and tradition. So the manner in which we are taking up Buddhist
practice and the kind of effort we are making is our contribution to
this shared task of translation. This is as important as, if not even
more important than, the work of translating texts. Can we become more
conscious of our contribution to this task as we make it?

I have often spoken about identifying what pertains to form in the
teachings, as opposed to what is in the domain of spirit. Mixing up
these things can mean that we put emphasis in the wrong place, and in so
doing we end up with results that we didn't expect. But sorting out such
matters is far from easy. The sparkling radiance of these exotic
teachings and techniques readily dazzle us, especially since we have
been in the dark for so long. We might feel contented to settle for that
initial bedazzled response to this new-found light. However, the Buddha
was consistent in his encouragement to not be fooled by the way things
appear to be; only after careful scrutiny should we fully accept
something to be true. The point of this encouragement was that we should
come to know directly for ourselves the benefit of the teachings. On the
other hand, it is not suggesting that we dismiss things because we don't
see the sense in them straight away. So how should we approach this
matter of discerning the spirit of the teachings?

\section{Discerning Essence}

The point of our taking up the Buddhist Way is to find support for our
heart's yearning to be free, and it is natural that we begin by
observing the way in which others engage in practice. But although a
particular technique or system has been applied successfully by one
person, it does not mean that it will work for everyone. It is wise to
ask, `What is important to me? What is it that is quickened in me when I
see a teacher, or hear a teaching?' I like to think about religious
forms as being like conventions around eating. If we are hungry, the
point of eating food is to become free from the discomfort of hunger.
Whether you go to a Japanese restaurant and eat with chopsticks, or a
Thai restaurant and eat with a spoon, or a place where you use a knife
and fork, the conventions are not the point. The point is that we are
fed. So it is with practice. The point is that our hearts no longer feel
hungry. So our task is to identify what it is that is nourishing, and to
focus on that. This is identifying the domain of spirit. If we give this
task priority, whatever this may mean in our case, then I feel that the
forms that support the spirit will evolve rightly. Not to give spirit
due priority means we might be missing out on what is most of value in a
religious tradition.

Something we could miss out on is a creative participation in our
enquiry. If our translation is going to be relevant, we have to be
creatively involved with it. Yes, we respect the forms that we inherit;
we have to begin with learning that which has been tried and tested. At
times this requires that we simply do what we are told; at this stage,
learning the form is the priority. For example, if we are learning T'ai
Chi, we don't question the master because the movements feel
uncomfortable, and then on our third lesson make some suggestions as to
how the form could be altered. No: although in the beginning we might
feel awkward and look a little silly, we simply learn the form and
humbly accept that it doesn't yet feel right, remembering that these
forms are supports for spirit -- in this case, working with the Chi. If
we practise the form with commitment then we eventually learn to relax
into the form. Then perhaps the Chi -- spirit -- starts to move, and we
are grateful.

So we are not dismissing forms. We take up the form and wait very
patiently until we are settled into it. Then we feel for the spirit
moving. When we are fully familiar with the spirit then that becomes the
priority. Now we are in touch with the essence. This way, we will be
able to change the forms without compromising or obstructing spirit. If
we attempt to adjust things too soon, based on our likes and dislikes,
we could be creating obstructions.

A friend of the monastery relates a story about a valuable lesson he
learned during his first year of training under a highly reputed
cabinet-maker. Starting out on his apprenticeship as a young man, this
friend had been given a brand-new, top-of-the-range hammer as a gift
from his father. It was perfectly balanced, with a wooden handle -- just
what an aspiring cabinet-maker would dream of. His master instructed him
numerous times on how he was to hold his hammer towards the end of the
handle so as to gain the best swing. But although a beginner, our friend
thought he knew better. If you are new at carpentry, it does feel easier
to hold the hammer nearer the head; you feel like you can be more
accurate. After a number of reminders, the boss one day took hold of our
friend's beautiful hammer and proceeded to saw half the handle off,
declaring that since he was not using that half he obviously didn't need
it.

\section{Holding Rightly}

We respectfully look at the practices that we take on, feeling for the
spirit. The teacher says practise this way, don't practise that way. We
do what the teacher says but, as we proceed, we are checking and
feeling. We do not just believe. It is necessary to trust our teacher,
but trust is not mere belief. There is a big difference between trusting
in what teachers are offering and believing in them and their
techniques. Many of us came into this path with conditioning from a
different religious tradition; one which holds up belief as the whole
point. Such an approach cannot be applied in Buddhist teachings.

In Buddhism, beliefs are functional. We believe in things like rebirth,
for example; we believe that when we die we are reborn. But most of us
don't know this to be objectively true. I don't know that it's true. I
believe it, but the way in which I believe it means that if somebody
says it is all nonsense, then we don't have to quarrel. I don't need
them to agree with me. I choose to hold a belief in the process of
rebirth, but I hold this belief lightly. The belief is not the end
point.

When our teacher tells us to practise in a certain way, we take this
teaching on trust. The Buddha used an image of a goldsmith purifying
gold to describe our effort to purify our relationship with the
teachings; it's a process of removing the dross over and over again
until we get pure gold. We purify our relationship to the teachings by
cultivating enquiry and feeling into how they work for us. When we are
practising various exercises and techniques and we find something is not
working, we start having doubts. That's fine. Doubts do not have to be
an obstruction in our practice. Doubts can also indicate that the spirit
of enquiry is alive within us.

Enquiry is something that comes naturally to us in the West, and we
should value it. This capacity for enquiry is one of the contributions
we are able to make to the task of translation. We shouldn't
automatically assume that, because our experience appears to be
contradicting what someone else is saying, they are right and we are
wrong, or vice versa. We listen. We feel for what is being said. We
patiently enquire. And if we proceed with a willingness to go gradually,
translating everything we experience into practice, then I trust that an
organic and lasting understanding will be borne out of our effort.

As we discover for ourselves what works and what does not, a confidence
grows, bringing benefit to us individually and to the community at
large. Discovering our own true way of practice is like finding a good
restaurant; the first thing you want to do is take your friends along.
My sense is that if we arrive at such confidence in a gradual way by
respectfully questioning as we go along, we spontaneously find our own
ways of expressing it. We are not just using other people's words. Such
confidence will spill over -- we won't even notice it happening -- but
others will.

\section{The Two Orientations of effort}

One way of illustrating this task of translating the practice is to look
more closely at how we internalise the teachings. If the kind of effort
we make is not coming from a place of confidence, not only are we
wasting energy, but we could actually be doing ourselves harm. I see a
lot of confusion in the way many meditators relate to the different
types of effort required in practice. There is sometimes quite a naïve
hope that by endlessly plugging away, doing what they have been doing
for years, something good will come out of it.

These days I feel convinced that there are basically two different and
distinct orientations of effort -- goal-orientation and
source-orientation. For many years I tried to practise by having a goal
`out there' to strive towards. My understanding of the teachings as I
heard them was that this was what I should be doing. I received
instruction in various techniques, which were oriented towards
realization of this goal. The goal was called `enlightenment' or `the
deathless' and so on, but it was always `out there in the future'. I was
encouraged to make great effort to achieve the goal and to break through
those things that obstructed progress towards it. And even when the
words didn't directly say that the goal was `out there', that was the
message that I heard. Eventually I found myself in a terribly
frustrating knot. At one point I felt that my whole commitment to
practice was seriously challenged. Gratefully, with some help, I came to
realize that the struggle I was caught in was about the very feeling of
having to get somewhere. I had internalised a sense that I had to fix
myself somehow, change what I was and get somewhere else. Clearly it
wasn't working, so I gave up. In giving up I experienced a feeling like
that of beginning a journey home. What a relief! Just as I was beginning
to wonder if the journey itself was about to come to a sudden and sad
ending, I felt I could settle into something perfectly natural. And with
this shift came a feeling, initially unnoticed, of being genuinely
personally responsible. This was new.

From this experience I developed a practice characterized by a strong
sense of trusting in that which already exists. This was altogether
different from striving towards achieving some goal. The effort that
this new appreciation spontaneously called forth was `not seeking'. My
attention was -- and is -- looking and feeling in this moment;
enquiring, `Where and when do I decide this situation is somehow
inadequate or wrong or lacking?' I found that I was able to notice quite
clearly when I was imposing on life some notion of how it should be,
thinking, `it shouldn't be this way, it should be that way'. My practice
became that of simply, but resolutely, being with this awareness. Now I
refer to this as source-oriented practice -- in which a trusting heart
intuits that what we are looking for is right here, not anywhere else,
not somewhere out there.

\section{Faulty Will}

Many of us start meditating with a faculty of will that is not doing its
job properly. In trying so hard and for so long to wilfully fix
ourselves up, we have abused the very faculty of will. If you abuse
alcohol for a period of years and become alcoholic, you can never again
have a social drink. In our case we have overused the will. Now we can't
help but habitually overdo it and interfere with everything that
happens. We often feel unable to simply receive a situation and gently
apply will to direct and guide attention. If we find something that we
think is wrong we tend to automatically slam an opinion on it -- that
`it shouldn't be this way', and then we set about wilfully trying to fix
it.

For those of us who suffer this dysfunction, engaging the will as the
primary tool of meditative effort just doesn't work. Whereas, if we
disengage from willing and abide in a mode of trusting in that which
already exists, trusting in reality and truth, if we simply stop our
compulsive interfering, then an accurate and conscious appreciation of
that which already exists will reveal itself.

If you follow a path of practice that is goal-oriented, you can expect
to have a clear concept of what you should be doing and where you should
be going. There will be appropriate actions to take for any obstacles
that you might encounter. But if your path of practice is
source-oriented it is not like this at all. Here you come to sit in
meditation and you might begin by checking bodily posture, making sure
the back is upright and the head is resting comfortably on the
shoulders, chest open, belly at ease; and then you sit there, bringing
into awareness the sense that you don't know what you are doing. You
simply don't know. All you know is that you are sitting there (and there
may be times when you can't even be sure of that). You don't hang on to
anything. But you do pay attention to watching the tendency of the mind
to want to fix things. You focus interest on the movement of the mind
towards taking sides, either for or against.

Usually when I sit in meditation I do nothing. I assume a conscious
posture and simply observe what's happening; maybe the mind is all over
the place -- thinking about the liquorice I had the other night at
somebody's house, or about how it's a pity the sun has gone in, or about
how I will be in Beijing this time next week, or about how the monks at
Harnham sent an email asking whether they should use gloss paint for the
doors in the monastery kitchen, and so on. Such thoughts might be going
through my mind; they're nonsense, but I do nothing with them.
Absolutely nothing, until I start to feel a little bit uncomfortable,
and then I watch to see where that discomfort is coming from. It is
always coming from the same place: `I shouldn't be this way. I should
be\ldots{} My mind should be clear, I shouldn't be\ldots{}' Once this movement is
identified, a settling occurs. When we identify that which takes us away
from our natural feeling of centredness, we come home. This is not the
same kind of effort one would be making in goal-seeking practice.

\section{Knowing for Yourself}

Most of us have a natural tendency to incline towards one of these two
orientations of effort. Some people are contented and confident when
they have a clear sense of the goal -- that is where they are supposed
to be going. Without a clear idea of where they are going, they become
confused and anxious. Others, if they focus on the idea of a goal, end
up depressed, feeling like they are failing: trying to stop thinking,
they fail; trying to sit properly, trying to make themselves happy,
trying to be loving, trying to be patient, trying to be mindful -- they
are always failing. What a terrible mistake! The worst disease of
meditators is trying to be mindful. Some quit, feeling they have been
wasting their time. However, if we realize that we don't have to do
anything other than be present with an awareness of the tendencies of
the mind to take sides for or against, then we settle.

These two orientations are not mutually exclusive. It is useful to
understand how each of them has particular merits at different stages of
practice. In the beginning, to build up some confidence, it is necessary
that we have a good grasp of techniques. Even though we may relate more
readily to source-oriented teachings and practices, if we haven't yet
found a firm foundation on which to practise, or if we have found that
firm foundation but our life is very busy, it can still be appropriate
at times to make effort to exercise will and focus.

I encourage people in the beginning to be very disciplined and to count
their breaths, one to ten, ten to one, every out-breath, one, two,
three, up to ten, ten, nine, eight, down to one, being quite precise in
the effort made. This way we get to know that our attention is indeed
our own. We are not slaves to, or victims of, our minds. If our
attention is wandering off and we get caught up in resentment or desire,
then we need to know that we are responsible for that. Our practice,
whether we are goal-oriented or source-oriented, is not going to
progress until we are clear that we are responsible for the quality of
attention with which we operate.

To reach this perspective it may be necessary to exercise a rigorous
discipline of attention for a long period of time. Yet we may reach a
point at which we sense that in continuing to make this kind of effort
we need to refine the techniques and systems to pursue a goal. But if we
encounter a deep conviction that to do so is no longer appropriate, then
we need to be ready to adjust -- to let go altogether of seeking
anything. If it is right for us to make this choice, then when we hear
someone talking about their differing way of practice, we can say, `Oh,
okay, that's fine.' We won't be shaken. It is really important that we
don't keep letting ourselves be shaken by somebody else's enthusiasm.

As we settle more comfortably and confidently into making our own right
effort it becomes easier to recognise the various strengths and
weaknesses of different styles of practice. In goal-oriented practice,
for example, it is probably easier to generate energy. With a clear
concept of what you are supposed to be doing, attention narrows, all
distractions are excluded, and you focus, focus, focus. By being so
exclusive, energy gathers; this way you readily observe yourself
progressing along the path. This in turn supports faith. As with
everything, there is a shadow side to this, which is directly related to
this strength. In being so exclusive you risk chopping out things that
could be useful or need to be addressed; there is a danger of denial. If
old neurotic habits of avoidance have not been addressed and you follow
a goal-seeker's practice, then these tendencies become compounded. This
is the origin of fundamentalism. And despite popular belief there have
been, and there are now, Buddhist fundamentalists.

One of the strengths of source-oriented practice is that as we release
out of the striving and the aiming for something other than
here-and-now, a balanced, whole body-mind relaxation emerges. And this
draws out our creativity. We have to be creative, since by not excluding
anything, everything must be translated into practice. There is no
situation that is not a practice-situation. However, unwise creativity
can harbour delusion. If we are so happy and relaxed that we are getting
lazy or heedless with the precepts, for example, then we need to
recognise what is going on.

Another danger in source-oriented practice is that when we really do get
into a pickle we could feel disinclined to do anything about it. This
tends to happen because we no longer relate to structures in the way we
used to. Faith for us is inspired not by a concept of what we hope lies
ahead, but by a sense that what we trust in is already essentially true.
However, if the clouds of fear and anger overshadow the radiance of our
faith we can tremble badly, and possibly even crumble. In this case it
is important that we have already cultivated spiritual friendship. To
have the blessing of association with others with whom we share a
commitment to conscious relationship is a precious resource. When we
gather in spiritual companionship, a special feeling of relatedness can
emerge in which we rightly feel safe. This relative security can be for
us what concepts and goals are for goal-striving spiritual technicians.

As we progress in our practice each of us has the task of checking to
see whether we are moving into or out of balance. But how do we assess
how things are moving? If we are moving into balance, it means we can
handle more situations, we can accommodate states of greater complexity.
If we are moving out of balance, it means we can handle fewer and fewer
situations: instead of spiritual practice liberating us and opening us
up to life, it makes us exclusive and painfully cut off.

So it is wise to examine our practice and see if we can find the
direction we feel we move in most easily, which orientation of effort
comes most naturally to us, what sort of language works for us. We need
to prepare ourselves with the understanding that teachers of these
different approaches use different ways of talking. So listen to the
teachings you receive, contemplate that which you read in books, and see
which orientation of effort makes sense to you. Once you know, I suggest
you go with what inspires you.

Hopefully you can see how this contemplation is an important part of our
contribution to the shared task of translating practice. May we all feel
encouraged to investigate the contribution we are making to this task at
this stage in its unfolding in the West. I am confident that our careful
enquiry will show up our weaknesses, individually and collectively, and
when we become quietly aware of our deficiencies we will become more
creative. We will be able translators of the practice. Adaptation will
happen where it is necessary and it will be in the service of Dhamma.
Possibly we won't even notice it. We will just know that the spirit of
the Way is alive within us and that our hearts are more at ease.

Thank you very much for your attention.

