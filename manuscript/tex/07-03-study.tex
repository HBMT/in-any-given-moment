\chapter{Why Study?}

\begin{quote}
  A master is one who has let go of all craving\\
  and clinging to the world;\\
  who has seen the truth beyond forms,\\
  yet is possessed of a profound knowledge of words.\\
  Such a great being can be said to have finished the task.

  \emph{Dhammapada 352}
\end{quote}

As I begin addressing the topic of `study', I would like to explain what
I mean by it in this context. Here I am referring to the whole territory
of acquiring the information needed so as to safely engage in the
practice; this includes, but is not limited to, the Pali Canon. Studying
could take the form of reading, but it could also come in the form of
conversation and listening. Reading books is one of my least favourite
activities, but I do like listening. Often these days, before retiring
at night, I will listen to an audio book via Bluetooth into my hearing
aids.

When I was leading that Q\&A session at the Buddhist Society Summer
School back in the 1980s and discovered some members of our monastic
community didn't know even the basics of the Pali tradition, it revealed
a gap in the training that we were offering. Anyone who was born and
grew up in a traditional Theravada Buddhist culture would have already
absorbed many of the basics before they put on robes. In the context of
the West we need to take care that adequate training in the theory of
\emph{Buddhadhamma} is being properly instilled, and not allow aspirants
to naively assume that feeling inspired is enough. The theory of
practice is like the highway code: we might own a wonderful car and be
enthused by the thought of driving to some stunningly beautiful
destination, but if we don't know which way we are supposed to turn when
we reach a roundabout, we might not arrive at that destination. It
matters that we are adequately informed before travelling too far along
the way. Without study, we leave ourselves vulnerable to falling into
delusion. As we regularly recite in our Evening Chanting,

\begin{quote}
  The Dhamma holds those who uphold it\\
  from falling into delusion.
\end{quote}

On the other hand, however, too much study and we could be feeding
delusion.

\section{How Much Study?}

So just how much study of the theoretical teachings do we actually need
before we can feel confident that we are sufficiently well-informed? Our
mental curiosity can be a support for our commitment to the inner
journey or it can create obstructions. The point of study is to
strengthen and protect ourselves from falling into delusion, not for it
to become another addiction to distraction.

Tan Ajahn Chah instructed his Western disciples that during the first
few years they should avoid reading anything other than the books on
monastic discipline. It would be a mistake, though, to interpret this
instruction as dogma that everyone must always abide by. There were also
occasions when he spoke about the importance of the right amount of
study. He likened it to a medical practitioner needing to have done a
certain amount of study before trying to heal people. Study alone is not
enough, but practice without study is likewise not suitable.

Presumably Tan Ajahn Chah had observed how tortured many of us were by
sceptical doubt and knew for himself how feeding mental curiosity, when
it is an obsession, is not the path to freedom. By saying we should
avoid reading, he was trying to show us that there is a `right amount'
of study, and that it is simply not the case that by reading more we
will find more contentment. Many of us tend to assume that we need more
information before we will feel satisfied. Hence Tan Ajahn Chah's
observation, `The reason you don't know anything is because you know so
much.' By `knowing so much' he was referring to our knowing `about'
things, not knowing in terms of the kind of insight that dispels all
doubts.

The study of the theoretical teachings is the first of the three stages
of training: study, practice and understanding (\emph{pariyatti,
patipatti, pativedhi.}) If we consider the practice of Dhamma to be like
watering and weeding our vegetable garden, and the arising of
understanding to be like eating the vegetables, then study of the
teachings is like digging the water channels so that when it rains the
seedlings will be nourished. Study is about preparation, and we bypass
it at our peril.

We do, however, need to be honest about our motivation: are we seeking
mental clarity regarding the path of practice so we can walk the way
with confidence, or are we indulging in our compulsive craving for
conceptual certainty? It is unlikely that anyone else can really tell us
how much study we need. They can make suggestions, but we have to keep
reading our own hearts and minds until we find our own answer. If we are
being resolutely honest, then study can serve as a support and not be a
mere gratification of intellectual cravings. The `right amount' is
enough to assuage sceptical doubt to the degree that we are able to
trust in the teachings and in ourselves. That trust then supports us as
we face our doubts, fears and confusion. It will help sustain us when we
reach that point where we are obliged to endure the (apparently)
unendurable.

\section{A New Rendering of the Dhammapada}

Towards the end of the 1990s I embarked on a project of producing a new
rendering of the Dhammapada\cite{dhammapada}.
As already mentioned, our annual Forest Sangha calendar
regularly used Dhamma quotes from translated teachings of Tan Ajahn
Chah. I also wanted to use verses from the Dhammapada. Issues around
providing proper accreditation of translators of those Dhammapada verses
was one of the main motivations for my deciding to attempt to compile my
own version.

There are a great number of translations already in print and many of
them by highly qualified Pali scholars. One thing that stood out for me,
though, was just how different most of them are. They might all have
been produced by skilled scholars who would have all been referencing
the same Pali texts, yet there is an evident lack of uniformity.

The other thing that stood out was that they read as if they were
teachings given for men only. I am aware that up until relatively
recently in the history of the English language, the term `he' was used
to refer to both men and women, but in today's use of the English
language that is no longer the case. I am not a scholar, but I am aware
that the Buddha stated his teachings were to be made available to
everyone. So with the help of someone who was well versed in Pali,
Thomas Jones, I embarked on a project of producing my own rendering.

For the most part, the recognized translations that I consulted were:
from Burma, the version by Daw Mya Tin and the editors of the
Burmese Pitaka Association\cite{burmese-pitaka}; from Sri Lanka a
version by Ven. Narada Thera (B.M.S. 1978), and also one by Ven. Ananda
Maitreya Thera (Lotsawa 1988); and in later editions I consulted
Ajahn Thanissaro's and Bhante Buddharakita's\cite{buddharakita} versions.
For each verse I looked into the Dhammapada Commentary\cite{commentary}
that is traditionally associated with it. Even
though these stories are generally considered to be apocryphal, I feel
they nevertheless contribute to understanding the spirit of the verse.

The exercise of producing this new rendering of the Dhammapada was
thoroughly rewarding. It was a privilege, and I am grateful for the
opportunity it gave me to study what the scholars and practitioners have
preserved. Here we are, after more than two and a half millennia, and
these teachings are still available for us to study and be guided by. I
do not claim that what we produced was a new translation of the
Dhammapada: in several places I have emphasised this is a new
`rendering.'

The Dhammapada is a good book to read for those who are new to Buddhism.
However, the book that I most frequently recommend is \emph{What the
  Buddha Taught}, by Walpola Rahula. This small volume,
along with Ajahn Sumedho's, provide a clear and accessible foundation
upon which anyone interested in answering life's great questions can
build their practice. When I stop to think about it now, I consider
myself extremely fortunate to have come across these teachings. Not
rarely I express the opinion that the Buddha's teachings on the Four
Noble Truths are the ultimate articulation of wisdom ever uttered by any
human being throughout all of human history. To be accurate, though, by
the time Siddhartha Gotama offered these teachings, he was no longer a
normal human being -- he was a Buddha.

These two books mentioned above come top of the list of required reading
for anyone who wishes to take up monastic training here at Aruna
Ratanagiri Monastery. There is a wealth of resources available these
days in the form of printed and on-line translations of the Pali Canon,
and many of these make up the remainder of our reading list.

\section{Extracurricular Studies}

When it comes to applying these Four Noble Truths in daily life, I am
indebted to a number of people who have worked in the field of
psychology. It is hard to find adequate words to express the
appreciation I feel for the understanding that they have so generously
shared. The associations and friendships I have developed over the years
with various psychotherapists, psychoanalysts and psychiatrists, have
helped me learn how to not just conceptualize about suffering and its
cause, but to truly meet it, here and now, in the body-mind. In many
ways they have demonstrated how to convert a chronic sense of feeling
obstructed into a more connected relationship with life. And I am not
just referring here to my own painful experiences of feeling obstructed,
but also to those of many other Dhamma practitioners I have met who have
been struggling to find realistic ways to deal with the difficulties
they encounter in practice. Considerable agility of attention is called
for as we are challenged over and again by the many coarse and subtle
expressions of delusion. Without agility we can easily be fooled by old
habits of clinging into forgetting about the Buddha and once more
falling into the trap of going for refuge to `my way'.

These days I almost never think about reaching the final goal of
complete freedom from suffering, but that does not mean I have lost
faith in it. It means that I feel my faith is sufficiently well
established, so that I don't need to always be dwelling on it. What I do
often think about, though, is how to be more accurate and sensitive in
my investigations of suffering, my own and others', where and when it
arises.

So what studies and strategies support the letting go of our misguided
identification with the body-mind? How can we more readily access
expanded states of awareness within which we can study our own hearts
and minds?

Perhaps we received instructions from a teacher who was full of
confidence and genuinely believed they knew what was good for us. Out of
respect we gave ourselves into following their instructions, only to
discover that our anxiety level increased and our mind was anything but
peaceful. Does that mean there is something wrong with us? In the
Visuddhimagga by Ven. Buddhaghosa\cite{visuddhimagga}, a well-known text
within the Theravada tradition, it is explained that there are various
character types -- at least six: greedy, hating, deluded, faithful,
intelligent, and speculative (\emph{The Path of Purification},
translated by Bhikkhu Ñanamoli, Buddhist Publication Society, Part two,
Chapter 3, para. 74). The text helpfully describes how a person's
posture when standing or walking reveals their character type; it then
goes on to explain how a particular style of accommodation will be
suitable and supportive of progress for one character type, but not
necessarily for another. One style of meditation could be suitable for
one person but not for another. From this we should understand that just
because one particular approach to practice works for one person does
not mean it will work for all. It is indeed an expression of
fundamentalism to insist that one approach will suit everybody.

Earlier in this book, when discussing my brief time at Waikato
University (Part 3, Chapter 3), I commented on the apparent conflict
between what Buddhists mean when we talk about letting go of the self
and what psychotherapists mean when they talk about the importance of
developing a strong sense of self. I offered an explanation of why that
apparent conflict need not be a problem if we look deeply enough into
what each discipline -- spiritual practice and psychotherapy -- is
saying. A few decades ago it was common for Buddhist meditators to speak
critically about psychotherapists, and some psychotherapists were
disparaging of Buddhist meditators. Thankfully, these days a level of
mutual appreciation has evolved whereby each discipline is better
informed as to the other party's perspective. Many meditators have now
come to realize that the aspiration to let go of attachment to the
body-mind can be inhibited by deeply held mental habits which do not
always disappear even after many years of meditation. Some
psychotherapists have realized that having a balanced and rounded
personality is no guarantee that they will remain cool, calm and
collected when confronted by the deepest and most difficult question:
what is life and death really all about.

It is also becoming apparent that the more imbalanced and confused we
human beings are, the more extensive a repertoire of skills is required
to untangle the knots of mental, emotional and relational complexity. It
is probably safe to say that those who have not struggled so much with
confusion require less complicated remedies; they might even have
difficulty understanding why traditional straightforward spiritual
instructions are not enough. Some people, and I count myself as one of
them, grew up to find they were carrying a burden of unreceived life, of
unacknowledged suffering -- perhaps what might be referred to as heavy
kamma. Those who find that the formula of `make your mind peaceful and
look at impermanence' fails to produce clarity and understanding, need
to feel allowed to be agile in how they approach their spiritual
practice. They need to be daring and brave, and at the same time gentle
and caring, and not be intimidated by those who were perhaps less
confused to start off with and who have trouble relating to their
struggles.

\section{Goal- and Source-Oriented Practice}

There is a small village in Yorkshire called Ampleforth which lies about
two hours drive south of our monastery and is where Ajahn Puñño grew up.
Just outside this village is Ampleforth Abbey and College\cite{ampleforth}
where Ajahn Puñño did most of his
schooling. On one of the occasions during the 1990s when I visited
Ampleforth, we walked over to the abbey and met with Father Cyprian
Smith. Besides being the novice master at the abbey, Father Cyprian was
also a respected commentator on Meister Eckhart and is known for his
book, \emph{The Way of Paradox}\cite{paradox}.

I believe it was in the conversation we had on that occasion that I
first became aware of the two distinctly different approaches to the
spiritual life found within the Christian tradition: the
cataphatic and the apophatic\cite{cataphatic}. The former is that expression of
Christianity with which most of us would be familiar, characterized by
positive affirmations about the nature of God and the spiritual journey.
The latter is an expression of the journey characterized by
non-affirmation -- exemplified, for example, by Meister Eckhart and St
John of the Cross -- a path sometimes referred to as `via negativa.'
That conversation later stimulated in me a contemplation which
eventually gave rise to the concept of goal- and source-oriented
practice.

I think that visit might have taken place around 1999 because a dear
Dhamma friend, Peter Fernando (known then as Samanera Dhammadasso), was
living at Bodhinyanarama Monastery, near Wellington, New Zealand. He
tells me that during the early months of the year 2000, I gave a series
of talks on the topic of goal- and source-oriented practice. (See
Appendix, `\emph{We Are All Translators}', for an edited transcribed
version of those talks.) I suspect those talks were fuelled by the
enthusiasm that emerged along with this way of viewing the different
approaches to practice.

Because of that conversation with Father Cyprian, I discovered a fresh
new perspective on how, not just in Christianity but in many of the
world's great religions, there are similar delineations: in Zen Buddhism
there are the Soto and Rinzai schools; in Tibetan Buddhism there are the
Dzogchen and the `gradual' approaches; in Hinduism there are Advaita
Vedanta and the more dualistic traditions. Within Theravada Buddhism we
have teachers who emphasize ardently striving to overcome all obstacles
in pursuit of \emph{jhanic} mind states which they trust will then
provide the environment within which deep letting go can take place; and
then there are those who advocate letting go of everything:

`There isn't anything and we don't call it anything -- that's all there
is to it! Be finished with all of it. Even the knowing doesn't belong to
anybody, so be finished with that, too! Consciousness is not an
individual, not a being, not a self, not an other, so finish with that
-- finish with everything! There is nothing worth wanting! It's all just
a load of trouble. When you see clearly like this then everything is
finished.' (Ajahn Chah, \emph{The Collected Teachings}\cite{collected},
p 478: Chap. 40, \emph{What Is Contemplation?})

My contemplations on goal- and source-oriented practice led to a more
confident appreciation of how different character types require
different approaches; as we see from the \emph{Visuddhimagga}, there is
nothing new about that. What a relief! I had spent many years feeling
frustrated because I simply wasn't convinced that a goal-oriented sort
of effort was appropriate. And yet many of the teachings within
Theravada are couched in a language that appear to endorse such an
effort. Gradually it became clearer to me that this apparent conflict
was not a conflict at all. The language used when talking about study,
and that used when talking about practice, are different. The idea of
suffering and the experience of suffering are worlds apart. If, for
example, I am sitting in my cottage of an evening thinking about the
tahini on toast that I might have for breakfast in twelve hours time,
those thoughts just cause me to start salivating and to feel hungrier.
The next morning, when I am actually eating breakfast, is an altogether
different experience -- it is nourishing.

It helps if we recognize the place of goal- and source-oriented efforts.
We don't have to judge ourselves or others because we read or hear
teachings about one style of practice and find them inspiring, and those
about another style not so much. Probably all of us started out with an
idea of a goal and that idea generated enthusiasm and motivated us to
take on practices. For some, it seems to be useful that they maintain
their relationship with an idea of the goal: it continues to inspire and
support them in their pursuit of freedom. For others, as they move
beyond the stage of initial faith in this path of practice, if they keep
focussing on an idea of the goal, they become more confused. What
inspires and supports them is not trying to get somewhere else, but
trusting in being acutely aware, here and now. Intentionally trusting,
not trying, replaces wilful striving as the primary motivator.

Those for whom trusting rather than trying makes sense need to prepare
themselves to include everything in their practice. For them, doubt does
not have to be an obstacle, it is something to get interested in: `This
is suffering. Can I sense the awareness in which this suffering is
taking place? Or am I totally identified \emph{as} that movement of mind
that feels like doubt?' Similarly with sadness: `Do I believe this
sadness is ultimate, or is there a spaciousness through which this
sadness is moving?' `Am I still projecting attention out into the
imagined future or can I exercise that subtle shift in focus that means
attention is more here and now?'

Earlier I was asking, `What sort of studies and strategies might support
the letting go of our misguided identification with the body-mind?'
Considering the difference between goal-oriented practice and
source-oriented practice can lead to our feeling confident about the
sort of effort we need to be making. From the outside, those who have
faith in a source-oriented style of practice might appear to not be
practising at all. In fact, sometimes I encourage meditators to spend
time `not-meditating'. For all of us, our effort in practice can become
subtly compulsive. \emph{Bhavatanha} can be insidious and creep in
without our realizing it, to the point where the deluded personality is
driving our meditation. By way of experiment, it can be helpful to
simply sit in a chair and consciously `not-meditate' for twenty minutes:
no special meditation posture, no meditation object, nothing in
particular to contemplate -- here and now, embodied awareness,
just-knowing.

\section{Agility of Attention}

My experience suggests that if we are given permission to be agile in
our approach to spiritual practice, there is a better chance we will be
able to keep practice constant; we won't just be filling in time until
we are next able to go on retreat. This applies both to those living in
monasteries and to those living the householder's life. So long as we
are holding tightly to an idea of progressing towards a goal, sometime
`out there', then we are not truly `all here'. We are not fully in touch
with what is happening, inwardly and outwardly, right now. We are not
properly receiving and processing the data that reality is giving us.
Worse still, we could be creating unnecessary limitations for ourselves.

Also, so long as we are not `all here' we are not able to draw on the
potential creativity that we have. If, after having spent a good number
of years following the instructions that we were given by our teachers,
we are still feeling obstructed, I would recommend turning attention
around and getting interested in that very feeling of being obstructed.
We need to be ready to look directly at the suffering, here and now;
holding onto hopes about being free from suffering in the future is not
enough.

One of the many forms of suffering with which I struggled during the
early years when I was living near to Tan Ajahn Chah was the hesitation
I felt about joining in with the other monks who were helping take care
of him -- such things as washing his feet when he came back from
alms-round and cleaning his kuti. I say hesitation, but it felt more
like a disability. I really wanted to be close to him but felt
obstructed. `Why can't I just grab his foot and wash it and dry it like
the other monks without making such a big deal out of it?' There were
possibly others who felt intimidated as I did; I couldn't tell. What I
did know was that I wanted to be part of the `in group' who seemed
unfazed by being close to the teacher.

It was unacknowledged fear of rejection that was getting in the way. In
those days, presumably because I was still in a striving and overcoming
mode of practice -- rather than consciously feeling those feelings of
fear and making an effort to sense the space in which those feelings
were arising and ceasing, I expect I just tried to get rid of fear -- as
well, of course, as indulging in seeing myself as a failure for having
such feelings in the first place.

So long as we are caught up in trying to transcend suffering and reach
the goal, we are ignoring the Buddha's teachings. \emph{`It is because
you don't see two things that you continue to suffer: not seeing dukkha
and not seeing the cause of dukkha,'} the Buddha said. When we hold too
tightly to the idea of freedom from suffering, we can mistake the
experience of suffering to be a sign of failure. In fact the experience
of suffering is the teaching. If I am feeling afraid that I will be
rejected by the teacher, then feeling that very feeling, in the
body-mind, is the message. That experience of suffering is ready, right
now, to be received into awareness -- to be met. If I refuse to meet
myself there, and instead keep striving to develop states of meditative
absorption, hoping that one day I will transcend suffering, I could be
throwing myself even more out of balance.

`But what about how the Buddha taught us to strive on with diligence?'
you might think. Yes, indeed, we can read in the recorded teachings that
the Buddha did say we should strive on with diligence, but what does
that \emph{actually} mean in practice? What does diligent effort look
like? The Buddha elaborated on the different ways we might approach
practice by giving us instruction of the `four right efforts': There is
the effort to give rise to so far unarisen wholesome states of mind, and
the effort to protect already arisen wholesome states of mind; then
there is the effort to avoid the arising of so far unarisen unwholesome
states on mind, and the effort to remove already arisen unwholesome
states of mind. When we look into what these truly mean in terms of how
we apply ourselves, it should become clear that we need to be adaptable.
For instance, the effort to protect an already arisen wholesome state of
mind does not necessarily mean that, having experienced a moment of
insight, we should be barging ahead aiming for the next one. It might in
fact mean we ought to change gear, slow down, and contemplate the effect
that this new experience is having on us.

Then, in the case of the kind of effort we make to remove an already
arisen unwholesome state of mind, it might mean gritting our teeth and
refusing to give vent to the impulse to speak unkindly to someone; or it
might mean cultivating the patient recognition that we are not always
able to get rid of mind states just because we don't like them -- there
are times when all we can do is bear with the unpleasantness and humbly
wait.

It does seem that there are some people who are already sufficiently
well grounded and integrated and have a genuine affinity with what we
could call the `transcendent approach'; but, again, it is naive to
suggest that because such an approach works for them, it will work for
everyone. Trying to force ourselves along a path with which we feel no
affinity, might initially give rise to an increased sense of aliveness
but eventually take us to a place of extreme vulnerability.
Consciousness might become brighter for a while, but if mindfulness is
not adequately embodied, when states such as fear arise, we cling to
them and are drawn down into a vortex of terror. Or, instead of fear, if
we encounter unacknowledged anger and cling to it, we risk becoming
possessed by rage. Such potentized suffering is much more difficult to
deal with.

We need not worry that spending time enquiring into our relationship
with fear is an abandonment of our aspiration for awakening. When we
fail to understand just where, when and how we are creating suffering by
indulging in telling ourselves stories and then clinging to the emotions
that appear as a result, we remain caught in a painful cycle, which can
seem endless. If, conversely, our practice is informed by an
understanding of how fear of rejection and desire for acceptance go
together, then perhaps we will stop trying to get rid of these
conditions and allow them to teach us how to let go. When we cling to
fear of rejection, we generate and cling to the desire for acceptance.
If we are lost in the desire for praise and appreciation, we tend to
become lost in the fear of criticism. Desire and fear go together like
this.

As I was saying, most of us begin practice motivated by an idea of
reaching some sort of goal. As we progress, we need to learn how to hold
those ideas more lightly. The suggestion that we should let go of our
idea of the goal might trigger the worry that if we don't hold fast to
our aspiration, we will lose it. That is not very different from how, as
children, we were afraid to let go of mummy or daddy's hand in case we
fell over. Without letting go of a relative form of security, we will
never learn to walk. Agility in practice means that, when necessary, we
are ready to turn attention away from any idea of a goal or of making
progress towards it, and are willing to fully feel our fear of failure,
or rejection, or sadness, or loss -- to skilfully and sensitively
enquire directly into suffering.

Earlier, when commenting on the two months I spent at Bodhinyanarama
Monastery in New Zealand, in the year 2000, I referred readers to the
edited talk (\emph{We Are All Translators}) that can be found in the
Appendix to this book. The transcribing and editing work on that
occasion was undertaken by Tan Abhinando who was living at
Bodhinyanarama Monastery at the time. We had met briefly before at
Harnham when he was still an anagarika, but those two months were the
first occasion of our becoming properly acquainted. Shortly after I
returned to Harnham he sent me a copy of a poem he had written about the
occasion of his seeing me off at the Auckland Airport. (See, `\emph{The
  Inner Distance}' in \emph{When Everything Is Said}\cite{everything} p 61.)

\begin{quote}
  Again I am standing unarmed.\\
  Again I am standing paralysed\\
  facing your last words.\\
  Your open gaze,\\
  our vulnerable silence;\\
  from farewell to farewell\\
  we feel for the inner distance\\
  of a reality\\
  that resists\\
  all feelings.
\end{quote}

Sometimes we use prose to describe the process of inner investigation;
at other times we write poems about it. Not long after that Tan
Abhinando came to live at Harnham and stayed with us for about fifteen
years. These days he is the abbot of Dhammapala Kloster in Switzerland.
Thank you, Ajahn Abhinando, for your friendship.

\section{Positive and Negative Projection}

Another psychological concept that I have found particularly helpful is
that of projection. I have not studied the subject extensively, but that
which I have gleaned from what I have read or heard has helped me a
great deal to integrate the Buddhist concept of clinging into daily-life
practice. For instance, what was really going on in my mind when I was
too afraid to join in with the other monks in performing attendant
duties with Tan Ajahn Chah? It wasn't as if he was going to whack me
over the head with his walking stick if I didn't dry his feet quickly or
carefully enough: so the reticence wasn't anything to do with him. Out
of unawareness I was mindlessly projecting onto him my longing for
approval, which gave rise to the fear I wouldn't receive the approval.
That mental process of compulsively projecting responsibility for that
which is actually ours onto external objects is happening all the time,
and it is very helpful to see it. I perceived that I needed Tan Ajahn
Chah's approval and, in so doing, became afraid of him. In Pali it is an
expression of what is called \emph{upadana}, (clinging), but we need
much more than the concept. We need to learn how to skilfully inhibit
that impulse to cling.

Not all projection is to be viewed negatively: there is such a thing as
positive projection. When a child perceives their parents to be
indispensable and they cling to their parents, that is suitable. If the
parents are doing their job well enough, little by little, as the child
grows, he or she will gradually learn that they have their own set of
abilities: they do not have to totally depend on their parents. In other
words, that which they projected onto those who cared for them is taken
back, and the child, or teenager, or eventually young adult, learns to
be independent. If the parents are not doing their job so well, because
of their own unawareness, they cling to their child and fail to
adequately mirror back his or her ability, and the development of the
child is interrupted.

It can be very helpful if Dhamma teachers also appreciate this
principle. Naturally, in the beginning students project their ability
onto those they perceive as being able to help them in their practice.
The teacher's job is to gradually and skilfully reflect back to their
students that ability; in the process the students learn to find their
own confidence and become independent of their teacher. When a teacher
doesn't quite understand this process, they can become excessively keen
to receive the adoration and respect that is being projected onto them,
and instead of supporting their students' becoming independent, they
encourage further attachments. It can feel very lovely to receive
admiration from others, but teachers are not supporting their students'
progress in practice if they are feeding on their projections.

Something similar can happen in the world of psychotherapy. It is
understandable and even functional for a client to project onto their
therapist that which they have so far not felt able to own within
themselves. If the therapist is skilled in their work, they will be able
to read the readiness of their client and, at the right time, reflect
back that which has been projected out. Again, in the process the client
grows stronger and more competent. If the therapist misreads the
situation, or for other reasons of unawareness, their client could
remain in therapy for much longer than is really necessary.

Equipping ourselves with an understanding of how we tend to project our
ability/energy onto others, can show us how, where and when we can
reclaim that ability/energy. Instead of making ourselves weak in
somebody else's company, and possibly blaming them for what is in fact
ours, we can apply mindfulness, restraint and wise reflection, and
discover where the source of competence really lies: it is not in
gaining approval from our teacher or being praised by others, it is in
seeing for ourselves, as we are doing it, exactly how we are causing our
own experience of limitation.

Perhaps progress on the spiritual journey could be described as a
process of gradually withdrawing our projections from external forms:
rituals, teachers and traditions. Having said that, though, I hasten to
emphasize the word `gradual'. And if it is happening in the right way,
it will not be `me' that is withdrawing the projection; it is more
likely to be a process of looking back and seeing, `Oh, look what
happened there.' The rituals, teachers and traditions are similar to our
parents and therapists: they support us so long as we need them. If we
have access to such precious supports in practice, we are truly
fortunate. When we bow down to the Buddha image, we project out, in a
positive way, our spiritual ability, and, if we are practising with
embodied awareness, then the admiration and gratitude we express is
reflected back to us, nourishing our confidence and commitment.

\section{Loving and Hating}

One of the trickier territories through which any human being ever has
to navigate, is the experience of falling in love. This too is an area
in which it helps if we understand the process of projection.

Most of us will be familiar with the phenomenon of falling in love and,
generally speaking, it is assumed that when it occurs it is a wonderful
thing. Without a doubt it can be a very powerful thing, but whether it
is wonderful or not, in my view, warrants careful consideration. It is
worth mentioning that wearing robes does not mean we have escaped from
having to deal with the experience referred to by the expression,
`falling in love'. I say this here so it is clear that I am not
unfamiliar with how extraordinarily beautiful, and how potentially
overwhelming the condition can be -- also how heartbreakingly painful it
can be.

From what I have seen, it is not love that we are falling into; it is
pleasure. I prefer to reserve the word love for that open-hearted state
that, for example, a parent might know when they gaze at their child. It
is undefended, undemanding, generous, caring, kind. This beautiful state
is more likely to happen with a newborn child, whose consciousness has
not yet collapsed into being identified \emph{as} the limited,
contracted condition of defended egoity, than it is when we are in the
company of an older person who has already assimilated the collective
assumption that we \emph{are} our ego. Yet that open-hearted state does
still keep happening, not just when gazing at innocent babies. It might
spontaneously occur whilst out in nature, or in an inspiring building,
or on a meditation retreat. And, obviously, sometimes it does
mysteriously occur between people who, much of the time, experience
themselves to be in a limited, closed-hearted state.

Whatever triggers it, when it does happen and the heart is happy, the
body feels pleasure. When two people simultaneously experience such
happiness, a sort of resonance can be struck up which intensifies the
happiness and potentizes the pleasure. It is this pleasure that, unless
we are thoroughly well prepared, we fall into; and in my view falling
into it is not at all wonderful. It is unfortunate; that is, if we
accept that by `falling' we mean being identified \emph{as} those
feelings and becoming lost in them. If we cling to, or become identified
as those pleasant feelings, we are no longer able to contain the
intensity, and our heart projects the happiness out onto the other. When
this happens for two people, at the same time, the experience is
intoxicating. They start believing that the other person has power over
them; and in a sense they do, because they have given them that power.

Because people rarely stop to investigate the reality of such an
experience, they assume lots of things about it that are not valid.
Falling in love, or falling into pleasure as I prefer to think of it, is
generally celebrated as a good thing and people want it to last forever.
But it never does last forever. In some cases, if both parties are
committed to integrity, the relationship could evolve into something
genuinely beneficial, but that requires a great deal of patience. Or
perhaps the relationship morphs into something that is more manageable
than amazing.

When the intensity of happiness which arises with the open-hearted state
becomes too much to bear and we project it out onto the other person, we
start saying such things as, `I can't live without you.' It sounds
irrational -- because it is irrational. We have fallen into a condition
of diminished responsibility. I am not saying it is necessarily wrong or
bad, just that we are not quite all there when it happens, and it would
be helpful if we understood that.

At the other end of the happiness-sadness spectrum, the same dynamic
also occurs, but with very different consequences. When two people
simultaneously experience so much sadness or anger or hurt that they
can't contain it, they project it out onto the other person and `fall
into' suffering. In this case they start saying such things as, `I can't
stand you,' or `I can't even be near you.'

Falling into \emph{sukha} and falling into \emph{dukkha} are really the
same thing; both end in intense disappointment. But if we have a degree
of wisdom, that disappointment can be turned around to our advantage.
This is similar to how somebody who has felt threatened by the
possibility of death yet survived might speak about how grateful they
are for the experience that they went through. At the time, their
suffering might have been difficult to endure, but because they were
well enough prepared, and they didn't merely believe in the way things
appeared to be, they received a precious lesson in life. So long as we
are still clinging to feelings -- agreeable or disagreeable -- we are
vulnerable to falling into delusion and will have to suffer the results.

Having shared my perspective on these matters, it might be assumed that
I am saying there is something wrong with love. That would be like
saying there is something wrong with the sun rising in the morning. The
point I am making is not that there is a problem with being loving --
being loving is an expression of selflessness -- the trouble begins when
we don't have enough clarity and understanding to accurately recognize
the causes of suffering; we fail to see how clinging spoils everything,
including love. To have learned how to effortlessly dwell in the
selfless, open-hearted state, would be to have learned one of life's
most important lessons. And anyone who helps us learn to see how
clinging obstructs all that is truly beautiful is a true spiritual
friend (\emph{kalyanamitta}).

To conclude these considerations on the place of study in the spiritual
life, I want to emphasize the importance of our refuge in the Buddha.
Our conscious commitment to the Buddha -- that is, to selfless,
just-knowing awareness -- provides us with a point of reference around
which we can explore and investigate our lives. Without that point of
reference, the many intriguing, often intellectually fascinating
theories about how to handle the difficulties of life, can turn into
further fuel for self-inflation: by investigating ourselves we could
become even more obsessed with our self-image. Personally, I consider
the tried and tested teachings contained within the Theravada Buddhist
tradition to be a reliable roadmap upon which we can depend as we
proceed on this journey. Without such a roadmap we are vulnerable to
becoming lost. With a well-developed commitment to the Refuges of
Buddha, Dhamma, Sangha at the core, we can trust ourselves as we enquire
into what earlier on I referred to as extracurricular studies, and see
which of them genuinely serve our aspiration for awakening. Also, we can
see how those aspirations translate into serving the well-being of the
world in which we live.

Often I have found it is not only the Dhamma books that I have read, or
the discourses I have heard, that have been so helpful in trying to make
sense of the madness that our human family is currently having to face.
The understanding found in the field of psychology has been tremendously
usefully in augmenting Dhamma teachings. Had there not been a degree of
spiritual aliveness at the core, however, it is questionable whether
much of what I studied would have been so helpful.

It is essential that we appreciate the profundity of the Buddha's
teachings on the nature and the cause of our suffering. We need the
basic principles of right view in place. With that right view at the
heart, there is a better chance that our efforts to resolve our many
difficulties will be productive. These days, I find I am not especially
intimidated when confronted by such challenging questions as, `Why the
increase in fundamentalism?', ` What is causing the current collective
identity crisis?', and `How did this present pandemic of narcissism come
about?', to name just a few of the topics that recently have people
wringing their hands and furrowing their brows.

Sometimes I fantasize about how psychologists might one day apply their
admirable skills to doing the R\&D on the subject of integrity. I try to
imagine what effect it could have on society as a whole, if, as
mentioned earlier, instead of IQ meaning merely Intelligence Quotient,
it stood for Integrity Quotient, and was a recognized metric used for
assessing the employability and overall worth of an individual. Surely
we can do better than assess somebody's worth by looking at their
parents' wealth, or their education, or their popularity on social
media. Similar to how the subject of mindfulness has made its way into
everyday life, a shared recognition of the importance of cultivating
integrity could bring many benefits -- individually and collectively. I
would suggest that if we did have a shared appreciation of how integrity
is conducive to self-respect and inner stability, it would lead to a
rebalancing in society, which, in turn, would redress injustice and
inequality: mental health issues would become more manageable and the
equitable distribution of wealth could be a natural outcome. Religion
used to serve the well-being of society by instilling the sort of
virtues that protected its members from becoming overly narcissistic.
For large portions of society now, conventional religion has almost no
place. For the sake of our survival, we need to find new ways of
protecting ourselves from our folly.

The right amount of study of Dhamma can provide the impetus and the
encouragement to cultivate our spiritual faculties (\emph{saddha,
viriya, sati, samadhi, pañña}: faith, energy, mindfulness,
collectedness, discernment) so that we are truly able to meet our
suffering, here and now, and let it guide us towards a clearer and
broader perspective on reality.

