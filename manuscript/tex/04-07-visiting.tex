\chapter{Visiting Luang Ta Maha Bua}

Towards the end of that year, or perhaps it was the beginning of 1979, I
returned to Wat Pah Nanachat. On the way back, I took the opportunity to
visit a young English monk friend who was living at Wat Pah Bahn Tard,
Luang Ta Maha Bua's monastery in Udorn province. There are three vivid
memories I have of that visit.

The first, I think, was from an incident that took place the day after I
arrived. As I wrote in the book \emph{Servant Of Reality}\cite{servant} p.12,

\begin{quotation}
It was the tradition in that monastery for all the monks to wait in the
main meeting hall before the morning alms-round. I recall being somewhat
taken aback when Luang Ta entered the hall, immediately approached the
shrine and performed three bows. He had a reputation for
being particularly fierce, possibly because in his earlier life he had
been a boxer. He was also reputed to be fully awakened. Somehow these
factors caused me to assume that when he entered the hall he was likely
to start barking orders at the junior trainees and not bother with
something as mundane as bowing. That was a mistake on my part.
\end{quotation}

Once again, similar to my perception of Tan Ajahn Thate and Luang Por
Kaaw, Luang Ta Maha Bua was the manifestation of gentleness, strength,
and dignity.

The second impression that stays with me is of the camaraderie of the
monks when they gathered in the dyeing shed for evening tea. It was a
thoroughly informal occasion with everyone wearing the minimum amount of
clothing, as befitted the sweltering heat. There was a regular supply of
sugar, tea and coffee and, unless I'm mistaken, plenty of cocoa. It
seemed like everyone helped themselves and could have as much as they
wanted. Without getting into making heedless comparisons, this did
contrast dramatically with teatime at Wat Pah Nanachat where everyone
sat in lines and talked very quietly. At Wat Pah Nanachat there was no
choice over what you drank; you took whatever was in the big aluminium
kettle that was passed down the line, or you went without. The
atmosphere in the Wat Pah Bahn Tard dyeing shed was even jovial, at
least at the time I was there. The overall approach to training in that
monastery was different; there was no morning and evening chanting, for
instance, and the emphasis was more on doing sitting and walking
meditation practice alone. In Tan Ajahn Chah's monasteries there was a
strong emphasis on group activities such as daily chanting, sweeping
leaves, and hauling water from the well.

Ajahn Paññavaddho, who, in terms of years in the robe, was senior to
Ajahn Sumedho, would sometimes join us for those tea sessions. I
remember how he seemed to listen in a way that meant I felt heard. That
wasn't always the case with senior monks. Presumably, it was over
evening tea that he found out about the difficulties I was still having
with my knees. He was obviously aware that at that time in the forest
monasteries in Thailand, sitting on a chair during sangha gatherings was
not an option. I suspect too that he sensed how threatened I felt
because of this physical condition. If I understand correctly, Ajahn
Paññavaddho was a trained engineer before becoming a monk; and he had
someone he knew who worked at the railways build a sitting stool for me.
It was almost strong enough to survive being run over by a train. That
he went to all the trouble of having it built was one thing, but then he
had it delivered for me to Wat Pah Nanachat. Thank you, Ajahn
Paññavaddho.

The third memory is of an occasion when Luang Ta Maha Bua offered a
formal teaching. If there was to be a Dhamma talk to the sangha, I was
told, it would usually happen on an evening when there wasn't a lot of
distracting noise drifting in from nearby villages, or from the wind and
rain. Although I was feeling pleased with my improved grasp of the Thai
language, unfortunately on that occasion I still had to depend on the
translation into English given by Ajahn Paññavaddho. That translation
happened with all the Thai monks still sitting there, and when he
finished, Luang Ta Maha Bua singled me out and asked if Tan Ajahn Chah
gave talks especially for the sangha, or just in general for everyone,
including the extended community of lay supporters. I expect I was like
a rabbit caught in the headlights of a car in my reply, saying that yes,
Tan Ajahn Chah regularly gave talks to the bhikkhu sangha after the
fortnightly \emph{patimokkha} recitation.

