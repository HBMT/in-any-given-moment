\chapter{Emphasis on Letting Go}

Ajahn Sumedho put a lot of effort into offering teachings to the
community, and there was evidence that his effort was appreciated. We
had a fairly constant stream of interested people, men and women, coming
to stay. Despite there being a lot of hard physical work, a rhythm of
morning and evening pujas was usually maintained. Also there was the
traditional weekly observance on the moon day when we would stay up at
least until midnight. There was no breakfast during those initial two or
three years, as we were trying to see if it was possible to keep to the
standards we observed in Thailand. Letting go was the teaching, and
letting go was the practice. Anyone who insisted on holding to their own
way of doing things soon stood out.

Not very long after I arrived in Britain my mother and father came to
visit. I think they found the few days that they stayed with us at
Chithurst rewarding. They then went on a whistle-stop tour around the
Continent. Soon after returning from that trip, while they were staying
with a Thai family who were friends of the monastery, in Hampstead,
north London, my father suffered a stroke. The wife of the couple was a
medical doctor and quickly recognized the symptoms. My father was
immediately admitted to a nearby NHS Hospital and underwent surgery.

The Thai family accommodated me in a converted garage adjacent to their
house, and I was able to support my mother and visit my father. On one
of those visits, when my mother and I were walking down a corridor at
the hospital, I was struck by her reaction to comments made by a young
doctor. As I recall, there were what looked like three junior doctors
walking towards us, and as they passed us one of them made a comment
referring to me: he said something like, `Take a look at that get-up'.
Without hesitation, my mother turned on him and said, `That is not a
get-up; my son is a Buddhist monk.' I don't remember his reaction, but
presumably he felt suitably upbraided. It was gradually dawning on me
that, although my parents found it hard to understand what I was doing,
they did respect it. A few weeks earlier they had met Ajahn Sumedho and
engaged in what they seemed to find a very rewarding conversation.

Staying in that garage with the thought of my father's medical condition
and my mother's disappointment over how their trip had turned out, was
not easy, and my attempts at meditation were hopeless. What I did find
helpful, however, was chanting. I was grateful that as a community at
Chithurst we had learnt to recite the \emph{Dhammacakkappavattana Sutta}
-- The Discourse on the Turning of the Wheel of Dhamma. This was the
Buddha's first discourse and contained the core teachings, including the
Four Noble Truths and The Eightfold Path. Besides serving to occupy my
mind with something wholesome, I seem to recall that this was when I
discovered the physiological benefits of chanting. The exercise of short
in-breaths followed by long out-breaths can result in feeling as if
energy is being drawn down from out of the head into the belly. It can
be calming both mentally and physically.

During this interlude in London an exceptionally friendly young fellow
called Chris would sometimes come to visit us, and on occasion kindly
accompanied my mother on outings to sites such as Buckingham Palace and
Madame Tussaud's. He later took up the training as a monk and was given
the name Karuniko Bhikkhu.

The generosity of that Thai family felt very supportive. Once again I
was struck by the beauty of a culture that is based in Dhamma
principles. Both of my parents were understandably extremely grateful.
My father made a good enough recovery, and several weeks later was
allowed to fly back to New Zealand. I returned to Chithurst.

The stage of painting and decorating the new shrine room was exciting.
By this time another Kiwi fellow had joined us, called Finlay, and he was
impressively skilled in carpentry and cabinet making.

The central piece of the shrine room was of course to be the shrine
itself. A massive oak tree had been extracted from Hammer Wood and it
was decided that the community would mill it ourselves and use it to
construct the shrine. What I think is called a `pit saw' was sent to us
from Thailand and Ajahn Anando and Tan Amaro together took on the task
of converting the log into a long thick oak plank.

The eventual arrival of a large gilded Buddha rupa from Thailand marked
a significant stage in establishing Chithurst, or what was to be called
Cittaviveka, as a monastery.

The ongoing renovation of the house took years, not months. There
developed a bit of a pattern whereby I would be given a room to stay in
and asked to fix it up at the same time. At least during one period I
recall how, after a day of scraping walls, filling holes and painting, I
would push the paint pots aside and lie down to sleep, wake up the next
day, and carry on. One of those rooms had probably previously been the
place where the nanny lived; once the redecoration was finished Ajahn Sumedho moved
in, and I was sent to live in the Granary.

That was a move I welcomed, as I was to be sharing the space with Tan
Kittisaro. He and I had been good friends already at Wat Pah Nanachat.
Sometime after that sojourn in the Granary, I moved into the loft above
the old Coach House: a very desirable residence, at least in terms of
the view over the walled garden and across to the South Downs. Other
than the bats that also occupied the Coach House I was alone there.
Eventually that semi-dilapidated building was to be replaced with the
current, very handsome, Dhamma Hall. Well before then I had moved back
into the main house and was living in the attic in a room where the
water tanks were situated.

I mentioned earlier that we were cautious about changing anything. In my
first or second year at Chithurst, I took the initiative to try and sew
a jacket. I think Tan Anando supported the idea. It seemed to me that if
we had a suitable jacket, that would mean we could wear pretty well
anything we wanted underneath it and still look presentable from the
outside. When Tan Ajahn Chah was visiting in 1979 and had seen the
variety of jumpers and t-shirts that community members were wearing, he
announced that when it was time for wearing the formal robes, they must
be worn covering both shoulders. Usually inside the monastery, and
always during puja, the main robe was worn with the right shoulder
uncovered. This originates from a custom in ancient India when baring
your right shoulder was seen as a sign of respect.

We obviously followed Tan Ajahn Chah's instruction; however, it appeared
to me a bit of a pity to lose such a long-standing traditional way of
wearing our robes. As it happened, a Sri Lankan gentleman who owned a
fabric dyeing mill in Leicester, had offered us a large amount of
polyester cloth dyed in our colour. It was with some of this cloth that
I experimented in making a jacket. In my mind I had a memory of how tidy
the jacket looked that my New Zealand friend, Bhikshu Ham Wol, wore. The
resultant garment received Ajahn Sumedho's approval, and in no time at
all the monks were wearing jackets, or English \emph{angsas}. Not long
after this, Ajahn Sumedho visited Thailand, and it was thought that, as
a courtesy, we should inform the elders there about this modest
adaptation. To that end, quite a lot of work went into producing a
rather fancy hardbound book with lots of photos, the idea being that
Ajahn Sumedho could present these to respected elders and hopefully gain
their endorsement. What I heard when Ajahn Sumedho returned from his
trip was that they had almost no interest at all in the book, the
implication being that they trusted us. That was reassuring news.

