\chapter{Expect the Unexpected}

In 1973, when I was experimenting with back-to-the-earth communal living
in Australia, and my Texan friend from Sydney persuaded me to join him
on my first meditation retreat, I definitely was not anticipating
discovering a wonderful new dimension of life. When I entered my first
Rains Retreat with Tan Ajahn Thate in 1975, it didn't occur to me that I
was about to have my world-view turned upside down and inside out. In
1991, when I joined the sangha at Amaravati for their Winter Retreat, I
was not anticipating the shift in perspective that introduced me to the
possibility of
\href{https://forestsangha.org/teachings/books/servant-of-reality?language=English}{\underline{living
life as a servant}} {[}131{]} rather than always trying to be a master.
I don't think anyone anticipated that between 2008 and 2010, thirteen
members of our siladhara community would disrobe or leave. The
unexpected can happen at any moment. And this is true for all living
beings. So what qualities do we need to develop in order to not just
survive, but to awaken to true wisdom and compassion? How do we make
ourselves ready so we can learn what we need to learn from life? In this
chapter I would like to reflect more thoroughly on the theme of agility.

\emph{Contrast and Agility}

In the world of photography, contrasting shades contribute depth to an
image, in music, contrasting tones produce richness. In our spiritual
life, contrasting modes of practice lead to agility and aliveness. In
the previous chapter I commented on the importance of cultivating
agility of attention. Without it we risk developing in an imbalanced
manner. For instance, we might feel confident when we are practising in
solitude, but become confused when in the company of others, and vice
versa. Well-established agility of attention supports our ability to
practice whatever the situation might be.

Tan Ajahn Chah was known for sometimes upsetting the rhythm of the
monastery by changing the routine, often without warning. He was aware
of our tendency to become complacent. With complacency comes lethargy,
and he was keen to keep everyone on their toes; he was encouraging us to
develop agility.

Anyone even slightly familiar with the Tan Ajahn Chah's teachings will
probably be aware of the emphasis he placed on the contemplation of
impermanence. There are Westerners who look to Tan Ajahn Chah as their
teacher, who have never visited Thailand and cannot speak the Thai
language, but from having studied his teachings know the words,
\emph{Mai nae}: `Not sure.' In
\href{https://forestsangha.org/teachings/books/the-collected-teachings-of-ajahn-chah-single-volume?language=English}{\emph{\underline{The
Collected Teachings of Ajahn Chah}}} {[}132{]}, there is a talk titled,
`Not Sure' (p 599), in which Tan Ajahn Chah says,

So I say, `Go to the Buddha.' Where is the Buddha? The Buddha is the
Dhamma. All the teachings in this world can be contained in this one
teaching: \emph{aniccaṃ}. Think about it. I've searched for over forty
years as a monk and this is all I could find. That and patient
endurance. This is how to approach the Buddha's teaching --
\emph{aniccaṃ}: it's all uncertain. No matter how sure the mind wants to
be, just tell it, `Not sure!' Whenever the mind wants to grab on to
something as a sure thing, just say, `It's not sure, it's transient.'
(p606)

Paying close attention to the universal law of impermanence is one of
the most important ways in which we can prepare ourselves to be ready to
meet the unexpected. So long as our minds are still under the sway of
unawareness, we continue to seek certainty and security in that which is
inherently uncertain and insecure. The conditioning that we have been
subjected to, ever since we were born, means we easily believe in the
way things merely appear to be. Without wise reflection we fail to see
beyond the apparent level to the truth of impermanence, and make the
mistake of blindly attaching to things that don't last: possessions,
opinions, preferences, relationships.

Sometimes I follow Tan Ajahn Chah's example and alter the monastery
routine without a lot of warning. Efficiency in the monastery is about
trying to maintain our hearts in a state of good-enough clarity. If, for
example, there is a lot of building work taking place and community
members are getting frazzled to the point where speech is becoming
heedless, we might just halt the project and have a week of noble
silence. Even if halting the building work at that stage is inconvenient
on a material level, it might be the right thing to do from the
perspective of training awareness.

Here at Harnham during the seven or eight weeks of noble silence that we
observe throughout the year, the answer-phone is switched on, the
internet in the office is switched off, and we remind ourselves that the
inner life is the priority. Just as we might stop eating food for a week
for reasons of physical health, abstaining from speech is a way of
putting our personalities on a fast. For those who are not used to it,
the first few days of noble silence might feel uncomfortable, but after
a while it can feel delightful. Then we shift back into talking mode
again, which offers an opportunity to study how our personality
reconfigures itself. The contrast can provide a useful perspective on
the dynamics of personality.

It is the same principle that determines group practice versus solitary
practice and living at the main monastery versus living in a kuti down
by the lake. This principle of contrast was behind the trips we used to
take up to the Scottish Highlands. One of our trustees owned a small
thatched roof cottage at the mouth of Glen Lyon and generously offered
us the use of it. For several years, usually during the summer, most of
the community would pack our alms-bowls and camping gear in the car and
head north for a week. The cottage was small which meant some of us
slept outside in tents. Observance of our monastic code of discipline
didn't change, but the kind of effort we made while we were hiking and
sitting around a campfire did have to change, and it was invigorating.
Then after a week, we would pack everything back in the car and return
to the monastery, to the same routine of morning chanting, morning
chores, group breakfast \ldots{}

We have an expression in English, `A change is as good as a rest.' This
is the same principle again: it is not necessarily the case that we have
to do something special like go on an intensive retreat before we can
feel refreshed and renewed; simply doing something distinctly different
can be enough. In a recent conversation with my good friend the abbot of
Wat Pah Nanachat, Ajahn Kevali, I asked why he kept sending junior monks
to train here at Harnham for a year or two, and he said it was because
he felt the contrasting experience was useful for them; it is not that
there is anything wrong with Wat Pah Nanachat, or that Harnham is a
better place to practice. Since he has been sending them here now for
several years, I think it is safe to assume the monks find it helpful.

\emph{Travel and Agility}

For the first ten or so years that I was living in this monastery, we
had major building projects ongoing. There was often a lot of noise and
a lot of mess, and having a break was one of the motivations for my
spending two or three months each year overseas. Also, when I was not
here to make all the decisions, the rest of the community had an
opportunity to learn. But on a personal level, when I reflect back on
those years of travelling, one of the most attractive aspects of that
time was the way the contrasting environments and cultures energized me.
It took energy too, not just making preparations to travel through four
or five different countries -- sometimes more -- but it also generated
energy.

I mentioned early on in this book how shocked I was when I witnessed the
contrast between the affluence and comfort of Darwin in Australia, and
the poverty of Portuguese Timor. That was the shock of awakening from a
dream that I was having, and though in that case it wasn't pleasant, it
was productive. The energy I experienced from travelling was something
similar: the newness of the situations in which I found myself required
letting go of assumptions and making an effort to stay open and alert.

For several years we would put our monastery car on the ferry at the
nearby Port of Newcastle upon Tyne and take an overnight trip across the
English Channel to Hamburg, in northern Germany. After a few days there
we drove down to Kandersteg, in Switzerland and stayed with Ajahn
Tiradhammo at Dhammapala Monastery. Sometimes those trips involved
giving talks to Buddhist groups in Hamburg or Heidelberg. I found it
invigorating to have to adapt to having my talks translated into a
language that I totally didn't understand, and found it interesting
learning to adjust to the ways different people presented themselves. I
recall that the Germans appeared riveted when I gave a talk and were
intensely keen on asking questions, which was altogether different from
the reserve of the British. On one occasion, not long after East Germany
had opened up, we took a detour via Berlin to Dresden where I saw the
\href{https://en.wikipedia.org/wiki/Zwinger_(Dresden)}{\underline{Zwinger}}
{[}133{]} that Jutta had helped rebuild after World War Two. On another
occasion it was a treat to walk the
\href{https://de.wikipedia.org/wiki/Philosophenweg_(Heidelberg)}{\underline{Philosophenweg}}
{[}134{]} in Heidelberg and reflect on the famous thinkers who had used
that path.

Besides finding energy in having to let go of preconceived ideas, I also
found a sense of vitality whenever an idea that I had long held onto, a
preconception, encountered the reality.

Seeing the Eiffel Tower in Paris was one of those experiences; something
shifts when the image in my brain that was produced by a two-dimensional
picture on a page collides with a multi-dimensional reality. There were
no longer just colours on a thin piece of paper, or fantasies in my
head; now there was the smell of Gauloise cigarettes, the sound of the
French language being spoken, and the site of this massive cold steel
construction -- the reality of Paris was thoroughly different from the
idea of Paris. Somehow in those situations, when a concept impacts with
the associated physical reality, it feels as if energy is released; such
contrasting perceptions are conducive to aliveness.

On a return journey from a visit to New Zealand, my Kiwi ex-monk friend
Mark invited me to stop over in Beijing, where he was working as a
doctor. He was keen for us to travel together to Japan. While we were
there I was struck by the composure and sensitivity of the women who
served tea on the train from Tokyo to Kyoto. Similarly, the dignity of
two businessmen bowing to each other on a sidewalk matched the
understated beauty of everything in the Muji Supermarket. Kyoto also has
their own
\href{https://en.wikipedia.org/wiki/Philosopher's_Walk}{\underline{Philosopher's
Walk}} {[}135{]}, and it was a privilege to be offered a place to stay
nearby.

Those impressions of Japanese culture powerfully contrasted with the
experience of being in China. The conduct of the Chinese was altogether
different from that of the Japanese, yet they were such close
neighbours. `How can human beings be so different?' Well, in many ways
we are that different; maybe I was beginning to see that letting go of
ideas about us all being the same was part of growing up. It can be a
symptom of mental laziness to ignore the complexity and uncertainty of
life.

Whenever a monk arrives at the main International Airport in Thailand,
he is greeted by the immigrations officers with hands held in
\emph{añjali}, and is invited to conveniently pass through the
diplomatic channel. In contrast, a twelve-hour flight later, on arriving
in New Zealand's Auckland Airport, it was likely that I would be taken
aside and left standing as my bags were thoroughly searched. And for
several years, whenever I entered Australia's Sydney Airport, I was
regularly told that my passport was listed as having been stolen.
Generally the immigration and customs officers in New Zealand and
Australia were courteous and, although having to face them after a long
flight did take some effort, I appreciated how they conducted
themselves.

And in comparison, my lasting impression of the United States of America
is that it is a country of extremes. During a visit to Los Angeles I was
invited to view an exquisitely beautiful and refined Japanese garden;
around the same time I was taken to visit the vast and extraordinary
\href{https://en.wikipedia.org/wiki/Monument_Valley\#:~:text=Monument\%20Valley\%20is\%20officially\%20a,equivalent\%20to\%20a\%20national\%20park.\&text=Parts\%20of\%20Monument\%20Valley\%2C\%20such,accessible\%20only\%20by\%20guided\%20tour.}{\underline{Monument}}
\href{https://en.wikipedia.org/wiki/Monument_Valley\#:~:text=Monument\%20Valley\%20is\%20officially\%20a,equivalent\%20to\%20a\%20national\%20park.\&text=Parts\%20of\%20Monument\%20Valley\%2C\%20such,accessible\%20only\%20by\%20guided\%20tour.}{\underline{Valley}}
{[}136{]}and \href{https://www.nps.gov/jotr/index.htm}{\underline{Joshua
Tree}} {[}137{]} National Parks. The memories I have of the three or
four times I entered the U.S. are all of feeling thoroughly unwelcome;
it seemed that the immigration and customs officers were trained to make
you feel like a criminal. Once I was in the country, however, every
American I met was exceptionally friendly and hospitable. Contrasting
with that overt friendliness was the unawareness of so many Americans
regarding anything that happened outside of their country. Despite its
phenomenal wealth, the majority of Americans have never travelled
abroad. In 1995 when the
\href{https://en.wikipedia.org/wiki/Oklahoma_City_bombing}{\underline{Oklahoma
bombing}} {[}138{]} took place, I happened to be staying in LA and was
surprised at the difficulty so many Americans had in accepting that the
perpetrators of that devastating blast were not foreigners. Then there
is the fact that the US has some of the most liberal laws on freedom of
speech, at the same time as being the
\href{https://www.statista.com/statistics/262962/countries-with-the-most-prisoners-per-100-000-inhabitants/\#:~:text=As\%20of\%20June\%202020\%2C\%20the,the\%20highest\%20rate\%20of\%20incarceration}{\underline{world
leader in incarcerating}} {[}139{]} members of its population.

India must be the overall world leader in contrasts of the senses:
colours, smells, sounds. I was grateful to have the chance to pay my
respects at the Bodhi tree in Bodh Gaya, but found my feelings for India
hadn't changed much since those days in Indonesia when my fellow
travellers were drawn there like iron-filings to a magnet; I just wanted
to go to Japan. The sensory overload, the glaring disparity of wealth,
and the way there always seemed to be someone trying to grab my
attention, made visiting there very hard work.

Giving talks in South Africa where everyone in the audience spoke
English, was strangely difficult. Giving talks in Italy, where only a
few in the audience spoke English, was a joy. I can't say I have a clear
sense of why, but it was noticeable. I recall that after a talk I gave
in Milan, one of the attendees approached, and, with a radiant smile and
exuberant gestures, poetically described how uplifted his heart felt
because of what I had shared, and how grateful he was.

Nothing like that has ever happened to me in the nearly thirty years of
my living in Britain. That is not to say that one is better than the
other; what I find interesting is the contrast and how it can quicken
useful contemplation. In case I sound like an ingrate, however, I do
want to comment on how good I consistently have felt when returning to
these shores after having been away. As mentioned already, I stopped
travelling several years ago, but I can still recall the sense of relief
that came once I was back in Britain. To avoid the risk of sounding
insincere I will be restrained here in my expressions of gratitude, but
I often regularly reflect on the privilege and pleasure I feel on being
permitted to live in this country.

\emph{Developing Agility\\
}

The way I have been discussing agility could sound as if I am saying it
is something new -- something outside of our Theravada Buddhist
tradition. This is not the case. In the discourse on the Four
Foundations of Mindfulness we are taught to exercise mindfulness
regarding the body; regarding feelings; regarding the quality of
awareness; and regarding those Dhammas that lead to awakening. Then
there are the four \emph{iriyapada} or `modes of movement': sitting,
standing, walking and lying down. Seeing Buddha images in these four
postures can serve to remind us that we should be making constant effort
in practice -- not just when we are sitting on our cushion. Also they
can remind us that awakening can take place in any situation, at any
time. Then there is a fifth posture, as displayed by the Buddha's
attendant, Ven. Ananda -- the between sitting and lying down posture.
After an extended period of making ardent effort to free his heart from
all remnants of unawareness, Ven. Anando accepted that it wasn't going
to happen, so he decided he would lie down and rest. Just before his
head hit the pillow, his heart was freed and he arrived at full
awakening.

Another way of approaching the development of agility based on what the
traditional teachings tell us, is to reflect on these two things: the
consequences of not having cultivated it, and the benefits of having
cultivated it.

I mentioned already the example of someone who is adept at practising in
solitude but struggles when they have company. Similarly, some people
will feel confident when conditions are conducive to maintaining a
degree of \emph{samadhi}, but when conditions are not conducive, they
struggle and fall into old habits of resistance. If our preferences are
never challenged, we remain vulnerable. Over the years when I was
travelling, my preferences were significantly challenged every time I
had to pass through Customs. This was especially true in New Zealand
where an extremely strict biohazard policy is enforced. One monk I knew
was fined something like \$200 because he omitted to declare his wooden
\emph{mala}-beads as he entered the country. After having travelled
there several times I learnt to not wait until we were about to land
before writing down whether I had worn my sandals on a farm in recent
weeks, and to list the herbal remedies I was carrying. However, I never
managed to feel relaxed as I passed through Customs. The experience did
serve to highlight limitations in my practice.

There were other occasions when I enjoyed receiving confirmation of the
benefits of the training I had been doing. On an occasion when I was
staying with my parents and was out for a walk, a carload of louts drove
by and one of them threw an egg at me. That was different from the way I
was used to being treated, but I was pleased to discover it didn't
disturb me too much. Fortunately I was on my own; the thing that did
disturb me was how upset my mother would be if she knew. As I recall, I
managed to get into the house and wash my robe without her finding out.
Later when I was describing the incident to Ajahn Karuniko, he reminded
me that the same thing happened to Her Majesty the Queen when she was in
New Zealand.

We don't need to be engaged in international travel to reflect on the
advantages of having cultivated agility and the disadvantages of having
not cultivated it. I have stopped flying now for many years, but there
is no shortage of situations in which my agility is tested. Every time
we don't get our own way is an opportunity to strengthen our commitment
to the training -- to deepening our refuge in the Buddha: selfless
just-knowing awareness. Every time I don't get my way is a time to stop
and check: am I going for refuge to the way of the Buddha --
\emph{Buddhaṃ} \emph{saranaṃ gacchami--} or am I going for refuge to `my
way' -- \emph{attaṃ saranaṃ gacchami}?

\emph{Applying Agility\textbf{\hfill\break
}}

Over the years, I have found myself in the middle of a number of intense
dilemmas: powerfully uncertain situations with potential for
far-reaching consequences, where I am required to make a decision.
Perhaps these were partly due to my having been put in a position of
leadership whilst still young and unprepared, or maybe it was just
happenstance. What I can say, though, is that those situations tested me
deeply, and in that testing I learned a lot. I would never have chosen
to have to face those dilemmas, but now I can feel grateful. On occasion
the situation involved telling somebody something that they did not want
to hear: it was my job to tell them, and yet I couldn't see how to do it
without hurting them. In most cases there was also a risk of my being
hurt in the process. It is not appropriate to describe here the details
of some of those incidents, but I think the situations are worth
mentioning by way of demonstrating the advantages of developing agility.
For example, occasionally it has fallen to me to tell someone that I
thought it is time for them to take leave of the community, and that is
really difficult. At least as far as I am concerned, wearing robes is
not a guaranteed formula for progressing in the spiritual life; indeed,
for some it can be a hindrance. Likewise, if I don't feel confident that
an applicant for ordination will benefit from the pressure they will be
under, I won't support their taking up robes in the first place.

On one occasion when I was staying at a monastery abroad, I had the very
difficult task of telling someone that I thought they had exhausted
their options in trying to make this lifestyle work for them. The monk
was not junior in the training and I did not want to be the one to tell
him. This was a time long before mobile phones and the internet, so
consulting with elders in other monasteries was not realistic. I didn't
know how to say it or when or where to say it. Thankfully, the years of
developing patient endurance, restraint and reflection enabled me to sit
with the not-knowing, feel the not-knowing, breathe through the
not-knowing -- until one day, almost without planning, the words were
said. Very quickly after that, things fell into place and, given the
potential for it being otherwise, the dilemma resolved itself without
too much difficulty.

Another situation occurred involving my parents. It wasn't so much a
dilemma, but nevertheless had the potential for triggering considerable
conflict. (I appreciate that those who are unfamiliar with the culture
of evangelical Christianity might not recognize the dynamics involved.)
It was a festival day at the Auckland Vihara on Harris Road and members
of the Auckland Theravada Buddhist community had invited me to receive
the midday meal and offer a Dhamma teaching. They also took it upon
themselves to invite my parents to participate. I was the only monk
there on the occasion, and it felt surreal to see my parents sitting on
chairs in the front row with almost half the floorspace of the room
covered with food offerings that were about to be given to me. This was
the first time my parents had seen me officiating in that capacity; they
had seen me in their home, but there, in their eyes, I was still very
much Keith. Here I was Ajahn Munindo and the Sri Lankans, Thais and
Burmese were utterly unrestrained in their expression of gladness and
devotion for `their monk'. I feel that my appreciation for and practice
with agility contributed to my being able to honour the occasion,
receive the gestures of devotion, accept the offerings of food, and
deliver a befitting Dhamma talk. I heard afterward that one of my
parents had commented about how impressed they were that I had managed
to deliver the equivalent of a twenty minute sermon without any notes --
they said, `He has got the gift of the gab'.

\emph{Meeting Dukkha with Agility}\\
~\\
All of us have, locked away in our basement of unawareness, varying
amounts of unacknowledged \emph{dukkha} -- unreceived suffering. If we
are fortunate enough to reach the point where we find it intolerable to
continue to deny it, and we feel inspired to sort it out, we should be
prepared to feel intimidated, on several levels at the same time; it can
be dark down there. Depending on how much we have stored away and for
how long, for some it can be very frightening. We have put ourselves
under pressure by engaging in such spiritual exercises as chanting,
concentration, restraint, fasting, and extended periods of silence, so
we should expect to meet that which we have previously denied. If we are
equipped with agility of attention we will be better placed to know
where, when and how to stop resisting the suffering -- how to fully
receive it and let go.

It is useful to be ready to enquire: is it present-generated
\emph{dukkha}, or old unacknowledged \emph{dukkha}, or adopted
\emph{dukkha}? What I refer to as `present-generated \emph{dukkha}' is
that which we are actively generating, right here and now, by resisting
what is. `Old unacknowledged \emph{dukkha}' is still suffering, but this
term refers to the backlog of suffering resulting from our having tried
to avoid it in the past. When we don't understand this particular aspect
of \emph{dukkha} we can become confused. For instance, a minor incident
of sadness can trigger an unexpected extreme reaction of deep grief.
That small moment served to open the door to the room in our basement in
which we had previously stored all the sadness we either didn't feel
ready or able to receive.

The expression `adopted \emph{dukkha}' is a term I use to refer to the
suffering we pick up, so to speak, from our environment. For example,
all human beings feel fear, and if, particularly in our early life, we
were surrounded by adults who carried within them a heavy burden of
their own unacknowledged fear, there is a chance our fear can become
potentized as a result. We are still responsible for it -- this way of
thinking about \emph{dukkha} is not a way of blaming others -- but I
find it helps in understanding why our struggles can, at times, seem so
onerous.

Analysing the type of \emph{dukkha} that we are dealing with makes
spiritual work more manageable. We also need to be ready to enquire into
the whole body-mind, not just the mind. Letting go of \emph{dukkha} is
what we are interested in, but holding on in ways that cause
\emph{dukkha} is constantly taking place on different levels.

Hopefully most meditators learn early on that we need to let go of the
stories we tell ourselves in our heads. Perhaps we have already made an
effort in that direction and are skilled in inhibiting the
story-telling, yet still do not feel open, trusting and engaged in life.
Especially for many men, it can come as a surprise -- and not an easy
task -- to admit that they have been numbing their hearts. To allow
sensitivity, without indulging in sensitivity -- without making it into
a `me' who is sensitive -- is a task requiring embodied awareness. That
can be challenging when we are so used to being identified with our
thinking mind.

If we have some competence in letting go of the story-telling in our
heads and are able to allow a balanced quality of sensitivity on the
heart level, then there is the third dimension of our body. Early on in
life, as we attempt to navigate our way through the uncertainty of our
inner and outer worlds, we often develop unconscious habits of
compulsive controlling of our breath which develops into chronically
contracted muscles. Learning how to let go on the physical level is a
different skill from letting go mentally and emotionally. We need
agility of attention to be able to discern where we are resisting. I can
recall around the age of eight or nine, when at school we stood in rows
for morning assembly and were taught to recite loudly, `Push back your
shoulders and hold up your head, and don't close the window when going
to bed.' The final phrase of the ditty might not have been bad advice,
but the two earlier phrases were disastrous, at least in my case. It was
many years before I discovered -- thanks to breathing exercises -- that
I had developed a rigid pattern of constriction in my back. I have heard
that pushing back one's shoulders and tilting the head upwards is a
means of blocking feelings: if you look at soldiers on parade that is
often what they are doing. I don't want to disparage the good teachers
at Morrinsville Primary School 1960, but it does help to see cause and
effect.

\emph{Rapprochement}

So far, in this chapter we have considered this theme of agility in
terms of it being a Dhamma principle, as well as the advantages of
having developed it and the disadvantages of having not developed it. A
lot of this consideration has been in terms of our subjective or inner
experience. Before ending this contemplation, I want to mention a couple
of examples of the benefits of bringing this aspect of practice into the
realm of our outer world -- our relationships.

Part way through the winter retreat of 1999, I woke up one morning from
a powerful and disturbing dream; it featured Luang Da Maha Bua. That
fact alone made it noteworthy; my dreams are not usually about Dhamma
teachers. The most noteworthy aspect was that in the dream the acclaimed
and virtuous teacher, Luang Da Maha Bua, was secretly running a fishing
business on the side. The recognition that my mind had created a story
about such an honourable being behaving so dishonourably really shook
me. Twenty years have now passed since that dream so the details are a
bit faded, but I suspect I didn't have much of a choice other than to
feel into what was behind this unsettling image. Very quickly my mind
presented me with the message that it was me who was behaving
dishonourably: I was presenting myself to the sangha and to the laity as
the abbot of Harnham monastery -- delivering Dhamma talks and offering
guidance -- and at the same time, I was harbouring thoughts of
resentment towards my teacher, Ajahn Sumedho. The disharmony between us
wasn't a total secret; it had been going on for a number of years.
Although I wasn't being overtly unpleasant to Ajahn Sumedho, neither had
I made amends for a disagreement that had occurred some years earlier.
The message sounded loud and clear: this is intolerable and
dishonourable and I have to try to make things right again.

To my great relief, I easily managed to get through to Ajahn Sumedho by
phone and, without attempting to explain anything, I simply asked if it
would be OK if I came down to Amaravati to see him. Although our
monasteries were on retreat he said that was fine. So with another young
monk, Tan Revato, as a companion, we took the two or three hour train
trip south. It was a cold and snowy winter and I was heading into the
deep unknown.

The only thing I remember about the meeting was that it quickly became
obvious that neither of us felt any need to talk about whatever had
happened in the past. I was able to bow and offer my sincere respects
and gratitude to my teacher once more, and it was over -- after years of
iciness, we were at ease in each other's company. Thank you yet again,
Ajahn Sumedho.

It could have very well been otherwise. Holding onto hurt feelings and
projecting the pain that we cause ourselves onto another is very common
in our world. My holding on for as long as I did caused disharmony in
our community and I feel remorse for that; I regret that it took me so
long before I was able to do the right thing. At the same time I feel
very grateful that I was eventually able to let go and make amends. That
the impasse had been resolved was also noticed by others; a senior monk
in Thailand who had been aware of the difficulties even sent a gift to
our monastery in appreciation of what he referred to as a rapprochement.

Another surprising and pleasing resolution occurred a few months ago. At
an early stage in preparing notes for this book, I realized there were
gaps in my memory, and probably the only person who could help fill
those gaps would be my sister, Jennifer, a pastor with the Assembly of
God community in New Zealand. We hadn't spoken in the nearly four years
that had passed since our mother had died. The recollections I had of
every interaction with my siblings over the past forty-five years were
all without exception unpleasant. As a result, any time the thought of
being in touch with them arose in my mind, I quickly dismissed it.
Initially, when I started visiting New Zealand to see my parents, I had
tried to have a normal, friendly interaction with them, but it never
happened. On this occasion, when the thought occurred to me that I could
contact Jennifer to discuss details, I was delighted to find there
wasn't the immediate reaction of, `No way -- that is not going to
happen'. It was not that I had suddenly forgotten the pain of always
feeling judged, just that this time -- and this was new -- those
memories and perceptions felt somehow more remote: almost as if they
belonged to a different era. It now felt OK to trust the impulse and at
least take the first step of searching the web for a phone number.

Without too much trouble I found a mobile phone number for her husband,
Guthrie, my companion of many years ago at the Ngawha hot springs near
Kaikohe. The roughly fifty-minute phone conversation with my sister that
followed was less than encouraging, but thankfully I found I had enough
alertness to be able to receive the things that were said (and not said)
without too much resistance.

A few days passed, during which time I allowed the pain that had arisen
as a result of that call to simply be there as long as it wanted to be
there. Then, once more to my surprise, I felt moved to call again; this
time using Zoom, which would have the advantage of our being able to see
each other more as people, and perhaps less as ideas. What a delight
that call turned out to be. I'm not sure now exactly how the
conversation unfolded, but I do clearly remember being able to say
things about the dynamics within our family over the years, and then
hearing my sister acknowledge that she could understand that I would not
want to have to keep defending myself all the time. Something began to
dissolve. How powerful just a few words can be. That call lasted about
forty-five minutes, and this time, when it ended, it was with a feeling
of lightness. I hadn't set out with any agenda to put things right, but
I was interested in being available if something new was ready to
emerge.

It was good to see that my mind was not overly interested in the idea
that this reconciliation should have happened sooner. Obviously I would
like to understand what causes and conditions contributed to such an
agreeable change in course, but asking `why' with an assumption that I
should understand, seemed pointless. We can't know the degree of
habitual resistance to reality that we carry within us; we don't know
what old kamma we have stored away. We can, however, come to appreciate
that often it takes time before the momentum of habitual resistance
slows down enough to arrive at the point where we can actually catch
ourselves, in the precise moment when we are about to create a problem
out of life.

The next morning I took the opportunity to send an email sharing with my
sister how good I felt about our meeting. I apologized for my part,
having in the past inflicted pain on our family. She replied very
quickly that she was delighted at being reunited with her brother again,
but also mentioned the respect she had for me and for the choices I had
made. She asked for forgiveness for any hurt that they had caused me and
suggested that we could have more Zoom calls since she was keen to find
out about the community in which I lived. Wow! That was not expected.
Thank you, Jennifer.

I like to think that the effort to cultivate conscious gratitude (which
was one of the main reasons why I started writing this memoir) along
with a daily ritual practice of dedicating the \emph{puñña} of my
practice to my teachers and to members of my family, contributed to this
happy resolution. Re-establishing communication with my sister feels
like a harmonious resolution of powerful, long-held misunderstandings. I
am glad that we are talking with each other again. Since then my younger
brother Bryan has also been in touch to wish me happy birthday. We ended
up having a video chat; our first conversation in eighteen years. He is
a competent photographer, particularly of birds, and we have been
exchanging photographs. Most recently he sent me some shots he had taken
of some
\href{https://www.eaaflyway.net/the-incredible-godwit-migration/\#:~:text=Bar-tailed\%20Godwits\%20spend\%20the,fly\%20back\%20to\%20New\%20Zealand.}{Godwits}
{[}140{]} which annually fly many thousands of kilometres from the
Northern Hemisphere to New Zealand, and back again. It occurred to me
that the relationship Bryan and I have has come a long way though it has
taken us about fifty years.

It is easy to indulge in the assumption we will feel grateful once we
have got what we are looking for. Another way of approaching life is to
intentionally dwell on feelings of gratitude for the goodness that we
already have, and to witness how attending to gratitude helps life to
flow.

