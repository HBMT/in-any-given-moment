\chapter{A Relief to Be in Britain}

Unfortunately, family commitments meant that Mason and I didn't manage
to meet up in Los Angeles, we only spoke on the phone. The monks at the
Wat Thai temple were helpful in providing a place for me to break my
journey. Then on 10th June I arrived in Britain. On exiting the airport
I had a sense of relief; somehow I already felt safe in this country and
was glad to be here. A friend of the sangha, Paul James, was waiting to
drive me to the monastery in West Sussex. It was not long before I met
Ajahn Sumedho, and he was still his ebullient self. He immediately took
me on a tour of the property, which included not just the impressive mock Tudor
Victorian mansion of Chithurst House\cite{cbm-house},
but also the nearby substantial forest and lake
property. On that particular day there was work being done on the weir
at one end of the lake. It was known as Hammer Pond, because in the
seventeenth century the water spilling over the slipway drove a hammer
which was part of an iron forge. One of the people busy working on the
repair of the ancient weir was Chris, a Kiwi fellow who later took
Precepts as a monk with the name Tan Thitapañño.

Earlier I have described the atmosphere during the first years at Wat
Pah Nanachat as a combination of focussed spiritual aspiration,
pioneering spirit, and New Age adventure. Life at Chithurst was much the
same, but was a great deal more `yang'. For one thing, the climate was
cooler, and there was almost an urgency to community activity. Instead
of everyone wearing only a lightweight waistcloth and flimsy
\emph{angsa}, here people wore a lot more clothes; instead of crochet
hooks and whittling knives, here it was chainsaws and concrete mixers.

The large room in which I would be staying overlooked the bucolic
countryside of that part of West Sussex. I would be sharing the room
with Tan Sucitto. He went out of his way to be hospitable and expressed
concern for my welfare. Tan Sucitto had spent time living in a
meditation monastery in Thailand but, at that stage, had never visited
Wat Pah Pong or Wat Pah Nanachat. As I said, he was very attentive to my
well-being; he did strike me as a bit of an enigma though. I had trouble
getting any sense of what made him tick. Over the years that followed, a
friendship developed between us, based on trust. It turned out that he
similarly didn't quite know what to make of me, and that didn't matter.
Trust and respect were more important.

Throughout all the years I have been living in the UK, I have felt
grateful for the privilege: grateful and extremely fortunate. Although
the English themselves excel when it comes to criticising their own
institutions and traditions, personally, I find it easy to heap praise
on the country; since arriving here I have never wanted to live anywhere
else. I do confess, however, that it took me about twenty-five years to
eventually admit how inscrutable I find not just the English, but the
British as a whole. As a country, they obviously understand each other,
but even after several decades I remain mystified. They almost never say
anything directly -- rather, there are hints and innuendos that one is
supposed to interpret. It is not surprising that I have been told more
than once that some find my Kiwi character a bit coarse. It was naive of
me to not admit sooner how often I have struggled to find my place in
situations in which I find myself. The truth was that I was feeling
excluded, and I didn't want to admit that. When eventually I did admit
it, it was a relief, and didn't seem to matter too much. This wonderful
country is full of misfits and, to me at least, there seems to be an
admirable willingness to tolerate people who are different. Of course,
at the same time I am aware that in Britain, as in other countries,
there is a lot more that could be done to address inequality on many
levels.

Perhaps this hesitation to admit feeling like I didn't fit in is similar
to the way we refuse to admit to ourselves that one day we will die; by
that, I mean that we habitually lie to ourselves about things we find
difficult to acknowledge. Out of wisdom and compassion, the Buddha
instructed his followers to regularly reflect on the inevitability of
their demise. Those who heed his instructions are sometimes surprised at
how good it can feel to own up to that fact. It can release a lot of
energy when we stop lying to ourselves. I intend to later reflect more
on this aspect of teachings by the Buddha and how they have impacted
upon me.


