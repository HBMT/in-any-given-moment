\chapter{Better a Monk Than a Drunk}

Bhikshu Ham Wol had arranged for the two of us to be staying at a
Tibetan Centre in Mt Eden, a particularly lovely part of Auckland. Like
much of New Zealand's largest city, this suburb is nestled on the edge
of a dormant volcano (there are roughly
\href{https://en.wikipedia.org/wiki/Auckland_volcanic_field}{\underline{50
dormant volcanoes}} {[}47{]} in Auckland alone, and many more volcanoes
throughout the country, several still active).

This Tibetan Buddhist centre was of the Gelugpa tradition associated
with Lama Yeshi and Lama Zopa. I had attended a session of talks by
these two Lamas some years earlier in Bangkok, but I couldn't say I knew
much about their practices. It did seem though that they placed a lot of
emphasis on study. Study seemed to take precedence over meditation.

Meeting and staying with this group in Mt Eden was enjoyable. They were
exceedingly considerate. I doubt whether any of them had ever met a
Theravadin monk before and would not have been familiar with our 227
rules. Bhikshu Ham Wol, who had arrived from Korea quite a few months
earlier, had a distinctly different take on the monks' rules. He had
spent time in Thailand and had initially taken novice (\emph{samanera})
precepts there, but had left and gone to Korea to take up the monks'
training. The most obvious difference between us, besides the fact that
his robes were grey and mine were brown, was that he handled money. He
was also comfortable cooking food; these two points meant that the
laypeople found him much easier to look after. For my part, there was no
way I was going to dilute my commitment to the discipline as it had been
taught to me by my teachers. Although Bhikhsu Ham Wol and I had not met
before, other than in written correspondence, we quickly settled into an
easy mode of cooperation and mutual support. I felt our rules served as
a protection and source of strength, while he seemed to perceive them as
creating an unhelpful distance between us and the community of
householders; but that difference never caused any issues between us. We
were both excited to be in New Zealand, and obviously this was a time
when interest in Buddhism was beginning to blossom. In the few months he
had been there he had already connected with many groups, and he was
keen that we travel around the country together and offer teachings to
them.

One of those connections was with the FWBO, or the Friends of The
Western Buddhist Order. As far as I could make out, this was an
organisation attempting to establish itself as a secular community of
people committed to the Buddhist teachings, contrasting itself with the
monastic communities found in Asia. Its founder, Sangharakshita, was a
contemporary of Ajahn Khantipalo; they had spent time together in India
but ended up parting ways. This group in Auckland was very energetic and
impressively well set up. Before I left to go north and spend time with
family, I visited their library and they generously allowed me to borrow
from their reference section the five volumes of the \emph{Vinaya
Pitaka}. While I was living at Wat Pah Nanachat I don't think we had an
English translation of the \emph{Vinaya} \emph{Pitaka}; in those days we
took our guidance from the Thai commentaries ‒ \emph{Vinayamukha,}
Volumes one and two.

My arrival back in Opua must have been hard for my parents. I was
painfully skinny to look at, which for a mother can be no easy thing.
She knew there was no room for negotiation over my eating in the evening
but I did agree to eat twice a day before noon. Thankfully, there was no
big welcome home party organised; I would have found that difficult to
handle. As far as I recall, I settled into a routine which included
doing morning and evening chanting, periods of formal meditation and
spending time talking with my mother. The role that external structures
play in containing internal chaos was something of which I was already
aware. My father and I had never had much of a relationship, but on this
occasion he did seem keen to take me out for walks. Because he was so
private, I never did find out from him what he really thought about the
way I was living. One day, though, I dared to ask my mother directly if
my father was embarrassed to be seen with me in robes, walking around
the village. She was emphatic that the way I appeared, and for that
matter what other people thought, was of no concern to him. Apparently
he had said that he would rather have me as a monk than a drunk. That
wasn't exactly a fulsome expression of appreciation, but coming from my
father it felt like a gesture of approval.

Regularly I would go out on my own for long walks in the beautiful
surrounding countryside. I was intentionally trying to build up some
physical strength. The food my mother prepared was nourishing, and there
was no need to be concerned about offending Thai customs. In Thailand
there is an accepted view that monks should not pay too much attention
to their physical strength. That has positive and negative consequences.
On the positive side it is obviously suitable that the emphasis be on
letting go of vanity, however, on the negative side, it means that these
days many monks suffer from obesity and diabetes. For those walks I
would pack the five volumes of the \emph{Vinaya} \emph{Pitaka} into a
bag and carry them on my back just as someone at the gym might lift
weights. I like to think I also studied the texts, though now, nearly
forty years later, I can't say for sure.

Being alone as a monk in that situation after having spent several years
contained within an intensely structured environment, triggered anxiety.
My mother had spent much of her life burdened with anxiety and perhaps I
had inherited some of those traits from her. Even without that, the
situation was odd: I was twenty-nine years old, the only Theravada
Buddhist monk in the country, and with an awareness of how the way I
conducted myself could have consequences for monks who would visit New
Zealand in the future. I was making more out of it than was needed, for
sure, but maybe that was better than being in a hurry to feel overly
relaxed.

