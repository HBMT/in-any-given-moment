\chapter{The Devon Vihara}

During the year of 1981, Tan Sucitto and a new young anagarika, Philip,
from York, had been sent by Ajahn Sumedho to see what could be
established in Northumberland. They went to a place called Harnham, a
few miles north of Newcastle and just south of the Scottish border. A
group of local people who shared an interest in yoga and meditation, had
earlier attended a retreat led by Ajahn Sumedho at Oakenholt, near
Oxford. After that retreat they visited Chithurst and upon observing how
hard everyone was working, offered to see if a place could be found
where community members might wish to go for some retreat time.

By 1983, anagarika Philip had already taken on the monks' Precepts and
was now called Tan Chandapalo; Ajahn Sumedho asked him to accompany me
and go to live in Raymond's Hill, near Axminster. We were being invited
to see what could be established in Devon.

We moved from a substantial four-storey mansion near the South Downs in
West Sussex, to a two and a half bedroom bungalow on a busy tourist
route near the South Devon coast. I loved it. Having an opportunity to
spread my wings, so to speak, was a relief. Not that living at Chithurst
wasn't agreeable, just that by age thirty, and having been a monk for
five going on six years, I welcomed having an opportunity to write my
own programme. Also, I think having a sense that Ajahn Sumedho trusted
me enough to take on such a project, mattered. The modesty of the
accommodation wasn't a problem, though conditions were cramped. The four
of us -- myself, Tan Chandapalo and two anagarikas -- were living in
very close proximity, and, thinking about it now, I admire the good
effort made by everyone.

The main heating was an old Rayburn stove, I think, fuelled by off-cuts
from a chipboard factory. My sensitive nostrils registered the
interesting smell as those off-cuts burned. The room I had, and the one
shared by the two anagarikas, were adequate, but Tan Chandapalo, who was
very tall, lived in an exceptionally small room alongside the boiler
tank. He never complained about it -- in fact, quite the opposite. More
than once over the years he has mentioned to me how much he enjoyed his
time there. He took the opportunity to learn to recite the Patimokkha,
which was no small feat. He and I both enjoyed walking, so would often
venture out into the surrounding countryside; at least once we made it
to Charmouth where we went for a swim.

The house was old but provided adequate shelter. Occasionally it
required a bit of renovation. One day, when I was taking a bath, I was
puzzled to see a dark black line gradually creeping up the wall besides
me. `What on earth!?' It turned out that it wasn't a black line creeping
up the wall -- the bath was sinking through the floor. Fortunately the
floor was only a small distance above ground level. Had that happened
upstairs at Chithurst, the consequences would have been very different.
I don't remember now whether we ever fixed the sunken bath. At a much
later time I found out that one of the trustees had taken out a
substantial loan to facilitate the purchase of that property, which was
typical of the extraordinary generosity of the group of people involved.

We had neighbours living very close by on both sides of the bungalow.
They could hardly have been more different. On one side lived Father
Straub, a retired Roman Catholic priest, who was a thoroughly sweet and
friendly person and someone with whom it was always a delight to spend
time. On the other side lived a fellow who held the view that human
beings were a virus or a blight steadily destroying planet earth. He
told me that he included Mother Theresa in that world view. He had a
less sweet demeanour and was less lovely to spend time with.

Whatever arguments one might make for being critical of the world we
live in, and the way our fellow human beings conduct themselves, it has
long seemed to me that having a pessimistic view of things only serves
to worsen the situation. Being naively optimistic strikes me as
irresponsible, which is part of the reason why I try to cultivate the
perspective of a strategic optimist. I am aware that the future could
turn out to be very difficult for everyone, but I am also aware that if
I dwell on negativity I become part of the problem. Choosing to assume a
positive outlook means that I am more likely to act constructively. It
is also possible that I was just born with a positive disposition.

One of the trustees, Geoffrey Beardsley, was a lawyer and had been
associated with the English Sangha Trust for some years. Two of the
other trustees, Douglas and Margaret Jones, were school teachers, and I
remember their being intensely committed to their Dhamma practice. On
the small farm where they lived, called Golden Square, they had
converted what looked like the old milking shed into a large meditation
room, and it was there that we regularly conducted retreats. I think it
was also there that I met that fellow who thought we ought to be wearing
saffron tracksuits.

I am confident that it is there that I met Sue Warren. Sue would have
been about eighty years old by this time. Had she met Aunty Mabel, Mrs
Parker and Mrs Gurusinghe in New Zealand, they would have got on like a
house on fire. Formidable! Sue lived within walking distance of Golden
Square. On occasion, when Tan Chandapalo and I would go there to receive
the midday meal, she would regale us with stories about what it was like
travelling around Germany in the 1930s, also about her years in the
\href{https://en.wikipedia.org/wiki/Women's_Royal_Naval_Service}{\emph{\underline{WRENS}}}
{[}59{]} during World War Two. She shared too how inspired she had been
many years before when she met the Thai Dhamma teacher, Dhiravamsa. Sue
was very committed to her meditation and Dhamma studies. Some years
later, after I went to live at Harnham in Northumberland, she sold her
house in Devon and moved to live in a retirement home in Newcastle where
we would regularly go to visit her. She was ninety-five when she moved,
and lived there until she passed away at one hundred and one. Despite
having had a rather privileged upbringing, and having moved from a
not-insubstantial house in a gorgeous setting in rural Devon, when it
came to spending her final years in a single room in the suburbs of
Newcastle, she was able to do so with ease. Sometimes she told me how
sorry she felt for the other residents in the home who had not prepared
themselves for that stage of life. When Sue died, it was as she had
requested, with someone reading to her from one of her favourite Dhamma
books by Ajahn Buddhadasa. A beautiful person who lived her life well:
it was a privilege to know Sue.

There was an occasion when Ajahn Sumedho was visiting us in Devon and he
and I were walking from Golden Square to Sue's place. I can't recall
what we were discussing as we walked along, but I do vividly recall at
one point his turning to me and saying rather sternly, `You don't have
to be like me you know!\emph{'} That didn't strike me as a
characteristic Ajahn Sumedho comment, and I hadn't been aware that I was
imitating him, so I was a bit surprised; however, I am glad he said it.
It seems to me that during our teenage years, imitation can be a valid
way of experimenting as we seek to find a meaningful direction in life.
But by the time we are in our thirties, we will hopefully have begun to
move beyond imitation. Growing up is really hard work, so thank you
again, Ajahn Sumedho, for that. Although I have no idea of what aspect
of my behaviour he was referring to, the fact that he said it was a
gift.

He was probably not referring to the routine we followed at the Vihara.
All branch monasteries of Tan Ajahn Chah's main monastery, Wat Pah Pong,
were expected to follow a somewhat similar structure. Hence, as at Wat
Pah Nanachat and Chithurst, our daily routine at the Devon Vihara
included regular morning and evening chanting. Early on we established a
practice of having public pujas on Thursday and Sunday evenings, during
which time I would offer a Dhamma talk. Within a very short period of
time a sizeable group of participants were joining us. Tan Chandapalo
was a very junior monk at the time, which meant all the talks were given
by me; similarly, I would lead all the retreats.

Because we lived in a built-up area, there were often occasions for
going out on morning alms-round and stopping for tea and convivial chat.
One of our regular stops was with Canon Horrocks and his wife. As with
Tan Chandapalo, Canon Horrocks originated from Yorkshire, and it
appeared he and his wife had moved to Devon when he retired. Another
regular stop was at a New Age community called Monkton Wylde. The group
of approximately ten residents always welcomed us warmly and often
offered food that was freshly picked from their impressive vegetable
garden into our alms-bowls. Several of the community members attended
our Thursday evening meetings.

Once a week we would walk down the hill to the village of Axminster. It
was a reasonably long walk and by the time we were in the middle of the
town, I welcomed the opportunity to sit for a while in the Minster. It
was in this Minster that during one period, possibly it was Lent, the
vicar organised a silent group sitting session. This was something we
felt we could participate in, and it was here that we first came in
contact with Harry and Mac. From then on our weekly alms-round to
Axminster always included a stop-off at their home with Harry's mother
participating in the offering of alms-food. Harry and Mac eventually
moved away from Devon up to Southampton, and became frequent visitors
and supporters of Chithurst.

It was around this time I heard that my good friend Mason Hamilton
(ex-Nane Dhamiko, Tan Jotiko/ Bill Hamilton) and his wife had been
killed in a motor accident in the U.S.; their baby daughter, Metta,
survived the crash. Obviously nobody saw that coming. Even though the
Dhamma teachings tell us that all things are impermanent, much of the
time we behave as if things are permanent. When something like this
happens, it can help us see how deep our refuge in Dhamma truly is.

After a year, Tan Chandapalo returned to Chithurst and was replaced by
another young monk, Tan Dhammapalo, and two new anagarikas. Tan
Dhammapalo was soon replaced by Tan Nyanaviro.

It was during my second year at the Devon Vihara that I received word
that Ajahn Sumedho had scheduled me to lead a lay retreat in
Switzerland. One of the new anagarikas, Jurgen, whom I had known as a
baker of bread in Brighton, was to accompany me. Since I had never
visited the continent before, I asked if we could travel to and from
Switzerland by train. Some of the strongest impressions I have of that
trip are the smell of chocolate wafting into our carriage as the train
waited at the station in Brussels, and the smell of cow manure that was
spread on the fields around the Stafelalp retreat facility.

As for the retreat itself, that was a combination of enthusiasm and
anxiety. The dedication and discipline of the Swiss retreatants were
inspiring and energizing. However, nobody had ever offered me any
guidance regarding leading a meditation retreat: what might be helpful
to bear in mind and what would be good to watch out for; it was a case
of sink or swim. Fortunately I found I was able to swim, though these
days I make a point, before and after sending a junior monk out to give
teachings, to check in with him and make sure he feels supported. For
most people, public speaking can be nerve-racking, and if the subject
matter is the most precious aspect of our lives, our spiritual
commitment, it can be terrifyingly difficult.

Around the same time that we moved to Raymond's Hill, a group of
committed Buddhist laypeople took up residence in a wing of a building
known as
\href{https://queenofretreats.com/retreats/the-sharpham-trust-england/}{\emph{\underline{Sharpham
House}}} {[}60{]}, near Totnes. Some of this group were associated with
the Vipassana community of the Insight Meditation Society, near Boston,
Massachusetts, U.S. There was already a meditation group in Totnes which
I regularly visited, and I expect it was through them that we met the
Sharpham House folk.

As far as I could tell, it was thanks to the largesse of Maurice and
Ruth Ash that a large portion of this grand villa was made available for
Dhamma-related activities. The charming grounds, which included a
sculpture by Henry Moore, were the location for several `Dhamma
picnics'. Besides being an opportunity to simply enjoy each others'
company, these occasions meant that members of the different schools of
Buddhism could meet and discuss shared concerns.

One of the things that contact with the Sharpham group brought into
focus for me was the nourishment that our community derived from
lineage. I confess that by that time, I might have started to take for
granted the benefits that come from being part of a long-established
community. At least one person in that group living in Sharpham House
shared with me their sense of uncertainty about the future of their
community. The fact that this caught my attention suggests that this was
not something I had even considered. The sangha had been around for such
a long time, like a great river that had flowed along the same course
for millennia; I saw our little group of \emph{samanas} at Chithurst and
at the Devon Vihara, as followers of the Buddha just joining in with the
flow. What the future might hold for us was not something that concerned
me.

As things turned out, a few years later, when seven Western abbots of
our various monasteries all disrobed within a period of five years, our
community did change size and shape, even if it didn't necessarily
change direction. We had issues that we needed to deal with, some of
which members of secular Buddhist groups perhaps didn't have to worry
about. They had certain advantages that we lacked. For instance, they
were sometimes more skilled in dealing with psychological and relational
matters. Often they were better informed when it came to discussing
issues around authority structures and projection. At that stage we
didn't even have a shared vocabulary with which to engage each other so
as to be able to discuss the tricky dynamics that inevitably occur
within communities.

Eventually, if I understand correctly, out of the people associated with
that group residing at Sharpham House, grew the development of a retreat
facility, on that same property, known these days as
\emph{\underline{\href{https://www.sharphamtrust.org/mindfulness-retreats/the-barn-retreat}{The
Barn}}} {[}61{]}. Some of the same group were also involved in the
purchase of an old Christian nunnery, the other side of Totnes, which
was to be developed into the large retreat facility called
\emph{\underline{\href{https://gaiahouse.co.uk/}{Gaia House}}} {[}62{]}.

There were other groups we visited in Devon. Perhaps the best
established was the one in Plymouth. One of the leading members of that
group, a Sri Lankan woman called Sushila Jayaweera, became a long term
friend of our sangha. Later, along with her husband and two children,
she relocated to Middlesbrough in North Yorkshire, and was a regular
visitor at Harnham Monastery. Sushila told me that she hadn't been
particularly involved with the Buddhist teachings until, in this
country, she came across translations of talks by Tan Ajahn Chah. These
teachings and her committed formal meditation practice sustained her
through a long period of cancer. On one of the final occasions I visited
her in hospital, I recall how she was more concerned with making sure
her husband had provided the monks with suitable refreshments. Of course
we were not interested in refreshments but that was characteristic of
the selflessness and strength of Sushila. Spending time with her during
her final days, gave me insight into how it is possible to die
beautifully. It was a great privilege to know her.

We received our midday meal at the Jayaweera family house in Plymouth on
the day that Tan Nyanaviro and I started our walk along the south Devon
coast. Also accompanying us was anagarika Jurgen. Some months later,
Jurgen would go on to take monks' Precepts, and was given the name
Khemasiri. He eventually spent several years as abbot of
\href{https://dhammapala.ch/}{\emph{\underline{Dhammapala Buddhistisches
Kloster}}} {[}63{]} in Switzerland. In those days my knees were still up
to what turned out to be a very demanding hike; even without backpacks
it would have been a workout.

Totally unexpectedly, one day back at the Devon Vihara, I found myself
pondering the fact that in no time at all everybody I knew would be
dead. Within perhaps a hundred years, nobody who I knew now would still
be here. That is amazing! At least it struck me so at the time. Indeed,
reflecting on it now can still trigger a sense of alertness. In a
hundred and twenty years, guaranteed, nobody alive on planet earth now
will still be here. Everyone, without exception, will be dead and gone.
It also struck me as very interesting that it felt so good to be
thinking about it. This wasn't anything to do with wishing myself or
anyone else to be dead. Upon investigation it occurred to me that the
good feeling arose out of ceasing from telling myself lies. Maintaining
our habits of denial consumes a huge amount of energy. Nothing could be
more certain than the fact that we are going to die one day. Every
person who has ever been born has died. So it will happen. Then why do
we deny it? And herein lies the reason for the Buddha's encouragement to
his disciples to
\href{https://www.accesstoinsight.org/lib/authors/gunaratna/wheel102.html}{\emph{\underline{regularly
reflect on death}}} {[}64{]}. Hence too, the recitation we perform as
part of our Morning Puja: \emph{I am of the nature to grow old; I am of
the nature to sicken; I am of the nature to die.} We are working on
dispelling the myths which we have been conditioned to believe. I was
aged about thirty-three at the time, and happy to find I could be a
little bit more honest about life, and death.

During a period when I was staying at Douglas and Margaret Jones' place,
possibly while leading a retreat in their converted milking shed, I came
across a book of transcribed and translated teachings by
\href{https://en.wikipedia.org/wiki/Ramana_Maharshi}{\emph{\underline{Sri
Ramana Maharshi}}} {[}65{]}. I am not sure whether I ever discussed that
book with my hosts but I do recall how glad I felt to find yet another
endorsement of the path of enquiring into `who'. The first time that
approach to practice had occurred to me was on retreat near Nimbin with
Ajahn Khantipalo. The second time was in Thailand, when someone related
to me an exchange between Ajahn Fun and his teacher Tan Ajahn Mun.
Apparently Ajahn Fun had been struggling with fear in his practice, and
approached his teacher for advice. Having listened to Ajahn Fun, Ajahn
Mun asked him, `Who is it that is afraid'? The third occasion of coming
across an affirmation of this avenue of enquiry, was in the translated
teaching by the Chinese Master Hsu Yun, in the books by Charles Luk that
Ajahn Sumedho had suggested I might read. Thank you, Sri Ramana
Maharshi, and Douglas and Margaret.

It was around the same time I came across a description of what had
happened in a monastery in Britain some hundreds of years earlier when a
Christian abbot required that the monks learn to chant in a new style.
Records show that this didn't go down very well -- to the extent that
some of the monks simply refused to follow the orders they had been
given. That had the regrettable consequences of archers being sent into
the abbey (I forget whether it was the King or the abbot who sent them
in) and one by one the monks were shot. The coincidence of reading this
description and my receiving news that at Amaravati and Chithurst a new
chanting style had been introduced, was fortunate. I was, and still am,
very fond of our chanting when it is done well. During my period as a
monk when I lived in Wat Boworn, I became used to participating in
beautiful chanting. This process of reinterpreting our chanting had
happened without my having been consulted, which, even without hearing
the new interpretation would have been enough to unsettle me.
Thankfully, that little lesson in British history helped prevent me from
making a problem out of what was really not a big deal. If I didn't like
the new style of chanting, that would only be a problem if I made it
one. That was helpful to reflect on.

After a little over two years at the Devon Vihara, my friend Tan
Kittisaro was sent down to replace Tan Nyanaviro. A few months later, in
1985, I returned to live at Chithurst. Shortly after that the trustees
sold the property at Raymond's Hill and the sangha moved to Hartridge,
near Honiton; the new place was given the name
\href{http://www.hartridgemonastery.org/}{\underline{Hartridge Buddhist
Monastery}} {[}66{]}.

