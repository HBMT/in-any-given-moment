\chapter{Preface}

This book has been compiled in large part because dwelling on thoughts
of gratitude brings happiness. Also, as I approach seventy years of age,
I find myself drawn to recollecting and reviewing earlier events in my
life and noticing how differently I now feel about them. As the writing
of these notes progressed, it became apparent that, in addition to
gratitude, I have been reflecting on two other themes: the dynamic of
spiritual community and ways of supporting our spiritual life.

The title, `\emph{In Any Given Moment}', means two things to me. One way
of reading it reminds me that in any moment there is the potential to
let go of our painful habits of clinging and consider the larger,
spacious context in which this drama of life is taking place. This is
how I understand, `Going for refuge to the Buddha': trusting that there
is selfless, just-knowing awareness.

In another way of reading it, the cover image of an open sky (thank you
Chinch) together with the title, suggests that whether or not we notice
the beauty of life in any given moment depends on how present we are for
it. When our faculties are obscured by self-centredness, we risk
becoming lost in memories of the past and fantasies of the future; as a
result our attention readily settles on what we perceive as lacking or
`wrong' with life, and we fail to notice the goodness and beauty right
here in front of us. If our vision begins to clear, if the dross of
unawareness is gradually removed and the gold of awareness revealed, a
thoroughly different perspective might emerge.

The timeline as it is presented here should not be taken too literally.
I have tried to be accurate; however, accuracy over dates and times was
not the main point of the compilation. I apologize if any inaccuracies
or inconsistencies cause confusion. The main point has been to reflect
on gratitude, community and sustaining spiritual practice. These three
themes are the foreground, with the incidents and events of my life as
the background; sometimes the background is not quite in focus.

The significant moments that I reference in these pages, both the
positive and the negative, are moments and events that stand out as
having been helpful in my effort to be freed from the addiction to
self-centredness. By no means have all the positive influences been
mentioned, and definitely I have not included many of the negatives.
Readers will find that the first six parts of the book read somewhat like a travelogue
interspersed with Dhamma reflections. Part seven is almost
entirely Dhamma reflections. It wasn't that I set out to write a book in
this style, it is just that this is how it unfolded. My hope is that
anyone who reads it will discover something beneficial for themselves
and perhaps find something that they want to share.

\bigskip

{\raggedleft
  Ajahn Munindo
\par}

