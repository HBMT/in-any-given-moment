\chapter{Shared Aspirations}

\enlargethispage*{\baselineskip}

As I begin writing the seventh and final part of this book, I find
myself pondering on how, by March 2021, I will have been abbot of this
monastic community for three decades. A reader who has been helpfully
checking what I have written recently recognized this fact and mentioned
it by way of appreciation. I am grateful for his thoughtfulness. Earlier
on in this compilation I described some of the challenges we have had to
deal with here on Harnham Hill, and maybe hinted at the many more that
have come our way. Attempting to establish and maintain a spiritual
community in twenty-first century Britain has not been easy work. There
have been many times when I have doubted whether I had what it takes to
keep going. However, at this point, what stands out for me is how
fortunate I am to have been put in this position. Thank you, Ajahn
Sumedho, for sending me here in the first place; thank you to all the
sangha members who put up with me as I have attempted to give shape to
this community; and thank you to the trustees and supporters who
provided the food, the fuel, and all the material means for maintaining
this venture.

In many unexpected ways, being put in this position has helped me learn
how to surrender more fully into serving the Buddha, the Dhamma and the
Sangha; it has supported the cultivation of wisdom and compassion. In my
early twenties, when I was wondering if I would ever manage to find a
way to live in this world that didn't feel foreign, I would sometimes
think in terms of a biological model I had been taught when I was still
at college; the concept of an ecological niche. I seem to recall the
teacher giving us an example that was something to do with
nitrogen-fixing bacteria living on the roots of clover plants. The
reason this biological model struck me as relevant, however, was because
it demonstrated how different organisms can find mutually beneficial
ways of coexisting, in a state of balanced reciprocity. It occurred to
me that finding such a `niche' for myself in life would mean finding an
environment that obliged me to develop those qualities that I truly
needed to develop, and, in the process, I would make a useful
contribution to the world in which I was living.

\section{Giving to and Receiving from Community}

It turns out that being a Theravada Buddhist monk, and, in particular, a
leader of a monastic community, has been my niche. The precisely stated
precepts of the monastic order require that, as monks and nuns, we
continually work on developing mindfulness, restraint and reflection. At
the same time, in the process of making an effort to cultivate those
qualities, we are contributing, in a modest but \mbox{nevertheless} significant
way, to the sanity of this world. Not everyone is going to want to live
in a spiritual community such as this; however, given all the support
and encouragement we continue to receive, it would appear that the
example of a renunciate community in the midst of a secular, materialist
society is still appreciated.

Dhammapada verse 122 says,

\begin{quote}
  Do not ignore the effect of right action, saying,\\
  ``This will come to nothing.''\\
  Just as by the gradual fall of raindrops\\
  the water jar is filled,\\
  so in time the wise become replete with goodness.
\end{quote}

The deep benefits that can be derived from living in spiritual community
are not always immediately obvious. It can take a long time for our
habits to change. As in the example mentioned earlier of a small,
apparently insignificant acorn turning into a glorious oak tree, we
don't necessarily see the change taking place in real time. Some of the
benefits might be immediately apparent, but others could take decades,
or even lifetimes, to manifest.

In the scriptures there is an incident recorded of the Buddha staying
with a small group of his disciples, and his commenting on how
harmoniously they were living together. Perhaps that state of harmony
was particularly pronounced for him since a short while earlier he had
walked out on a group of monks who refused to stop bickering with each
other. Indeed, they had told him to leave. The Buddha enquired of these
monks who were living harmoniously how they managed it. They replied
that they all made an effort to watch out for each other. They explained
how they regularly recollected the benefits which accrue from keeping in
mind the well-being of those with whom they are living.

As reported earlier, when Bani Shorter, a Jungian psychoanalyst friend of our monastery, once
spoke to me about her appreciation of spiritual community, she referred
to it as `a harmonious resonance of shared aspiration.' At the time we
didn't discuss her comment in detail so I can't say for sure that I
fully understood what she meant by that expression, but I can
confidently say that over the years I have felt glad for the many
benefits derived from living in cooperative spiritual community. There
have been times when I have felt as if harmonious spiritual community
generates a sort of rarefied atmosphere that nourishes beings who live
within it. Almost as if being part of such a community contributes to an
elevated quality of awareness. I don't want to sound too flaky here, but
I do think the point is worth considering. If we appreciate how this
particular quality of goodness might emerge from beings who share an
aspiration for awakening and are living together in harmonious
communion, then we will want to contribute to and protect that
communion.

So long as our insight into the Buddha's teachings is not so well
established that we can totally depend on ourselves, our spiritual
companions help protect us from falling into delusion. In Dhammapada
verse 354, the Buddha points out, \emph{The gift of Dhamma excels all
gifts.} Being blessed with the company of true spiritual friends, and
having access to a well-practising sangha are part of this gift of
Dhamma.

True spiritual companions do not necessarily conduct themselves in the
way our worldly friends might; they don't encourage us to indulge in
heedless habits. For instance, if we are caught up in a flow of speaking
in an unkindly manner about someone, a Dhamma friend might just go quiet
for a while and leave us feeling awkward. In that quiet space which we
have been offered, hopefully we will catch a glimpse of the consequences
of compromising our commitment to training in right speech. The
contracted, deluded character that we often experience ourselves to be
was conditioned by the relationships that surrounded us in our early
life. To untangle ourselves from misidentification with this conditioned
character, we benefit greatly from the company of those who share an
aspiration for awakening. Hence the Buddha's emphasis on cultivating
spiritual friendship\cite{friendship}. If we are concerned with finding structures that
will sustain us on our journey, building and protecting spiritual
community ought to be high up on our list.

Such work will, inevitably at times, be difficult, but it is worthwhile.
It requires that we be willing to go against the current of our
conditioning. Although it can be hard work, when we truly surrender
ourselves into it, we grow in self-respect. It is easy to follow
habitual ways of acting and speaking, the same as it is easy to scratch
a wound that feels itchy when it is healing. Without skilful restraint
we won't build up the potential required for a new way of seeing to
emerge.

\begin{quote}
  Whoever is intent on goodness\\
  should know this:\\
  a lack of self-restraint is disastrous.\\
  Do not allow greed and misconduct\\
  to prolong your misery.

  \quoteRef{Dhammapada v.248}
\end{quote}

It is not always because of a lack of good intentions that spiritual
communities fall into disharmony; often it is a lack of skilful
restraint. And, conversely, it can be the presence of skilful restraint
that means communities succeed.

It is tempting to think that we will live happily and harmoniously
together if we all like each other. That is never going to happen, at
least not for very long. Sooner or later we will find ourselves in the
company of someone we find irritating. With skilful restraint, we will
be in a position to reflect on how liking and disliking are conditioned,
unreliable and unstable. It is embodied mindfulness, skilful restraint
and wise reflection that build harmonious community -- not being
surrounded by people that we like.

In the early days of living in monasteries in Thailand I observed how
Tan Ajahn Chah built the monks' meditation huts (kutis) with a
significant distance between them, emphasising isolation. For young Thai
monks I imagine this was quite difficult to tolerate, since they are a
very social people; they find it hard to be alone. The Westerners who
came to join the monastery, on the other hand, often loved not having to
see anyone or talk to anyone; we tend to be rather more individualistic
and can find it annoying when we have to put up with others. It occurs
to me that, when considering whether we might benefit from living in
community or living in solitude, we need to know our motivation. Are we
avoiding relationships out of a balanced appreciation for the benefits
of quietude, or are we strategically avoiding the pain we feel in the
company of others? And from the opposite perspective, are we associating
with others because we genuinely find the company of spiritual
companions nourishing, or are we using them to distract ourselves from
the pain we feel when we are alone? Are we capable of both living with
others and living alone? I am reminded of what Tan Ajahn Chah would say
if one of his monks was complaining about not being able to practise
because he had been sick. `If you can't practise when you are sick, then
you can't practise when you are healthy. If you can't practise when you
are healthy, then you can't practise when you are sick.' Likewise if a
monk had to spend time away from the monastery in the city, and
complained that he couldn't practise while he was there: `If you can't
practice in the city, then you can't practise in the forest. If you
can't practise in the forest, then you can't practise in the city.' In
other words, we should not cling to ideas about suitable conditions for
practice, but determine to work with whatever is happening. Obviously
this is an ideal that he was presenting and not something to be made
into a fixed position. Such ideals are for helping us orient our lives.

Sometimes when Tan Ajahn Chah spoke about living in community, he
compared the frustrations that are inevitably experienced to the
friction that stones in a riverbed undergo as they are tumbled; in the
process their sharp edges are gradually ground down; those stones become
smooth and agreeable to hold. Likewise, those who choose to work
creatively and constructively with the frustrations of community living
can learn how to fit in with, and are welcomed by, any community they
visit.

Acknowledging this point about developing a willingness to keep
practising, whatever our circumstance might be, does not mean that we
should expect everyone to want to live together. That, again, would be
idealistic. There is such a thing as natural affinity. We all approach
spiritual community with our own set of abilities, assumptions and
limitations. It took me a number of years before I realized that many
people join a monastery because they want to avoid having to deal with
the difficulties that arise out of human companionship. While I was
enthusiastically interested in finding ways of living cooperatively with
my fellow \emph{samanas}, many of them were equally enthusiastically
seeking solitude.

\section{Practising Together and Alone}

Accordingly, some monasteries emphasise solitary practice; other
monasteries encourage group practice. From what I have observed,
generally speaking, during the seven years that constitute the first
three stages of training -- \emph{anagarika, samamera, navaka bhikkhu} --
it is to everyone's advantage if trainees spend most of their time in a
monastery that emphasises group practice. This observation is based
partly on my having noticed how, sadly, a number of monks become quite
senior in the training before they recognise that they have some untidy
loose ends that require attention. By that I mean they might, for
instance, have unacknowledged authority issues or biases (more on that
later). They could even be in a position of leadership and be exercising
authority over a community before they get around to owning up to such
unaddressed aspects of their character.

Living with others is bound to put us under pressure, and it is when we
are under pressure that weaknesses in the system show up -- as when a
doctor who wants to check the strength of our heart prescribes a stress
test. Community practice stress-tests us. Once we know our strengths and
weaknesses then we are more likely to make skilful choices as to whether
we go on living in community or spend more time in solitude.

Also, generally speaking, there are two ways of using solitude: one, as
a means of accessing ease and well-being so as to deepen our practice;
two, as an \emph{upaya} for turning up the pressure so we get to see
what unreceived aspects of life have been stored away in unawareness.
Or, as the Buddha put it in Dhammapada verse 239,

\clearpage

\begin{quote}
  Gradually, gradually,\\
  a moment at a time,\\
  the wise remove their own impurities\\
  as a goldsmith removes the dross.
\end{quote}

Being surrounded by others, friendly or otherwise, can serve to support
us or undermine us, depending on where we are at in practice.

Mentioning as I have that generally speaking, in my view, junior monks
and nuns should spend the first three stages of their monastic training
living in community, is very much in keeping with how I understand Tan
Ajahn Chah's teachings. He himself benefitted from spending time alone;
however, as far as I could discern, his emphasis was on each individual
finding out what works: what takes us to the point where genuine letting
go of fixed positions takes place. People are different, but
particularly in the early years of training, it is wise to follow what
our teachers recommend. Of course, this also accords with the spirit of
what the Buddha laid down when he required that monks, during their
first five years, spend the Rains Retreat period living with an elder
monk.

The tricks that our minds play on us can be very deceiving. When we live
in community we are more likely to receive helpful feedback from others
and are hopefully less vulnerable to believing in any deluded notions.
When we live alone, we might be entertaining ideas about how well
established we are in practice -- even be delivering profound Dhamma
talks to ourselves -- but that doesn't mean that when we are under
pressure we won't crack. We might enjoy living alone for a number of
years, but then if we were to suffer an accident, who do we assume would
take care of us? Or if we needed our teeth done, who would be paying for
the work? If we are assuming there is someone who will take care of
those things when they are needed, then we are already indebted to them,
and part of right practice means honouring those on whom we depend.
Living alone can be very appealing so long as we are reasonably young
and healthy; it might not be the case once we are older. Then again,
perhaps it will work out OK. What matters is that we are motivated by
honesty and gratitude. We all depend on others who have protected these
teachings and passed them on to us; our job is to show appreciation by
sincerely surrendering ourselves into the practice.

\section{Attention to Structures}

One of the first decisions that Tan Vipassi and I made when we initially
arrived here at Harnham was to reconfigure the breakfast routine. As at
Chithurst, things here were set up so that everyone -- sangha and lay
guests all together, with monks and novices wearing our formal robes --
sat in lines, with the abbot on a raised platform at the head. To my
mind it generated a rigid atmosphere and didn't at all conduce to ease
of communication. The midday meal was arranged in a similar manner and I
was confident, for a number of reasons, that that was how it should be.
But I wasn't confident that it was helpful to have this earlier meeting
set up that way. I had observed that there were no other opportunities
during the day when just the monastic residents were all together
without guests present, which struck me as a pity.

When at first the change was made and the sangha started meeting
informally, with the laity taking breakfast in their accommodation next
door, I heard that it hadn't gone down so well with all of our guests. I
understand that visitors find it supportive to spend time with the
sangha, but if the monastic community is not being properly nourished --
and by that I mean on an interpersonal level,
not just with power-porridge\cite{porridge}
-- then it could lead to disharmony. Within a very short
period of time, I was informed that the guests were now enjoying a more
relaxed atmosphere at breakfast and appreciated not having the sangha
there all looking so serious. Ever since then, breakfast time has become
our regular house meeting. With a structure that is less formal and an
atmosphere more relaxed, it is easier to check in with each other. That
is not to say breakfast is a consistently jolly occasion, but it is
better than it was.

\enlargethispage{\baselineskip}

When monks and nuns fail to develop a sense of belonging within their
community, they are more likely to become disaffected. This is partly
why I have long considered it a priority to pay attention to such
dynamics as how we meet with each other. The excessive fascination that
some \emph{samanas} have for using social media is, I suspect, an
expression of their search for relational sustenance.
That is just an observation on my part, since as far as I
know nobody has done a survey on the subject. Thankfully, generally
speaking, there is still a good level of restraint exercised within Tan
Ajahn Chah's branch monasteries in the west and we regularly review the
influence technology is having on us.

Most people will be familiar with the conventional saying, `the survival
of the fittest'. The assumption that `the fittest' means the toughest,
is unlikely to stand up to scrutiny. Being tough is no guarantee that
you will survive; sooner or later you will need to depend on the support
of others. Maybe it is better to understand `the fittest' as meaning
those who have learnt how to cooperate.

\enlargethispage{\baselineskip}

\sectionBreak

Earlier I attempted to explain how fundamentally important I have found
it to be that all of us, those living the monastic life and those living
the life of a householder, find ways of meeting our conventional need
for community -- the need to feel like we belong. Monks and nuns who
`have gone forth from home to homelessness' are not magically released
from such relative psychological needs. Maybe they will eventually
arrive at a state of awakening whereby they are freed from the painful
consequences of identifying with their personality, but if they attempt
to override that condition by clinging to ideals, they might find they
slow down their progress towards liberation. In the following pages, I
would like to offer further reflections on some of the skilful ways in
which we can support ourselves and each other. Also, I hope to be able
to usefully identify some of the obstructions and distractions we might
meet along the way.
