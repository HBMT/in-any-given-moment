\chapter{Sitting in the Buddha's Waiting Room}

There is a story traditionally associated with \emph{Dhammapada verse
331}\cite{dhp-331} above, in which \emph{Mara} tries to
entice the Buddha to take on the powers of a king and become a worldly
ruler. There was no way that was going to happen; the Buddha was not
interested in having power over others. He was interested in showing
people how they could have dominion over their own lives. All of his
teachings point to the true source of inner strength and security:
selfless just-knowing awareness. Most of our practice is about learning
how to make that just-right kind of effort so we are gradually more
aligned with this inner refuge.

We all like to feel reassured that we are making progress in practice
and that our efforts are not wasted. However, there comes a time when we
need to let go of requiring such reassurance and choose instead to
trust. We should not see this as abdication or resignation. When it is
trusting in a wise way, it is based on an understanding that
persistently checking up on ourselves undermines us. Tan Ajahn Chah
illustrated this habit of always seeking proof of progress: he said it
was like planting a sapling and every few days pulling it out of the
ground to see whether it was still growing. Obviously, to do so is
unhelpful. As with equanimity and patience, trust is not assertive, and
from the outside it can appear weak. In truth, trust is potentially a
very powerful source of support.

% \clearpage

A visitor to our monastery who was clearly committed to the spiritual
life once asked me if I would explain the correct approach to practice.
The image that came to my mind and which I shared with her was one of
our sitting in the Buddha's waiting room. I suggested that our trust in
the possibility of awakening is like having an appointment to meet the
Buddha. In such a situation, while we were waiting to be called, what
would we be doing? Would we be complaining to the Buddha's secretary,
Venerable Ananda, about how long we were having to wait? I don't think
so. Would we be pacing up and down fuming with impatience? I doubt it.
Would we indulge in getting upset when we saw someone else go in ahead
of us? Unlikely. I didn't elaborate to that extent in the conversation
with the visitor, but I had the impression that the image helped her
clarify how she should be approaching her practice. Her approach and my
approach would not be the same; we are not coming from exactly the same
place. What we do share, though, is faith that there is an end to
suffering and that the Buddha realized it.

When I think about how I personally would conduct myself were I waiting
for an appointment to see the Buddha, I am sure I would be careful and
try to not indulge in heedless mind states. Probably the thing I would
focus on would be the five spiritual faculties. During his lifetime the
Buddha gave many different teachings to different types of people. Some
who came to listen to his teachings were suffering a great deal, and
others not so much; some needed a lot of explanation before they got the
message, and others needed only a few words. It is not necessary that we
study and understand everything that has ever been recorded about what
the Buddha taught. What is necessary is that we reach a point where we
can let go of doubting the validity of the path and surrender ourselves
-- whole-bodily, wholeheartedly -- into the spiritual training. For me,
surrendering into the training means working on the five spiritual
faculties: \emph{saddha, viriya, sati, samadhi, pañña} -- faith, energy,
mindfulness, collectedness and wisdom -- coming back, over and over
again, and refining my relationship with them.

(Some readers might find it unhelpful that I so frequently use the Pali
words instead of the English equivalents. I take this decision partly
because I suspect there is a better chance the Pali words will come less
encumbered with assumption. That we don't quite know what the Pali words
mean could help us be more open to enquiring.)


One way of approaching these five faculties is to think of them as a
pyramid: the four sides of the pyramid have \emph{saddha} and \emph{pañña} facing
each other, and \emph{viriya} and \emph{samadhi} facing each other. Then, from the
apex of the pyramid to the base, right down through the centre, there is
the axis of \emph{sati}, maintaining balance.

\section{Saddha (faith)}

\begin{quote}
  The fragrance of flowers or sandalwood\\
  blows only with the prevailing wind,\\
  but the fragrance of virtue\\
  pervades all directions.
  
  \quoteRef {Dhammapada 54}
\end{quote}

This Dhammapada verse fifty-four is about \emph{sila}, or integrity; I also
think of \emph{saddha}, or faith, as being like a beautiful fragrance. To use
the example of honeysuckle: it is possible to feel the texture by
touching it, and we can capture the shape and colour with a camera, but
how do we sense the fragrance? The scent of the honeysuckle has a
somewhat amorphous quality to it which we can't quite capture. In the
same way, \emph{saddha} or faith (or trust) cannot really be grasped;
however, it can affect us if we allow it.

As we contemplate the faculty of faith, it is helpful to appreciate this
amorphous quality. Sometimes we make the mistake of only valuing those
things that we think will make us feel sure. It is not in the nature of
faith for it to make us feel sure; however, and most importantly, it can
help us to be more at ease with feeling unsure. We are not sure that
every chair we sit on is secure and won't collapse under us; we trust.
We are not sure that our car will start next time we need it; we trust.
In our heads we have an assumption about what will or will not happen,
but that assumption is based on how we feel -- in this case, a sense of
trust. This ability to trust is a faculty which can be, and needs to be,
cultivated. It has the potential to sustain us when we are confronted
with intense uncertainty. Having embarked on this journey to awakening,
it is guaranteed we will travel through territory that is uncharted,
uncertain, unknown. A well-developed sense of \emph{saddha} means we can feel
uncertain, we can enter enter into the unknown better prepared and less likely to sink into the swamp of fear and dread.

\emph{Saddha} in the context of the Buddha's teachings is not the same as
belief. We believe in ideas in our heads. \emph{Saddha} or faith involves our
whole body-mind. To use another metaphor: when we are swimming in the
ocean and roll over onto our back and float, it is not the fact we
believe that floating is possible that prevents us from sinking; it is a
whole body-mind effort. It is an embodied sense of trust. We are not
sure, but we allow ourselves to trust. It is helpful to find our own
word for the feeling we have when we engage with this faculty of
\emph{saddha}. For me the key word is `surrender'.

Because the power of \emph{saddha} is sometimes not as obvious as such
qualities as concentration and insight, we risk undervaluing it.
\emph{Saddha} is like a reliable secretary who works behind the scenes to
make sure the visible aspects of our life function efficiently.
Sometimes it inspires us, at other times it sustains us. \emph{Saddha} can be
good at protecting us from taking ourselves too seriously. When we
recite the phrase, \emph{Buddhaṃ} \emph{saranaṃ gacchami} -- I go for refuge to the
Buddha -- we are saying that I trust that awakening from unawareness is
possible and that the Buddha was fully awake. When we recite the phrase,
\emph{Dhammaṃ saranaṃ gacchami} -- I go for refuge to the Dhamma -- we are
saying that I trust that the teachings of the Buddha lead to awakening.
When we recite the phrase, \emph{Sanghaṃ saranaṃ gacchami} -- I go for refuge
to the Sangha -- we are saying that I trust that there is a community of
awakened beings who have followed those teachings. \emph{Saddha} encourages
us to let go of `my way' and go for refuge to the Buddha's way, the
Triple Gem: to align ourselves with timeless truth and in so doing learn
how to hold our views and opinions more lightly.

Besides having trust in the Triple Gem, we also need to have trust in
ourselves, which requires learning to be increasingly honest with
ourselves. The more honest we are with ourselves, the more we trust
ourselves. The more we trust ourselves, the more inner strength we have.
The Triple Gem serves as an outer representation of wisdom and
compassion. So long as we are still under the sway of unawareness, we
need external supports that protect us from falling into vortices of
delusion. Trust in the Triple Gem provides us with a frame of reference
that supports trusting in ourselves. Without such a frame of reference
there is the risk that we could misperceive self-confidence and cling to
it; instead of being freed from the suffering of deluded egoity, we
become more identified as it. \emph{Saddha} is precious and as practice
progresses we come to see the wisdom of protecting it, of guarding it,
of treasuring it. Here is a poem about \emph{saddha} by Kittisaro that he
shared with me. He wrote it while he was on a year's retreat in
Chithurst forest in 1989.

\begin{quote}
  Faith\\~\\
  
  Trust is precious\\
  A treasure trove of gold.\\
  Guard it with all your heart\\
  And you'll never grow old.
  
  It's not a question of this or that\\
  Believed or disbelieved,\\
  But rather letting where you're at\\
  Be silently received.
  
  The heart of faith,\\
  The heart that knows,\\
  Leaves no trace,\\
  And neither comes nor goes.
\end{quote}

\section{Pañña (wisdom)}

Elsewhere I have mentioned the conversation that the Buddha had with his
son, Rahula, in which he asked Rahula what the purpose of a mirror was.
Rahula replied that a mirror was for seeing our face in. The Buddha went
on to say that when we want to see our heart, we use wise reflection.
Towards the very beginning of this book, where I was explaining the way Tan Ajahn Thate taught about wisdom, I described \emph{pañña} as being a self-reflective capacity which has the
function of revealing the reality of that which appears within
awareness. It is \emph{saddha} that gets us started on this journey to
awakening, and it is wisdom that shows us where we need to go.

If we have too much \emph{saddha} we can become lost in naivety. The
confidence that comes from \emph{saddha} has the potential to make us
complacent. That which counters naivety and complacency is \emph{pañña}, or
wisdom. Wisdom functions in relationship to faith; they cooperate.
\emph{Pañña} likes to enquire: it asks questions; it is not satisfied with a
surface level of understanding: it wants to look deeper. \emph{Pañña} is a
disruptor -- a constructive disruptor. It dismantles and dissects, but
not out of heedless reactivity. True wisdom accords with reality and
deconstructs in pursuit of the understanding that liberates us from
conceit and confusion. My personal key word that connects with the
concept of \emph{pañña} is `discernment'.

We begin by using the wisdom of those who have taken the journey ahead
of us. By paying close attention to their teachings, we make progress
and avoid too many serious pitfalls. We don't yet know what they know
and we don't see what they see, but we benefit from what they share.
Similarly to how we would use a satellite navigator in the car: to begin
with we are not sure that it is reliable, but as we travel along we see
that indeed there is a bridge where there should be, and we pass through
a village at just the right time. In the process we learn to trust in
the instructions that the device gives us. Those instructions are not
the journey, and are definitely not the destination, but they can be
very helpful. Likewise we can learn to trust in the wise instructions
that our spiritual teachers give us. What they offer are approximations
-- they are not the journey itself and not the goal -- but their guidance
is beneficial.

For example, wise teachings point out the importance of heedfulness and
integrity. On the surface level we might think that so long as nobody
else knows that we are being dishonest then we will get away with it.
What we don't see is the damage we are doing to our self-respect.

In recent years several of the building projects here at Harnham
Monastery have been overseen by an architect friend who lives in London,
Rion Willard. When I was first getting to know Rion he shared with me
how he reached a point in his life where drinking alcohol no longer made
sense. He had participated several times in meditation retreats at our
sister monastery, Amaravati, and had probably heard on a number of
occasions the `precepts talk' given just prior to retreatants leaving
the monastery. The talk on the five precepts is aimed at helping those
who have been in the supportive environment of a retreat to integrate
whatever benefits they have accrued, as they merge back into situations
that are less supportive. Without the protection of a commitment to
restraint and harmlessness we are exposed to the influence of old habits
of resisting reality; in the process, dark shadows of unawareness return
and obscure the clarity that perhaps we enjoyed while on retreat.

Rion explained to me how one year he had spent time during the retreat
pondering: `Why do I continue to drink alcohol when it costs so much
money, causes me to behave heedlessly, and I end up feeling horrible?'
That year, after the precepts talk was given, he joined in with the
group recitation and made a firm resolve to give up all alcohol. He
hasn't drunk since. At that time he was working in an architectural firm
and, as would be expected, was often invited to attend social gatherings
where alcohol was consumed. Having made that resolution at the end of
that retreat, he drank juice, water, coffee, or soft drinks. Not
only did his mental clarity benefit but often others would notice and
were inspired to ask how he managed it. Some of them had made big
mistakes in their lives while under the influence of alcohol. Many
rewarding conversations occurred as a result. The confidence and energy
that was released as a result of making that resolve contributed
significantly to his establishing his own architect company. These days
Rion runs a consultancy business, and as part of the contract that
clients sign, there can be no consumption of alcohol for twenty-four
hours prior to their meeting. We could say that it was wisdom that recognized the relevance of the precept talk and compassion that meant the wisdom was shared. Dhammapada verse 290 says,

\begin{quote}
  It is wisdom that leads to letting go\\
  of a lesser happiness in pursuit\\
  of a happiness which is greater.
\end{quote}

It is easy to be inspired by the compassionate example of those who have already taken the journey -- who have done their work -- but we would be mistaken if we think we can have the wisdom without doing the work.

In 1967 I was fifteen years old, and I can recall sitting in the living
room in our house in Morrinsville, watching on our black and white
television set a global TV link-up\cite{satellite} -- the
first occasion when many countries around the world all simultaneously
tuned in to the same program. One of Britain's contributions to that
event was the Beatles singing \emph{All You Need is Love}. They were backed
by a large orchestra and joyously sang out a refrain about love -- over
and over again. It was a catchy tune and sounded very hopeful.
Unfortunately, that global link-up and the Beatles song were not enough
to transform the insanity and suffering of the world. In reality, when
what is referred to as love is not associated with wisdom, it can be
very selfish. A song called \emph{All You Need is Wisdom} is not so catchy,
but the message is closer to the truth.

Wisdom understands, for instance, that reality is multidimensional. The
waves on the surface of the ocean are only part of the reality of the
ocean: there is stillness in the depth. For example, we might think we
are acting with good intentions as we try to persuade others that we
know what is best for them, but perhaps we don't have the depth of
discernment to realize that it is delusion that is driving us. If we
look at the results of many of the religious crusades over the centuries
we see how, despite what they thought were good intentions, they left a
trail of disaster behind them with far-reaching consequences. Without
true wisdom, delusion can be running riot and our actions can be causing
a great deal of suffering, for ourselves and others. It takes wisdom to
see through self-delusion, and without it the spectre of
self-centredness creates a massive amount of trouble.

\emph{Pañña} doesn't shy away from hard questions. We need to learn how to
turn our attention around and truly face \emph{dukkha}, and ask, `Where does
this suffering come from?' This is what the Buddha in his wisdom is
encouraging us to do. Because of habits of heedlessness we find it
easier to turn away and distract ourselves -- to blame external
conditions. Even the development of goodness can become a form of
distraction. I attended a public talk once in London where the teacher
was comparing different Buddhist traditions, and commented that, in his
view, the Theravadins focused too much on suffering. I wouldn't want to
comment on all Theravadins, but I would say that it is also possible to
focus too much on the aspects of practice that give rise to
surface-level good feelings. (See the example of Tan Ajahn Thate who was
locked into \emph{samadhi} for several years without progressing towards
wisdom.) Indeed, we need to be able to draw on the strength that comes
from our storehouse of goodness, but we also need to be careful that we
are not merely indulging in pleasant feelings. It is possible to be
dwelling on thoughts of kindness and gratitude yet at the same time be
completely lost in pleasant feelings. Pleasant feelings that arise from
focusing on goodness can be intoxicating. Once again, it is wise to
reflect that the Buddha pointed out: \emph{You continue to suffer because you
fail to see two things -- dukkha and the cause of
dukkha}. When wisdom is well-developed it is less likely that we will
make the mistake of indulging in agreeable feelings. There is a better
chance we will engage the strength and resilience that goodness gives
us, and use it to fearlessly face \emph{dukkha}, to drill down into it and
enquire: `What is this suffering? What is the cause of this
suffering?'

\emph{Pañña} sees through facades -- our own and others. We might catch
ourselves midway through telling a familiar story about how great we
are, then suddenly see our own falsehood. Or when listening to another
person talking about the drama of their life -- they could be utterly
convinced about how unfortunate they are and justified in blaming
so-and-so for their unhappiness, but all you hear is somebody totally
lost in a dream. With wisdom you won't be pulled into heedlessly
believing in, or reacting to, their drama. Without wise reflection we
tend to become lost in habits of reactivity, taking sides for and
against opposing views and perspectives. Wisdom shows us how to pull
back from heedless reactivity and see the situation from a broader
perspective, one of expanded awareness: one that has the space to
accommodate the \emph{dukkha} -- our own and that of others. Without such a
perspective it is not likely that we will be able to really change
anything.

So long as we totally believe we are our conditioned personality, our
ego, there is very little hope of our finding any happiness other than
that which arises from mere gratification of desire. One who is
searching for satisfaction and security but still believes their
personality is who or what they are, is like someone who is hungry and
eats some photographs of food. Those colourful photographs approximate
food but they are not the real thing. \emph{Pañña} sees the games we play and
sees through the fronts that we erect and hide behind: the powerful one,
the entertaining one, the sensitive one, the spiritual one. Wisdom has
the potential to lead us in the right direction of dismantling those
fronts and learning how to stand firm on our own two feet, without
hiding.

We are fortunate to have the benefit of the Buddha's all-encompassing
wisdom, as there are many opportunities on this journey for us to become
distracted. Even avenues of apparently profound significance can be a
complete waste of time. The classic teaching we have from the Buddha
illustrating this point is where he was with a group of monks and
scooped up a handful of leaves from the floor of the forest and asked
them which was greater: the handful of leaves or all the leaves on all
the trees in the forest? The monks replied that the leaves on all the
trees were greater. The Buddha then explained that the truths that he
had realized were much greater than those which he had taught. However,
what he had taught was what mattered to anyone who was interested in
awakening to freedom from unawareness.

The traditional presentation of the teachings that lead to the arising
of wisdom involves an analysis of phenomena according to the three
characteristics: \emph{anicca, dukkha, anatta} (impermanence, suffering, and
not-self). We are taught that investigating experiences -- mental,
emotional, physical -- in terms of these three characteristics gradually
leads to our letting go of habits of clinging. The tradition also
suggests that we might well find an affinity with one characteristic in
particular, in which case we ought to follow that line of enquiry.
Essential to that enquiry, however, is that we engage it with an
authentic quality of interest; we are not blindly applying a technique
because someone said it was good for us. The Buddha's own motivation to
turn away from a life of habitual distraction and to pursue liberation
began when he truly saw the consequences of his behaviour -- when he
truly saw that this life is fraught with \emph{dukkha}. It was at that point
that disillusionment arose in him. This recognition triggered in him a
deep interest in searching for an escape from the terrible tedium of
always trying to avoid old age, sickness and death. The great question,
`What truly matters?' arose in his heart and with it the energy to
embark on the great journey.

\section{Viriya (energy)}

\begin{quote}
  Those who are energetically committed to the Way,\\
  who are pure and considerate in effort,\\
  composed and virtuous in conduct,\\
  steadily increase in radiance.

  \quoteRef{Dhammapada 24}
\end{quote}

Let us now turn to the third face of the pyramid, the faculty of
\emph{viriya}. The usual translation of `\emph{viriya}' is energy, or sometimes
vigour, or effort. The word I find helps form a connection with this
Dhamma principle is `motivation'.

\emph{Viriya} gets things done. It can help get you out of bed in the morning
and motivates you to do your morning exercise before settling into
sitting meditation. \emph{Viriya} is needed to endure through difficulties.
Even after experiencing significant insights, it can take a lot of
effort before feeling able to fully live from that place of new
understanding.

\emph{Viriya} is needed to overcome inertia and to take initiative; it means
we don't settle for the status quo. Without \emph{viriya} the Buddha-to-be
wouldn't have embarked on the journey to awakening. Without \emph{viriya} he
might have accepted one of the the invitations from his first two teachers to settle
in and help them run their communities, and might even have abandoned
his aspiration to realize full and final freedom from all suffering.

Without \emph{viriya}, Ajahn Sumedho might have remained at Wat Pah Nanachat
and not spent years initiating and supporting the development of many
monasteries in the West. Without \emph{viriya}, the abbot I mentioned earlier who found himself on the receiving end of somebody else's projected pain wouldn't have taken the time he needed to thoroughly attend to how he had been affected.

In the early days of Chithurst there was an occasion when I was tasked
with moving Ajahn Sumedho's belongings from a small room in the main
house over to the redecorated Granary. There wasn't very much to move,
and I paid what I thought was enough attention to boxing things up. I
like to think I was particularly careful in packing up his shrine. As it
happened I should have been more careful, because somewhere between the
main house and the Granary the head of Ajahn Sumedho's carved rose
quartz Buddha rupa was broken off. What I particularly remember about
that day was Ajahn Sumedho offering a very helpful Dhamma talk in which
he described how, instead of on the one hand pretending that he wasn't
annoyed, and on the other hand indulging in the annoyance, he made an
effort to simply bear with the suffering that arose upon learning his
lovely Buddha rupa had been broken, until the suffering faded. It is not
necessarily the case that practice will take us to a point of profound
insight and suddenly all our suffering disappears. That might happen for
some, but what is more likely is that the arising of insight is a
beginning of a new way of relating to suffering. Insight gives us a new
perspective. From that point onward \emph{viriya} is required as we endure
the burning that is the purification and the integration.

While we are contemplating the spiritual faculty of \emph{viriya} we should
look again at the Buddha's teachings on the Four Right Efforts. It is
easy to memorise and rattle off the list of the four right efforts, but
what does it \emph{actually} mean to make these four kinds of effort?

Let's begin with `making an effort to protect already arisen
wholesomeness'. As an experiment, bring to mind some positive aspect of
your character, and then ask yourself, `what do I need to do to protect
this good quality?' For example, perhaps some time ago you decided to
take the precepts seriously -- not merely repeating them in Pali without
any intention of observing them -- and you have managed to honour that
resolution. However, now the festive New Year season is approaching and
you feel afraid you could end up compromising yourself. One way of
making an effort to protect the already arisen wholesomeness is to call
on a Dhamma friend to bear witness to your resolve to maintain the five
precepts. It doesn't matter whether we really understand why it makes a
difference having someone else that we respect know about the effort we
are making; we can just try trusting in it and see if it helps. Or,
another example, perhaps we have reached a point in our meditation
practice where we find the benefit from regular sittings is spilling
over into daily life and we are experiencing increased clarity and calm.
One way of making an effort to maintain that benefit of practice is to
determine to keep to a regular sleeping routine: setting an alarm for
ten o'clock at night and being in bed by ten-thirty, for instance. We
know that irregular sleeping patterns are unhelpful, and that staying up
late dealing with emails is disruptive; to make a resolve to be in bed
by a certain time can be supportive.

Considering now the second right effort: what is involved in `making an
effort to give rise to so far unarisen wholesome states of mind'?
Perhaps you are someone who finds it easy to be generous, but finds it
very difficult to forgive those who you feel have harmed you. One way of
making the effort to develop the virtuous quality of forgiveness could
be to focus attention on how much pain we cause ourselves by indulging
in resentment -- not merely mentally, but feeling the resulting pain and
the tension in the body. Only once we realize that we are the ones
responsible for making ourselves unhappy will we be motivated to stop
doing it. And perhaps upon acknowledging the consequences of indulging
in resentment, we will discover that we can enquire more clearly into
the mental processes involved. Maybe we come to see that memories of
past hurt are not actually a problem. The suffering of unforgiveness
comes with our investing ill will in those memories. The memories and
our ill will are not the same thing. We can't necessarily free our mind
from unpleasant memories, but we do have the potential to stop
compounding the unpleasantness by adding resentment. The ill will is
extra. As a result of seeing this, forgiveness grows.

The near right effort is described as `making an effort to remove
already arisen unwholesome states of mind'. It is beneficial to
familiarize ourselves with what the Buddha said about the five ways of
removing distracting thoughts\cite{thoughts}. Also I
would recommend reading what Ajahn Tiradhammo wrote in his book,
\emph{Working with the Five Hindrances}\cite{hindrances}.

In my own experience I have found it useful to bear in mind that the
kind of effort required to deal with an already arisen obstruction
depends on the intensity of the obstruction. It seems to me there are
three approaches. When an obstruction is of a low level of intensity we
can afford to simply ignore it -- to not give it the energy of our
attention. Sometimes this is enough for the obstruction to disappear. It
is similar to choosing to not answer the phone when it rings. I call
this the `cutting through' approach.

When we encounter an obstruction that is charged with more energy,
attempting to ignore it or cut through it could lead to making things
worse. It might seem like it disappears, but that doesn't mean it has
gone away; it has gone into unawareness and might be more difficult to
deal with when it returns. For this level of intensity we need to turn
around and face that which is troubling us and use our faculties to
investigate. We could call this approach `seeing through'. We use our
mental, emotional and physical faculties to enquire as to the source of
this obstruction. How do we feel in our heart as we face this sense of
being blocked from progressing? Where do we feel the tension in the
body? In other words we build a relationship with it: the opposite of
ignoring it. We might even strike up a conversation with it: `What do
you want? How can I help you? Sorry I have been ignoring you.' As we
become more acquainted with the whole body-mind sense of the
obstruction, not only will our mental acumen be available to support the
investigation, but also our intuition. When we feel confronted with a
real conundrum, we need to be listening to all of our being, including
our gut. In the process we might find that we grow tired of trying to
figure out a solution and head outside for a long walk in the woods, or
go swimming. Physical exercise is an important concomitant in this
process.

A different kind of effort is required when dealing with the most
intense type of obstruction, which I call `burning through'. In my own
case it often feels like physical burning involving a lot of heat. If we
find ourselves in such a situation, there is not much that we can do
other than feel the fire, stay present in the body-mind, stay soft and
open, and be consciously willing to bear with it, especially when it
feels unbearable.

Now to the final of the four right efforts: what is involved in `making
an effort to avoid the arising of so far unarisen unwholesome states of
mind'? Let's take the example of witnessing how unpleasant it is to be
in the company of someone lacking empathy. Having noticed how much hurt
can come from such a lack of emotional development, we decide to make an
effort to avoid becoming like that. Just because we happen to meditate
regularly does not guarantee that we are protected from falling into the
trap of insensitivity. There are many meditators around who become so
caught up in trying to solve their own suffering that they become
obsessed and short-sighted: while making an effort to attend to their
pain they have been pulled down into the vortex of their pain. This is
one of the very real dangers of meditation practice. To avoid this
danger we can turn up the volume of compassion.

As an exercise in formal meditation we can imagine the face of another
person and think to ourselves, `Just as my eyes have cried many tears,
their eyes also have cried tears. Just as I suffer, they too suffer. May
all beings be free from suffering.' We can perform the same exercise in
daily life: sitting on a train or waiting in an airport, look at the
faces of those around you and imagine tears rolling down their cheeks.
It is safe to assume that everyone has cried, and when we feel how we
feel when we recognize that fact, the barriers we construct around
ourselves can begin to dissolve. Maybe we start to sense that we are all
in this together -- men and women, young and old, rich and poor -- we all
suffer and long to be free from suffering. Mindfully empathizing with
the suffering of others gives rise to compassion and can protect our
heart from becoming cold and insensitive.

Although there is a great deal more that could usefully be discussed on
this topic of \emph{viriya}, there is at least one point in particular that
should be mentioned. We have considered the importance of generating
energy, and we also must be ready to accord with energy that arises
spontaneously. Here I am referring to the intensity we feel when faced
with a dilemma.

When we are in the middle of a dilemma and feel frustrated, it is the
easiest thing to indulge in wanting the \emph{dukkha} to disappear. The same
applies to when we are shocked -- when something totally unexpected
occurs and our bubble of uninspected assumption bursts, and we
experience a great release of energy. If our wanting to be free from
\emph{dukkha} is informed by wisdom and restraint it will help motivate us to
find the cause of the \emph{dukkha} and the way out of it, but often our
wanting is laced with clinging and only serves to stoke the fires of
frustration. It is skilful to prepare ourselves in advance for such
occurrences in order to not miss the precious opportunity to make
progress on the path. A dilemma or a shock should be seen as free energy
that has been made available to fuel the purification of our gold. And
we prepare ourselves by wisely reflecting in advance. The perception of
intolerable intensity that arises with such experiences is the result of
our imposing limitations on awareness. When we decide that we can't
handle the intensity, there and then we are imposing limitations on the
heart of awareness: we are turning away from our refuge of trusting in
the Buddha, and instead believe in the story in our heads that tells us
we can't handle it. Wisely reflecting in advance is one way of nurturing
the mindfulness and restraint that have the power to prevent us from
forgetting the refuge in the Buddha -- in edgeless, selfless,
just-knowing awareness. If we remember the refuge, then the energetic
intensity that manifests upon feeling frustrated or shocked is a gift
for which we can feel grateful. It is our habit of clinging that creates
the perception of limited awareness, and it takes energy to free
ourselves from that habit. How we view energy when it hits us determines
whether or not we benefit from it.

As with gravity, we don't have to know what energy actually is to be
able to accord with it. What matters is that we know how to access it
and generate it so that when it is needed we are not caught unprepared;
and when an unexpected wave of energy does appear, how to meet it
without judgment, without the contraction of fear -- how to benefit from
it.

\clearpage
\section{Samadhi (collectedness)}

\begin{quote}
  On hearing true teachings\\
  the hearts of those who are receptive become serene,\\
  like a lake: deep, clear and still.
  
  \quoteRef{Dhammapada 82}
\end{quote}

As \emph{saddha} and \emph{pañña} compliment each other, so do \emph{viriya} and
\emph{samadhi}. While \emph{viriya}'s speciality is getting things done,
\emph{samadhi's} speciality is skilful not doing.

The Dhammapada verse above speaks of a deep, clear stillness that can
appear upon receiving true teachings. This image fits well with how we
might usefully contemplate the cultivation of \emph{samadhi}. Particularly
for those whose native approach to practice is primarily source-oriented (refer Appendix I, \emph{We Are All Translators}), developing \emph{samadhi} is
not so much about making the mind peaceful, as about allowing the mind
to resume peacefulness: we are not `doing \emph{samadhi} meditation', but
`allowing stillness'.

When we first start out in meditation practice, most of us benefit from
precise instructions on what to do and what not to do. When I give
beginners meditation instruction I usually encourage them to count the
out-breaths. We are so used to always doing something to get somewhere,
that beginning with `not-doing' is perhaps asking too much. However, it
seems to me particularly important that students of meditation learn
early on that the attitude with which they approach practice will
determine the result of their effort. If they relate to their practice
with an attitude instilled by a culture of consumerism, and they feel
entitled to get the results they desire as and when they wish, they may
not get very far on the journey. I am not saying that everyone ought to
adopt a source-oriented approach, and that goal-oriented practice will
not be productive -- obviously for some it can be -- just that if our
striving to make our mind peaceful is not working, then we should be
ready to consider adjusting the kind of effort we are making. Applying
focused attention on counting the breaths, for instance, can introduce
us to what is possible; it can nurture faith and motivate us. But once
we recognize the potential that we have for inner peace, we need to
attend closely to the attitude with which we engage the spiritual
exercises.

The sort of attitude we need to have when disciplining attention is
similar to that of a gardener as he or she trains their runner
beans\cite{beans} to run along the frame which they have
erected. The gardener gently guides the beans to grow in a certain
direction so they get maximum sunlight and are easy to pick once they
are mature. The gardener is aware that if they are not careful they will
damage the tender young shoots. And they understand that, with enough
water and warmth and time, the vines will produce beans of their own
accord. They are not trying to squeeze the beans out of the vine -- it is
not up to the gardener to force the plant to produce beans.

Personally, I have found that when I approached practice with a
striving-gaining attitude, my mind became more disturbed, not less. I
spent many years trying to make my mind peaceful because that is what I
understood the teachers were telling me to do. Eventually, when I came
to realize that not everyone was out of balance in the way I was, I was
able to accept that I needed to adjust my approach. And upon reflection,
it seems that not all the teachers were advocating a goal-oriented kind
of effort anyway -- just that that is how I interpreted what they were
saying.

If I were to compare myself with how I understand some other meditators
relate to \emph{samadhi}, I would say that my mind is all over the place -- my
\emph{samadhi} is hopeless. However, that would be a heedless assessment. It
is indeed true that my mind is not as still as I would want it to be,
but it is not all over the place. There is a sense of containment, and
with that comes a degree of clarity that I did not use to have. With
that increased clarity comes an ability to contemplate life, and that is
what really interests me. I am not drawn to `making the mind peaceful',
but I am drawn to stewarding attention in a way that inclines the mind
towards stillness, and such stillness invites deepening of enquiry. This
approach to the development of \emph{samadhi} is perhaps best described as an
effort to stop causing disturbance: to stop taking sides, and to let go
of the compulsive judging mind.

Many of the approaches to meditation that have been taught in Buddhist
centres in the West originated in monasteries in the East. These
teachings emerged out of minds that were conditioned in ways very
different to ours. Casually comparing one culture with another is of
course unhelpful and disrespectful, but to ignore how different our
cultures are, and the effects those differences have, is naive. The
effects of being raised and educated in Judaeo-Christian culture, where
there is an emphasis on competing and comparing, are very different from
the effects of growing up in a traditional Buddhist culture where the
law of kamma and rebirth is accepted, and where guilt and self-loathing
are generally unfamiliar concepts.

For many years now in my meditation practice I have used a reflection on
the compulsive judging mind: observing the tendency of the conditioned
mind to take sides for and against, and observing how a confused sense
of self is sustained by that process. Regularly I hear meditators talk
in very critical tones about their practice. They might have been
practising for many years and making admirable effort, but because they
still don't see the undermining effects of the compulsive judging mind --
of taking sides for and against the conditions that arise -- they don't
receive the fruits of their good efforts. They are addicted to
`becoming' -- to \emph{bhava}.

Twice a year in our monastery, we meet for a fire risk assessment. One
of the major risks that requires regular mention at those meetings is
the overloading of extension cables. Extension cables come in various
types: some can be used for operating a lamp or a laptop but must never
be used for a hot water kettle or a heater. Others are designed to carry
a heavier load and can be used for running more power hungry appliances.
If the wrong sort of cable is used there is a real risk of starting a
fire. In the spiritual life, regular mention is required regarding the
immodest efforts of meditators who are hell-bent on attaining elevated
states of mind. Without modesty and contentment their heroic efforts can
lead to an overload of their nervous systems and, sadly, sometimes cause
meltdown. Many people come to this path of spiritual practice with
wholesome aspirations but regrettably don't receive adequate instruction
in developing the right attitude. Our aspirations are a form of energy
and that energy can take us either in a direction of increased balance
and ease, or to increased confusion. We would do well to remember the
teaching that the Buddha gave to Bhikkhuni Mahapajapati where he
included modesty and contentment as two indicators of right practice.

We are bound to have been affected by the greed-fuelled consumer culture
in which we grew up. I recommend posting the words `contentment' and
`modesty' in places where you will easily see them, or any other words
that you feel could serve to counterbalance the effects of rampant
consumerism. Approaching the cultivation of \emph{samadhi} with an attitude
that is rife with self-centred greed is setting ourselves up for great
disappointment, or worse: it can sow seeds of discontentment deep within
our hearts.

If you have had a taste of \emph{samadhi} and then get greedy, it is possible
you will become hypersensitive and won't want to listen to what anyone
else has to say: you become inflated with self-importance. When practice
is proceeding in a balanced way, inevitably you experience an increase
in sensitivity -- mentally, emotionally, physically. However, contrary to
what we might imagine, increased openness and sensitivity does not
necessarily immediately make us feel more calm and balanced. It can in
fact make us feel more exposed and unstable. The amount of time it takes
before we feel comfortable with increased openness and sensitivity will
probably depend on how contracted and out-of-balance we were to begin
with. The point here being: as tempting as it can appear, it is not
always the case that the more \emph{samadhi} the better. \emph{Samadhi} should be
viewed as a medication that can be skilfully used in support of
increased well-being. It can also be abused in support of habits of
addiction. If you notice that you are still entertaining attitudes of
untamed greed in the way you hold your meditation, try dwelling on how
you would hold a newborn baby -- gently, softly and lovingly.

Another approach to stillness -- and one with which we might prepare
ourselves by contemplating in advance -- can come with sickness. A few
years ago a good friend of our monasteries contracted Lyme
disease\cite{lyme}. As can happen, his condition went
undiagnosed for a long time. Then it took a great many months before he
could say he was back to anything like normal again. When he was
somewhat recovered, he shared with me how at one stage during his
illness he didn't have enough energy to even lift himself off the bed.
And in that debilitated state, there was a period when he was so drained
of energy that even the effort required to maintain a sense of personal
self was beyond him. He related how when he reached that point, the
individual self disappeared, and what was left was a perception of vast
expanded awareness and connectedness -- there was no fear. He felt that
if that was his final breath he was going to take, that would be OK.
This friend had been meditating for many years before falling ill, so we
can assume it was not through sheer luck that he stumbled upon this
life-changing experience.

\section{Sati (mindfulness)}

If the four faculties of \emph{saddha} and \emph{pañña}, and \emph{viriya} and
\emph{samadhi} are functioning in an optimum way, they will exist in a state
of balanced tension. If there is too much faith we tend to lose our edge
in practice and become heedless; if there is too much enquiry we end up
questioning absolutely everything and risk becoming possessed by doubt.
If there is too much energy we will feel restless, and if there is an
over-emphasis on stillness we could become susceptible to delusion.
\emph{Sati} serves to oversee balance. \emph{Sati} does many other things besides,
but in the context of this contemplation of the five spiritual faculties
this particular function warrants mentioning. Ideally there will be a
dynamic tension between the faculties, which strengthens and deepens our
effort. \emph{Sati} manages our life. My keyword for \emph{sati} is watchfulness.

Tan Ajahn Chah had an interesting expression: `kaad \emph{sati} muea rai,
bpen baa muea nan' which translates as `moments when you are without
\emph{sati} are moments of insanity'. \emph{Sati} is that central, not only to the
Buddha's path of practice, but to life. Sometimes Tan Ajahn Chah made
jokes saying that Westerners have `stupid feet' because we would always
be stubbing our toes as we walked through the forest. We might have
appeared very clever intellectually since most of us had spent more time
in school and at universities than the Thai monks, but we were clumsy
and inattentive. When ceremonies take place, the Thai monks seem to have
a way of knowing exactly when to act and what to do, and it isn't just
because they are familiar with the protocol and we aren't. Even very
junior monks and novices seem to be attuned to what is happening and can
respond without someone having to tell them what to do. They are
attentive and more `embodied' than we are. Owing to their better
developed sense of spatial awareness they are picking up on more
information. I doubt that mindfulness will ever be commodified in
Thailand the way it has been in the West; it would be like trying to
commodify breathing. The concept of \emph{sati} is so thoroughly embedded in
their culture.

At Wat Pah Nanachat, particularly during the seven years that constitute
the first three stages of training, as a postulant, novice and junior
monk, the monastic training offered is largely a process of assimilating
the principle of embodied mindfulness. There is an emphasis on the
cultivation of mindfulness in all aspects of life -- mental, emotional,
physical, relational. It is a gradual whole body-mind training which is
altogether different from `me' performing a technique so that `I'
improve `myself'. The understanding behind this traditional approach of
embodying mindfulness is that to be able to untangle the knot of deluded
egoity -- to be able to investigate \emph{anicca, dukkha, anatta,} and awaken
-- requires that we are mindful in the whole body-mind. Mindfulness
training is not merely a mental exercise.

Earlier I described how during my time as a junior monk in Thailand I
was hesitant to join in with the other monks performing attendant duties
with our teacher, Tan Ajahn Chah. I suggested that my hesitation was
because of a fear of rejection, but it might have been more complicated
than that. I suspect that I also sensed on some level that I simply
wasn't up to the task. When there is embodied mindfulness we can
function with ease; intuition will be informing our actions. In my case,
because of a lack of embodied mindfulness, I was functioning from a
place of perpetual controlling. I was always thinking, `What should I be
doing now?' No wonder I was so exhausted so much of the time. It wasn't
that I was bad, I was just a little bit crazy. Fortunately I wasn't so
crazy that I couldn't learn from my mistakes, and that is really what
matters.

We all fall short of how we would want to be. We all have so much to
learn. However, when we have mindfulness we are more able to learn. The
root of the word `\emph{sati}' means `remembering', and perhaps when we are
translating `\emph{sati}' a better word than `mindfulness' would be
`presence'. We can be too trusting, too energetic, too tranquil and too
inquisitive, but we can never be too present. The more consistently
present we are in the whole body-mind, the better.

Regularly reflecting on these five spiritual faculties is an activity I
find thoroughly rewarding. If I were sitting in the Buddha's waiting
room contemplating as we have been doing, I like to think I would be
protected from falling prey to too much heedlessness. I also like to
think that by sharing these reflections, readers might find a few hints
that will help them as they progress along the way. It is a huge good
fortune to have come across this way as explained by the Buddha. It is a
privilege to find companions with whom one can share the journey. None
of us know how much time we have and what challenges lie ahead. But
right now it is my conscious wish that we remember to dwell in gratitude
for the benefits we have already received.

