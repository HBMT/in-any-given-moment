\chapter{The Forest Sangha Calendar}

When we had received word that our teacher, Tan Ajahn Chah, might not be
with us much longer, it occurred to me that we could mark the occasion
with a pictorial calendar -- something that could be printed and
distributed around the world to the increasing number of branch
monasteries and their supporters.

In part, my motivation was to find a beautiful way of honouring the life
of our teacher. Also, in equal part, it was to produce something that
would offer the extended community of lay disciples of Tan Ajahn Chah a
sense of belonging. We all benefit from feeling like we belong
somewhere, and if that somewhere is a spiritual community for which we
have respect, then all the better. It seemed to me that having a
calendar hung on the wall throughout the entire year, one that you could
look at regularly and be reminded of the community of which you were a
part, would be beneficial. A young fellow called Pete, who was
frequenting Chithurst in those days and was a graphic designer, kindly
offered to assist me in compiling such a calendar.

This was the beginning of the annual Forest Sangha calendar\cite{calendar}
that has been produced each year, except one,
since 1990. The design has altered somewhat over the years, and the
production and distribution has changed, but as far as I can tell, the
function has remained much the same. It has been a privilege and a
pleasure to have been involved in this project all these years. I say
`involved', because it depends on many more people besides myself to
produce it. The final selection of photos and quotes has been my
contribution, but I have had the assistance of a good number of others
when it comes to design, layout, printing and distribution. The process
of acquiring the astronomical (and astrological) dates for many years
depended on when the royal palace in Thailand would release them. These
days, thanks to an algorithm skilfully generated by Tan Gambhiro and
colleagues, we are able to calculate the dates with excellent accuracy,
without having to wait to hear from others. Initially the calendars were
printed in this country and shipped abroad. For many years now, thanks
to the generosity of the Kataññuta Group in Malaysia, they have been
printed and distributed from there.

Not everyone in our sangha agrees with my personal preference
(influenced by Marshall McLuhan's \emph{The Medium is the Message}) for
black and white images, and my view that they more effectively
communicate the message, `less is more'. The world is intoxicated, in my
opinion, with excess sensory stimuli. The idea of producing a
full-colour calendar could be tempting in the same way that over-eating
of cakes and cookies could be tempting. Our message, as far as I am
concerned, is: if we are seeking clarity and contentment, either as a
samana or as a householder, then simplification is what is called for --
not proliferation. This is one of the central themes in the teachings of Tan Ajahn Chah,
as I understand them. It is not merely a matter of aesthetics (although
I acknowledge that is a factor).

Then there have been a variety of opinions about the sort of photos to
publish. The first year consisted exclusively of images of Tan Ajahn
Chah. If we had continued doing that, it could have fed into the notion
some people had that we were a sort of cult. For a while it seemed that
nice pictures of nature would be a good way of representing the Forest
Sangha. One year we used images of shrines in the various branch
monasteries. On occasion I would receive an objection because somebody
didn't think a photo I had selected looked quite right. Then there have
been requests that we have more photos of monks' and nuns' faces; at
other times, requests that there be less photos of faces. For many years
now the photos have simply been of people doing things in monasteries.
Since we regularly receive messages of appreciation and there are 18,000
copies printed annually, it must still be serving a useful
purpose.

Selecting suitable photos for these calendars is always fun; however, an
equally rewarding part of the project is finding the `just right' Dhamma
quote: one that resonates with the image. In the early years we used
extracts from translated teachings of Tan Ajahn Chah. More recently we
have been alternating, year by year, Tan Ajahn Chah's teachings with
verses from the Dhammapada. When our efforts are successful, the image
generates an atmosphere that makes users of the calendar susceptible to
the message contained within the Dhamma quote.

Towards the end of 1990, I was told that I would be moving to
Northumberland to take over leadership of the community at Harnham. With
the development of more branch monasteries in Britain, Switzerland,
Italy and New Zealand, a pattern was beginning to emerge whereby abbots
would be moved on roughly every two years. The idea behind that was to
try and avoid anyone becoming overly attached to one place. This wasn't
the only reason Ajahn Sumedho sent me to take over at Harnham in
Northumberland, but it fitted in with the pattern. A second monk was to
join me, Tan Vipassi. Before heading north, he and I spent the Winter
Retreat together at Amaravati.

