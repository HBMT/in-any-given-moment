\chapter{Doctor Albert Schweitzer}

Around the age of fourteen I was given a copy of a book, possibly as a
Christmas present, about the French-German philosopher, theologian,
musician, writer and doctor of medicine, Dr.~Albert Schweitzer\cite{schweitzer}.
I'm not sure which came first, the book or
the idea of my entering into the Rotary Club speech contest that ran
each year at our school, Morrinsville College. Whether it was my
parents' idea or mine, I also don't know, but at some stage in my early
teen years I assimilated the idea that I would grow up to become a
preacher. It isn't hard to imagine how such a suggestion might have
become lodged in my mind. Entering that annual contest and speaking
about Dr.~Albert Schweitzer, fitted with that vision. As far as I recall
I made it into the semi-finals or perhaps even the finals of the
competition, but I didn't win. What matters now though was the good
fortune of having been made aware of this extraordinary human being.
Dr.~Schweitzer started out studying for, and achieving, a PhD in philosophy,
then the next year a PhD in theology; he was a world renowned organist,
an expert on and writer about Johann Sebastian Bach, and that was all
prior to studying and graduating as a medical doctor so he could spend
50 years serving the sick and needy in a remote medical facility in
equatorial Africa. When in 1952 he was awarded the Nobel Peace Prize,
the prize money went straight to the hospital. I feel fortunate to have
been introduced at that formative stage of my life to such an example of
selflessness. There are some who have written critically about
Dr.~Schweitzer, and I am sure he himself would have been critical of his
failings; however the manifest goodness of this man is truly worthy of
admiration.

Many years later when I read and heard talks by the Jungian analyst
Robert Moore, I became aware of the power of intentional admiration. By
consciously admiring a particular quality in another, that quality can
be nourished within ourselves. I have no recollection whether
Dr.~Schweitzer's being a vegetarian impacted on me at the time, but perhaps
it did. His \emph{reverence for life}, as he referred to his commitment
to harmlessness, meant that not only was he vegetarian, but he also
refused to kill insects, which in malaria-ridden Africa was a radical
statement. Vegetarianism, and eventually a purely plant-based diet,
became an important part of my life some years later.

I think it was also when I was fourteen years old that I contracted
hepatitis. In the summer of that year, probably 1966, as was usual for
our family we had gone away for the holidays. Almost without exception,
every year we would go to a rented cottage (`bach' in Kiwi parlance)
near a beach somewhere, and spend many hours swimming in the sea and
playing in the sand (totally unaware in those days of the risks that
come with excess exposure to UV). That year, we went to Ohope, near
Whakatane. I remember that we were eating a lot of fish at the time and
when I fell ill it was assumed that somehow the fish had given me food
poisoning. It was only once we returned to Morrinsville that my illness
was diagnosed as hepatitis. Horrible as it was, it only meant I was
bedridden for a few weeks and missed some classes at school. One of the
enduring unpleasant memories of that period, however, was being visited
by one of the Sunday school teachers who appeared to me to be there
under duress. I can't say what her intentions were but somehow I didn't
like her visiting me. Possibly by that stage I had already learnt to
pretend to be virtuous in an attempt to win approval, and in the
process, betrayed myself. Maybe I thought I sensed something like that
in her.

A few years earlier I had made the mistake of lying to my parents about
giving my heart over to Jesus. I observed how when my older brother
David announced he had made such a commitment to the Lord, he received a
lot of approval and affection. I wanted some of that. Unfortunately, the
importance of impeccability was not something I had learned about so I
wasn't aware of the painful consequences of telling such an untruth. It
was only years later that I began to see how, when we lie, we create a
fissure in our minds which leads to instability. We then compensate for
the feeling of instability by becoming rigid, which obstructs the flow
of life. We turn ourselves into someone untrustworthy: a recipe for
self-hatred and guilt. For fourteen-year-old Keith Morgan, sadly, the
darkness was already beginning to descend.

Auntie Nessie is mentioned several times in these notes, which indicates
the special place she has in my heart. If I had shared with her back
then how badly I felt about the lie I had told my parents, I am sure she
would have listened carefully, looked at me with her kind eyes, and said
something that would have made all the difference. It was generally
understood that Auntie Nessie was a faith healer and I have the
impression that she was held in high regard in various parts of society.
She would never take personal credit for any ability she had, and was
always quick to ascribe all goodness to her Lord. A few years later in
1973 she received an MBE from Her Majesty The Queen, by way of
recognition of the work she had done in Arohata Womens' Prison. One of
the favourite memories I have of her these days is, once when she had
come to stay with us, we children begged her to perform a Maori haka\cite{haka}
for us before we would go to sleep. In
my mind's eye I can still see her standing at our bedroom door, slapping
her knees, with her tongue out and eyes bulging. We loved it and I loved
her. A heartfelt thank you to Auntie Nessie.

When I was about fifteen years old, members of our local church began
discussing the possibility of replacing the church building. With my
parents' encouragement I spent a lot of time producing designs for the
construction. Probably this was the first time I entertained the fantasy
of one day becoming an architect instead of a minister or a missionary.
Over the next few years, however, I discovered that, although I took
considerable pleasure in producing designs, I was utterly hopeless when
it came to mathematics. Working with numbers, and for that matter even
reading books were, and continue to be, a chore (more on that later).
The love of imagining, and sometimes having the good fortune to design
actual buildings, remains with me. It is hard to explain the pleasure I
derive from creating a space in my mind that feels right and then seeing
it manifest in form. As far as I am concerned, successfully designing
living spaces is primarily not about how the space looks, but how it
feels when we are in it. After all, we don't live in the things, we live
in the space. Many people, it seems, don't recognize this, which is
presumably why they fill their living spaces with so many things. They
assume that including yet another nice thing will make the place look
nicer, when in fact that extra thing might well offend the space,
regardless of how attractive the item in itself might have appeared.

As a present on my sixteenth birthday I was given a course of private
art lessons in oil painting by a nationally recognized painter, Violet
Watson. Not that I was any good at painting but I imagine it was better
than pretending I wanted to play rugby. Morrinsville produced several
rugby players who featured in the national All Blacks teams -- Don
Clarke and Ponty Reid are two names that come to mind. On one occasion,
when the All Blacks had done particularly well overseas, the players
were honoured with a procession down the main street, Thames Street.

Already I have mentioned that I learned to play the violin. I can't be
sure now about details, but I think that this lasted for about three
years. The school provided the instrument and the instruction. As I
recall, the music teacher at Morrinsville College was very encouraging.
The most enjoyable aspect of it came when I was accepted into the
Waikato Youth Orchestra. Each Friday afternoon after school, I would
hitchhike, carrying a violin, the twenty or so miles to the nearby city
of Hamilton. Being part of that body of people, with a shared fondness
for making music, was a highlight of my week. Many years later, when a
Jungian psychoanalysis friend described spiritual community as `a
harmonious resonance of shared aspiration', I felt like I could get what
she was talking about. Perhaps my time playing in that orchestra helped
with my understanding. After these sessions of practice finished I would
take a train back to Morrinsville.

Thinking back about it now, it surprises me that I didn't feel more
bothered than I did by the fact that I wasn't fitting in with the other
boys. One explanation might be that because our family, for obvious
reasons, was considered churchy and thereby somewhat different, I
perhaps assumed that not fitting in was somehow normal. Children naively
assimilate all sorts of assumptions. Another explanation could be that I
was learning to hone down my skills in self-deception. I was betraying
myself by not truly feeling what I was feeling and as a consequence
gradually becoming more fragmented within.

