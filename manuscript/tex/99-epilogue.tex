\chapter{Epilogue}

\begin{quote}
A sweet-smelling and beautiful lotus\\
can grow from a pile of discarded waste.

\quoteRef{Dhammapada v.58}
\end{quote}

Although our goal in practice is clear seeing -- wisdom -- it is faith
that ignites our aspiration and enables us to embark on this journey of
awakening. We have faith that there is more to life than that which
appears on the surface; we are keen to look more deeply. Faith helped
pique my interest and led to my joining my first meditation retreat;
faith meant I have been able to endure apparently unendurable ordeals
and burn through layers of habitual resistance. Faith illuminates the
way ahead when it seems that there is no way. To have faith in the
Buddha's teachings is to have wealth.

This wealth nourishes us, often in unimaginable ways. If we could
accurately imagine what lay ahead we wouldn't need faith, but we can't,
so faith is essential. Faith has the power to transform pain into
understanding and confidence. On this journey I have seen how the pain
of the loss of a sense of belonging can lead to learning how to let go
of false securities. I have learned that the pain of feeling judged can
lead to enquiring more deeply than I thought possible. Early on in life
I was told that we are all born damaged goods, and in my case it took a
lot of effort before I could even begin to see that story as a story.
Like many others I was taught that we need someone else to save us, but
that is like being told someone else can take away our shadow. Everybody
has a shadow. What is needed is that we understand the nature of our
shadow and we develop the spiritual faculties until we are able to fully
receive all aspects of who and what we are. For the teachings that point
to that understanding I am sincerely grateful.

Alongside the teachings of the Buddha, spiritual companionship is that
which I hold most dear. Without the clarity of the teachings, there
would be no path for us to follow, but without good friends progress
would be limited. I have already mentioned some of the names of those
friends upon whom I have depended, but there are more whose names, for a
variety of reasons, have not been mentioned. Before this series of
reflections comes to a close I want to express my heartfelt appreciation
to you all. The thought of how this life would have been without your
friendship is very unappealing.

Sometimes when gratitude appears it is familiar and expected, like how
we feel when we take off a heavy backpack at the end of a long day's
walk. At other times gratitude feels both familiar and surprising at the
same time, such as when, towards the end of a long dark winter, the warm
sunshine might suddenly break through, triggering a release of the fragrance
of hyacinths and jonquils. Then there are times when gratitude comes as
a total surprise, such as when an old friend whom we haven't seen for
many years, unexpectedly arrives for a visit.

However it manifests, a sense of gratitude is always welcome. Gratitude
nurtures hope: not hope of the naive kind, as we have discussed, but of
the kind that conduces to insight, of the kind that teaches us that,
whatever is happening -- be it agreeable or disagreeable -- every moment
is always new, even if we feel or think otherwise: every moment is a new
opportunity to learn how to let go and trust in that which is, and
always has been, simply true.

Thank you.

