\chapter{The End of the River}

Approximately eighty-five miles south of Auckland, in New Zealand's North
Island, there is a small town called Te Awamutu. This is where I was
born in September 1951 and was given the name Keith Morgan. The Maori
name of the town, Te Awamutu, translates into English as \emph{the end
  of the river}. In various online resources\cite{early-history}
it is explained that it wasn't that the river
Manga-o-hoi actually ended there, it was just that beyond that point it
became unnavigable by canoe. I'm guessing that in 1951 the town had a
population of about 5000. The area had a history as a place where
battles had been fought between opposing Maori tribes, where an early
group of Christian missionaries had established itself, and as a
settlement used by the British military during the Waikato wars. By the
time my parents, Pearl and Ian Morgan, moved there, Te Awamutu had found
its identity as a service centre for the surrounding farming
communities.

Christianity was a defining element in our family. My father was the
youngest of six children in a family headed by a Presbyterian minister,
Rev. Richard Morgan and his wife Grace Morgan. My mother was the only
child of a Baptist minister, Rev. Alfred Dewe and his second wife, Sadie
Dewe, or `Nana' as we knew her. Rev. A. Dewe died young and so eventually
Nana remarried another Baptist minister, Rev. Christopher Wilfrid
Duncumb, after spending a number of years as housekeeper to a
Presbyterian minister, Rev. Lloyd Wilkinson. Auntie Nessie, my father's
older sister, was a deaconess in the Presbyterian Church, Uncle Roy was a Baptist
minister and my younger sister, Jennifer, went on to become a pastor,
who, along with her pastor husband Guthrie Boyd, ministered within the
church of the Assembly of God. Recently I found out that my younger
brother Bryan, is ministering these days as a lay preacher in the Paihia
Christian Fellowship.

We lived in Te Awamutu for about two years before moving to a similar
sized town, Morrinsville, about twenty miles away. I imagine my father's
work was the reason for the move. Although for much of my life I have
struggled to find my place in this family, this does not mean I don't
value it. To be born to parents who worked so tirelessly to raise their
four children in a wholesome environment was indeed a blessing. At later
stages in my life it became apparent that growing up in that environment
was a mixed blessing and it did take some skill and discernment to
decipher which aspects were truly valuable and which needed to be left
behind.

When I think back now about my father, I have huge admiration and
gratitude for his integrity and kindness. Besides his Monday to Friday
job working in Hawkes Motors Ford garage, initially as a mechanic and
eventually as the manager, he would spend many hours after work and on
the weekends cultivating a substantial vegetable garden that he had
planted out the back of our house. Always on Sunday he would drive the
family to the morning church service and for some time led the
Sunday School of which he was superintendent. Regularly after Sunday
lunch, he would drive out to remote village halls -- places like Tahuna,
Ngatea, Patetonga -- to conduct a church service for the farming
families who couldn't manage to get into town.

Similar feelings of appreciation arise when I remember my mother's
dedication and how she would spend days on end in the kitchen throughout
the hot summers, bottling and preserving vegetables, apples, and peaches
and pears to keep the family well fed through the year. She also sewed
many of our clothes.

Bible readings and prayers at the evening dinner table were normal. The
summer camp I was sent to was a Christian Youth Camp (CYC) at
Ngaruawahia. Since my parents were teetotal, and drinking at the pub was
a national pastime in New Zealand, we had very few visitors. The only
visitors I remember coming to our house were relatives who were equally
if not more devout Christians, and other families who attended our
church. Devotion to religion from such an early age served to instil
virtues in me which I continue to value. I can't recall my parents ever
arguing and, with only one very minor exception, nor did I ever hear
either of them speaking critically of anyone.

Music was another central aspect of our family life. Both my mother and
father were vocal leaders during hymn singing in church on Sunday; not
just singing loudly, but with fervour and enjoying the opportunity to
break out into harmony. My mother sometimes played the organ in church.
At home, our idea of a good time amounted to us children standing around
singing praise to Jesus while Nana played the piano. Auntie Nessie spent
many years working as a missionary to various Maori communities, and
leading a choir of Maori singers.

David, my older brother by two years, learned to play the cello. I
played the violin for a while, and I seem to remember that Jennifer, my
younger sister, played the piano. Compared with what most families these
days might think of as having a good time, our hymn singing sessions
probably sound rather tame, however I recall them as a source of
considerable happiness: the togetherness, the sheer pleasure of making
beautiful music and the delight in praising the Divine. When I was old
enough to have my own bicycle I would regularly ride out into the
countryside to be alone. I frequented the woodlands along the banks of
the nearby Piako River where I would passionately sing my heart out.
Something about unrestrained adoration of the Almighty triggered
tremendous joy within me: it was thrilling, even electric and
exhilarating.

