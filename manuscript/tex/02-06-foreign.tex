\chapter{A Very Foreign Country}

Andrew and I arrived in Sydney, Australia, shortly before my 22nd
birthday, September 1973. One of the first impressions I had was one of
being shocked at the sight of police in Sydney's airport carrying lethal
weapons. The only real guns I had seen were for hunting rabbits or wild
pigs; in New Zealand the police did not carry lethal weapons.

When we are shocked by something (and I don't mean merely surprised) it
is a sign a bubble we were living in has just burst. The bubble I had
been living in was the idea that the rest of the world was some sort of
slight variation on New Zealand. Unconsciously I assumed everywhere
would be as peaceful and parochial as New Zealand. Although I guess I
wasn't aware we were parochial. Of course I had seen people-killing guns
in movies and on television but only now realized that there are places
where these guns are actually used. Eight or nine months later, another
bubble would be burst when I flew from Darwin in the Northern
Territories of Australia and arrived in what was then Portuguese Timor.

Besides feeling somewhat shocked at the airport I imagine I felt excited
to be somewhere genuinely new. Not long before taking leave of New
Zealand we had hosted a fellow, I think Canadian, at our place in Owen
street. He must have given us the address of where he expected to be
staying in Sydney since, even though the address was a bit vague, we
easily located the place and found ourselves being received at their
apartment. An American guy called Bill Hamilton was part of the group
living there and he mentioned that we could probably find work at the
Darling Harbour Railway Station. He had a cushy job there spending much
of the day folding tarpaulins. As it happened, I did get a job there
without any difficulty; however, mine was loading beer kegs onto railway
carriages.

Within a few days Andrew and I had found our own place to live in the
suburb of Coogee (which is the Aboriginal word for `stinky', because
large quantities of stinky seaweed sometimes wash up on the beach
there). Also, soon after we arrived in Sydney, an Australian fellow
called John, took me to visit a Buddhist monastery, Wat Buddharangsee,
in the suburb of Roselands. Maybe when we were staying in Bill
Hamilton's place, I had mentioned that I was interested in meditation
and John had picked up on that. Meeting Buddhist monks for the first
time was intensely intimidating. I imagined they would be able to read
my mind and see what a fake I was. Part of me desperately wanted to be
`seen', but a big part of me felt like I couldn't handle it. On that
occasion I had the good fortune of meeting Tan Phra Khru Mahasomai who
was Laotian by birth but had received his monastic education in
Thailand. He had a big scar across his face that you couldn't help but
notice; he also had an even bigger and very warm smile.

That was the only time I visited Wat Buddharangsee until about 5 years
later. The next few months were spent accumulating funds so that I could
travel north and join Jim and Roselberry on their commune near
Mullumbimby. On one occasion when working at Darling Harbour Station
there was another of those moments that stood out as significant. In
truth I can't say that the moment stood out at the time: it was only
later that I appreciated how something valuable had been triggered. What
was quickened in me was an appreciation for inner work: psychological
and spiritual work. Of course the concept of doing your inner work was
not new to me, but on this occasion something clicked and a sense of the
significance of being committed to doing your inner work took on new
meaning. The person I was assigned to work with that day was mumbling
and grumbling about something that had upset him until he suddenly
caught himself and said words to the effect, `I have to be more careful.
I had forgotten that in situations when I feel let down by the tools
that I am using, I easily overreact'. It was just that example of
self-awareness that struck me. According to the conditioning I had
received, the only things that really mattered in life were what you
believed. If you professed to believe in the right things then all would
be well, and if you believed in the wrong things then all would not be
well. Although I would not have been able to articulate it at the time,
now I see how that observation usefully brought into relief the lazy
attitude that can become established in our minds if in our early
education we fail to receive wise instruction. Although I can't recall
the name of my co-worker that day at Darling Harbour Railway Station,
thank you.

When later on I became familiar with instructions offered by the Buddha
I began to internalise faith in the law of kamma. It is not the case
that anyone else can take responsibility for our actions; we alone carry
that responsibility. The Buddha rarely criticized other religions but he
was critical of teachings that undermine confidence in the law of kamma.
What he endorsed was what he called \emph{citta bhavana}, or cultivation
of the heart, cultivation of awareness: doing our inner work.

After a few months in Sydney I packed my belongings into a backpack and
headed north. I forget now how I travelled to Mullumbimby, probably by
train, but arriving there felt like how I imagine Muslims must feel when
they arrive at Mecca, or Christians when they go to Jerusalem.

Roselberry had told me to make my way to The Sunflower Cafe and ask for
directions to the Narada commune, which I did. The intensity of emotions
I was feeling was dizzying: joy at seeing beautiful, colourful,
long-haired, bearded members of the back-to-the-earth community mingling
with the local population; along with hope that perhaps I was about to
find a place where I might fit in, mixed with the fear that it might not
work out. A potent cocktail of expectation.

The image I have now of The Sunflower Cafe is that it was full of the
fragrance of patchouli oil, beautiful smiling men, women and children,
bright colours, stained glass, shelves filled with organic produce,
pottery, hand-woven shawls, and walls displaying artwork for sale. The
folk working there helpfully pointed me in the direction I needed to go
to reach Upper Main Arm where I would find the Narada commune.

It puzzles me that I can remember almost nothing about my arrival at
Narada, but this was my first experience of the Australian countryside,
and probably the sun, the heady aroma of the eucalyptus trees, combined
with the sounds of the otherworldly wildlife, all were somewhat
overwhelming. I do however have an impression that Jim and Roselberry
might have given the community members an exaggerated report of me since
they welcomed me almost like family.

There were five or six couples on the commune which comprised
approximately 200 acres of undulating land. It had previously been
farmland but was now covered in lush, subtropical, secondary growth,
mostly varieties of eucalyptus with dense undergrowth of lantana. Mount
Warning was not far away to the north and there was a stream that
meandered along one boundary of the property. Except for a couple who
lived in the main communal house near the entrance, all the others had
their own portion of the property that they occupied and their own
residences. All were Australian, several from Melbourne, some
ex-teachers and others artists, and most had known each other for quite
some time. Clearly they were embracing ideals of cooperative living, but
there was also a decent dose of common sense. For instance, each member
of the commune held shares in the company that owned the property. It
wasn't automatically assumed that I would be invited to live there but
they went out of their way to include me in activities. I wasn't treated
like an itinerant visitor.

The thing that took time and considerable effort to accommodate was the
wildlife. The screaming kookaburras weren't any trouble, they were
fascinating; it was the snakes, the goannas, stinging ants, poisonous
spiders, leeches and ticks that got to me. New Zealand bush, by
comparison, is totally benign. In New Zealand you can walk out into the
bush, pitch a tent and lie down almost anywhere without any bother
(other than from sandflies). I was used to camping and roughing it a
bit, at least that is what I thought, but here, in this very foreign
land, there was always something trying to sting me or bite me. Many of
the creatures and critters were deadly poisonous!

Jim took me to a spot a wee way further along the ridge from where he
and Roselberry lived and said it would be alright if I wanted to
construct a small residence there. From that ridge it was possible to
look down the Upper Main Arm Valley over Byron Bay to the ocean. I had
brought a modest sized tent with me, so over the next few days, using
canvas, bamboo, plastic sheeting and a variety of posts and poles, I
constructed myself a dwelling. This was to be my home for the next few
months.

Life at Narada was a mixture of fun and hard physical labour. The hard
labour in my case mostly involved hauling water up the hill to my
dwelling. This was the first time in my life I had to genuinely exercise
restraint in the amount of water I used. It became clear that even here,
living the beautiful life with beautiful people, took a lot of effort.
Maybe I actually started to grow up at that point and appreciate that
what you put in, you get back. Also there was work assisting other
community members with projects they had on the go. One of the residents
was still finishing his geodesic dome and needed assistance.

Nearly all the fun times involved music. The Main Arm Valley was strewn
with New Age communities of different shapes and sizes. A variety of
excellent musicians passed through. There was little need for shopping
other than for food. That required only a very occasional trip into
Mullumbimby; once I ventured as far as Lismore to source some hardware:
cooking equipment, tools etc. There was an abundance of fruit available
growing in the valley, especially bananas -- so many bananas! Every so
often there was a barter market where members of the different communes
would meet and exchange produce. As I recall, the ideal was that no
money would be exchanged, though I suspect that was one ideal not held
too tightly.

After several weeks of settling in, I received a visit from Danny, a
very short, very jolly chap from Texas, whom I had met in Sydney. He was
on his way to participate in a Buddhist meditation retreat due to take
place a few miles away to the west, at Nimbin. I think there was some
connection between Danny and John who had taken me to visit Wat
Buddharangsee. This retreat was to be led by an English monk, Ajahn
Khantipalo, who was usually resident at that temple in Sydney. Danny
wanted me to join him on retreat since a spare place had opened up.
Although reading \emph{The Way Of Zen} the year prior had inspired me,
the idea of getting involved with another organized religion was as
appealing as a splitting headache. Danny was very persuasive however and
I did end up joining him. That seven-day silent meditation retreat
turned out to be one of the greatest gifts I ever received.

