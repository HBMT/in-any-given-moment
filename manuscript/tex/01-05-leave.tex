\chapter{Getting Ready to Leave}

\enlargethispage{\baselineskip}

It was indeed good fortune to have been born into a family that
appreciated the importance of goodness, but when I was seventeen years
old, which was my age in 1969 when we moved from Morrinsville to
Kaikohe, I was completely unaware of the power of wisdom. I was also
unaware of why we were moving. Perhaps it was because of my father's
work or perhaps there were other reasons. He had been given a new job
managing another Ford garage. David had already left home. For me,
Jennifer, and our younger brother Bryan, it was just a case of being
told that this is what was happening: a new town, a new school and of
course a new church.

There are only a few things I remember about Kaikohe since I was there
for less than two years. I do remember that there are thermal hot pools
just outside of the town, called Ngawha Springs. It was worth putting up
with the almost overwhelming stench of rotten eggs for the sake of the
soak in the steaming hot water. I had my driver's license by then and
although I can't be sure, I suspect my mother lent me her car so I could
go out there. At the time I was also hoping that the mud we slopped all
over ourselves would cure my embarrassing acne. Usually I would go there
with a guy called Guthrie whom I had befriended from school. We were
about the same age and went to the same church. These days he is married
to my sister.

That was the year I was admitted to hospital with mumps, meningitis and
suspected pancreatitis. During the school holidays I had gone back to
Morrinsville for a visit and ended up in the intensive care unit of the
nearby Waikato Hospital. The condition was as serious as it sounds and
there were good reasons for putting me into intensive care. As part of
the regime that the doctors put in place to manage this threatening
condition I was on pethidine, which possibly contributed to the fog I
now think of when I try to recall the episode. Thankfully I recovered,
though according to practitioners of traditional Chinese medicine, the
high fevers that go with hepatitis and this latest batch of illnesses,
might have done damage to my kidneys. At the time of writing this I am
nearly seventy years old so the damage mustn't have been too severe.

\enlargethispage{\baselineskip}

As part of the education offered at the college in Kaikohe our class was
taken to visit the nearby Moerewa abattoir. The grotesque sights I
witnessed on that occasion remain etched in my mind. Now I find it takes
effort to register that members of the human race of which I am a part
would choose to conduct themselves in the ways that the workers at that
abattoir did, and probably still do. These days, when the subject of
vegetarianism comes up, I am cautious in what I say since it is easy for
groups of people to become divided according to views on matters such as
diet. I do, however, sometimes subtly suggest that anyone who eats meat
ought to visit an abattoir and see where what they consume comes from
and take on board the implications. The hideousness of the industry, the
harm it inflicts on countless animals, is only part of the story; there
is also the damage the workers are inflicting upon themselves.
Sadly, such conduct has been considered normal for so long that many people
never stop to question it. I believe that for there to be any hope of rescuing
the human race, and planet earth, from the self-destructive trajectory it is on,
it will require a radical transformation of the attitude of entitlement many
humans appear to have. It seems to me not feasible that we can continue to treat
other living beings in such an insensitive and disrespectful manner without
disastrous consequences.

It was also around 1969 that I was introduced to the thinking of the
Canadian philosopher Marshall McLuhan\cite{mcluhan}.
Through the Presbyterian church in Kaikohe I had
become aware of a Christian youth conference to be held in Christchurch
in the South Island. It was on the theme of communication. The poster
announcing the conference quoted from Marshall McLuhan and it piqued my
interest. It said something to the effect, `I am an investigator. I make
probes. I have no fixed position.' This contrasted appealingly with much
of the dogma that, up until that point, I had been obliged to go along
with. So thank you, Marshall McLuhan.

I attended the conference and still recall how inspired I was to learn
about the process of communication. It was pointed out, as I recall,
that communication is not the same thing as expression. We can express
ourselves, but that doesn't mean those who bear witness to our
expression know what we are saying. If we wish to effectively
communicate we need to begin by clarifying for ourselves the message we
intend to get across; then there is the process of encoding that
message, of transmitting the message, receiving the message and finally
decoding it. At any stage in that process there can be disruption
resulting in miscommunication. Whether I read his work at that
conference or later on I am not sure, however I am still influenced by
what Marshall McLuhan had to say in his \emph{The Medium is the
Message}. It was he who coined the term `global village' and readers of
his works these days might be surprised to find how pertinent the points
he makes are in light of our current global crisis.

That period in Kaikohe was the end of my high school education and, once
again, I entered the Rotary speech contest. This time the topic of my
speech was, \emph{Is the purpose of religion to be comforting or
challenging?} I believe I gave a good speech and again reached at least
the semi-finals if not the finals, but the judge scored me down because
I argued the point that religion is supposed to be both: religion ought
to equip us with well-being in order that we can meet the inevitable
challenges of life, or something along those lines. He said my talk
wasn't accurately addressing the topic. Never mind, it gave me some good
experience so that later on when I found myself as the abbot of a
Buddhist monastery I wasn't totally unprepared for speaking in public.

