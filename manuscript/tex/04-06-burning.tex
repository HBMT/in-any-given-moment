\chapter{Burning, Burning, Burning}

Compared to Wat Pah Nanachat, at Wat Doi we were a relatively small
community of about seven monks and two novices. Nobody could speak
English, and that was a good fortune. I already had the basics of the
spoken language down, and now I was obliged to make use of it. During
this Rains Retreat something seemed to click, and I felt like I reached
a level of proficiency which made speaking with Thai people enjoyable,
not merely a struggle; I wasn't having to try so hard. Most days,
someone from the nearby village would drive me the two or three miles to
the thermal spring where, for perhaps an hour, I would soothe my legs.

During this period, the Canadian monk, Tan Tiradhammo, was living with
Tan Ajahn Chah at Wat Pah Pong. That year there were a large number of
junior monks in residence, and many of them were from Bangkok. This
meant that Tan Ajahn Chah regularly offered Dhamma talks spoken in the
Central Thai dialect. Most of the Western monks who had learnt to speak
Thai had learned that dialect; only a few were fluent in the Isaan
dialect. Tan Tiradhammo was thoughtfully sending me audio tapes by post,
and it was a delight to discover how well I could now understand them;
also I was motivated to start to work on translating at least one of
them into English. This was the talk now called,
\emph{Reading The Natural Mind}, and is printed in
\emph{The Collected Teachings}\cite{collected} (Chapter 22, p.237).

The exercise of translating proved particularly rewarding. It required
using my head to access conceptual meaning and word equivalents in both
languages, as well as the heart to sense the essential meaning that the
teacher was putting across. I vividly recall how in that \emph{Reading
The Natural Mind} talk, Tan Ajahn Chah was helpfully pointing out the
difference between the way wise beings and unwise beings relate to
wanting. Awakened beings relate to wanting with clear understanding so
they don't suffer. The rest of us still find identity by clinging to
wanting, and suffer accordingly. That opportunity was another gift.

Part way through that three-month retreat period, the monastery was
threatened with a forest fire. Fortunately Ajahn Koon had had the
foresight to create a firebreak around the vulnerable perimeter of the
monastery. After a lot of energetic sweeping the firebreak clear of
leaves, and skilful extinguishing of fires, the monastery was declared
safe again. Thinking about it later, firefighting struck me as a fitting
metaphor for the spiritual life. When we are on the front line dealing
with the flames, we can't be thinking too much about the bigger picture;
we can't know for sure the overall extent of the fire. However there are
those, our teachers and guides, who do have an overview; they can see
more than we can. It doesn't serve us well to be worrying about whether,
in terms of the bigger picture, we are succeeding in stopping the fire;
sometimes our task is to deal with the flames here and now, right in
front of us, and keep trusting.

It might have begun earlier, but from what I remember, this was the
first time that I registered another type of fire: an intense physical
sensation of burning. Sometimes my whole body felt like it was on fire,
at other times it was just my head. Where did all this heat come from?
Was it because of all the sugar I was consuming? There did seem to be a
correlation between taking a very sweet drink in the evening and shortly
afterwards having disturbingly strong symptoms of sweating and heart
racing. Or was it the kammic consequence of my misspent youth? If so, it
seemed a high price to pay for what, by comparison to others, was a
moderate amount of heedlessness. Maybe it was the result of how
unskilfully I approached meditation in the beginning, without an
appreciation for precepts and restraint. Or was this past life kamma?
Then again, if you trust in the theory of epigenetics\cite{epigenetics},
perhaps it was related to how my ancestors conducted themselves?

I definitely didn't know what it was. It took a very long time -- and
here I mean years, not weeks or months -- to even begin to learn that
right practice meant training the whole body-mind to be able to simply
receive such sensations of being on fire, along with the not-knowing
state, and let it be. Trying to get rid of it, or to get over it -- and
often our attempts to understand are another sort of trying to get rid
of it -- only provides fuel which it feeds on.

It was during that period that one night I rolled over in my sleep and
landed on a scorpion. Understandably the scorpion stung me. I sat bolt
upright, putting my hand around to my back to find what had happened;
the critter must have taken that as another threat so it then stung my
hand. This was about 1 or 2 o'clock in the morning. I had received bites
from stinging ants before which were nasty, but this was my first
encounter with a scorpion. I was only guessing it was a scorpion as I
couldn't see anything. My heart was racing as was my mind: is there
anything I should do about it? The monastery I was in was a long way
away from any significant medical facility. Is there a chance I might
die? Should I be reciting \emph{Buddho, Buddho}? After some time the
pain subsided and I probably ended up going back to sleep. The next day,
back in my kuti after morning alms-round, I reached for a book that was
on a ledge above my bed and, just in time, I saw there was a scorpion
heading for my hand. I reacted quickly by throwing the book out the
window. I did feel bad about having possibly caused damage to the book;
but when I went outside and picked it up, I didn't feel terribly bad on
seeing it had landed in a way that meant the scorpion had been squashed.
My level of compassion was still not very well-developed.

