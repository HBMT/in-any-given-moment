\chapter{First Encounter with the Forest Sangha}

It was that meeting with Ajahn Sumedho that motivated me to venture out
of Bangkok up to Wat Nong Pah Pong, near Ubon, in the North East of
Thailand.

After a long and slow train ride I arrived at the monastery at a time
when the community were still out on morning alms-round. Just as I can't
remember much about arriving at the Narada commune in Australia, my
memories of arriving at Wat Pah Pong are very hazy. I was left though
with an impression that the sangha at Wat Pah Pong were not in the
business of trying to attract newcomers. The first Westerner I met
looked at me and said something like, `What are you doing here?' My
accommodation was a thin grass mat and a blanket in a corner of the
meeting hall (\emph{sala}) and no mosquito net. The linoleum-covered
concrete flooring was a far cry from the plush red carpets of the temple
in Bangkok. Not that I was overly enamoured of those plush red carpets,
but this was towards the other end of the spectrum. Perhaps the
reception I received had something to do with my wearing the robes worn
in city monasteries, combined with my only being a samanera and
belonging to the Dhammayutta Sect. I had never been in the military, but
this was how I would imagine it felt to be in boot camp.

I was told that Tan Ajahn Chah's mother had died not long before, so
regrettably he was not there during my stay. It was a relief to meet
Ajahn Sumedho again and he was ebullient as ever; and of course it was
good to meet up with Nane Dhammiko. Some of the other Westerners were
veterans from Vietnam who would have known about real boot camp.

The daily routine began with the morning bell being rung at 3am. I had
to quickly gather up my bedding and store it away before morning
chanting. Here chanting was a lot longer than what I had become used to,
as each line was recited in both Pali and Thai. After chanting we sat
together in meditation for a painfully long period, before heading out
on alms-round.

For the once-a-day meal, we sat in two long lines; I was almost at the
very end with a group of young boy-novices below me. From the food we
had received when we went out on \emph{pindapat} (alms-round), we were
allowed to keep as much rice as we wanted and the rest was given over to
the community. Here it was the glutinous `sticky rice' variety and
everyone would make a large ball, perhaps four or six inches in
diameter, and place it in the centre of their bowl. Several senior monks
would then walk down the line and ladle a variety of curries, pieces of
vegetable and, maybe fruit, into our bowls.

In the afternoon, after sweeping leaves and hauling water, the second
chanting and meditation session began at 3 pm, and lasted for about one
hour. From time to time, the silence of the sitting was interrupted by
the eerie screech of a tokay gecko\cite{gecko} --
\emph{tokay, tokay, tokay, tokay, tok, tok, tok.}
When the rest of the sangha all departed, the
Westerners remained in the \emph{sala} for another four hours,
alternating sitting and walking meditation. Sometimes we would be
invited to gather around Ajahn Sumedho and listen to a reflection on the
teachings.

Those Dhamma teachings were what I was looking for, however, I was
having grave doubts about my ability to hack the austerities. The
haziness of my memory around arriving at Wat Pah Pong extends to the
full four or five days I was there. I do, however, recall that Nane
Dhammiko's travelling companion, Randal (I forget his Pali name), was
seriously unwell, and the only medicine he was receiving was a thermos
of hot water in the evening which he could mix with a spoonful of
Marmite. Whatever thoughts I might have had about my first visit to Wat
Pah Pong, the feeling was one of intimidation. Several of the Western
monks struck me as disturbed to the point of being frightening. One of
them in particular was extra scary, and what made it worse was that he
was being held up as an exemplar of mindfulness and restraint!

Did it really have to feel this heavy? Was there something wrong with me
that meant I found it heavy? Was I going to be able to do it? I
desperately wanted to be able to do it. Maybe this experience was
helping me get just a little bit clearer about what it was I was looking
for: it wasn't the form. It wasn't about another religious tradition or
organization. One distinct impression I do still have is that when it
came time to leave, I said that I looked forward to coming back: that
might not have been completely true.

On the train trip to Bangkok I managed to get an object stuck in my eye.
It was extremely uncomfortable. After a troubled night it was obvious
that I needed medical attention. Almost the whole day was spent riding
around Bangkok attempting to find the right person with the right
equipment who was available to help. The offending object was a small
piece of metal which, by that time, had already started to go rusty.
Eventually we found the right person and, with my head clamped into a
device that made sure I didn't move, the eye specialist carefully drove
a magnetic needle directly into the eyeball, gently extracting the
slither of metal. This incident fittingly symbolized the long and
painful process I had ahead of me of extracting many offending mental
and emotional objects from my mind.

Despite that disappointing first encounter with the forest sangha, I
don't believe I was put off at all. A facility for burying
disappointments was part of my character. Also I did have faith in the
practice. Perhaps there were other ways of approaching the meditation
monk's life. For the time being anyway, it felt good to settle back into
the less austere surroundings of Wat Boworn. The noise of the rowdy
tuk-tuk taxis driving by outside the monastery were tolerable, as were
the smells of food being cooked in the evening just on the other side of
the monastery walls. I persisted in trying to learn the Thai language.
An elderly monk who used to be a school teacher attempted to instruct me
in reading and writing, but I really couldn't get my head around it at
all. When it came to repeating words that I heard, I was quite quick in
picking them up; and since Thai, like Chinese, is a tonal language, that
makes all the difference. I have met several Westerners who have an
impressive grasp of the written Thai language, but they haven't managed
to get the significance of the tones. To a Thai person, even one who is
well educated, they can have trouble imagining what a Westerner is
trying to say unless the tones are accurate. One word in Thai, such as
`kaaw' (pronounced cow in English) can have five entirely different
meanings according to the tone -- high, low, rising, falling and neutral
-- and the length of the vowel sound.

In April 1975 both Cambodia\cite{fall-of-phnom} and Vietnam\cite{fall-of-saigon}
fell to the Communists. Presumably this news caused trepidation
in my parents in New Zealand. I was periodically sending them an
aerogramme; from their perspective, my predicament must have seemed
dangerous. Even though Bangkok is not that far from Phnom Penh and
Saigon, the general population in Thailand appeared to be getting on
with business much as usual. Perhaps if I had been able to communicate
in Thai more competently I would have had a different impression. It
does seem though that the Thai way is to try to avoid making any sort of
problem out of anything unless it seems absolutely necessary. One of the
first expressions everyone who visits Thailand learns is, \emph{mai ben
rai}, which translates as `never mind, it is not a problem'.

