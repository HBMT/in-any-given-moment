\chapter{Different Perspectives}

Bangkok used to be known as the Venice of the East because of the
network of canals that criss-crossed the city. In the first half of the
last century, in an attempt to eradicate malaria, many of the canals
were filled in and replaced with roads. There were still two canals
which passed through Wat Boworn and there were still plenty of
mosquitoes, although I believe malaria was no longer a big concern. It
was a delight to watch the many turtles that populated those canals. It
wasn't such a delight to one day notice an enormous snake that appeared
just outside my accommodation. I would estimate it was at least eighteen
to twenty feet long and sixteen to twenty inches in circumference.
Nobody else made much of it. There seemed to be a belief that such
snakes were connected to the spirit world and harming one would bring
misfortune. Indeed, it almost seemed as if the snake was considered
venerable. It was generally understood that this massive reptile had
been resident in the monastery for a very long time (somehow living in
the canal) and would emerge from time to time to catch a dog.

There were plenty of dogs, most of them with the mange. Where I was
raised, dogs were used to help round up the sheep or were pampered pets,
in both cases much appreciated and cared for. Here there were dozens of
dogs and nobody appeared to own them, almost nobody really cared for
them; they just lived there with their mange. I heard somewhere that
Thai people would rather deposit unwanted puppies at the local monastery
than put them in a sack with some stones and throw them in the river.
Seeing so many dogs looking so wretched wasn't pleasant, but the thought
of heartlessly drowning them was much worse. The monks and temple boys
did feed them though with leftover alms-food. Temple boys were children,
or teenagers, who lived in the monastery and helped with cleaning and
running errands for the monks. In many cases they were from poor
families up-country and were there primarily to receive an education.

It took quite a while before I was able to go on alms-round in the
morning without feeling afraid that my robe would fall off or my
alms-bowl might slip out of my hands. The faithful lay folk, who got up
early in the morning to reverentially place food in our bowls,
considered it a privilege to have a monk to whom they could make
offerings. Although, as recipients, we would acknowledge their gifts
with words in Thai or the traditional language of Theravada Buddhism,
Pali, we weren't saying thank you. To say thank you would risk offending
the donor since that would imply that we assumed their offering was made
to please us. In fact, what they were doing was being intentionally
generous so as to build up their own storehouse of wholesome potential,
known in Thai as \emph{boon} or in Pali as \emph{puñña}. Words or verses
that we recited in response to their offerings were forms of expressing
\emph{anumodana,} which means, `I acknowledge and celebrate your good
efforts; may this act bring you true benefit'.

There is an ingrained appreciation in Thai culture of the Buddha's
teaching on developing goodness so as to eventually reach Awakening. To
think one could reach Awakening without having already accumulated a
wealth of wholesomeness would be the height of folly. One of the easiest
ways of generating goodness was through acts of generosity. Hence the
ubiquitous spirit of giving and receiving throughout Thai culture. It
would be almost unheard-of for someone to visit a monastery without
taking something to offer. It didn't have to be a substantial sum of
money or an impressive object; what mattered was the spirit of giving.
Often offerings were totally mundane items such as cleaning products and
postage stamps. Here I was beginning to see that there was another
language besides Thai that I needed to learn. Perhaps this was how
people communicated with each other in spiritual communities. Religious
forms obviously have their place, but the spirit profoundly matters.

On one occasion, early on, I recall my assumptions about giving and
receiving being challenged in a very beautiful way. It involved
witnessing an exchange between three women. It began with one woman
presenting a gift of a bag of oranges to another woman. What pulled me
up short was seeing how the woman who received the gift, immediately
separated out a substantial portion of it, and then, right in front of
the woman who had given the gift, passed that portion on to her friend
sitting next to her. `Surely the woman who gave the gift of a bag of
oranges is going to be upset on seeing that what she just gave was being
given away' I thought, but no: the original giver was genuinely
delighted. Not only did she have a chance to accumulate goodness by
giving a gift to her friend, she also gave an opportunity for her friend
to herself perform an act of goodness. This, I learned later, is what
the Buddha referred to as \emph{mudita}, or taking delight in the
well-being of others. I was very familiar with the opposite of mudita,
which is jealousy, but this was my first lesson in the virtue of
vicarious delight in the wholesomeness of others.

There was so much to learn. Some lessons were exquisite, such as this
one of witnessing mudita. Others were hard work. It wasn't obvious to me
at the time, though I can see now that what I was having to adjust to
was not being in control. Not unusually, I imagine, I was used to
operating on the assumption that it was all up to me: I had to strive
and struggle to get what I wanted out of life, to become who I wanted to
be in life, or I would fail. It was an intensely assertive and tiring
way of relating to life. Now, as a samanera, I was having to let go of
excessive striving and struggling and learn how to graciously receive
what was offered. When we walked on alms-round we were not allowed to
ask for offerings. We could stand and wait, but never ask; `eyes cast
down at plough's length ahead'. The same principle applied to other
material requisites such as robes, shelter and medicines. Certainly I
had my preferences though the training made it clear that we were to
accept whatever requisites we were given, without complaint.

This encouragement to patiently wait and receive, was contrasted with
the principle of giving and serving (though at the time I wasn't
thinking in terms of contrasting principles; everything was so new).
Every evening at chanting, we recited the words, `I am a servant of the
Buddha, I am a servant of the Dhamma, I am a servant of the Sangha'. We
were called to offer ourselves completely -- physically, mentally,
emotionally -- in service to the Triple Gem. In practice this might mean
turning up on time for the chanting whether we wanted to or not. It
could mean going out barefoot on alms-round every morning regardless of
how we felt about the filthy streets of Bangkok or what the weather was
like. Sometimes it meant accompanying an Elder who needed a chaperone
while he spoke with a supporter seeking advice with difficulties in his
or her life. On a more subtle level, it related to how fully we were
able to give ourselves into formal meditation practice. We were expected
to be able to sit upright and still for long periods of time, and
without a cushion, since cushions were only for placing the head on.

Making such an offering, of our whole body-mind, regularly involved
enduring a considerable amount of pain. However, this wasn't
masochistic; this was about learning to find the point of balance
between being assertive and yielding. Most of us are out of balance in
one direction or another, either too assertive or too yielding. Finding
that point of balance required mindfulness, concentration and a great
deal of patient endurance. Years later I was introduced to the Qi Gong
exercise of `pushing hands' which involves two people standing opposite
each other, literally pushing each other's hands, leaning forward and
leaning back. It is an excellent exercise for embodying this principle
of balance.

The changes I was having to make were considerable but I was enormously
glad to be there doing what I was doing. From what I heard later on, my
family back home in New Zealand were not so glad. Of course my decision
to take robes in the Buddhist religion and live `the homeless life' was
going to be hard for them to accept. Perhaps if we had had Skype and
FaceTime back then I might have been able to manage it more
symathetically and have caused less upset. In terms of what I was
experiencing though, for the first time I felt like I had discovered
something I genuinely wanted to be doing. Yes, it was challenging, but
these were challenges that not only was I willing to face, I truly
wanted to.

In a matter of only a very few weeks I had gone from being a confused,
disoriented backpacker wandering through South-east Asia, to being an
alms-mendicant, living on offerings made out of faith in the Buddha's
teachings. How did that happen? I don't know how it happened, but I am
very grateful. I don't think it an exaggeration to say that a deep
process of healing had begun. The words `wounded' and `healing' are
rather hackneyed these days, however sometimes there are no other words
that fit. The wounding I had suffered by betraying myself, by not acting
and speaking honestly, had significant consequences. The medicine
prescribed by the Buddha -- integrity, tranquillity, and insight -- were
the remedy.

Still I was struggling to regain the clarity and calm I had known during
the period immediately following the retreat in Australia. At one point
I think I even wrote to my friends at Narada asking that they post me
the copy of Alan Watts' book, `\emph{Nature, Man and Woman}', hoping to
find that place of inspiration again. However, even though I didn't have
the joy and confidence of those earlier months, what I did have, and
what I was so glad for, was a sense of being part of a community that I
respected. With hindsight, I now see it provided a subtle but
significant feeling of belonging.

There wasn't anybody at Wat Boworn with whom I was friendly on a
personal basis. Bruce had left after only a short while, presumably to
return to America. Also, at the time there were a number of
English-speaking monks who were getting ready to disrobe. They had a lot
to say about what was wrong with this or that teacher, this or that
monastery. From where I was at, they just sounded like they were missing
the point. Probably they saw me as a wimp who didn't know what he was
getting himself in to. For me, it was enough to feel a part of something
that encouraged me to develop wholesomeness and to let go of
self-centredness. Going on daily alms-round was a religious ritual with
meaning. Regularly joining in chanting was providing structure to my
life that was relevant. Even having our names called out at the end of
puja\emph{,} to make sure we were all there, and having to reply in
Pali, was fine by me -- \emph{akato bhante}.

There were other English-speaking monks around who were not suffering
from disillusionment. One evening, as I was coming out of puja, I met
two young American monks who were visiting from Wat Nong Pah Pong, that
monastery up-country where Bill and Randal had gone. They were Tan
Pabakharo and Tan Anando. I can still see them standing there: upright,
clear faced, neither showing off nor hiding, just standing there in
their dark brown forest monk's robes (city-dwelling monks usually wore
brighter coloured robes). There was an air of impeccability to them. How
appealing! They were staying at another temple nearby, Wat Saket. I
can't remember what we spoke about. I don't recall why they were in
Bangkok or why they were at Wat Boworn that evening, but I am very glad
to have met them. These days they occupy a vivid place in my memory
alongside that `Awareness' poster.

Then there was my meeting with Tan Varapañño. Tan Varapañño, later
well-known by his lay name Paul Breiter, was a prolific translator of
Tan Ajahn Chah's teachings, and one day he dropped by Wat Boworn. Maybe
he was even staying there, though I doubt that since Tan Ajahn Chah's
monastery belongs to the Mahanikaya Sect and Wat Boworn belongs to the
Dhammayutta Sect, and usually the monks didn't mix. I took the
opportunity to ask Tan Varapañño about Tan Ajahn Chah's style of
teaching. There were many teachers and many monasteries in Thailand and
it was becoming clear that my staying in this rather noisy city
monastery was probably not the best choice if I wanted to develop
meditation practice. I was interested in knowing in which monastery, and
with which teacher, it would be most useful to spend time. One of the
questions occupying my mind then was regarding the correct understanding
of the Buddha's teachings on the first factor of the Eightfold Path. So
I asked Tan Varapañño how Tan Ajahn Chah taught about \emph{sammaditthi}
or right view. His reply was brilliant. He explained that Tan Ajahn Chah
pointed to two aspects of right view: there were the views themselves,
but then there was the way we related to those views. Even the Buddha's
teachings on right view, that is the Four Noble Truths, were wrong view
if we were clinging to them. Wow! That was different. It wasn't only
about having the right view regarding reality that mattered -- it also
mattered how we held those views. Thank you, Tan Varapañño. Approaching
practice from such a perspective highlighted the danger in clinging to
views and opinions -- a risky territory for all organized religions.

As I adjusted to this new life there was one issue that was constantly
troubling me: food. It wasn't so much the fact that Thais were fond of
eating meat. As monks we were generally obliged to accept whatever food
was offered us but we weren't obliged to eat it. (I say generally,
because there is a rule that states, if a monk sees or suspects that an
animal has been specifically killed to feed him, then he is not
permitted to accept it.) The thing that was causing me difficulty was my
digestion. Ever since that bout of illness in Bali, my stomach had not
been right. Although the food was so considerately and generously
offered, much of it simply didn't agree with my condition. Add to that
my often feeling anxious about upsetting my hosts and the fact that I
was greedy, and consequently mealtimes were intense and uncomfortable.
The bio-flora in my intestines seemed to be struggling to handle the
daily onslaught of rich, oily and spicy foods. Occasionally while I was
staying at Wat Boworn, Her Majesty Queen Sirikit would invite the entire
community to a vegetarian meal. That was indeed welcome and encouraging,
but rare. The food issue was a struggle throughout my time in Thailand
and for a good many years afterwards.

At some stage, John from Sydney, who had taken me for my first visit to
Wat Buddharangsee, arrived at Wat Boworn. I think he had aspirations for
joining the sangha. However, like Bruce, his sojourn in Thailand was
brief. He did stay long enough to take me to meet John Blofeld, an
Englishman who had lived many years in China before the revolution, and
now had an impressive traditional Thai house on the outskirts of
Bangkok. Well, it used to be on the outskirts when it was built, but
these days there were houses and factories surrounding it. Meeting John
Blofeld was like meeting a perfect combination of an English and Chinese
gentleman; such dignity and such modesty. Clearly he had accomplished a
lot in his life, but talking with him you wouldn't know it. It was only
after I left and eventually read his book
\href{https://www.shambhala.com/the-wheel-of-life.html}{\emph{\underline{The
Wheel of Life}}} {[}27{]}, that I became aware of what an extraordinary
life he had led. He showed us the original \emph{tanka} painting that
features on the cover of that book. Although I wasn't aware of it when
we met, in his book he describes his visits with the great Chinese
meditation \href{https://www.emptycloud.net/}{\underline{Master Hsu
Yun}} {[}28{]} who, some years later, was to have a significant
influence on my practice.

John from Sydney also wrote into a notebook I had, the treatise known
as, \emph{On Trusting In Mind}, originally written by the great Chinese
Master Tsen Tsan. There weren't many texts at the time that I had read
that truly spoke to me, but this one did. Although these days that
notebook is falling apart, I still have it with me.

Another short but significant text that spoke to me was,
\href{https://forestsangha.org/teachings/books/the-collected-teachings-of-ajahn-chah-single-volume?language=English}{\emph{\underline{Fragments
of a Teaching and Questions and Answers with Ajahn Chah}}}
{[}29{]}\emph{.} This was a small booklet of translations of Tan Ajahn
Chah, compiled, I believe, by Jack Kornfield. What I recall in
particular from that booklet was how Tan Ajahn Chah taught regarding
doubt. He didn't make it into a problem, he used it as an object of
contemplation. Little by little I was coming to recognize that Tan Ajahn
Chah's approach to practising Dhamma was much more here-and-now and
applicable and less theoretical. Although my samanera Preceptor was an
abbot in the Dhammayutta Sect and I was part of that tradition, I was
feeling inspired by the community living under Tan Ajahn Chah.

The non-Thai monks and novices at Wat Boworn were accommodated together
in a building called \emph{Ganna Soung}, which means the Tall Section.
One day, around the beginning of the year 1975, there was a knock on the
door to my ground floor room at \emph{Ganna Soung}. Standing there was a
leaner version of someone I ought to recognize; it was Samanera
Dhammiko, previously known as Bill Hamilton. Nane Dhammiko, as he was
now called, was staying nearby at Wat Saket accompanying Ajahn Sumedho,
the senior western monk living at Wat Pah Pong. Ajahn Sumedho had a
chronic medical condition stemming from an old injury he had received
during his time in the Peace Corps in Borneo some years earlier. The
injury meant that his left foot would sometimes swell up dangerously,
and he was in Bangkok to see if there was anything that could be done
about it. Nane Dhammiko was keen on my meeting Ajahn Sumedho so we
agreed I would come around to Wat Saket.

In terms of significance, that meeting with Ajahn Sumedho, and the
conversation we had, comes right up there next to that first meditation
retreat at Nimbin. In my book,
\href{https://forestsangha.org/teachings/books/alert-to-the-needs-of-the-journey?language=English}{\emph{\underline{Alert
To The Needs Of The Journey}}} {[}30{]}\emph{,} I wrote the following,

`On the first occasion of my meeting Ajahn Sumedho, I was struck by the
simple, but beautiful way in which he was able to say no to a second cup
of coffee. That sounds like a small and insignificant thing, but it left
a vivid and meaningful impression on me. We had enjoyed an initial cup
together, and then his attendant, Nane Dhammiko, offered him a second
cup. Somehow, he seemed able to say `No' in a way that I had never
witnessed before. His manner wasn't that of a self-conscious somebody
doing something special to get somewhere, which was probably what I
would have expected from those living the religious life. It was a plain
and simple `No, thank you'. It was new and delightful to meet someone
with both a sense of humour and clear discipline. I had known people who
were fun to be with but not particularly principled. And I had known
those who were seriously disciplined, but not much fun. Here was someone
who appeared able to honour a commitment to spiritual training, but
without denying life. Here was the result of wise cultivation. Later,
when I met Tan Ajahn Chah, I found that he too had both infectious
laughter and an evidently sincere commitment to discipline.'

