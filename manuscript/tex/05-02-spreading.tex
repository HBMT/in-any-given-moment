\chapter{Spreading the Word}

By the time I went back down to Auckland, Bhikshu Ham Wol had prepared
an itinerary for us that included Hamilton, Palmerston North and
Wellington. I do remember that we hitchhiked at least some of the way to
Palmerston North where my good friend Jutta had arranged accommodation
for us with a colleague of hers from the school where she taught. This
acquaintance had a large house with a separate building out the
back that Bhikshu Ham Wol and I were invited to use. She also had a
swimming pool which must have been tempting, but perhaps by that time
the weather was already turning into autumn.

Once we arrived in Wellington we were again hosted at a Tibetan Buddhist
Centre associated with the same group as the folk in Auckland. They were
likewise gracious in their hospitality, support and interest.

Bhikshu Ham Wol said there was one person in particular in Wellington
that he wanted me to meet; everyone knew her as Aunty Mabel, but her
full name was Mabel San Nyein (or her Burmese name, Daw Aye Myint).
Aunty Mabel owned the Monsoon restaurant in Upper Willis Street,
opposite a very colourful shop called `The Merchant Adventurers of Narnia'.
As was later explained to me, when Aunty Mabel and her four children had
left Burma during a period of political unrest, they managed to make
their way to New Zealand to reunite with Aunty Mabel's brother and his
children, but with only the minimal amount of funds allowed by the
Burmese government. Aunty Mabel was well known to some of the folk who
ran the `Narnia'; she would regularly invite them to her house for a
meal. They had a lease on a property which they used as a warehouse but
which was no longer large enough. Zeke, one of the partners in the
`Narnia', apparently encouraged Aunty to open a restaurant and take over
the lease, probably in part so they would have ready access to her food.

Immediately upon our entering the Monsoon, it was obvious that Aunty
Mabel was a woman of considerable faith in the Buddha, Dhamma and
Sangha. She hadn't seen a Theravadin Buddhist monk for a long time and
with joyous self abandon, threw herself down on the floor and made
prostrations. I was very moved by her devotion. From that time onwards
she became a devoted supporter. I soon learned the Burmese words for
someone of her standing in the community and henceforth referred to her
as Da Ga Ma Gyi (Burmese for devoted supporter of the sangha).

Buddhists expressing devotion was not unfamiliar to me from my time in
Thailand, but somehow it had always felt as if it was filtered through
layers of cultural assumption which I didn't really understand. Aunty
Mabel had been brought up speaking English in British-ruled Burma, which
meant we could speak in a much more open, unfiltered manner; and here we
were in my home country. I think on that first meeting we were offered
the midday meal, and I might have let it slip how impressed I was with
the pumpkin curry and coconut rice. For years afterwards, we were served
pumpkin curry and coconut rice and I didn't mind at all.

One day, as Bhikshu Ham Wol and I walked through downtown Wellington, we
were approached by a photographer. He politely asked if he could take
our photograph and we said it would be fine. That photograph ended up
being printed in \emph{The Dominion}, a leading newspaper in New
Zealand's capital city. Not long afterwards, a phone call came through
for me at the Tibetan Buddhist Centre from the Royal Thai Embassy,
enquiring if the two of us would be available to receive a meal
offering. As it happened, also staying at that Centre on that occasion,
was an Australian man who was training as either a novice or a monk in
the Tibetan tradition, so in the end the three of us, in our very
different robes, attended the invitation.

Typical of the way Thai people enjoy sharing opportunities to accumulate
goodness, a large group of Buddhists from a variety of countries had
been invited. I believe it was there that I first met Mrs Parker, a Thai
woman married to a New Zealander. I also met Mrs Gurusinghe, a Sri
Lankan woman married to Dr. Gurusinghe who, along with their children, had
settled in the Wellington area.

Perhaps as soon as the next day, Mrs Gurusinghe visited me at the
Tibetan Centre and was asking quite a lot of questions. Only later did I
come to appreciate that she wanted to see what sort of Buddhist monk I
was. She would have been aware from having grown up in Sri Lanka, that
some monks are more committed than others to observing their rules and
practising with sincerity. It was those three women -- Aunty, Mrs Parker
and Mrs Gurusinghe -- who formed the core group of what later came to be
known as the WTBA, the Wellington Theravada Buddhist Association. When I
think of them now, the word that comes to mind is `formidable': such
commitment, strength and determination, combined with a wholesome, happy
disposition. The WTBA managed the purchase and eventually saw to the
building of Bodhinyanarama Monastery\cite{bodhinyanarama}.

As I met more monks and novices from other Buddhist cultures, along with
their lay supporters, I was beginning to appreciate that there was
another kind of translation taking place -- not just that of the texts,
but of the traditions. It became apparent that I myself had picked up a
set of assumptions during the years spent in Thailand, and I felt as if
there was a `right way' to do things: a right way to bow, to chant, to
make offerings to the shrine, to make offerings to the sangha. It
shouldn't have come as a surprise to discover that each country had
their own set of assumptions. It was going to require mindfulness and
probably patience to discern the essence of these traditional practices;
in the process I would hopefully learn how to exercise skill in their
observance.

One matter that had already become clear was the importance of
Theravadins and Mahayanists making an effort to get along together. The
last thing the world needed was another religion with its members
squabbling with each other. It was one thing to speak about each other
in less than complimentary terms when we lived thousands of miles apart,
but now, because of the ease of travel, we would be regularly
encountering each other. This situation wasn't totally new; records show
that in the time of Nalanda University in India, various different sects
of Buddhism managed to live harmoniously in close association with each
other. Now we had the potential advantage of access to technology which,
if we were wise, could help us with the task.

Of the roughly six months I spent in New Zealand, it was at about this
point that my personal lack of restraint meant I had once again become
hooked on smoking cigarettes. It is common in Thailand to see monks
smoking; however in Tan Ajahn Chah's monasteries it was not permitted. I
heard that he said he didn't want the faithful lay supporters spending
the little money they had on purchasing tobacco to offer to the sangha.
At that time there wasn't the same amount of information available about
health hazards associated with smoking. Partly I had resumed my old
habit of smoking in an attempt to handle feelings of anxiety that I was
having; and probably rationalized going against the Wat Pah Pong
standard because I wasn't staying in a Wat Pah Pong monastery. It was a
humiliating situation that I had got myself into. I wasn't smoking in
public and that meant there was the added sense of being dishonest.
Attempts to wilfully stop the habit had failed. Eventually I decided to
draw upon the strength of making a vow -- \emph{adhitthana}. Kneeling in
front of the shrine, I made the formal determination, `So long as I am
staying in New Zealand I will not smoke cigarettes.' I expect I added,
`May the Buddha, Dhamma and Sangha bear witness to my determination.'
After that there was very little or no struggle at all. I never smoked
in New Zealand again.

This was a real eye opener for me. During the early years of training in
Thailand, we were strongly encouraged to develop the spiritual muscle of
\emph{adhitthana}, but now I was seeing the benefit. Some of the monks,
on hearing that encouragement to make a special effort for the sake of
strengthening this ability, decided to take up the daunting practice of
not lying down to sleep, or strict adherence to the practice of eating
only food received on alms-round. I was cautious to not make vows that I
thought I might not be able to keep, so I would generally decide on such
practices as sweeping out my kuti every day. Even that minimal level of
resolution I could forget; when that happened and I was already in bed
before remembering my resolve, I would have to get up again and sweep
out my kuti. In the process, I was training my mind with the perception
that if I determined to do something, I would honour that resolve. At
the time of performing such practices it is not necessarily obvious that
a gradual accumulation of increased ability is taking place. I am very
thankful indeed that the teachings highlighted the benefit of training
in \emph{adhitthana parami}.

Maybe it was on our way back up to Auckland that Bhikshu Ham Wol and I
stopped to spend a few days at a Christian Abbey. I'm not quite sure why
we ended up there; perhaps my companion had a certain zeal for spreading
the word. I don't think I was against meeting the monks; at some point
in Thailand I had come across translations of the Desert Fathers\cite{desert}
by the Trappist monk, Thomas Merton, and was very
inspired by them. It was interesting to notice how at this monastery,
during meal times, one community member read loudly from scripture or
commentaries. Maybe this was a way of helping keep their hearts and
minds focused on the spiritual quest and not be distracted by
sensuality. I can sort of appreciate the thinking that was perhaps
behind that. I didn't appreciate, though, their attitude towards raising
stock for slaughter. It struck me as terribly sad that people who were
otherwise committed to goodness didn't see how their attitude towards
animals caused so much suffering.

Our hosts in the nearby town of Napier were once more beautifully
generous and accommodating. They were part of a Zen group there. We also
ended up staying one night at the home of the liberal vicar from my
Gordonton commune days; he was now living at nearby Havelock North.

By the time we arrived back in Auckland, we were thinking about how we
might mark the traditional celebration of the birth, Awakening and final
passing of the Buddha. The Awakening is universally celebrated within
all Buddhist traditions on the full moon of May, or \emph{Vesakha}, as
it is called in the Pali language. Only the Theravadins observe all
three on the same day. Luang Por Mahasamai, from Wat Buddharangsi in
Sydney, had accepted an invitation to join us, so we organised an event
that included members of the Thai, Cambodian, Sri Lankan, Burmese and
Laotian communities.

The gathering took place at Patsy's house in Parnell. One of the things
I remember from it was the shared sense of enthusiasm. Either at the
time, or shortly afterwards, some of the Sri Lankans there on that day
began discussing how they might set up a group that would look into
establishing a Theravada Centre in Auckland. Yet again, the level of
excitement and energy, and the harmony between the groups, were a
delight to behold. That occasion was the beginning of what later became
the ATBA, the Auckland Theravada Buddhist Association. It was this group
that bought and developed a property on Harris Street, known as the
Auckland Vihara\cite{auckland}. After many years and a huge amount of hard work, the
same group went on to purchase and develop Vimutti Monastery\cite{vimutti},
just south of Auckland.

The months were passing by, and soon it would be time to decide where I
would be spending the three months of the Rains Retreat. My parents let
me know they were not happy with the idea of my returning to Thailand. I
had told them about the developments in Britain, and they said they
would offer me a ticket to go there if I wished. After some
correspondence with Ajahn Sumedho, I think I got the impression that
they could use help with painting and decorating the large Victorian
mansion into which they had recently moved.

There are two routes by which one might travel from New Zealand to
Britain: westwards, via Singapore, for instance, and eastwards, via Los
Angeles. Since my friend Mason Hamilton was in those days living in Los
Angeles, I elected to go via LA. He helpfully wrote a letter endorsing
my application for a stopover visa in the USA.

On 7th June 1980 I departed New Zealand. The time I had spent there had
given me my first glimpse of what Buddhism in the West might look like.
The potential for benefit seemed very real, and I was leaving with a
feeling of inspiration. Thank you, New Zealand and thank you, Bhikshu
Ham Wol. It would be approximately another ten years before I would
return.

