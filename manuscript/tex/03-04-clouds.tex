\chapter{Dark Clouds Descending}

Travelling further west, from Bali to Java, there was an undeniable
culture change. The population was still Indonesian; however, whereas
Bali was predominantly Hindu, Java was distinctly Muslim. Java felt like
a different country. Whilst staying in Surabaya, Ross decided he had
enough of travelling and began the journey back to Mullumbimby. From now
on I would be on my own. Ross was never really into meditation, which
for me was more important than anything else. I liked seeing sights,
sort of, but I suspect the call to the inner life was the stronger pull.
By this stage of my journey it was dawning on me that I was not a
traveller. However, since I was in Indonesia, and there was no pressure
to be elsewhere, I decided I may as well look around.

For a brief while in Java I did toy with doing batik painting, but I
don't recall if the fantasy ever progressed beyond a conversation with
someone in Yogyakarta. While staying in Yogyakarta I checked out the
Dieng Plateau, from which it is possible to view both the Java Sea to
the north and the Indian Ocean to the south. I visited
\underline{\href{https://en.wikipedia.org/wiki/Borobudur}{Borobudur}}
{[}20{]}, an ancient, monumental Buddhist stupa originating from a
period prior to the arrival of Islam. The site definitely didn't feel
like a spiritual sanctuary as it had tourists scampering all over it
taking photos. Somewhere I had read that a beach called Parangtritis, a
few miles south of Yogyakarta on the coast of Java, was worth visiting.
Perhaps I had been spoiled growing up in New Zealand because I found the
beach rather uninspiring.

Gradually I was having to admit to myself that travelling around looking
at stuff was boring. What was the point if inwardly you weren't at ease.
I clearly was not at ease. The calm and clarity I had known a few months
earlier, now felt very remote. Not that I had lost faith in it, not at
all, but I had lost sight of it.

I did find people interesting. To my surprise I had managed to pick up a
functional grasp of
\href{https://en.wikipedia.org/wiki/Indonesian_language}{\underline{Bahasa
Indonesia}} {[}21{]} which meant I was able to enjoy some sense of
connection with many of the folk I met. Most of them were warm and
friendly and did want some connection. I say, `to my surprise', because
I was not very successful at learning French in my third and forth form
years at Morrinsville College. This left me with the impression that I
was without ability when it came to speaking any language other than
English. It would seem that in those days New Zealand's education system
was still modelled on that of the British. Why else were we not learning
Japanese or Chinese? I couldn't see the point in learning French. In my
Report Card for 1965 it is written, `Half-hearted in his approach, Keith
could work harder.'

It similarly surprised me to realize that the people I found least
appealing, and indeed sadder, were fellow travellers. I must have
assumed we were part of some sort of global community and would all view
each other warmly. This assumption showed up just how wet behind the
ears this Kiwi really was. We were all obviously financially better off
than many of the locals, yet that didn't deter the Western travellers
from becoming heated, and sometimes downright insulting, in how they
haggled when making purchases. The interaction between customer and
seller, I discovered, was in that culture a significant part of making
any transaction. Nearly all the buying and selling I witnessed amongst
the Indonesians was always amicable. At times it might have become
animated but there was a civility to the engagement. Unfortunately,
civility often went out the window when it was a certain kind of
backpacker doing the dealing: how sad.

Perhaps the saddest thing I saw during that period was on a train trip
from Yogyakarta to Jakarta. At first I couldn't understand why a young
French couple in our carriage were behaving so rudely; it wasn't just
disrespectful, it was revolting. They were rolling in the aisle of the
carriage, moaning and thrashing about. I think it was only when another
passenger told me they were going
\href{https://americanaddictioncenters.org/heroin-treatment/cold-turkey}{\underline{cold
turkey}} {[}22{]} from a heroin addiction that I began to understand.
They looked educated, affluent, probably experienced in the ways of the
world, yet here they were without anyone to care for them. No welfare
state to rescue them, no friends to support them: how tragic.

Being in Jakarta also triggered sadness in me. For the first time, in
this sprawling metropolis, the disparity between rich and poor became
offensively evident. I had no companions with whom to discuss the
questions and concerns that were flooding my twenty-two-year old mind.
Indignation and sorrow disturbed my already clouded heart. I was
beginning to doubt what I was doing. Without friends, without a clear
direction, and without the inner resource of centredness, I was
flailing.

On the boat ride from Jakarta to Singapore, several of the group of
travellers I was with helped each other cut their hair. We had heard
that the authorities we should expect to meet at immigration wouldn't
even allow you into the country if the length of your hair breached
their strict regulations. The only other thing I knew about Singapore
was that it became part of
\href{https://en.wikipedia.org/wiki/Malaysia_Day}{\underline{Malaysia}}
{[}23{]} on my birthday, 16th September; although that arrangement
didn't last long, with Singapore soon becoming independent. The other
memory I now have of my stay there is of waking up one morning in the
hotel to find everything was flooded. I had a room on the ground floor
and it hadn't occurred to me the night before that I should put my
backpack up on a chair. There had been heavy rains overnight and the
city's wastewater management was shown to be seriously lacking. My old
passport from that time indicates that I left Singapore on 8th September
1974.

Whether hitchhiking was actually an acceptable thing to be doing in
Malaysia or not I don't know; perhaps it was another example of a naive
Kiwi assuming everywhere was similar to his own habitat. It did work
though, and I received rides from the border to Kuala Lumpur and then, a
day later, on to Penang. Penang was charming. My accommodation this time
was a simple grass roof hut right on the beach. Lovely as the location
was though my inner world was becoming darker. Meditation wasn't giving
me any ease, and the distractions that other travellers seemed drawn by
were unappealing. The most radical thing I did there was to treat myself
to a Guinness beer and a slice of cherry cheesecake by way of marking my
twenty-third birthday.

The one proper book I was carrying with me during that journey was
Basho's
\emph{\underline{\href{https://www.goodreads.com/book/show/175626.The_Narrow_Road_to_the_Deep_North_and_Other_Travel_Sketches}{The
Narrow Road to the Deep Nort}h}} {[}24{]}. I loved reading Haiku. Three
lines of text didn't challenge me in the way a large body of text could.
And something in the way Basho used words -- something about how he
writes about wandering and wondering and observing nature -- was
reassuring. Thank you, Basho. Words are so powerful. When we relate to
life primarily from our heads, we readily forget that words are symbols
that human beings have crafted over millennia to communicate a richness
of meaning. More than 300 years after Basho died, the craftsmanship of
the man is still admired and appreciated around the world.

When I was hitchhiking from Penang towards the Malay-Thai border I was
again picked up by a couple of British expats. They were very friendly
and even invited me to stay in their home. Later it became apparent that
they needed a babysitter. I was happy to oblige, though thinking about
it now, I confess it strikes me as daring on their part, almost to the
point of dereliction, for them to abandon their child to someone they
had just picked up hitchhiking. I was grateful though and apparently so
were they. The next day they arranged for me to be driven to the border.
I was dropped on the Malaysian side of the border and walked across into
Thailand.

