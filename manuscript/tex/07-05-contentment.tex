\chapter{Contentment}

\begin{quote}
  While in the midst of those who are troubled,\\
  to remain untroubled\\
  is happiness indeed.

  Dhammapada 198
\end{quote}

Continuing this enquiry into structures that sustain our spiritual
practice and support harmonious community, I would like to consider the
topic of contentment.

In response to a request by the first Buddhist nun, Bhikkhuni
Mahapajapati\cite{mahapajapati}, the Buddha offered a concise eight-point
summary of all of his teachings. Around the middle of this discourse he
stated: \emph{That which leads to contentment is Dhamma and that which
leads to discontentment is not Dhamma}.

Earlier this year I gave a talk titled, Unapologetic Pursuit of
Contentment\cite{pursuit}.
A friend of the monastery warned me
at the time that currently many people equate the pursuit of contentment
with being irresponsible: since the world is in such a precarious state,
contentment is the last thing we should be thinking about -- everyone
ought to be striving to find ways to fix this terrible mess.

In that talk I spoke about awareness as a multidimensional reality, not
a singular thing: just as the ocean can have waves thrashing about on
the surface and at the same time there can be perfect stillness in the
depths. The point I was making was that so long as we believe that the
surface-level of turmoil is all there is, we will continually struggle
and risk ending up in despair. I support the enthusiasm for finding
creative solutions to the current global crisis, but we will only be
successful if we factor in all dimensions of reality, not just the
immediately obvious surface-level. A broad-minded and open-hearted
approach is necessary -- one that supports sustained effort and an
ability to view the situations in which we find ourselves from varying
perspectives.

We are all familiar with surface-level contentment -- the feelings we
experience when we get what we want: when the weather is agreeable, our
health is good, and the friends we are with are telling us nice things.
Such feelings are obviously desirable, however, we know that they don't
last. Our Dhamma teachers tell us that there is another quality of
contentment -- one which does last. This is an aspect of our being with
which those who have looked more deeply are familiar. They know that,
like the ocean, even when conditions on the surface of the mind appear
wild and unruly, at the same time, on a deeper level in their hearts,
there can be peace. To have such a perspective on reality leads to
confidence, or faith, and can be a powerful source of support.

If we focus on merely the surface or material dimension of existence,
then indeed the world is in a terrible mess, and this readily gives rise
to feelings of hopelessness. However, our life is much more than
materiality. Even some scientists and philosophers are these days
talking about `the hard problem of consciousness'. The fact that they
are acknowledging the possibility that there may be more than mere
materiality is significant. Not only are they acknowledging that
potentially there is another dimension that is profoundly relevant to
our lives, they are admitting that they know next to nothing about it. I
find this very hopeful indeed. It is like when someone who has been
suffering with physical pain for a long time, but refuses to go and see
a doctor, eventually comes around to admitting that they need help.

Part of the help that all human beings need is the recognition that the
real source of contentment is to be found within awareness itself -- no
amount of rearranging material conditions can truly protect us from
sorrow, loss and despair. Herein lies the value of the discipline of
attention, or formal meditation. Even an entry-level familiarity with
meditation can acquaint us with hitherto unappreciated inner ability. It
would be a mistake to assume that the benefits of meditation can only be
found after many hours, weeks, or years of practice. After only a few
months of twenty-minutes-a-day, six days a week, or even a lot less,
meditators can begin to taste the benefits. So long as we are unaware of
this potential inner resource, we are likely to keep striving to
rearrange conditions on the surface. As already mentioned, even when we
do manage to make things agreeable for a while on that level, part of us
knows that conditions could change at any moment and become
disagreeable, which in itself is a sort of suffering.

In case there be any doubt, I should emphasise that I am not saying we
can afford to completely ignore the conditions of the outer world and
turn a blind eye to such matters as injustice and abuse of power. What I
want to emphasize is that, since surely we aim to be effective in
addressing issues of injustice and abuse, we have to properly equip
ourselves for the challenge. If we were to take on the challenge of
climbing Mount Everest, we would invest a lot of time and effort in
preparing ourselves in advance. Taking on the challenge of transforming
the suffering of existence into wisdom and compassion, likewise requires
that we are ready for the task.

Before considering some of the skilful ways in which we might equip
ourselves, I would like first to look into why it is that, despite our
longing for freedom from pain and confusion, so many of those who start
out on this journey end up disillusioned. It would be helpful if we
understood some of the hindrances to accessing the resource of deep
contentment.

There is a pattern that regularly shows up in monastic communities
whereby people start out determined to reach the goal of unshakeable
peace, only to return to their old habits of distraction; despite
initial zeal for the practice and gratitude for the opportunity to
commit to the inner work, they forget what it was that motivated them to
begin on this journey. The community they are in has not necessarily
changed, and others are still benefiting from being there, so what is it
that overshadows their wholesome aspirations? Dhammapada verse 344 says,

\begin{quote}
  There are those who have begun\\
  on the path to freedom,\\
  yet run back to being chained once more.
\end{quote}

Recently a young fellow who was visiting our monastery with the
intention to request monastic training asked me if I had any advice on
how he should prepare himself for what lies ahead. My advice was: take
note of your motivation -- the aspiration that you feel right now at the
beginning of the path -- and regularly reflect on it.

\section{Unaddressed Authority Issues}

During the year 2007, I received a warm and friendly message from the
new abbot of Wat Pah Nanachat, Ajahn Kevali, inviting me to visit
whenever I was available. Around the beginning of 2008 I accepted that
invitation and, en route to New Zealand, stopped off in Thailand to meet
Ajahn Kevali and spend time with the sangha at Wat Pah Nanachat. As was
usual when a senior sangha member from another monastery was visiting, I
was invited to take part in a Q\&A session at an evening tea gathering.
One of the questions I was asked on that occasion was, `What would you
say was the number one challenge within our Western sangha?' My reply
was immediate: `Authority issues.' After a good number of years of
observing the dynamics within our various communities, unaddressed
authority issues stand out for me as one of the main obstructions to
harmony and contentment; and disharmony and discontentment are amongst
the main reasons for people leaving. We all value the opportunity to
live in spiritual community, yet somehow we still act in ways that
undermine ourselves. What is behind this?

I am very cautious in what I say on this topic of authority as I know
others have studied the subject more thoroughly than I have. However,
much of what has been said is from the perspective of social psychology:
for example, the consequences of the industrial revolution which led to
`the absent father' and the terribly damaging effect that has had on
boys and young men. Here I wish to reflect more from a spiritual
perspective.

In the early days of developing Chithurst monastery, around 1981- 1982,
Tan Ajahn Chah sent a message from Thailand expressing his well-wishing.
He commented on the importance of working together and helping each
other, and spoke about some of the difficulties he himself had
experienced in the early days of developing Wat Pah Pong. It was a
tape-recorded message and it was a joy to hear him mentioning our names
and, like a caring father, giving us some pointers. The part of that
message that stayed with me, and on which I have often reflected, was
where he said,

`To be the abbot of a monastery can be compared to being a rubbish bin:
those who are disturbed by the presence of rubbish make a bin, in the
hope that people will put their rubbish in there. In actual fact what
happens is the person who makes the bin ends up being the rubbish
collector as well. This is how things are -- it's the same at Wat Pah
Pong, it was the same at the time of the Buddha\ldots. everything gets
chucked in the abbot's bin! One in such a position must therefore be
far-sighted, have depth, and remain unshaken in the midst of all things.
They must be consistent -- able to persevere.'\cite{seeing}

Every abbot of our branch monasteries in the West with whom I have
spoken has developed serious spiritual muscles in the performance of
their duties as a rubbish collector. And the rubbish I am referring to
here comes not just in the form of a novice monk who might be sharing
the difficulties he is having in missing his evening meal; sometimes it
comes in a form that feels more like you have been kicked in the
stomach. There are numerous occasions I can recall from over the years
when suddenly, out of the blue, a community member has thrust upon me an
unexpected amount of their personal pain. An abbot from another
monastery recently mentioned to me that he had given himself a few days
of private retreat in order to reorient himself after being on the
receiving end of strong blame and unfair criticism. Taking some time out
was, of course, the right thing to do; pretending he had not been
affected by the experience would have made things worse.

When the leader of a community feels like he or she has just received a
kicking from someone who offloaded their \emph{dukkha}, it is important
that they understand accurately what has taken place. From the
perspective of the Buddha's teachings, the leader has been given a
message. The Buddha was very clear: \emph{you continue to suffer because
you fail to see two things -- not seeing dukkha and not seeing the cause
of dukkha.} He didn't say that we suffer because someone spoke to us
unkindly. If we feel hurt because someone projects their \emph{dukkha}
onto us, we should (if at all possible) take whatever time is needed to
meet ourselves, to receive ourselves, there, in that very place, in the
whole body-mind. This is the time and place where we learn. Reacting
with counter-accusations of how ungrateful the student is and how he or
she should show more appreciation for all that they have received, is
not being the sort of rubbish bin that Tan Ajahn Chah recommended. It is
not truly taking responsibility for ourselves.

In the process of observing how it feels to be on the receiving end of
someone else's projected negativity, we will hopefully also recognize
that we have a choice: we can feel the pain and react with resistance,
or we feel what we feel, take a long, slow, deep in-breath, and create a
sense of expanded awareness within which that pain can be received, and
witness letting go. In such moments of letting go we might even discover
an unexpected sense of gratitude for the attacker who helped us deepen
in awareness (although that might take a while longer).

So far we have considered such interactions from the perspective of the
one who received the projection. We could also usefully consider why the
projection occurred in the first place. Why is it that some people seem
capable of handling more pressure than others? In my own observation and
experience of what actually takes place when someone blames another for
their suffering, it is because they feel they have more pain than they
can handle -- their ability to take responsibility for the \emph{dukkha}
of life has reached its limit. The point is made perfectly clear when
they say, `I just can't stand it any more'. It is not, however, the
\emph{dukkha} they can't stand, it is the lack of inner space in which
to receive it. Their heart is already choc-a-bloc full of suffering; it
can take only a little bit more before there is an explosion.

Everyone has to deal with \emph{dukkha}, but not everyone understands
that we can cultivate the capacity for dealing with \emph{dukkha}. When
I suggested above that we respond to feelings of being attacked by
taking a long slow deep in-breath, I meant we do something that reminds
us that we are not victims of the experience of limited capacity or
limited awareness. We do whatever we need to do to be able to feel what
we are feeling in the moment. If you are someone who still finds your
identity in the thinking mind, then the long slow deep in-breath might
not be enough, as you will probably be busy in your head trying to
figure out what to do. Instead, what might be needed is a more vigorous
form of physical activity -- something that takes you out of your head
and back into the body where you can feel the intensity and the heat and
learn to accept it.

When one experiences the threat of being overwhelmed by \emph{dukkha},
this contributes to a lack of self-confidence, which in turn leads to a
lack of personal authority. As long as we lack a stable inner sense of
authority, we will struggle. If we perceive ourselves to be under the
authority of someone else, we tend to suspect that those above us are
abusing their power, and we resist, either aggressively or passively. If
we see ourselves as having authority over others, we are susceptible to
feeding on a form of `borrowed' authority that comes with the role. When
our actions are motivated by a sense of borrowed authority, we come
across as untrustworthy, because in fact we are. The various roles that
we play in life are simply social structures that provide us with a
convenient degree of predictability. Those who lack an inner sense of
authority easily fall prey to taking their roles too seriously.

Authority can also be borrowed by clinging to our personality structure
-- to our deluded ego. Instead of training awareness so that we keep
learning to meet ourselves and let go of ourselves, we cling to the
perception of `me' and `my way'. With varying degrees of intensity, this
is the disposition of someone lost in a sense of self-importance. It can
produce an impression of being authoritative, but only those people who
want someone else to take responsibility for their lives will go along
with them. These people project their own potential for inner authority
onto so-called leaders because it makes them feel safe. If the leader no
longer provides them with a sense of security, the projection can
quickly be withdrawn.

Another significant cause for the lack of inner authority is the
compromising of integrity. The mobile telephone I was using earlier this
year had been faulty, probably for as long as I had had it. I was sure
there was a manufacturer's defect, however I couldn't be sure that I
hadn't dropped it. I think I dropped my last phone several times, but
was uncertain about this one. Having put up with poor quality phone
calls for many months, I asked someone to take it into the phone shop
and see about having it repaired. When it was mentioned to the shop
assistant that there was a possibility that the phone had been dropped,
she said that if that was the case then there was no way the
manufacturers would even look at it, but she insisted, in a
matter-of-fact way, you didn't have to tell them it might have been
dropped. She gave the impression that reporting any possibility of the
phone having been dropped was unthinkable. She was confirming what sadly
these days appears to be widely accepted: that it is alright to lie.

What people who habitually lie don't understand is that even if nobody
else knows, they themself know. And just as if we were to catch someone
lying to us, we would no longer trust them, when we know that we lie, we
stop trusting ourselves. Every time we compromise integrity, we
contribute to inner fragmentation -- we weaken inner stability.

We need to commit to cultivating impeccability, not because God will
punish us if we don't, but because when we don't, we undermine
ourselves; we do damage to inner harmony. Often when I speak with
psychiatrists and psychotherapists I look for an opportunity to
encourage them to find ways of supporting their clients in understanding
the value of integrity. I find it truly tragic that strong medication
purporting to help people with their suffering can be prescribed before
any sort of a conversation regarding ethical matters. Of course I
appreciate how difficult such a conversation might be, but to hand out
mind-numbing drugs without raising the matter is also difficult. To me
it is the same as a GP providing a prescription for insulin to help a
patient manage diabetes without offering any advice on dietary matters.
Low self-esteem is a logical consequence of acting and speaking in ways
that are dishonest, and it contributes to a sense of lacking inner
authority.

Conversely, those who are equipped with the self-respect that comes
naturally with impeccable conduct express an authentic sort of
authority. Anyone who keeps the company of those who are committed to
integrity is very fortunate. To live life surrounded by people who think
`everyone lies' is a great misfortune. However, there is no benefit in
dwelling on our misfortunes; as soon as we recognise the link between
the way we conduct ourselves and self-respect and inner authority, we
can resolve to be more careful.

\section{Reshaping of the Self}

As recently as one hundred years ago, the majority of inhabitants of
England periodically engaged in a plethora of spiritual exercises that
helped protect them from the danger of excessive self-importance. On a
regular basis they would shake hands with and feel seen by someone who
they perceived to be in close communication with the Almighty; they sang
rousing hymns of praise and gratitude to the Almighty; listened to
sermons that highlighted virtues such as forgiveness, generosity,
modesty, and humility; and each time the offerings plate was passed down
the pew, they would invest some of their hard-earned money in their own
personal relationship with the Almighty, the Lord, the Omnipotent, that
which is inherently secure and `changeth not'; in so doing their
personality was relativized. They didn't experience themselves as the
centre of existence; they saw themselves as living in relationship with
that which was all-loving and imperturbable. I don't know how far back
records go, but it is probably safe to say that, up until the beginning
of the last century, something like this had been happening in most
countries throughout most of human history. During the last century,
materialism, scientism and humanism (including communism) have changed
all that.

These days, a large percentage of the world's population live their
lives without ever thinking about the importance of instilling virtue in
themselves or in their children. Their sense of self is left dangerously
vulnerable to the powerful forces of delusion. They rarely if ever meet
anyone whose life is committed to serving the transcendent reality. For
many, the mere thought of a transcendent reality is dismissed as being
so passé as to not even warrant consideration. Individualism is their
religion, and sadly, so far, they haven't seen that they are worshipping
a false god.

Although I haven't researched the data, it wouldn't surprise me if
statistics showed that over the last fifty years there has been a steady
increase in use of anxiety medication; one that coincides with the
increasing rate of suicide\cite{suicide}.
What does surprise me is that I almost never hear anyone
talking about the effect that this sudden absence of religious belief
must be having on our sense of self.

We are not born with a sense of self. We are born with undifferentiated
awareness and it takes about seven years to evolve a sense of separate
identity. I sometimes fantasize that if neuroscientists had mapped the
personality part of the brain over the last century, they would have
seen a radical change in shape taking place. And further, if we accept
that the decline in attendance at a place of worship corresponded with a
loss in faith in any overriding principle, I would suggest that that
decline would coincide with the gradual increase in anxiety. With that
increase in fear and confusion comes a decrease in a personal sense of
authentic authority.

Obviously there are other powerful influences that humanity has
undergone over the last century, but the loss of faith in true
principles must have major consequences. (The loss of a sense of being
part of a community would surely count as another such powerful
influence). If questioned, I would assume many people would say that
giving up a belief in God has made them stronger and more confident. But
from what I have observed, the opposite is the truth: the degree to
which people have become ego-centric corresponds to their loss of inner
security. Being self-centred leads to disorientation, which is expressed
in desperate attempts to find a secure identity. The belief that I am
the centre of everything makes my thoughts, my feelings, my preferences,
my opinions, terribly important, even though on some level I know that
my thoughts, feelings, preferences and opinions are continually
changing. No wonder there is such an increase in conspiracy theories and
so-called culture wars, to name just two examples of people's often
frantic struggle to find out who they are.

Of course I am not suggesting that those spiritual exercises in which
the general populace of this country were engaged a hundred years ago
were ultimately beneficial, but I do think the impact they had and the
consequences of their disappearance deserve careful and thorough
investigation. Many of those people who now believe that their current
secular approach to life has liberated them from `primitive' beliefs
seem to me to be wandering around in a wilderness -- lost and homeless.
Neither the beautifully decorated house in which they live, nor the
impressive car that they drive, protects them from feeling threatened by
the current pandemic. Even if there is a return to normal, no amount of
material possessions will provide deep contentment. In Dhammapada verses
288-289 the Buddha says,

\begin{quote}
  As you approach death\\
  none of your fond attachments will protect you.\\
  See this, then, wisely restrained by virtue\\
  and unwavering effort,\\
  hasten to clear your path to liberation.
\end{quote}

The words `\emph{wisely restrained by virtue}' here refer to the
practice of equipping ourselves with the right understanding and
appreciation for such principles as integrity, generosity, forgiveness,
kindness and gratitude. These are what protect an inner sense of
psychological integration. Earlier in this book (Part 6) when we were
considering `Symbols and Rituals' I explained why we had the
\emph{lokapala} (protectors of the world) depicted on the front doors to
our Dhamma hall: those who wish to enter the sanctuary -- the place
where we go to contemplate our deepest concerns -- must have prepared
their hearts and minds with the virtues of \emph{hiri} and
\emph{ottappa}.

\section{Technology is Not the Problem}

Technology is another of those powerful influences that has had a
profound effect on our lives. However, to blame technology for our
discontentment is yet another example of heedless projection. All
technology does is amplify where we are at. We, the users of the gadgets
and devices, are the ones who are responsible for the effects of
technology, including the data presently being churned out by algorithms
over which humans now have limited control. The initial algorithms were
a product of human awareness.

Technology amplifies the consequences of both our foolishness and our
virtue. It is great that I can write this book on gratitude and
spiritual community in only a few months, have it typeset, and
distributed as an ePub, all within one year. It is amazing that an
aspirant for acceptance as a trainee in the monastery can turn up
already prepared with questions that arose out of their having listened
to Dhamma talks that he or she had downloaded from the internet. Also it
is terrifying how easy it is for one disturbed individual to acquire a
following of likewise disturbed individuals and, in no time at all,
start a movement which is equipped with deadly weaponry.

Whether or not technology contributes to the harmony and contentment
within a spiritual community, a family, or society at large, depends on
our relationship with it. It is appropriate that we feel afraid of it,
in the same way we should be afraid of getting too close to a source of
radiation: it has the power to cause harm. However, to react with panic
and blame technology, and possibly even try to ban technology, is
similar to what the Luddites\cite{luddites}
did back in the 1800's out of fear of the fabric-manufacturing
machines. Their disruptive actions didn't stop the mechanical production
of cloth. Technology is not going to disappear. What would help is to
develop a wise relationship with it and learn to understand where the
real causes of suffering lie.

\begin{quote}
  Ably self-restrained are the wise,\\
  in action, in thought and in speech.

  \emph{Dhammapada 234}
\end{quote}

The Buddha's encouragement to strengthen our capacity for restraint
(\emph{indriya samvara}), is not some neurotic religious injunction that
leads to blind repression. Through the development of embodied
mindfulness, we are encouraged to learn how to contain compulsive
reactivity; perhaps restraint might better described as a form of
conscious composure. All deluded personalities are addicted to their
preferences. Most of us manage to get by in life without causing too
much damage, by controlling our preferences using will-power: we don't
allow ourselves to do or say anything that is too harmful. But such
wilful controlling is energy-extravagant and takes its toll on our
nervous system. If, instead, we employed mindfulness and wise reflection
on that which leads to increased well-being and that which leads to
harm, perhaps we would come to understand what the Buddha meant by
\emph{indriya samvara --} sense-restraint or conscious composure -- and
appreciate the power it has to genuinely protect us.

Without such restraint we are victims of our habits of reactivity: the
phone rings or we receive a notification and we \emph{have} to pick it
up. But do we really have to pick it up? Is it not possible to develop a
quality of restraint that overrules tendencies to react, without tipping
over into repression? Once more we can consider the benefits of formal
meditation. We learn by trial and error how the way we react whenever
our mind wanders from the object of meditation has a direct effect: if
we are judgemental and critical of ourselves for not being as good as we
want to be, we increase our suffering; if we respond with gentleness, as
a parent would when teaching their child to walk, we experience a
lessening in our suffering. Little by little we learn what `ably
self-restrained' means, and perhaps find we are better equipped to make
skilful use of technology.

In our monasteries here in Europe, for several years now, we have had an
ongoing discussion about our relationship with technology. As someone
who enjoys communication, but who is also cautious about being defined
by the tools that we use, I fall somewhere around the middle, or perhaps
to the cautious side of the middle, when it comes to making decisions on
what is suitable and what is not. Around 1995 I was looking into the
potential benefits and dangers of our monasteries having websites. At
the time, others in the community were more cautious than I; they
assumed that having websites would increase our workload. I settled on
the view that websites could simplify things and save us from having to
reply to lots of letters.

Around the same time, with the support of the Elders' Council in Europe,
I produced the first iterations of the websites www.dhammatalks.org.uk
and dhammathreads.org.uk, which provided centralized free access to
audio and written materials from within our family of monasteries. It
concerned us that other groups or individuals might be posting our
materials online without our knowing about it, and possibly even
charging for them. Later those two websites were combined to form the
current www.forestsangha.org\cite{branches} website.
(A considerable amount of time and effort went into
consulting around our world-wide family of branch monasteries to ensure
there was sufficient agreement that I construct such a website. At that
stage of the evolution of our global community there was no body of
Elders who had the authority to endorse any such proposal. In fact it
was only in 2016 at an International Elders' Meeting at Amaravati, that
the BAM group accepted the responsibility of being a decision-making
body). I am indebted and sincerely grateful to a good friend of the
monastery, Kris Quigley, for his generous offering of skill and support
in that project. I remain convinced that such websites provide a useful
service to both the sangha and the world-wide community of friends and
supporters. Currently Kris and I are considering the possibility of
producing a smartphone app that corresponds with the content of the
www.forestsangha.org\cite{branches} website.

My early adoption of Facebook was not such a good idea. I was looking
for a means of distributing fortnightly Dhammapada reflections\cite{reflections}
without triggering users' spam filters. It
quickly became apparent that Facebook was not the right tool.
(Eventually we learnt about bulk emailing services which now serve the
purpose.)

Currently in our communities here in Europe there is an ongoing debate
regarding which of the various video-distributing platforms we ought to
be using. During much of this pandemic there has been a sort of
moratorium in place so monasteries can conveniently make teachings and
services available to the wider community. But the jury is still out in
terms of an overall policy. My personal view on the subject is
influenced by Marshall McLuhan's `the medium is the message'. In the
ever-increasing mania of our out-of-balanced world, I believe the
sangha's message of stillness, silence and space, is rare and must be
protected. It is part of our duty as inheritors of this ancient
Theravada tradition to be cautious, and not engage in adaptations until
it is clear they will not harm that with which we have been entrusted.
At the beginning of this chapter there is a quote from the discourse
given by the Buddha to Bhikkhuni Mahapajapati, mentioning the place of
contentment in his teachings. Alongside contentment, the Buddha also
mentions: \emph{that which leads to modesty is Dhamma and that which
leads to self-importance is not Dhamma.} It seems to me great caution is
called for when using the various video-distributing platforms to
address matters of the heart.

The electronic distribution of audio files, however, is something I do
very much support. We tend to use our eyes to search outwards, and when
it comes to studying Dhamma it is good to remember that the solutions to
our apparent problems are to be found inwards. When listening to Dhamma
it can be helpful to close our eyes -- to release ourselves from the
scanning, straining, liking-disliking mode -- and simply receive that
which is being offered. This is what I understand Tan Ajahn Chah was
pointing to when he encouraged us to listen to Dhamma talks with our
hearts.

It is alright to say no to so-called technological advancements, even if
part of us wants to say yes. In fact I teach the young monks here at
Harnham to practice saying no just so they know they can do it.
Obviously I don't want them to say no when they are asked to do the
dishes or rake the grass that has been cut. I mean say no to an extra
cup of coffee or to agreeing to have their photograph taken. Sometimes
we are asked to appear on television, and I make it a rule to always say
no. Regardless of how charming or persuasive the contact person or the
interviewer might be, I tend to distrust television editors. Their
agenda is generally always financial gain, while sangha life is
predicated on principles that are different from theirs. A television
company once asked if they could paint the front door and window frames
of part of the monastery a different colour because they were filming an
episode of a Catherine Cookson movie here on Harnham Hill, but I said
no. On another occasion we were asked if a television company could
borrow some monks robes to use in a film that was being made, but I said
no. They told us that it was our fault if they got it wrong. On yet
another occasion, the teachers of a school group that was planning a
visit told us a local television crew had asked if they could come along
and film the visit, and I said no. I confess on that occasion I was
concerned that the teachers were going to be upset with me. As it
happened, they mentioned how pleased they were that I rejected the
request; they also didn't want the television people involved. The
timeless principles of Dhamma are too precious to subject them to the
vagaries of worldly preference.

The initial driving force behind social media may well have been
increased ease of communication and facilitating community; however the
way it has evolved, it has turned into a destabilizing influence. The
lack of integrity in the media industry in general is part of what has
brought humanity to a point where cynicism is not only accepted, it is
normal. It is now even normal to question the accuracy of news that is
broadcast by long-established outlets. This is not the fault of
technology, it is the fault of the users. It is a very sad state of
affairs and we don't have to contribute to it by always going along with
it. We can train ourselves by turning off our gadgets, and by
recognising and inhibiting the impulse to always react when we hear a
beep. When we have suffered a wound to our body and it is healing, it is
natural that we want to scratch the itch, but we know that to do so
risks making things worse. Maybe humanity can learn to apply that
understanding to restraint on the level of mental activity.

We have the option of investing in our refuge in the Buddha -- in
selfless just-knowing awareness. Going for refuge to the Buddha is not
aligning ourselves with a group of people who all believe that the
Buddha knew best; it is disciplining our attention so that in a moment
when suffering arises, we remember to turn directly towards the
\emph{dukkha} and enquire into the cause of \emph{dukkha} -- get
interested in the reality of \emph{dukkha} and not follow our habits of
resistance. The situation that we are in as a species is already so
severe, it is hard to imagine how we will survive without a fundamental
shift which involves recognising the primacy of the cultivation of
awareness.

Earlier I referred to the expression that we have in English: `the
survival of the fittest'. There is an equivalent expression in Chinese,
\emph{Shi zhe sheng chun,} which translates roughly as `the one who
adapts is the one who survives'. The word `adapts' here holds the key:
if we want to survive we need to adapt to the information and evidence
that we have available. A few years ago climate change deniers managed
to refute statistics and get away with it; the disastrous effects of
climate change are now very difficult to deny. Before the pandemic many
people insisted individuality was all-important; now it is evident that
our very survival depends on cooperation. Whilst it is sensible to be
afraid of technology, it is not helpful if we become lost in fear. We
can have faith in the potential to train our faculties with accurate
understanding of the advantages and disadvantages of technology, and
choose to use it for good.

\section{Replenishing Our Storehouse of Goodness}

Some readers might find that in these pages I have used the words `thank
you' just a few too many times; however, I hope that we all agree that
it is not possible to actually feel too much gratitude. In my
experience, dwelling on gratitude begets gratitude and has within it the
power to dissolve obstructions and transform our view on life. This
principle applies not just to gratitude, but to goodness in general.
When we focus on our misfortune we easily sink into feeling sorry for
ourselves and see only the things that we think are wrong; when we focus
on the goodness of our lives, our hearts are buoyed up and we notice
even more goodness. Dhammapada verse 118 says,

\begin{quote}
  Having performed a wholesome deed\\
  it is good to repeat it, again and again.\\
  Be interested in the pleasure of wholesomeness.\\
  The fruit of accumulated goodness is contentment.
\end{quote}

It is such a simple fact that we could dismiss it: when we remember our
good actions, we feel good. It takes very little effort to recollect the
goodness of our lives. The gladness that arises in the process naturally
manifests as gratitude which, in turn, can be expressed as generosity.
If we want to make a difference to our inner and outer worlds, regularly
replenishing our storehouse of goodness is essential.

Giving is one way that leads to contentment. As monks and nuns we don't
have much in the way of material things that we can give, but we can
share things that we have learnt. In this regard, giving Dhamma talks
can be a source of much happiness. I wrote earlier about Ajahn Sumedho
supporting me in giving talks quite early on after I arrived in Britain.
I can't pretend it was easy in the beginning, but I am hugely grateful
now for his encouragement. I have been told that I come across as
confident when I speak in public, but even after many years I still find
that it is work. Other monks tell me that they find it easy but I have
more or less given up expecting it to be easy. Even though it is work,
it is work I love doing. It reminds me of many years ago when I used to
dabble in throwing clay pots on a wheel: there is a joy in crafting
something beautiful out of that which was raw and unrefined.

An oft-reported incident, that apparently occurred in the early years
when Ajahn Sumedho lived at Wat Pah Pong, involved Tan Ajahn Chah having
a firm word with Ajahn Sumedho about his preparing in advance for a talk
that he gave. It is reported that Tan Ajahn Chah told him that he should
never do that again. This incident is well-known within our family of
monasteries and is presumably the reason why many monks and nuns attempt
to give talks without any preparation. For some that seems to be a
useful approach; they find they can handle whatever emotional reactions
they might be having, maintain mental clarity, and ably offer a Dhamma
reflection at the same time. For others I suspect it is a terrifying
experience; certainly from the tone of their voice and the content of
their talks, it would appear that they are caught in an intense and
often humiliating struggle. I don't believe it has to be that difficult.
Tan Ajahn Chah was aware that he could appear inconsistent, and
personally I am not convinced that he would insist everyone abide by the
same instructions he gave to Ajahn Sumedho when he spoke to the local
villagers in North East Thailand in the late 1960's.

When I look out into the Dhamma hall, here in twenty-first century Great
Britain, I see people who have made considerable effort to attend the
gathering and who are seriously seeking meaning in their lives; I want
to give them something helpful that they can take away and ponder on. It
does happen that sometimes I give a talk that is totally unprepared,
particularly in spontaneous Q\&A sessions, or perhaps a talk given
specifically to the resident sangha. However, for many years now, when I
know that I will be offering a Dhamma talk, I prepare my thoughts in
advance. Typically I make notes on an A4 sheet of paper, folded into
four, with two points listed in each quadrant, usually adding more
detail. Committing those eight points to memory means I feel ready to
offer a reflection to the listeners without worrying that I am going to
be wasting their time.

Sometimes I am told that I sound a bit heavy when I talk. Maybe this is
because as far as I am concerned, life is not a picnic. There is also
the possibility that it stems from very many years ago when I was
struggling to overcome fear in advance of those Rotary speech contest
talks. I recall one year in particular, as I was preparing my speech, my
parents insisted I go next door to where my Nana and Grandad (Rev.
Duncumb of the skewered moth on the sofa incident) were living. Since
Grandad was a preacher, he was presumably considered a competent speaker
and it was thought a good idea that he listened to my delivery. I went
over as I was told. Now, more than fifty years later, I can still
vividly recall standing there in their living room, mute. Intense fear
prevented me from even beginning my speech. My sweet and caring Nana
told Grandad he should let me go because I was obviously struggling, but
he insisted I deliver the talk; he said I couldn't leave until I did. I
never did deliver that talk in front of them. I just stood there, rigid
with fear -- petrified. By the time I eventually got around to taking a
closer look at that pocket of fear, there was already enough awareness
to be able to appreciate that my Grandad wasn't to blame for inflicting
that trauma on me -- unawareness was the cause. He didn't know what he
was doing and I wasn't ready. I might not feel thoroughly equanimous
about it, but neither do I feel passionately indignant.

That is what unawareness does. This is not to say that Grandad didn't
rack up some negative kamma for himself; I suspect that out of
insensitivity and heedlessness he might have done. But it is helpful to
understand that many of the obstructions that we encounter don't come
from our having been `bad' in the past; they come from our having been
unaware. One of the things we can do about that is invest in the
goodness that has the power to outshine feelings of limitation. I can't
say I feel grateful to Grandad for that ordeal, but neither do I resent
him. It hasn't stopped me from finding my own way of sharing the good
fortune of my life with others.

My advice to anyone who is feeling anxious before speaking in public is
to be honest with yourself. Ask yourself, `what is the cause of this
suffering?' Or, since we already know that uninformed desire is the
cause of all forms of suffering, ask, `What do I want?' And listen
carefully to the answer. If it is, `I want to get out of here', tell
yourself you are absolutely allowed to want to get out of there. But
what is happening there that is so bad? Probably the only thing that is
happening right there and then is anticipation about the future.
Anticipation is just a movement of energy; it is not a rabid dog that is
about to bite you, which would be a perfectly valid reason for wanting
to get out of there. Simply becoming aware that it is anticipation that
we are struggling with, can take the sting out of the struggle. If we
deal honestly here and now with our anticipation, we might find that
there and then things are not so bad.

Or when we ask ourselves `What do I want?', maybe we hear the answer, `I
want everyone in the audience to be impressed by my talk'. Again, all we
need to do is listen to that voice, be honest, no judgement -- receive
that movement of energy, and see what a difference it makes when we stop
resisting it. I would not advise that you share with the audience the
answers to your question and tell them how petrified you feel, or that
you hope they will be stunned by your erudite teachings. If we do that,
it is expecting others to take responsibility for that which is ours.

It was possibly towards the end of 2009, again during a stopover in
Thailand en route to New Zealand, that I visited Ajahn Nyanadhammo at
Wat Ratanawan near Khao Yai National Park\cite{khao} in central Thailand.
It so happened that
while I was there, a supporter of his monastery was visiting and was
discussing with Ajahn Nyanadhammo the possibility of reprinting the
translated talks of Tan Ajahn Chah. Various booklets of Tan Ajahn Chah's
teachings had been published over the years in a variety of formats, and
a number of them were out of print. In the course of that conversation,
I was asked if I would possibly be interested in heading up such a
project. What a delight that production turned out to be! It was a big
project and it felt like a big privilege. Much of the good feeling that
I experienced came out of knowing that I was helping to make Tan Ajahn
Chah's teachings available to others. I was already familiar with much
of the content, but reading the talks was not the only source of
delight: much of the good feeling came from the process of collecting
all the available material, seeing that it was adequately proofread by a
global group of readers, finding sponsorship, and arranging for
distribution. Due to the generosity of the Kataññuta Group in Malaysia,
Singapore and Australia, we were able to produce both a box set and a
single volume version. Neil Taylor, who has been helping me over the
years with the graphic design aspects of our annual Forest Sangha
calendar and other publications, was a significant support in the layout
and presentation of what ended up being called
\emph{The Collected Teachings of Ajahn Chah}\cite{collected}.
Tan Gambhiro was responsible for the phenomenal amount of work that went into
typesetting. Thank you, Tan Gambhiro, and Neil, and to all the other
good friends for being part of the team that produced this wonderful
collection.

The way I felt about being part of the group that published these
teachings is also the way I felt about designing and managing the
www.forstsangha.org website. I have a similar feeling regarding the monthly
Dhammapada Reflections\cite{reflections} that we distribute. This later project was
initially inspired by a conversation I had with Ven. Myokyo-Ni during
one of my visits with her at her Fairlight Zen Temple\cite{fairlight}
in Luton. We were discussing a program
that she had been running for many years whereby she sent out, by mail,
a Dhamma teaching to a number of recipients around the world. This gave
rise to what I called our \emph{Dhammasakaccha} program (Dhamma
dialogue) whereby I sent out, via email, a short commentary on a theme
of Dhamma, and recipients replied with their own reflections on what I
had written. The program quickly became popular to the point where it
was taking up a considerable amount of time.

After that project came the current
Dhammapada Reflections\cite{reflections}
program which has been running now since 2007.
It started with sending out a verse from the Dhammapada and an
associated reflection every fortnight to coincide with the new- and
full-moons, but eventually that also became more work than I could
manage. These days the reflections are offered in seven languages and
emailed out once a month on the full-moon day.

The comments associated with the Dhammapada verses are not aimed at
explaining the verse itself -- readers can do their own research if they
wish -- rather, my reflections are an effort to encourage followers of
the Buddha to develop their thinking minds in service of deepening in
Dhamma. Many Western Buddhists seem to have picked up the teachings in a
way that causes them assume that the point of practice is to make their
minds peaceful by focussing on a meditation technique. In the West our
minds have been programmed since very early on to be discursive, and
while being compulsively discursive is indeed painful, the way back to
stillness is not necessarily by wilfully concentrating on the end of
your nose. When the Buddha asked his son Ven. Rahula what the purpose of
a mirror was, Ven. Rahula replied it was used for seeing the face. The
Buddha then told him that wise reflection is what we use for seeing the
mind. A degree of mental calm and clarity are important, but there is
much more to our practice than just that. So each month I find a verse
that feels apposite, pick up whatever contemplation is stimulated in my
mind, and then share that contemplation. It is an attempt at acquainting
readers with an appreciation of their own ability to contemplate. It is
also a source of happiness for me.

Like gratitude, giving can lead to contentment. Traditionally, in all
Buddhist countries there is an emphasis on the cultivation of generosity
(\emph{dana}), and there is much talk about `making merit'
(\emph{puñña}). When we perform wholesome acts, however, it is
unfortunate if we dwell only on thoughts of how much merit we are
making. This is an understandable mistake if we have not been taught all
the benefits of developing \emph{dana}. We can trust that there is such
a thing as \emph{puñña} which can be accumulated, but rather than
thinking of it as credentials which define our worth, it is wiser to
view it as potential. This is similar to how refraining from consuming
junk food and filling our bodies with harmful chemicals makes us
potentially less likely to become sick because our immune system is in
good shape. Every time we act generously we let go of a little bit of
the sickness of selfishness. In Dhammapada verse 118 it says accumulated
goodness leads to happiness, but we would do well to pay attention to
what we lose, not just what we gain. The less selfish we are, the
happier we are. As Tan Ajahn Chah said: `If you are not careful, you
will make so much merit that it will be too heavy to carry.' He went on
to say: `The point of accumulating merit is for the sake of
realization.' Accumulated \emph{puñña} nourishes our potential for
awakening. It is what gives us the strength and resilience to do the
work.

Also in Dhammapada verse 118 we are encouraged to be interested in the
pleasure that arises from such wholesome acts as generosity. For some
readers, the mere suggestion that we should take delight in our own
goodness could set alarm bells ringing. `Aren't I at risk of becoming
conceited?' There is a big difference between mindfully taking delight
in the natural sense of well-being which arises when recollecting our
own good deeds, and heedlessly indulging in the ego-centric thought,
`Aren't I wonderful'.

The helpful reminder by Tan Ajahn Chah that \emph{puñña} is not the goal
-- it is the fuel that propels us on our journey towards the goal -- is
the basis of a significant part of my personal morning ritual routine.
After bowing to the shrine and reciting some verses in Pali, I make the
conscious wish: `May whatever happens today be for the development of
goodness and wisdom.' We need goodness, we need fuel, and we also need
wisdom. A well-stocked storehouse of goodness sustains us as we burn
through the layers of habitual resistance to reality in the pursuit of
wisdom.

After spending nearly five years living in Thailand back in the 1970s, I
felt as if I had absorbed an appreciation for the skilful use of rituals
and symbols; it seemed to happen without my trying. In the beginning my
rational mind had made a bit of a problem out of them, but somehow
eventually the resistance fell away. I suspect that initially I was
afraid I would lose the (false) sense of security I felt by remaining
aloof -- rather arrogantly looking down on those caught up in
superstitious rituals. But it was undeniable that they were the ones who
were happy, not me. And I also think Ajahn Sumedho's willing
participation in the rituals helped give me permission to experiment
with letting go of my resistance. He was comfortable going along with
the way things were done, and yet obviously had lost neither his sense
of humour nor his critical faculties. I am grateful that I managed in
those early years to let go of the conceited view that progress on this
path depends solely on being rational and reasonable. Many of the
obstacles I face in practice are unreasonable and irrational, so why not
surrender myself to these tried and tested rituals? During our evening
chanting, when I ask the Buddha, Dhamma and Sangha to bear witness to my
acknowledgment of fault, I am not imagining some deity listening to me
and being pleased by my obeisance; but \emph{I} am listening to me, and
it feels good to hear my acknowledgement of fault. Sometimes rituals can
communicate what our hearts want to say better than linear logical
dialogue.

When the first substantial meeting hall (\emph{sala}) at Wat Pah
Nanachat was completed, around 1977, Tan Ajahn Chah paid us a visit. He
complimented us on how the area in front of the main shrine had been
laid out. It seems he was pleased with the way we had arranged it so
that everyone who came into the hall -- sangha members and laity alike
-- had direct access to the main shrine; everyone was able to make
offerings and pay their respect to the Buddha. In many monasteries, the
area immediately in front of the shrine is designated for only the Ajahn
to sit. For reasons that I can no longer recall our \emph{sala} had been
designed with Ajahn Sumedho and all the sangha members sitting off to
one side.

This might have been the first occasion when I began to contemplate
matters of authority: who has the power and how is it exercised? In most
theistic forms of religion there is a mediator -- a priest, or rabbi, or
holy man or woman -- located between those who are seeking and the
Godhead. I interpreted Tan Ajahn Chah's comments on that occasion as
saying the space between the followers of the Buddha and the Buddha
himself should be open. Years later I followed his example when we
designed our Dhamma hall here at Harnham so that all who entered had
direct access to the main shrine. More recently, because my mobility has
become impaired, during pujas I sit on a chair in the middle in front of
the shrine, but as soon as I leave the chair is removed.

When I was describing earlier my first visit back to New Zealand as a
monk, I mentioned how I engaged in a ritual practice of making a vow
(\emph{adhitthana}). On that occasion it proved very helpful. I am aware
of occasions when overly zealous monks have made vows that were beyond
their ability to keep. We do need to exercise caution and not be too
idealistic. For instance, it sometimes happens that, while on retreat,
meditators experience a lot of enthusiasm and feel motivated to make
firm resolutions. It should be understood that resolutions that are made
when the mind is clear and the heart is open -- when we are in a state
of heightened and focused energy -- can have very far-reaching
consequences. As long as we are surrounded by friends who are walking
the same path, have access to teachers who are offering us reassurance,
and we recognize the great privilege of being able to commit to this
practice of purification, we might be tempted to say, `Bring it on
\emph{Mara}. I am ready.' The sincerity of our resolve means the message
goes deep. Unless we have experience in such matters, it can be
difficult to know whether our enthusiasm is grounded in our own matured
awareness, or is an effect of feeling held by spiritual community. If it
is the latter, it would be better to discuss our intentions with a
teacher before being too adventurous in making vows. The point is,
making vows is like turning up the heat. We might be keen on the image
of purifying the gold, but not be ready to handle all the dross as it
comes to the surface. What I am saying is: we need to be careful to not
bite off more than we can chew. Naivety can make progress on the path
more difficult than it needs to be.

\section{Equanimity}

At the beginning of the previous chapter on expecting the unexpected, I
quoted Tan Ajahn Chah saying, `I've searched for over forty years as a
monk and this is all I could find. That (\emph{aniccaṃ}) and patient
endurance.' I ask forgiveness for assuming that I know what went on in
Tan Ajahn Chah's mind, but I think it is safe to say that he was also
well acquainted with patient endurance's close friend, equanimity
(\emph{upekkha}). Patience and equanimity are like companions on the
journey that work together to help us meet the many obstacles on the
path to freedom. For example, patience and equanimity can tame our
excessive striving and protect us from the danger of craving for results
in practice. In a talk by Tan Ajahn Chah titled, \emph{Two Faces of
Reality}, he says,

`I used to think, about my practice, that when there is no wisdom, I
could force myself to have it. But it didn't work, things remained the
same. Then, after careful consideration, I saw that to contemplate
things that we don't have cannot be done. So what's the best thing to
do? It's better just to practise with equanimity\ldots. If there's no
problem, then we don't have to try to solve it. When there is a problem,
that's when you must solve it, right there! There's no need to go
searching for anything special, just live normally.
(\emph{The Collected Teachings of Ajahn Chah}, 2011, p500)

The word that quickly comes to my mind when contemplating equanimity is
`even-mindedness'. When we read or listen to what our teachers tell us,
it should be clear that true equanimity is an expression of wisdom. In
the lists of traditional Theravada teachings we find equanimity
consistently comes last. It is last in the four divine abidings:
\emph{metta, karuna, mudita, upekkha}; it is last of the seven factors
of awakening: \emph{sati, dhammavicaya, viriya, piti, passadhi, samadhi,
upekkha}; it is last in the ten perfections: \emph{dana, sila,
nekkhamma, adhitthana, sacca, khanti, pañña, viriya, metta, upekkha}.
There are possibly other explanations as to why equanimity comes last,
but I trust that in some ways at least it is because there cannot be
true equanimity without true wisdom; in other words it is difficult to
develop.

The formal cultivation of this virtue has its place in our monastic
routine as one of the recitations during morning \emph{puja},

\begin{quote}
  I am the owner of my kamma,\\
  heir to my kamma,\\
  born of my kamma,\\
  related to my kamma,\\
  abide supported by my kamma;\\
  whatever kamma I shall do,\\
  for good or for ill,\\
  of that I shall be the heir.

  All beings are the owners of their kamma,\\
  heirs to their kamma,\\
  born of their kamma,\\
  related to their kamma,\\
  abide supported by their kamma;\\
  whatever kamma they shall do,\\
  for good or for ill,\\
  of that they shall be the heirs.

  \emph{Chanting Book}, Vol. 1, p.56\cite{chanting}
\end{quote}

Reflecting thus on the law of kamma is a way of instilling the
understanding that we cannot take responsibility for the intentional
actions of others, and nobody else can take responsibility for our
intentional actions -- nobody can take away our kamma. Even the Buddha,
with his profound wisdom and limitless compassion, could do no more than
`point the way'. It is important that we equip ourselves with this
understanding, otherwise we could live our lives without growing up
properly -- always expecting someone else to take responsibility for us.

It also matters that we train ourselves with this perspective so we are
protected from being overwhelmed by emotions, and not just painful
negative emotions. The first three of the four divine abidings are
positive emotions -- kindness, compassion and empathetic joy -- and
without equanimity we are at risk of becoming lost in the pleasure of
positivity. Being positive is not enough. We might feel compassion for a
drug addict and sincerely try to help them come clean from their habit,
but what do we do if they decide to not come clean? Compassion fatigue,
an expression sometimes used in the caring professions, is not the
result of too much compassion; it is the result of too little
equanimity.

It is also possible to have too much or an immature kind of equanimity.
Without kindness, compassion and empathetic joy, so-called equanimity
can be a form of cold-hearted indifference. A mere intellectual
understanding of the law of kamma without warm-heartedness is not a true
source of support on this journey.

One useful way of developing equanimity can be to allow ourselves to
feel foolish in front of others. For instance, if in a group meeting we
say something that we think is funny but it falls flat like a lead
balloon, instead of rushing to cover up the embarrassment, allow those
feelings to be there: fully feel what it feels like to feel embarrassed.
It takes a well-developed quality of awareness to be able to allow such
feelings to be there without defaulting to strategies of self-defence.
Equanimity is an expression of inner strength that from the outside
might appear as weakness. However, an individual who is equipped with
such strength will be able to listen to criticism and consider whether
there are grounds for it or not; they will be able to make mistakes and
learn from them; be disliked but remain committed to true principles;
feel judged by others and not fall into despair.

As with developing any Dhamma principle, we need to exercise caution and
not be overly idealistic. We once had a guest staying who appeared to
have turned equanimity into an idol, possibly because they had
experienced some benefit from attending a series of meditation retreats
where there was a lot of emphasis on this particular Dhamma principle.
Appealing as the thought of realizing unshakable equanimity might be,
the reality is that if we cling to an ideal of the goal, then we create
obstructions; we need to be more subtle in our approach. Many determined
Dhamma practitioners have found themselves obstructed in practice
because of clinging to ideals: by wanting wisdom, by longing for
tranquillity, by hoping for freedom from anxiety -- but the way they
were wanting lacked equanimity.

There is a lot more that could be said about equanimity, however my own
lack of development in this area means I probably shouldn't try to say
too much. There have been a number of occasions over the years when I
have given a series of talks on the ten \emph{parami} and, when it came
time to address the topic of equanimity, I have turned to the second
monk, Ajahn Abhinando, and asked if he would address it. The ten
\emph{parami} provide a useful framework for a series of talks, and for
nine of the ten I usually have something to say, but consistently when
it comes to the last in the list -- \emph{upekkha} -- I feel
unqualified. My lack of equanimity regarding even something as mundane
as smells has long been a challenge for me. After forty-four years as a
monk, the smell of certain detergents used for washing clothes and
bedding still disturbs me. The fragrances that people wear when they
attend our evening puja sometimes disturb me. Thankfully, though, the
fact that I am still disturbed by these things is something about which
I can work to feel equanimous -- developing equanimity for my lack of
equanimity is a beginning.

\section{Contentment in Old Age}

Earlier this year I received a phone call from Luang Por Sumedho in
Thailand. He had been due to arrive in Britain to lead a retreat at
Amaravati, but his trip had been cancelled due to the pandemic. He
wanted to be in touch to see how our sangha here at Harnham was doing,
and to wish us well. During that conversation he explained to me that
these days he is having to learn to walk with a cane; since he is now
eighty-six that isn't surprising. We spoke about the benefits of working
on cultivating contentment. He commented on the fact that it is work,
and if we don't do our work, then we risk the restlessness of the
physical body taking us over.

I have been thinking about what a rare and wonderful thing it is to have
been acquainted with such a person as Luang Por Sumedho for nearly
forty-five years. Also, earlier this year I spoke with Luang Por Pasanno
in the US and we discussed how folk at Abhayagiri were coping with the
forest fires and earthquakes; and with Luang Por Viradhammo in Canada to
talk over publishing a photograph of him; and with Luang Por Tiradhammo
in Australia in a conversation about the new book he is writing on
`Beyond I-making'; with Luang Por Sucitto at Cittaviveka to consult on
points of \emph{vinaya}; with Ajahn Vajiro in Portugal to discuss
distribution of the FS calendars printed in Malaysia; with Ajahn
Jayasaro in Thailand to talk over a translation of teachings by Tan
Ajahn Thate; with Ajahn Amaro at Amaravati to consult on the
\emph{upasampada} of two of our novices; and with Ajahn Candasiri in
Scotland to discuss protocols around live-streaming via the internet.
All of these people I have known for forty or more years. How fortunate!

Being seventy is not exactly old, but clearly it is not young, and I am
glad for that. Thinking back to the first thirty or forty years of my
life, it feels as if a lot of the time I was in a fog. I suspect that
much of how I conducted myself -- by body, speech and mind -- was
influenced by fantasies about the future: `what will I do with my life?
what really matters? am I up to it?' I expect that most of the time I
wasn't even aware how much I was being affected by thoughts of the
future. One of the many advantages of being older is the way such mental
patterns naturally reveal themselves.

Those who have confidence in the Buddha's teaching would do well to read
what he had to say about recovering from the `three intoxicants: youth,
health and life' (\emph{Anguttara Nikaya}, Sutta 3: 39, Wisdom
Publications 2012). Probably, again due in large part to the influence
of technology, humanity is presently more intoxicated by youth, health
and life than ever before. It is not that the Buddha is saying there is
anything wrong with these aspects of existence -- Buddhism is not a
life-denying religion, it is a reality-affirming religion -- rather, he
was highlighting how, without wise reflection we misperceive the life we
are living: we create problems out of things that are perfectly natural.
Old age, sickness and death are no more wrong than the changing seasons,
but because of a lack of wise instruction, we develop attitudes and
behaviours that contradict that which is natural.

In a recent conversation with one of our monastery's trustees, who
happens to be around the same age as myself, I mentioned that these days
when I lock the door to my kuti at night, I always take the key out of
the latch, so if I die during the night my attendant monk will be able
to use the spare key to unlock the door and avoid having to break the
glass to get in. It turned out that the trustee does the same thing each
night in their house in London. It is always refreshing to find there
are those whose reflections on life have taken them to a point where
they no longer lie to themselves about their mortality. The opposite is
also true: it is sad to find that there are many who prefer to hide
behind the myth of immortality. Obviously they don't believe that they
are literally immortal, but they behave as if they are and invest huge
amounts of material and mental energy in trying to maintain the myth. So
long as we entertain the idea that old age, sickness and death are
problems, we will struggle to find contentment. Contentment increases to
the degree that we are honest with ourselves. That doesn't mean the
topic of death should be raised at a dinner party (unless all who are
there are clearly willing to discuss the matter, in which case it could
be quite productive).

It would be disingenuous of me to not mention the difficulties that come
with old age. Sometimes I struggle to find where I have left my glasses,
and that is not fun, neither is fumbling to replace the miniscule
batteries in my hearing aids, or regularly forgetting my lines when I
lead the chanting. I even forgot to light the candles for evening puja
recently. But these are all very minor matters compared to the relief I
feel about no longer being quite so concerned with what others think of
me. Even in my fifties it still sometimes seemed the young monks and
novices felt as if they had to compete with me -- as if it was a sign of
their weakness to simply go along with what I was asking them to do.
These days it is obvious they don't see me as one of them, and there is
a greater willingness to accept what I tell them.

When recently we took the decision no longer allow monastery funds to be
used to purchase animal food products, I found it pleasingly easy. A few
years ago I might have been more worried about what others would think.
At this stage, although such concerns haven't disappeared, they do seem
to have less momentum. For a long time I have felt dismayed by the lack
of effort made in our monasteries to carefully dissuade supporters from
offering meat at the meal time. Only very rarely did I say anything
about it, out of a fear of causing division between our communities. Our
monastery here in Northumberland hasn't been purchasing meat for many
years, but more recently we took it to the next level and agreed to also
stop purchasing dairy products. I have already described the horror of
what happens in abattoirs and the impressions left in my mind from
having visited one. Now that we have taken this decision I feel an
increased sense of ease. We don't have a big sign up saying which
offerings are accepted and which are not -- everything is accepted --
but gradually our community of supporters have learnt that we are more
likely to eat plant-based food if they bring it. As far as I am aware,
our decision hasn't caused any rift or difficulty within the monastic
communities or amongst the laity. Perhaps people don't feel so
threatened by those who are older and allow us to be a bit different.

There was a period when Tan Ajahn Chah started eating vegetarian food.
At the time it was quite a radical statement to be making. He didn't
hide the fact, but neither did he push it -- there was no shaming or
pressure of persuasion. He made it clear that everyone in the monastery
should feel free to eat whatever food had been offered. I took strength
from that example of skilfulness and from his willingness to go against
the grain.

For much of my adult life I have enjoyed the exercise of consciously
admiring competence; be it of a skilled craftsperson, a caring
professional, a clever computer programmer, or an experienced cook. I
find a particular delight in observing the results of sincere commitment
and dedication. Although in my own case I see plenty of evidence of a
lack of competence, I remain enthusiastic in making an effort to improve
upon the abilities I do have. These days, however, I find there is a new
kind of difficulty that comes with discovering I no longer have the
equipment with which to work. However hard I might try, I keep
forgetting the chanting and find it challenging to memorize new verses.

When I notice people around me my own age, or older, employing
strategies to hide the signs of their deteriorating faculties, I wonder
if they are aware that they are doing that, and if perhaps the same
thing is happening to me. Am I aware of my increasing incompetence? And
when I hear a young monk talking about the mental decline of a senior
sangha member in their monastery, I wonder if the junior monks here are
having the same conversation about me.

I am reminded of something Tan Ajahn Chah said when he was describing
self-view. He had read in the scriptures about the different types of
conceited self-view -- how we see ourselves as being better, equal or
inferior to others. He spoke about the struggle he had to equate what he
read in the scriptures about getting rid of conceit, with his own
experience, `The fact is I can sew robes better and chant better than
many of the other monks.' He couldn't deny that reality. Then he
realized that the problem was not the perception of being better, equal,
or inferior to others, it was clinging to those perceptions -- it was
finding identity \emph{as} those perceptions of self. In themselves, the
perceptions are `just-so', they are not a problem. The suffering of
conceited views arise from our clinging to the sense of self.

Perhaps contemplating this Dhamma principle can help in coming to terms
with seeing oneself as less competent. It is not pleasing to find that I
have trouble to hear without hearing aids, or see without glasses, and
likewise, it is disagreeable to find my memory less accurate. But at
what point do these perceptions of loss become a problem? At the point
of clinging! If we can remember there is a larger reality in which these
perceptions of oneself as being competent or incompetent is taking place
-- if we remember the refuge in the Buddha -- then maybe we won't fall
into the trap of resenting deterioration. I suspect from here on out, a
large part of my practice is going to be about learning to be
competently incompetent.

In this contemplation of contentment we have considered how important
this particular Dhamma principle is, how obstructed we sometimes feel in
our efforts to access it, and how supportive it is to be building up our
storehouse of goodness. Most fundamentally we need to own up to the fact
that discontentment is not happening to us -- it is something we are
doing -- we are the agents of discontentment. If our attempts to
cultivate contentment seem to be getting us nowhere, I recommend using
the word `contentment' as a mantra: on each good long out-breath,
quietly think the word, con-tent-ment.

To end this chapter, here is one more helpful verse from the Dhammapada,

\begin{quote}
  The timely company of friends is goodness.\\
  Fewness of needs is goodness.\\
  Having accumulated virtue at life's end is goodness.\\
  Having dispensed with all suffering is goodness.

  \emph{Dhammapada verse 331}
\end{quote}

