\chapter{Heading For Asia}

Before departing Australia I returned to Narada to dismantle my dwelling
up on the ridge and bid farewell to friends. I think at that stage it
was more excitement than trepidation that accompanied any ideas I had of
what might lie ahead; truthfully, of course, I didn't know. As a way of
marking the departure and moving on, and also out of gratitude, I~performed
a little ceremony which involved lighting a candle and saying
thank you for the place, the time and the precious opportunities that
the community there had given me. I was leaving Narada a very different
person from the one who had arrived a few months earlier. It had been my
intention to pass on the usable canvas portion of the construction to
someone else living at another community nearby; however in the process
of dismantling the structure, part of the tent touched the candle and
immediately caught fire. I rushed inside through the flames to save my
backpack and passport, managing to escape just in time to sit on a log
and watch the whole thing be consumed. I was lucky to get away with only
a few burns on my forearms where molten burning plastic had landed;
also, fortunately, I didn't start a forest fire.

A fellow from across the valley called Ross had expressed an interest in
accompanying me at least as far as Indonesia. So, with a mixture of
adolescent enthusiasm and hidden anxiety, the two of us set off for
Darwin. The journey wasn't straightforward. Hitchhiking wasn't as easy
as we had hoped. At one point we gave up and boarded a train for part of
the journey. It was primarily a goods train but there was one passenger
carriage tacked on at the end. After travelling for quite some time, in
what felt like the middle of nowhere, the train suddenly stopped. There
was no explanation. Then after a while we discovered that at least one
of the carriages ahead of us had come off the tracks and the train
engine had uncoupled and gone on without us. Eventually, after a long
wait and considerable uncertainty, another train did appear, presumably
one that managed to lift the derailed wagon back onto the track, and we
progressed to the next town where we disembarked.

We managed to get a ride on a
delivery truck. After a long and rather uncomfortable journey sitting in
the back amongst boxes of fruit, we arrived in Darwin. The place was hot
and dusty and was not at all like Mullumbimby. In fact it felt rather
red-necked and perhaps even unsafe. Besides the regular residents of
Darwin, there were tourists there to observe the interesting ways of the
Aboriginal people and a large number of backpacking travellers,
presumably about to depart for, or returning from, Asia. One day, to my
surprise, I spotted Bill Hamilton, the American from Sydney, sitting
with a group in one of Darwin's parks. We recognized each other, however
we weren't really friends, so a passing nod of acknowledgement was
enough.

The cheapest flight headed in the direction we wanted to go was to Dili,
the capital of Portuguese (East) Timor. West Timor was part of Indonesia
as was West Papua. On arriving in Dili, as mentioned earlier, another
bubble burst. For the first time I saw people who were actually poor,
desperately poor. The rude awakening I received on entering Australia
was on this occasion amplified manyfold; this was real culture shock. I
was almost nauseous with a feeling of disorientation. This young man
from Morrinsville was starting to struggle.

Not far out of town the authorities had provided a very basic shelter on
the beach where backpackers like us could stay. As far as I recall it
was free, or at least very cheap. Concrete floors, three walls and a
roof, but it had a stunning view. Timor was the kind of location where
they would film a James Bond movie: pristine sandy beaches, overhanging
palm trees, sun and coral. Shopping in the open market was an adventure.
Almost nobody spoke any English. It was the first time I had seen people
chewing betel nut. Initially I thought they must have some terrible
mouth infection and it was blood that they were spitting out. On the
surface at least, everyone was smiling and seemed very friendly and
accommodating.

Somebody or something had encouraged me to bring a pair of flippers, a
snorkel and goggles with me to Timor -- maybe because of the great
diving available. I knew nothing about snorkelling but had equipped
myself with the basics. One day I wandered quite far to a lonely beach,
presumably so nobody would see me, and tried on the gear. Gradually I
made my way out into the water and dived under. What a treat: the
extraordinary beauty of being immersed in such an environment. The water
was warm and I readily took to it, quickly becoming adventurous.
Possibly the benign and relatively harmless world I had grown up in
meant I was susceptible to feeling more confident in the water than I
should have.

At one stage I noticed a large, attractive conch shell lying on the
ocean floor. It was a long way down but greed and innocence together
meant that I went for it. It turns out I misjudged how far down it was
and how long I could hold my breath. I reached the conch shell and was
on my way back to the surface but the shell was heavy: it was huge in
fact, and kept pulling me down. I reached a point where I realized I had
a choice: either hang onto this lovely object and risk drowning, or let
go immediately and maybe make it back to the surface. Fortunately the
latter motivation prevailed and I made it to the surface. Being a
complete novice at snorkelling, as soon as my head was above the
surface, I breathed in, forgetting to blow out the snorkel; instead of
taking in air I took in a large gulp of seawater. That unsettled me even
further, prompting me to make a dash for the beach. Unfortunately I
omitted to take care walking over the coral, so the soles of my feet
were shredded.

During the time living in the bush on the Narada commune I had regularly
used Hypercal Tincture (a herbal remedy made up of Hypericum and
Calendula). I believe it was this remedy that protected me from the sort
of severe infection that some of the others suffered from because of
leech bites and ticks. I have carried Hypercal Tincture with me ever
since, and still use it. I ascribe the rapid healing of my painfully
wounded feet on that occasion to that remedy.

The loss of access to calm and ease during periods of meditation I
ascribe to my wrong understanding of how to skilfully approach
meditation practice. Through wilful concentration of attention I had
found out how to touch into some delightful inner spaces. However the
way I had approached them, and the manner in which I related to those
initial experiences, meant I couldn't fully appreciate them. It is like
a young person working hard at their first job, making some money, and
then going out and spending most of it on getting drunk. Whatever hadn't
already been spent on alcohol was then stolen while they lay unconscious
in a stupor. At that time in Timor I was confused and I didn't really understand why. I
hadn't been drinking alcohol. Part of what was happening, however, was that I was feeding on sights, sounds,
smells, tastes etc. in a way that meant my heart was being drained. I
recall keeping a journal during that period of travelling and at one
stage, I think a few weeks later, noting down, `I feel as if my eyes are
locked in staring mode.' The lack of restraint meant that my heart's
energy was being stolen. It turned out that getting a rush from staring
into flowers was not such a skilful spiritual practice after all!

Ross and I didn't stay long in East Timor and made arrangements to
travel west, in the direction of Bali. When I think back now about how
we travelled, it strikes me as rather foolhardy. We had heard or read
somewhere that it was possible to hitch a ride on a military barge that
would sometimes travel from Dili to a small East Timor enclave pocketed
within West Timor. Incredibly (it seems now) the Portuguese military
agreed to take us. So, along with a small handful of colourful fellow
travellers and a barge-load of ethnic Timorese soldiers, we spent much
of one day gently coasting along the north shore of East, and then West
Timor, until we reached the enclave. The weather was perfect, the views
stunning, the company somewhat questionable, but we made it. From there,
a group of us hired a Four-Wheeler that drove us to Kupang, the capital
of West Timor located at the western-most part of the island.

\enlargethispage{\baselineskip}

While waiting at the airport for a flight out of Kupang to Bali, once
again I noticed Bill Hamilton. He had a connecting flight already
confirmed so we just nodded and that was that. On arriving in Bali, Ross
and I headed for an area called Legian, which is a short distance west
from the most famous part, known as Kuta Beach. Already back in 1974,
Bali was inundated with Australians, and Kuta Beach was the most densely
populated. Legian, though still relatively nearby, was much quieter. We
took a room in a \emph{losmen} just back from the beach and started to
settle in. A \emph{losmen} is like a motel and ours was very modest but
more than adequate. I imagine I was hoping that now we were somewhat
more comfortable, my meditation would again give access to the clarity I
had enjoyed during those first few weeks following the retreat.

One day, while I was relaxing on the porch of our \emph{losmen}, a young
Canadian fellow who was occupying the room next door, approached me and
struck up a conversation. His name was Randal McCaw. He said he had
noticed me sitting on the beach in the early morning and wondered what I
was doing. I let him know I was meditating. He was happy to hear that
and explained how he and his travelling buddy were also into meditation.
In fact they had just done a course with this English monk, Ajahn
Khantipalo, at Nimbin. Shortly after that he introduced me to his
companion -- it was Bill Hamilton. They must have been on one of the
other courses that had been led around the time I was there. After all
the earlier occasional meetings with Bill, it was good at last to make
more of a connection.

\enlargethispage{\baselineskip}

My hope at the time was to be able to stay on in Indonesia for the full
three months that were allowed, which involved a monthly visa renewal.
Besides taking in the sights I might already have warmed to the idea of
learning to do batik painting. Randal and Bill planned to stay for only
one month and not renew their visas; they were keen to get to Thailand
as quickly as possible. They might have been even more impressed than I
was by what they gained from the meditation retreat. After Thailand,
they planned to travel on to India. A few days later, when we parted
company, I did not expect to see them again. My sights were still set on
Japan.

Riding around on a motorbike was a normal way of seeing the island of
Bali. After I had somehow managed to push aside the memories of the
Gordonton accident, Ross and I hired a bike each. We headed off,
initially to the village of Ubud. A German couple I had met during my time at Narada
had lived near Ubud for an extended period of time and described it as
an especially delightful spot favoured by artists. On the way there we
stopped in the main town of Denpasar and bought some refreshments at the
market. It wasn't long though before we were very much regretting buying
those delicious-looking cold drinks.

Our time in Ubud was spent almost entirely in bed. We were both
violently ill. It went on for days. As I recall the only medicine that
we had access to was of a home-made variety, generously offered by our
host, the owner of the \emph{losmen}. I'm not sure what substances he
chewed up in his mouth before spitting it and massaging it onto our
bellies; it smelt like garlic. Only years later was I introduced to the
more effective medicine Ciproflox (Ciprofloxacin) and have carried it with me
ever since on any significant trip I have taken.

Once we were well enough again to travel, we departed Ubud and spent a
few days exploring the island. We were not, however, in very high
spirits. That illness had taken a toll. Fortuitously, we met a couple of
British expats who took pity on us and invited us to stay in their very
lovely home where we were able to relax and regain some strength.

