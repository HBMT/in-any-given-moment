\chapter{Out Into the World}

A vocations advisor I was taken to see during my time at the college in
Kaikohe, recommended me for a job as a trainee laboratory technician in
Holeproof Fabric Mills. I was probably considered suitable because I had
an eye for colours and did reasonably well at chemistry. The job
involved calculating dye formulae by matching small pieces of cloth with
precise colour swatches, and the `recipes' we produced would then be
used in dyeing vast quantities of fabric. So in 1970 I left home and
moved in with the McLean family who we had known in Morrinsville. Dr.
McLean had moved from Morrinsville to take up a post in a psychiatric
hospital in Auckland and it was conveniently near Holeproof Mills.

One day my attention was drawn to a small book on a shelf in the room in
which I was staying. Maybe it was the title that caught my attention,
I'm not sure. It was a book by Professor Carl Gustav Jung. As I recall,
as soon as I started reading that book I was taken aback by the fact
that he was criticising the Christian church. In my mind the only people
who would openly criticise Christianity were people of no consequence.
Yet here was someone of considerable consequence, a well respected, well
published, world renowned psychoanalyst. This was the first time I felt
I was allowed to start questioning the culture, or even the cult, in
which I had been raised. Thank you very much, Professor Jung and Dr.
McLean.

For the next few years a depressingly dark mood dominated my life. Had I
known then what I know now about depression, I might not have made such
a problem out of it, and, even though I didn't use the word
`depression', I did make a problem out of it. The perceptions of the
world for anyone who is having to endure depression are obviously
negative. However, if instead of demonizing depression and expecting
those suffering from it to get over it, perhaps we could consider it as
a sort of `holding pattern' that is protecting the sufferer from an even
worse state of disintegration. With such a perspective we might be able
to view the condition more creatively, less judgmentally. Because our
society holds up happiness as an indicator of our personal worth, we
readily fail to appreciate the functional value of pain. Not all pain is
pathological. If, for instance, I stub my toe and don't feel any pain, I
won't inspect the wound and see if it is likely to become infected. Pain
can serve a helpful purpose. It says, `Pay attention here.' Besides, the
pain of depression could well be preferable to the much worse
possibility of a psychotic breakdown. During a period of so-called
depression we could be building up strengths such as the self-respect
which comes with cultivating integrity, the confidence associated with
developing mindfulness, and that special kind of nourishment that comes
with cultivating wholesome friendships. Once those strengths are in
place the depression might have served its purpose and dissolve.

Those around me back then didn't seem to notice how despairing I was. I
like to think I was quite successful in acting as if everything was OK.
In my case I would say that put me at an advantage, since if others had
known how negative I felt they would probably have tried to fix me. (I
realize that this doesn't apply to everyone). Sometimes we really need
to endure through the pain of life until we get the message -- and I do
see pain as a message -- which is one of the many reasons why I have
such great faith in the Buddha. At least within the practice of
Theravada Buddhist teachings there is a consistent emphasis on paying
attention to suffering. It doesn't encourage practitioners to dwell too
much on fantasies of possible future states of enlightenment. Many years
later, when a book called \emph{I Am That,} which records teachings by
the Advaita Vedanta teacher Nisargadatta, was popular within our
monastery in England, I read just one, or perhaps two pages, before I
put it down. Not that the content wasn't appealing: it was too
appealing. I was afraid that my mind might start trying to imitate what
the teacher was saying and that would get in the way of making my own
discoveries. This is not to say others shouldn't read Nisargadatta and
the like if they find it helpful. People are different and I am one of
those people who likes to find things out for themselves. Hearing about
the realizations of others can indeed be encouraging, however we need to
be careful not to feed on the good feelings that arise out of receiving
such encouragement. Suffering is what we have, what we can know here and
now. Right practice means preparing ourselves so we are ready to
accurately receive suffering when it impacts us, and to see beyond it:
seeing in a way that leads to letting go.

