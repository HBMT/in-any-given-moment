\chapter{Time in Thailand Coming to an End}

The difference in training style didn't cause me to think, even for a
minute, that I would be better off living at Wat Bahn Tard. Tan Ajahn
Chah's employment of group practice as a training tool made very good
sense to me. Years later I adopted a similar attitude in how I ran the
monastery in Northumberland. In the first few years of monastic
training, most of us want to be able to do what we want, when we want --
something like, `doing your own thing, in your own time.' However, if
doing our own thing had been so great, we wouldn't have gone to live in
a mosquito-infested forest with one meal a day and excessively sweet
drinks in the evening. We went because we understood, at least to some
degree, that a commitment to `my way' doesn't work. Hence, training
involves going against `my way'. And living together with other monks,
often with very different preferences, is an excellent way to get in
touch with `my way'. Getting in touch with `my way' is the first step;
then we have to learn to let go of `my way'.

I was pleased to be back again at Wat Pah Nanachat with what felt like
family. During my absence, the BBC had visited Wat Pah Pong and produced
a film for the Open University, called \emph{The Mindful Way}. Not long
after that film was shown in Britain, an elderly English woman who saw
it was so inspired that she quickly caught a flight to Thailand and
found her way up to Tan Ajahn Chah's monastery. On the day she was due
to depart for Britain, I happened to be at Wat Pah Pong waiting under
Tan Ajahn Chah's kuti, ready to join him on morning alms-round. An
American \emph{maechee} who was living there at the time, brought this
English guest to pay her respects to the teacher and, at the same time,
asked if he would kindly say something like `goodbye' into her tape
recorder. Tan Ajahn Chah took the recorder in his hand and delivered a
profound and loving message lasting about fifteen minutes.

Since the \emph{maechee} was only able to speak Laotian, and not Thai, I
volunteered (or was asked) to translate this short talk into English.
After returning from alms-round I rushed over to my kuti to translate it
in time to give it to the departing guest. That talk is now printed in
`\emph{The Collected Teachings of Ajahn Chah}\cite{collected}',
page 233, with the title,
\emph{Living with The Cobra}. It is a brilliant, succinct summary of the
right way to approach practice. My favourite part of the talk is where
Tan Ajahn says,

We extinguish fire at the place at which it appears. Wherever it is hot,
that's where we can make it cool. And so it is with enlightenment.
\emph{Nibbana} is found in \emph{saṃsara}. Enlightenment and delusion
exist in the same place, just as do hot and cold. It's hot where it was
cold and cold where it was hot. When heat arises, the coolness
disappears, and when there is coolness, there's no more heat. In this
way \emph{Nibbana} and \emph{saṃsara} are the same.

I have listened to this talk many times over the years and continue to
enjoy the vitality and the compassion in Tan Ajahn Chah's voice. When I
reflect on it these days, what I hear Tan Ajahn Chah saying accords with
how I have come to think about the two different ways of approaching
practice: one way could be called a `goal-oriented approach' and the
other a `source-oriented approach'. People are different and naturally
have different ways of picking up the practice. Some, it seems, benefit
from having a clear idea of the goal spelt out for them, and the stages
of reaching that goal. They feel energised by the perception that
they're making progress along the path. For others, ideas of a goal can
serve a useful purpose in the beginning, but the deeper they go in
practice the more such ideas get in the way. For them what is more
important is letting go of all approximations of a goal, letting go of
even wanting to progress, and instead growing in ability to be more
intensely and accurately present in this moment with whatever is
happening. This orientation of effort is what I understand Tan Ajahn
Chah was pointing to by emphasising that suffering and liberation exist
in the same place. We need to resolutely release out of all habits of
being someone trying to get somewhere.

It was around this time of that year that Tan Tiradhammo began compiling
the first collection of translated talks given by Tan Ajahn Chah which
were eventually printed in a small booklet called \emph{Bodhinyana}. A
fledgling sangha was beginning to settle in, in Britain, but there was
no idea that in only a very few years a worldwide community of
monasteries\cite{forestsangha} would be established. Here
at Wat Pah Nanachat we were still sweeping leaves and pulling water from
the well every day, and dealing with the almost constant challenges that
such a simple and disciplined life is guaranteed to produce.

On occasions I would still become caught up in worry about my health,
especially to do with the consumption of sugar. Eating only once a day
around 8 o'clock in the morning, and sweating a lot because of the heat,
meant that the temptation to gobble vast amounts of sugar in the evening
was strong. I felt sure it was not a sensible thing to be doing, but I
guess I was addicted. At some stage that year I underwent a medical test
to assess my sugar metabolism, and the result indicated that I either
had a tumour on my pancreas or was suffering from `functional'
hypoglycaemia. After further tests it was agreed that there were no
signs of my having a tumour, so the doctors prescribed a regime for
dealing with hypoglycaemia. They recommended I eat several small meals
throughout the day, including in the evening, and see if the sugar
metabolism would stabilize.

This event coincided with a fad passing through the monastery inspired
by a book called \emph{Water Of Life: A Treatise on Urine Therapy}, by
John W. Armstrong. Eating several small meals a day, and especially,
eating in the evening, were not options, so I determined to abstain from
consuming any sugar at all for the duration of the three months Rains
Retreat of that year; also I committed to drinking urine and often
received a massage of fermented urine. I have only a very vague memory
now of undergoing another test at the end of the three months but what I
recall is that my sugar metabolism was by then perfectly normal. Not
only is the drinking of urine mentioned as an allowable medicine in
Buddhist scripture, but it turns out that it is a well-known practice,
especially amongst yogis in India. The book mentioned that the practice
was not suitable for anyone with high blood pressure, and also I imagine
it says it shouldn't be used by anyone who is taking other medication.

After that Rains Retreat I asked Ajahn Jagaro if I could go back and
spend some time in New Zealand. I had completed the initial five year
period of training (\emph{navaka} stage), and besides, my parents had
offered me a flight ticket. In correspondence with them I said I would
like to accept, but only on the condition that I could return to
Thailand. I don't imagine Ajahn Jagaro had any objections, because I
went over to see Tan Ajahn Chah who, at that time, was staying in a
nearby village monastery, Wat Gor Nork. It was a pleasure to be in his
company, and I don't recall him raising any concerns about my going for
a visit home.

On another occasion at Wat Gor Nork, Ajahn Jagaro himself went over to
speak with Tan Ajahn Chah. He was accompanied by Tan Thitiñano, a
French-Italian monk, and Tan Gavesako, a Japanese monk. I wasn't around
on that occasion but I have a tape recording of the conversation. The
three young monks were questioning Tan Ajahn Chah about what exactly is
meant by the term `Original Mind' and what exactly is contemplation. A
translation of this conversation is printed on p 475
in \emph{The Collected Teachings of Ajahn Chah}\cite{collected}.
Here is my favourite extract from that dialogue,

\question{Q:} I still don't understand. Is true contemplating the same as thinking?

\answer{Tan Ajahn Chah:} We use thinking as a tool, but the knowing that arises
because of its use is above and beyond the process of thinking; it leads
to our not being fooled by our thinking anymore. You recognize that all
thinking is merely the movement of the mind, and also that knowing is
not born and doesn't die. What do you think all this movement called
`mind' comes out of? What we talk about as the mind -- all the activity
-- is just the conventional mind. It's not the real mind at all. What is
real just IS, it's not arising and it's not passing away.

This teaching inspired me to develop the meditation practice I have used
for many years, of enquiring, \emph{in WHAT is all this taking place?}

I departed Thailand for New Zealand on 25th November 1979, with a ten
day stopover in Sydney. On landing in Sydney airport and exiting the
plane, the cooler air seemed to trigger a sensation which gave me a
feeling like my brain had started working again. It was as if I had been
partly brain dead.

This was my second visit to Wat Buddharangsee where Tan Phra Khru
Mahasomai was still living. He was still smiling. During those few days
there I was encouraged to give a talk at one of the local Buddhist
societies, which didn't exactly fill me with delight. As I recall,
however, I might have been forewarned, either by Ajahn Jagaro, or
another New Zealand monk who had been training in Korea and was also
currently home visiting, Bhikshu Ham Wol. Bhikshu Ham Wol and I had been
in correspondence, and he was arranging accommodation for me in Auckland
before I ventured north to the Bay of Islands to see my parents. Whoever
it was who warned me about possibly giving a talk in Sydney, also
helpfully advised me to approach it as giving voice to my own inner
contemplations: instead of just inwardly pondering on a theme, give your
ponderings a voice and that will be your talk.

On 5th December I took a flight to New Zealand and landed in Auckland.

