\chapter{A Reorientation}

Previous moments of insight and reflection that had spontaneously
occurred in my life were all surface level shifts in perspective
compared to what happened on that retreat. They had been like a thick
fog momentarily clearing, indicating that there was in fact some sort of
a path worth following. The shift that occurred on the third day of that
retreat was like a real signpost, and it precipitated a fundamental
reorientation of my life. It indicated clearly the goodness and clarity
that are potentially available. Looking back now I see this episode as
representing the beginning of my learning to communicate in a new
language. It was the language of the heart and not just of the head. It
is not really possible to live the spiritual life if we are still trying
to find our sense of identity/security in our heads, in mere
approximations.

Without my knowing it at the time, this was also the beginning of my
first concrete experience of the pitfall known these days as `spiritual
bypassing' (more on that later).

It is difficult to be objective about what happened. For some beginners
in meditation such an experience might not have seemed special at all,
just a brief acquaintance with a somewhat deeper sense of contentment.
In my case, by comparison with the ordeal I had been going through, it
was a wonderfully significant encounter with inner potential -- one I
had not the vaguest idea was possible.

Ajahn Khantipalo was very enthusiastic and generous in his efforts to
encourage the practice of formal meditation. The retreat I attended was
just one of a series of five or maybe seven retreats in a row that he
taught there at that old farm house just outside Nimbin. We began each
morning sitting quietly listening to a tape recording of a group
reciting one of the names of the Buddha, to the tune usually used in
Pure Land Buddhism in praise of Amitabha: \emph{Sakyamuni, Sakyamuni,
Sakyamuni, Sakyamuni, Sakyamuni, Sakyamuni, \ldots.} This was followed
by alternating periods of sitting meditation, walking meditation and
supportive talks by the teacher. There was a break for breakfast and
another for lunch. It was my first experience with the disciplines of
not eating after midday, keeping silent, and not making eye contact. I
don't recall resisting the discipline. That is not to say I found the
seven days easy. Perhaps I already had a degree of faith in the Buddha
because I did take it seriously.

On the third day, during a period of walking meditation, having been
diligent in my effort to concentrate, I noticed that my mind was quieter
than usual. At that point the voice inside me that likes commenting on
everything, came up with the observation, `there is just awareness', or
it might have been, `there is just knowing\emph{.'} This was almost
immediately followed by an enquiring voice, `but who is aware?', or `who
knows?'\emph{,} at which point it felt like the mind dropped. That shift
in perspective was the signpost. A quality of peacefulness appeared that
seemed to require no maintenance: it was just there, and I fell into it.
There were no drugs involved, no group therapy dynamics; the main thing
seemed to be an adjustment in the quality of attention.

When thinking about it afterwards, as I definitely did, the image that
came up was that our experience of existence is similar to focussing the
lens of a camera. What is most important is not so much having to keep
changing the objects of attention, but rather whether or not we can view
those objects clearly; and that clarity, or lack of it, does not depend
on anyone or anything outside of ourselves.

Great gratitude to Ajahn Khantipalo, and thank you, Danny. What would
have happened in my life if I had stubbornly resisted the invitation to
go on that retreat?

