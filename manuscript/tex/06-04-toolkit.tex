\chapter{Our Spiritual Toolkit}

Arriving back at Chithurst I felt renewed and revitalised. From now on,
my practice was more about working with a quality of feeling awareness,
in touch with the body, a much broader perspective than viewing life
from my head. (Of course I hadn't previously been aware of the degree to
which I was identified with my thinking mind). It no longer mattered
quite so much what the sensations were -- gladness, sadness, joy or
sorrow -- the task was how to receive them, how to allow them. Gradually
my ideas about what awakening meant were changing; now I was more
interested in `unobstructed receptivity of everything', or `unobstructed
relationship with everything'. The idea of striving towards some
imagined experience in the future really made little sense to me. This
didn't mean I abandoned all notions of a goal; it meant my relationship
with those notions was changing.

It felt as if up until that point in time I had been listening to music
with the bass turned down. Now the bass was turned way up! Aliveness.
Instead of trying to be free from painful feelings such as anticipation,
for instance, I was now interested in how to feel whatever feeling I was
feeling, without adding or taking anything away from it: learning what
it meant to be \emph{free to} feel that which I was feeling, rather than
struggling to be \emph{free} \emph{from} certain feelings. The feeling
of anticipation, for example, is just a feeling; but there is a space in which that
feeling is arising and ceasing. The feeling is not ultimate; there is
also awareness of the feeling. Now I felt like I had a powerful new tool
in my spiritual toolkit: embodied awareness.

This new tool didn't suddenly absolve me of the pain of guilt and
self-doubt, however. There were still periods when I struggled with a
sense that I was about to be overwhelmed by pain. Sometimes I would have
to tell myself, `Just because I feel bad, does not mean I am bad'. The
bullies of guilt and self-doubt, along with many other apparent
obstructions, didn't disappear, but there now seemed to be a chance that
we could get to know each other.

As a craftsperson will have a variety of tools in their toolkit, so
those committed to awakening require a variety of techniques and skills
to deal with the many challenges he or she is going to encounter on
their quest. Since everyone is different in temperament and talent, we
need to equip ourselves according to our own conditioning. In my case,
it became very apparent that I had been seriously out of touch with my
body, so I needed skills that addressed that particular imbalance. The
breathing discipline I learnt that year in New Zealand helped.

So, too, did frequent visits to see a Vietnamese acupuncturist in
London, called Thong. That he was a Buddhist monk within the Mahayana
tradition and a Tai Chi teacher were also significant. For a period
during the Vietnam war he had been imprisoned, as a monk. After having
been released from prison he disrobed, and, before leaving Vietnam to
come to Britain, married and had a family. Once his family had grown up,
he again requested the monks' Precepts. By that time Thong already had
an acupuncture clinic established in London and was well-known as a
skilled Tai Chi teacher. After many years, I continue to practise the
Qigong form that he taught me. And I believe I continue to benefit from
the many sessions of acupuncture and the traditional Chinese herbal
remedies that he kindly offered me.

Whatever understanding of the Buddha's teachings we might have, if our
body is not in harmony, then life will be a struggle. Maybe some of
those struggles are kammic and unavoidable, but perhaps some of them are
not necessary.

In Theravada Buddhism it is taught that there are three types of
illness: one from which you will recover whether or not you take any
remedy; another from which you will recover if you take an appropriate
remedy, and from which you won't recover if you don't take the remedy;
and the third, where whether you take any remedy or not makes no
difference, since the illness is a result of kamma.
(For a more literally accurate interpretation of what the Buddha said, see
Bhikkhu Bodhi's translation of the Anguttara Nikaya -- `\emph{The Numerical
  Discourses}', Somerville, Wisdom \mbox{Publications}, MA, USA, 2012, Book of Threes,
`Patients', page 217).
Thank you, Thong,
for those treatments, and remedies; for teaching me the Qigong form;
also for your strength and gentleness.

Somewhere I heard or read that, within certain schools of the Tibetan
Buddhist tradition, they won't even introduce you to meditation practice
until you have completed one hundred thousand prostrations. It makes
sense to me now why, in Zen Buddhism, so much attention is paid to the
sitting posture during meditation; if you begin to droop it could result
in your receiving a wack. When Tan Ajahn Chah returned from a visit to
America, he spoke enthusiastically about stories he had heard of the
Chinese Patriarch monk Venerable Bodhidharma. Tan Ajahn Chah was
impressed by how, if Ven. Bodhidharma asked you a question and you
answered it wrongly, you received a whack; if you answered it correctly,
you received a whack; if you didn't answer it at all, you received a
whack. `As for us Theravadins', Tan Ajahn Chah said, `we just carry on
talking about Dhamma, saying it is like this and it's like that, and so
on.' Tan Ajahn Chah also wanted us to get out of the head and come back
into the body.

At one point, I think it was in 1989, Ajahn Sumedho received word that
Tan Ajahn Chah appeared to be dying, and so he quickly departed for
Thailand. As it turned out, it wasn't until January 1992 that Tan Ajahn
Chah eventually passed away. Just before leaving us on that occasion,
however, Ajahn Sumedho had turned to me and, almost as a passing
comment, said that he wanted me to be his substitute at a one-day
seminar due to take place at the Buddhist Society in Eccleston Square,
London. The theme for the day was, `Several Schools, One Way'. By then I
should have been used to how Ajahn Sumedho would occasionally throw a
googly, not just to me, but to anyone in the community. I have never
figured out whether he did that sort of thing as a strategy to test our
agility, or if perhaps he wasn't even aware that he was doing it.
Personally, I wouldn't have expected a relatively inexperienced monk
like me to be standing in for someone of Ajahn Sumedho's stature on such
a public platform -- at least not without some sort of a discussion. An
august line-up of very senior teachers had been planned, including Ven.
Myokyo-Ni representing the Zen tradition, a famous Rimpoche for the
Tibetans, and a well known elder from the Pure Land School. This was an
invitation that did intimidate~me.

On the day of the seminar, I discovered that Ven. Myokyo-Ni was to speak
first and then I was to follow. Ven. Myokyo-Ni gave an inspiring talk,
as always, but all I can recall now was that it was on the Four Noble
Truths. From other conversations I have had with her, I know she never
wanted to be described as a Zen practitioner -- rather she insisted that
she was a Zen Buddhist practitioner. Before being inspired by Christmas
Humphreys and then following Japanese Zen Buddhism, she was already studying
Theravada Buddhist teachings, and always maintained that having a good
grounding in the original teachings was essential. That was all well and
good, of course, but on the occasion of that Several Schools, One Way
seminar, what was there left for me to talk about. One of the things I can still remember about my contribution on the day, was that when they pinned
a microphone onto my robe at chest level, I imagined it was going to
amplify the sound of my heart pounding.

During an interval between the talks, Ven. Myokyo-Ni and I went outside for
a walk by the Square. I took the opportunity to ask for her thoughts on
the situation in which we found ourselves: practising traditional forms
of Buddhism in the West, in an environment that was not always welcoming
or supportive. She turned and focused her gaze on me
and said, `Venerable, when you are doing the real practice, it can
feel like too much, too soon!' Thank you again, Ven. Myokyo-Ni.

On another occasion when Ajahn Sumedho asked me to accompany him to the
Buddhist Society, (not just to deputize for him), his talk had to be
interrupted because there was a distracting amount of smoke rising up
from behind where he was seated. As usual, he was sitting on a zafu in
front of the shrine in the main meeting room of the Society. There was
no mistaking it being smoke, and it wasn't a small amount which could
have come from the sticks of incense. What had happened, it seemed, was
that when he had lit the candles and incense as a preliminary to the
occasion, a spark must have dropped onto his zafu, igniting it. Somebody
helpfully removed the smoking zafu and the talk continued.
Interestingly, quite a long time afterwards, when we went to leave the
Society buildings, we found a bright glowing orb sitting on the pavement
outside the front door. That helped me appreciate why kapok filled
zafus, although preferable to sitting on a polyester-filled one, are
considered a fire hazard; once they start burning it is difficult to put
them out.

