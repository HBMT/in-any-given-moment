\chapter{The Missing Ingredient}

After having lived at Wat Hin Maak Peng for several weeks, one evening,
part way through a perfectly normal puja, a shift in perception suddenly
occurred. Without trying to do anything in particular other than follow
along with the chanting as usual, a joyous `just-so' appreciation of an
altogether new perspective on experience manifested. Years later, in an
attempt to find words for that altered perception, I described it as a
shift from finding identity as someone having this or that experience,
to abiding \emph{as} the context of all experience. Witnessing from this
altered perspective, any and every experience that could ever arise,
would be seen simply as `content' within that `context'. It was joyous,
unexpected, and felt profoundly significant, but at the same time
utterly normal. There was nothing dramatic about this new perspective
though at the same time it felt like it changed everything.

When I reflect back now on what happened that evening, I see there never
really was any missing ingredient in my life. The struggle to find
something that I hoped or assumed would make me feel OK, was never going
to provide true satisfaction. What was needed was a shift in the way I
viewed experience. That new perspective was the greatest gift I have
ever received.

When the chanting session was over, I attempted to share with the
translator monk something about what had occurred. I don't think I said
very much, but it was enough for him to suggest that I should explain
what had happened to Tan Ajahn Thate. We gathered as usual at Tan
Ajahn's kuti. Initially everything went as usual, until my translator
began to describe what I had shared with him. At that point Tan Ajahn
stopped the massage and sat up. That was unusual: having the Ajahn
interrupt his massage and address one of the most junior monks in the
community. By this stage I expect my mind had returned to finding
identity as `content' again, nevertheless I tried to convey what had
happened. Tan Ajahn Thate was encouraging and he said something along the
lines of, `from now on your practice should be about remembering quicker
and to not be caught up in all the activity of mind.'

The next thing I can recall was walking back to my kuti, and, similarly
to what had happened at Wat Boworn a few weeks earlier, I started to
reflect on the things I had just been saying. Very quickly my mind
became possessed with self-doubt and guilt at the thought that I might
have exaggerated the extent of my understanding and, in so doing,
transgressed one of the monks' rules about making an unfounded claim to
a supermundane state. The terror that was released was indescribable.
What had happened at Wat Boworn was like a heavy rain storm compared to
this full-on hurricane, and the ferocity of this storm was beginning to
wreak serious havoc throughout my body-mind. I had no idea such an
intensity of mental and emotional anguish was possible. This was terror.
The contrast between the sublime, selfless, incontrovertible OK-ness of
earlier that evening and this devastating sense of being drawn down into
hell, could hardly have been more pronounced.

Many years later, I came to appreciate that such an unexpected but
profound opening as that which occurred during that evening puja, is not
uncommon. Even without any specific spiritual preparation, it is not
rare for people to find themselves spontaneously having to come to terms
with such a completely new perspective on reality. It seems that for
some people the heart remains open and they learn how to integrate this
new way of seeing. Then there are others, like me, who go through a
powerful shift in perspective and sense the beautiful possibility of
living in expanded awareness, but then find their heart closes again,
and they have to endure a period of painful disorientation. In the case
of the latter, although the clarity that came with the expanded state
was no longer accessible, something precious still remained. Perhaps it
is like this: if we were to throw a pebble into a very deep well and
listen until eventually we heard the sound of the splash as the pebble
entered the water -- afterwards, even though the sound of the splash has
gone, an appreciation of the depth of the well remains. Whatever one
might say about it, or however one might attempt to explain it, that
shift is something for which I am deeply grateful.

The torture that followed did indeed wreak havoc in my physical,
emotional, mental and relational worlds. Looking back now with
hindsight, I can view it as at least in some ways cleansing. Some people
carry with them what would classically be called a lot of old negative
kamma. I prefer to think of it as old, unmet life. If you find
yourself encountering such difficulties, the most helpful thing to do is
resolve to learn how to meet it, receive it, allow it, until the lesson
we need to learn has been learnt; at that point letting go might happen.
In other words, resolve to take full responsibility for it. Desperately
trying to let go of old unmet suffering is another way of trying to
get rid of suffering; if we do that, then the burden is likely to become
heavier. We would be wise to reflect on how the Buddha taught: \emph{`It
is through not seeing two things that we remain lost: not seeing
suffering and not seeing the cause of suffering.'} We need to learn to
view suffering as our teacher, not as an obstruction -- and to not
merely pay lip service to this new attitude. What is called for is a
radical re-education of perception so we come to view suffering
constructively, not as a symptom of failure.

My burden was painfully heavy. Where did it come from? Was I that bad a
person that I deserved to suffer so intensely? Fortunately, I was
prepared enough to be able to endure it without making too big a mess. I
could have been better prepared, but I could also have been much worse.
The sense of belonging to a community that was worthy of respect was
precious -- likewise, the training in restraint that comes with the
monastic discipline (\emph{vinaya}). Then there was the structure of the
daily routine: the chaos that prevailed in my inner world was made more
manageable because, on the level of the outer world, things were so
predictable and stable. The significance of this last point cannot be
overestimated.

Fortunately I didn't try to tell anyone what was going on. If I had, I
imagine they would have been unnecessarily worried about me. This is not
to say that having the right person who shared the same language and an
appreciation for the psychological factors involved, wouldn't have been
helpful. Indeed that could have helped, but there wasn't any such person
around.

On the feeling level I was condemned to excruciating hell as a result of
having done something unforgivably bad -- that is, making an unfounded
claim to a supermundane state. Conceptually I was intensely confused.
Yet, in some dimension, perhaps we can say at a heart level, there was a
feeling of trust that everything was OK. I had a dream during this
period that gave me confidence. In the dream, which I can still recall,
I gave birth to a child. As far as I was concerned, all was well, but
what puzzled me was that my friends who were standing around were
disappointed. They had expected the child to be a boy when it was a
girl, or maybe a girl when it was a boy, either way they were
disappointed. What I am not able to recall accurately now, is whether my
translator told Tan Ajahn Thate about the dream, or I told him, or I
came up with my own reading of what the dream was saying. Whatever
actually happened, the message of the dream, as far as I was reading it,
was that I had given birth to my practice, but the consequences would
not accord with my preferences; things would not turn out how I had
hoped or imagined.

The weeks and months, indeed years, that followed were either varying
degrees of hell, or feeling as if I was about to fall into hell. I
recall, after one painful period of sitting meditation inside my kuti,
that when I walked outside, I saw that the Mekong river was still
flowing by -- and that provided me with some reassurance. The word that
comes to mind when I try to recall the rest of my time at Wat Hin Maak
Peng is `intense'. On one day I felt convinced that the biggest issue I
had to grapple with was anger. Then on another day it would be fear,
then greed. My digestion was not good, and gradually I was becoming
emaciated. In the afternoon, some of the monks and novices would
occasionally gather to brew up various concoctions to drink. If we were
lucky there would be cocoa, and the brew-up sometimes extended to making
fudge out of sugar, cocoa and salt. The day of my twenty-fourth
birthday, approximately midway through the Rains Retreat, was one such
occasion. I remember at the time thinking I was overdoing the
consumption of fudge; I still had a lot to learn about restraint. In the
middle of the night that followed, I was awakened to find the walls of
my kuti seething with ants. Everything was covered in them -- I was
covered in them -- and trying to brush them off was futile. Were they
attracted by all the sugar I had greedily consumed? In that already
heightened state of shame and anxiety, this incident only served to take
my anguish to another level. Eventually I abandoned my kuti, took my
robes and went to the main meeting \emph{sala} and lay down on a mat on
the concrete floor in a state of despair. I reflected on how this
birthday marked the end of the second twelve-year cycle of my life.
There seemed to be some sort of poignancy to this and to how I was
feeling so racked with despair: there were thoughts of death in my mind.

I managed to survive the full three months of the Rains Retreat, though
my emaciated condition had not escaped the attention of the other monks.
Almost immediately after the retreat ended, the parents of my translator
monk companion kindly offered to fly me down to Bangkok. When I took
leave of Tan Ajahn Thate, his parting words were something like, `you
are in a vulnerable condition, be careful.' I didn't hear his words as
judgemental, but full of kindness and concern.

Soon after arriving at Wat Boworn I was taken to the \mbox{Chulalongkorn} Hospital
and placed in a private room and prepared for all sorts of tests in an
attempt to find out what was causing my physical symptoms. Apparently
the King kept a fund at Wat Boworn that took care of the monks if they
were unwell, and at some stage I was told that my stay was being paid
for from that fund. I was unwell, but the nature of my illness was not
obvious. Was it a parasite, or digestive problems, or was it old
unreceived life coming to the surface waiting to be received? The
clinicians were thorough and extraordinarily considerate, but in the end
they found nothing. I returned to Wat Boworn feeling utterly dejected.
This was the beginning of an approximately fourteen-year-long holding
pattern of contained disorientation.

