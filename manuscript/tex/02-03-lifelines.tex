\chapter{Lifelines}

Around 1971 I enrolled to study psychology, sociology and education at
Waikato University in Hamilton. Choosing that university had nothing to
do with my brother David living in Hamilton -- he and I had never been
close. Perhaps it did have something to do with my having heard Waikato
University had a psychology department that favoured a Humanistic
approach. I had felt encouraged after hearing a bit about Gestalt
therapy which made it seem a more practical approach than that of the
Behaviourists.

Early on, after enrolling at the university, as a result of an
unexpected turn of events, I had met Professor Jim Ritchie who invited
me to join in with a regular extra-curricular sensory awareness group
that met at his house on Wednesday nights. I wasn't totally unfamiliar
with the theory of sensory awareness exercises and encounter groups, but
I had no experience of them. We were a group of about eight to ten,
mostly students at the university with perhaps some ex-students, and we
would experiment with techniques employed in Gestalt therapy. I suspect
those meetings contributed significantly to my beginning to find that I
had my own voice: I had my own desires and opinions. Prior to that, the
conventional sense of who I was had been moulded into a virtual human
being: I said what I thought I was supposed to say and did what I
thought I was supposed to do. In today's parlance it would be said that
I was totally inauthentic. No wonder I felt so confused. I had no
dependable frame of reference.

Although it is a digression, perhaps at this point I could say something
about what I mean by `the conventional sense of who I was' or about how
I understand `self'. In Buddhist teachings we hear a lot about not-self,
and all of the world's major religions emphasize `selflessness'.
Contrasting that, in the world of psychotherapy, and perhaps especially
so within Gestalt Therapy, there is a strong emphasis on developing
`self'. On the surface these contrasting perspectives could appear to be
in conflict, however, if we look deeper there need be no problem. If we
can appreciate the sense of self (or the ego or the personality) as
predictable patterns of mental and emotional activity that give an
individual a fulcrum or a frame of reference by way of which they can
relate to the world, then we can appreciate its relative function. The
problems arise when we take this structure as ultimate or as permanent
-- as who and what we are: we identify \emph{as} it. In the early stages
of life, if all goes well enough, a conventional sense of a separate
self is configured in an individual and they get on with the various
physical, mental, emotional and relational aspects of their life,
hopefully without too much trouble. That doesn't mean they won't
struggle with the normal suffering of loss, of disappointment, and
eventually of death that life gives us all; addressing those issues is
the domain of a deeper dimension: what we might call the spiritual life.
Real difficulties do appear if during the early stages of life things
don't go well enough, and if the child experiences more distress than
they are ready to handle. What psychotherapists can potentially do well
is to help someone who finds themselves with such unacknowledged
difficulties to come to terms with their past conditioning. A skilled
therapist can help the individual reach a stage of having a well enough
integrated sense of individuality, or self. From that stage onward, they
will be better prepared to deal with the vicissitudes of life. Also,
they will be more ready to take on the tasks of the spiritual journey --
that is, cultivating selflessness. The cultivation of selflessness means
learning to truly see that the self is not-self -- it is not ultimate:
what we refer to as the self, is activity taking place within a larger
context.

Now back to Hamilton.

Jim, as I came to know him, had a holiday property at Whale Bay, near
Raglan. He and his wife Jane and their children would often spend time
there. Also some of the Wednesday group spent time there. Whether it was
with them or on some other occasion, I have a strong memory of one day
noticing a poster on the wall in one of the cabins on that property. All
I recall of the poster was the word printed large, \textls{AWARENESS}. Perhaps
the text under it gave some extra meaning to the word, however, this was
one of those significant moments when something registered: this
mattered. There is awareness. I can't say what it was about that moment
that meant so much, but nearly 50 years later, I am still able to
visualise that word on that poster and recall the moment -- so thank
you, Jim Ritchie.

Jim had a wide circle of friends. He and Jane were well known for their
work studying child-rearing patterns in New Zealand. Also he was close
to some of the local Maori communities and worked hard to get them
recognition. One day when I was at his house, James (Hemi) Baxter turned
up and I can recall the two of them rolling around the living room floor
hugging and playing like two small boys. James Baxter was one of the
best known and most highly regarded poets of New Zealand. He had made
the unconventional choice of living in a commune called Jerusalem, near
Whanganui.

During this period in Hamilton I was offered three lifelines. The first
came from a friend, Jim Smith. Jim Smith and I had met before I came
down to Hamilton and he was also studying at Waikato University. He had
a child to support and I remember helping him one evening (maybe more)
on a night job he had as a cleaner in a school. He gave me my first
Buddhist book: \emph{The Way of Zen} by Alan Watts. What a gift! Here
was a spiritual teaching that honoured the heart's yearning for insight
into that which matters most in life, and it didn't demand naive
acquiescence. There was one phrase from that book that stood out for me;
as I recall it was presented as a summary of Zen Buddhist teachings. It
said something to the effect, `When you sit just sit; when you walk just
walk; above all don't wobble!' Although it was already quite late at
night when I read that, I was so inspired I simply got up and went
walking around the streets of Hamilton. Of course I was thoroughly
wobbly, not just when I was sitting or walking, but here was another
hint of a useful direction in which I could be seeking.

It might also have been Jim Smith who gave me a book by Krishnamurti.
There was a Krishnamurti Society at our university but for some reason I
was not drawn to it. What I did value though was reading in one of
Krishnamurti's books where he made the observation that most people
assume gratification of desire and satisfaction are the same thing, when
they are not. I don't know what else he said. I suspect I didn't read
more than the first page, but that much was helpful and has stayed with
me. It fits with so many other teachings I have received over the years,
including of course, teachings by Tan Ajahn Chah. With gratification we
get momentary relief from the pain that arises from being caught up in
wanting, being identified with wanting. True satisfaction is what arises
in those who have let go of all identification with the movement of
wanting. They know, and can wisely and compassionately accord with, the
reality of wanting.

Although I have said reading was challenging, now that I think of it
there were a few books that I found nourishing. RD Laing's \emph{Sanity,
Madness and the Family}, which I probably had to read as part of the
psychology course, also his, \emph{The Divided Self}, and \emph{Knots};
Alan Watts' \emph{The Taboo Against Knowing Who You Are}; Paul Reps'
\emph{Zen Flesh, Zen Bones}, and his \emph{Square Sun, Square Moon}.
None of these came close to giving me the missing ingredient, however
they did provide a bit of colour at a time when my life was painfully
grey.

Music was also providing some colour, though these days I am less clear about
what I was listening to. I can however fondly recall \emph{The Incredible String
  Band} and in particular \emph{The Water Song}. I am happy to still have the
words about flowing floating around in my head.

The second precious lifeline thrown my way came from a Canadian woman,
Lynn, whom I knew from the Wednesday group meetings at Jim Ritchie's
place. She generously paid for me to receive initiation into
Transcendental Meditation. It might in fact have been her partner David
who paid, I'm unsure. TM represented something to me. I don't know if
the mantra I was given was genuine or something that the business-like
gentleman who gave it to me had just made up. I confess that I failed to
abide by all the `terms of use' prescribed for TM meditators, and from
time to time I continued to indulge in various forms of unwholesomeness.
Nevertheless, it represented and introduced me to the discipline of
attention. Thank you very much, Lynn and David.
I cannot possibly estimate the value of that introduction to the world of meditation.
These days Lynn and I are still in
contact, although usually only once a year. She has been a dear Dhamma
friend.

Just before the beginning of my second year at university I joined a
group of about a dozen other friends and acquaintances in moving to live
in a couple of houses on a farm near Gordonton, a few miles outside of
Hamilton. It is hard to define what it was that drew us together,
probably a shared longing for something meaningful, something new. My
sociology lecturer and his wife and children were part of the
`experiment'. Also there was a super liberal Anglican minister who had
previously run a communal house in Hamilton. Raewyn was my best friend
there and she worked in town as a secretary. It is probably safe to say
that most of us were influenced by the book The Greening of America\cite{greening}
and maybe by The Whole Earth Catalogue\cite{earth}.
We were looking towards communal living, I expect, as a solution to the sense of there being something
missing -- what I have been referring to as the missing ingredient.

\enlargethispage{\baselineskip}

Around this time the New Zealand Government was experimenting with what
was called \emph{Ohu}\cite{ohu}, a Maori word having to do with community activities. James
(Hemi) Baxter had become somewhat of an inspiration for communal living in New
Zealand, and this wasn't long after Woodstock had happened in America.
In Nimbin, New South Wales, Australia, they had their own version of
Woodstock. However, communal living in Gordonton didn't turn out to be
much fun. As far as the lecturer and his wife were concerned there
needed to be standards kept that were suitable for raising children.
Some of our group could probably have been described as anarchists. I
naively spoke in praise of Mao Tse Tung (though I knew nothing about him
or the horror he was inflicting on his country). Essentially I was
hoping that if we could live communally all society's problems would be
manageable.

I regret speaking in praise of Mao but I don't feel overly judgemental
of the naivete. How could it have been otherwise? It would have been
better if we had had a modicum of humility, but we didn't. We felt
frustrated and desperate to do something, anything. I had converted the
hen house into my accommodation and was energized by fantasies of the
study of psychology at university and finding meaning in communal
living. Unfortunately, as far as I was concerned, the dominant mood in
that commune was one of desperation.

Before long, disillusionment also set in regarding what was really on
offer at university, combined with the difficulties I was having keeping
up with the reading expected of me, which led to my dropping out at the
end of the first term of the second year. That was Easter, and in New
Zealand by that time winter was rapidly approaching. Raewyn and I
decided we would go to Christchurch for the Easter weekend.

\enlargethispage{\baselineskip}

A few weeks after returning from that trip, it was on Raewyn's motorbike
that I was riding when I had a serious accident. That morning when I
borrowed her bike to ride into Hamilton without a crash helmet and
without a license was to impact the rest of my life. Although it was
around the middle of the day, the conditions were very foggy; the
country road was narrow so I took the liberty of riding nearer the
centre of the road than I should have. I had my lights on, but when I
cut through an `S' bend I ploughed into a car coming in the opposite
direction. The next thing I remember was regaining consciousness on my
way into the operating theatre.

On the negative side of things, most immediately it meant that I spent
quite a long time in hospital recovering from the damage done to my
head, shoulder and ankle, and it resulted in many months of having a
full length plaster cast on my right leg. My parents, as you would
imagine, were not delighted when they heard I had dropped out of
university, but were relieved I hadn't been killed in what was a serious
motorbike accident. Raewyn's bike was a write-off; my lovely hen house
accommodation had to be abandoned as it wasn't suitable for the winter;
the insurance company of the person whose car I hit sued me and it took
nearly all the money I received in sickness benefit to pay that off.

\enlargethispage{\baselineskip}

On the positive side, I received another lifeline. A few months earlier,
at an encounter group-style gathering at Whale Bay that Jim Ritchie had
instigated, I had met an Australian guy, also called Jim, who was currently
living on a commune near Mullumbimby in northern New South Wales. While
I was laid up in bed recuperating I received a copy of a book from Jim
and a mutual friend Roselberry: it was \emph{Be Here Now}, by Ram Dass.
This wasn't just a gift -- it was a treasure. Ram Dass was someone with
bright eyes and a beautiful smile, talking about meditation,
macrobiotics, chanting, pranayama, yoga and spiritual community. True,
he had a strange name and wore robes, but when I read his words they
were like music, beautiful music.

