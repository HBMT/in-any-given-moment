\chapter{Difficult Lessons}

There was plenty that happened during the nearly fifteen years we lived
at 81 Studholme Street, Morrinsville, that was not particularly pleasant
and that left a strong impression. One of my favourite pastimes in those
days was cycling with a friend to a disused quarry on the edge of town.
For most of the year there was plenty of surface water which meant it
was a great place to collect tadpoles and bring them home to watch as
they turned into frogs. On one occasion a group of other boys were at
the quarry and they were more interested in torturing the tadpoles and
frogs. These days I have an impression of the group setting up our glass
jars in which we had collected the tadpoles and using them for shooting
practice with their BB guns. Another painful impression of that day is
of those boys blowing up frogs using firecrackers. Whether these events
actually took place or not, I can no longer be sure; perhaps they just
spoke about doing it. However, what I do remember vividly, was speaking
to some adults about it, perhaps my parents or grandparents, expecting
them to sympathize with my hurt feelings at what had happened, only to
be disappointed when I found out they seemed to think nothing of it. I
still feel that brutality of any kind, including that done by small boys
against tadpoles and frogs, should be called out. Small barbaric acts
desensitise our hearts and can lead to bigger barbaric acts.

On another occasion I was surprised and disappointed when my
grandfather, Rev. Duncumb, one day took me into the living room in our
house saying he had something that I would like. I imagine he knew that
I was fond of collecting things like bird skeletons and dead spiders. To
my dismay, the thing he wanted to show me -- indeed, I think, give to me
for my collection -- was a handsome moth which he had skewered with a
pin to the back of the sofa. How could a man of God do such a thing?
Obviously I had a lot to learn about life. Thinking about it now, it
might have been that he found the moth already dead and just put it
there for me, but I doubt it. The idea that animals don't have souls,
that only humans have that privilege, is one of the intensely
regrettable aspects of many theistic religions. In my mind, the pain and
distress humans have caused and continue to cause to living beings, on a
mass scale, is one of the reasons why our human family is in such a
tragic state. Every time we intentionally cause harm, the native
kindness and sensitivity of the human heart is obscured. Not only do
other living beings suffer, but we hurt ourselves in the process.
Because we weren't taught this truth, or at least didn't really
understand it, we go on to deny the pain and the loss of self-respect
that we are causing ourselves, and then feel confused by the gradual
sense of deadening and unhappiness that is taking us over.

It should be emphasized at this point that despite the example of my
being shocked at my Grandfather's actions, in no way am I suggesting
that I was beyond wanting to cause harm. One day a friend invited me to
go hunting rabbits with him and I agreed to go along. If I am honest I
think part of me was really excited by the idea. I wasn't consciously
aware at that stage of an inner conflict between very base instinctual
impulses such as wanting to fight and to conquer, and the more refined
impulses human beings have, such as the inclination towards compassion
and humility. On that occasion I did actually shoot a rabbit, however I
only wounded it, which meant that by the time we had reached it, it had
crawled into a wood pile. I felt ashamed.
It was similar when we used to go fishing: I liked the
thrill of the challenge, but when it came to seeing the eyes of the fish
as I cut its throat, I struggled. On one occasion at least I recall
refusing to do it, and gave the fish I had caught to my brother. It took
me many years and a lot of effort to even begin to find the right kind
of strength to be able to hold the inner tension between the wild,
unruly animal aspect of our nature, and the love and joy one might feel
towards that which we trust is inherently beautiful, without condemning
the struggle. Now I see such a struggle as inevitable for all those on
the journey of awakening. It is important we understand that we do have
a choice: either we struggle in a way that leads to further struggle, or
we struggle in a way that leads to the end of struggling. If we are
inspired by the latter then we are obliged to train ourselves to
skilfully inhibit the base impulses and cultivate the more refined.
Admitting this is not a form of abdication, it is simply being honest.

When I was given my first camera I was excited at the possibilities it
presented. It turned out, however, that not everyone shared my passion
for photographing toadstools and fungi. It was yet another surprise to
be told I shouldn't be so wasteful and ought to be photographing
sensible things. I guess by that they meant family photos taken after we
had come back from church.

In my sixteenth year Nana died. As with many members of our family it
seems, it was cancer that took her. She was staying in our house at the
time and, although I was asked if I wished to see her corpse and say my
goodbyes, I said I didn't. That was another lie, as I expect I very much
did want to see her. Nana was my favourite member of our family, along
with Dad's sister, Auntie Nessie. The warmth of their hearts had not
been overshadowed by the cold insensitivity which comes with clinging to
dogmas. Sadly, on that occasion my commitment to playing the game was
stronger than the impulse to be honest. I didn't feel able to allow
myself to be seen in my upsetness. After the funeral, when a minister
poked his head through the car window and tried to be helpful by saying
to us children that, `Never mind, Nana has gone to a better place', I
thought the comment worse than pointless. In a way that I probably
couldn't quite define at the time, it felt somehow insulting. He was
trying to be supportive, but as became increasingly obvious to me,
trying to be good is not enough. Without matured and considered empathy
we misjudge situations.

Without mindfulness and empathy we always run the risk of projecting our
own ideas and needs onto others. At a later stage of life, when I came
across some studies on the psychology of fundamentalism, it helped me
make sense of what I have often experienced as insensitivity, even
arrogance, on the part of followers of this or that religion. People of
all religious persuasions, including Buddhists, run the risk of trying
to impose their convictions and preferences onto others. Just because we
find a belief gives rise to good feelings within us personally, does not
mean that belief is ultimately true. Studying a little bit about
fundamentalism helped me see how all unawakened beings are at risk of
compensating for a lack of inner security by clinging to forms; that
includes material forms such as drugs and patterns of behaviour as well
as to beliefs. It would be very useful if a course in the psychology of
fundamentalism were to be taught in schools.

I think it was also in my sixteenth year that I lost all my teeth.
Probably that sounds like a serious thing to happen to such a young man,
and it obviously wasn't insignificant, however I was relieved. I
understand that my mother and grandmother had both lost their teeth when
they were thirteen, and since I was around that age I had been in a
great deal of pain. The dentist wasn't in a hurry to take all my teeth
out, but by sixteen it was obviously the right thing to do. A deformity
of the roots (as far as I recall) meant the teeth were going rotten and
the nerves were being pinched. My teeth looked alright from the outside
but because they were rotten, I was being poisoned. Having dentures at
that age was yet another piece of conditioning that caused me to feel
like I didn't fit in. Somehow, though, I didn't succumb to self pity.
Maybe I even thought I was special. I had been taught to think that
those who believed in Jesus were special. They were saved and would be
going to heaven when they died, whilst those who didn't believe were
headed for a very sorry destination. Another example of arrogance born
out of unwise assumptions.

By that age I think I was already becoming aware of a deep, dark sense
of guilt. However, my inability to acknowledge the extent to which I was
feeling guilty meant that that invidious form of pain was accumulating
in unawareness, generating potential for future confusion. The more I
lied to myself, the more I became divided within myself. Part of what
made it so difficult to own up to was that I didn't really know where
all this pain was coming from. Of course I blamed myself for the most
part. Only many years later was I able to understand how as children we
readily take on our parents' pain in an effort to make those caring for
us appear more competent.

Despite my mother's dedication to caring for all of us, and despite her
fervent love for Jesus, I sense she was deeply unhappy. I don't
think she knew how to love herself or be kind to herself. In our family
we never spoke about anything personal -- conversations were always on
the level of what one was supposed to say or feel. At least that is my
reading of it; perhaps my siblings perceived it otherwise. So I am
guessing when I say that I imagine my mother was carrying a huge burden
of guilt for much of her life. I still feel sad when I think of it. How
can people who are trying so hard to be honest, kind and generous end up
being so unhappy? As I found out later when I encountered the Buddhist
teachings, it takes more than goodness, it takes wisdom: it takes wise
reflection to see beyond the way things appear to be to that which is
real. For instance, the idea that it is somehow virtuous to hate
ourselves for having made a mistake, leads to poisoning ourselves, to
self-harming. While aversion is natural, when we make it `my' aversion
through clinging, it turns into hatred and becomes toxic. Guilt is a
distorted form of aversion and manifests as self-hatred, obscuring any
possibility of real contentment. We try to make ourselves feel better by
hating ourselves for having been bad. We are taught that God casts those
who have sinned into eternal hell and that looks like an act of hatred.
So in an attempt to be virtuous, we play God and condemn ourselves for
the things we get wrong. On a level of conditioned thinking this is
sometimes how it works. With wise reflection, however, there is the
possibility of letting go of such wrong thinking and arriving at an
appreciation of what the Buddha said in Dhammapada verse 5,

\enlargethispage{\baselineskip}

\begin{quote}
  Never by hatred is hatred conquered,\\
  but by readiness to love alone.\\
  This is eternal law.\cite{dhammapada}
\end{quote}

Before reflecting on the next stage of this journey, perhaps I could say
something about what I mean when I use the word `wisdom'. In 1975 I was
living in a monastery called Wat Hin Maak Peng, which was perched on the
banks of the Mekong River, just north of Nongkhai in North East
Thailand. My translator at the time reported to me a conversation that
the teacher, Tan Ajahn Thate, had had with his monks. Tan Ajahn Thate
asked the question, `Since the face uses a mirror to see itself, what
does the heart use to see itself?' Apparently none of the monks replied
so Tan Ajahn Thate answered his own question, saying: `The heart uses
wisdom (\emph{pañña}) to see itself.' Wisdom in this context refers to a
self-reflective capacity within awareness that is activated when other
conditions are sufficiently powerful. At the very least there need to be
the strength and confidence which arise out of integrity, and the
steadiness which is the expression of \emph{samadhi} or well-disciplined
attention. In other words, wisdom is not an accumulation of information;
rather it's a dynamic of awareness that has the function of revealing
the reality of that which appears within awareness. Above I have
mentioned that if we truly wish to be protected from the forces of
delusion, mere goodness is not enough. It requires wisdom.

Now back to the journey.

