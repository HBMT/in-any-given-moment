\chapter{Riding the Waves}

As I mentioned in the previous chapter, there was a lot of joy in
designing and decorating the sanctuary that is our Dhamma Hall. Of
course, as is to be expected, there were also times when things were not
so joyous. Looking back now, though, I can see that even the truly
challenging conundrums with which we were presented brought benefits. It
would be foolish to assume that everything was going to be smooth
sailing. The skill that is worth developing is the ability to accord
with the ups and downs of life -- being able to ride the waves of
samsara and, in the process, learn what we need to learn. Dreaming about
a life free from ups and downs is just that: dreaming.

One of the most challenging issues we were obliged to deal with during
the early stages of development of Harnham monastery was the wastewater
system. Farmer Wake and I had a meeting with his solicitor to see if we
could come up with a mutually agreeable plan that would accommodate both
his needs and those of our growing community. It was an amicable meeting
and I thought things were going in a good direction: we seemed to agree
that the monastery could pay for the construction of an eco-friendly
reed bed system on his land in the gully directly behind the property we
occupied; he would have use of it so long as he was alive, and a formal
agreement would be set up which meant future owners of the land would
share in the running costs. Unfortunately, Farmer Wake passed away in
1996 without anything having been signed. His heirs were not at all
interested in a cooperative arrangement and went ahead with putting in
their own separate system.

This was doubly difficult for us because by this time we were well
underway with renovating another property, Kusala House, part way down
the hill. This building would mainly be used as lay guest accommodation.
We followed the suggestion of the new landowners and employed a
professional consultant to see what they could come up with, only to
have the expert's idea rejected. We had reached a point where we simply
could not see any way of dealing with the wastewater. We couldn't undo
the work already done and we couldn't see a way forward.

The monastery's trustees decided that they should consult the wider
community of lay friends and supporters, present them with the
situation, and see how they felt about it: should we proceed with the
building work and trust that a suitable plan would eventually emerge, or
should we close the project down? This wasn't the first time that the
trustees and the wider community had faced the unknown. When the reply
came back, I was inspired to hear it was a resounding endorsement to
carry on and keep trusting. We hadn't made it this far in establishing
the monastery because we felt certain about how things would turn out;
for much of the time it had been down to trust.

Work on Kusala House continued, and after some time we received an
encouraging message from Bill, the project overseer. He and a group of
friends -- a men's group associated with the monastery -- had been away
hiking for a few days in an area of exceptional botanical interest. They
had spent the night in a Youth Hostel, called
\href{https://www.yha.org.uk/hostel/yha-langdon-beck?utm_source=google\&utm_medium=maps\&utm_campaign=google-places}{\underline{Langdon
Beck}} {[}107{]}, in higher Teesdale. In the morning, on waking up, Bill
had a light bulb moment when suddenly it occurred to him that this Youth
Hostel, like the monastery, has to handle a fluctuating number of
residents and is also surrounded by land which they didn't own. He set
about investigating the system they had in place, came back to the
monastery, and presented us with a plan. The monastery's wastewater
system has now been working efficiently for more than ten years. Thank
you, Bill, and all the friends and supporters, for having a good sense
of how and when to trust.

There was a similar sort of issue involving an electricity pylon that
was situated slap-bang in the middle of our Kusala House building site.
The power lines were carrying electricity to several houses, not just to
the monastery, and it was proving difficult to find an agreed relocation
of the pylon. Any suggestion that we put to the neighbours was viewed
negatively. This went on for years and, once again, it looked like it
was going to be impossible to find a solution. Meanwhile, construction
work on other parts of the building steadily continued, trusting that we
would eventually find an answer. And, indeed, an answer was eventually
found. In a conversation with a neighbour one day, Bill asked what they
thought about a new configuration he had come up with and, almost as an
aside, they simply said `fine'. That was it. Thank you again, Bill, for
your patience and creativity, and also to our neighbours.

In the year 2001 the community on Harnham Hill was virtually isolated
because of the
\href{https://www.bbc.co.uk/news/magazine-35581830}{\underline{Foot and
Mouth}} {[}108{]} disease outbreak. It was the only time I have seen our
trustees genuinely distressed, even in tears. After many years of
working really hard to build and maintain the monastery, very little
support was coming in, Vesak had to be cancelled, and we were surrounded
by massive cremation sites as many hundreds of animals were being
slaughtered and burned. Talk at the Trust meeting included the
possibility that we might have to close the monastery down. Cars were
not completely forbidden from using the road leading up the hill to the
monastery, but those that did venture in had to drive over sacks soaked
in disinfectant. All visitors to the monastery were obliged to wash
their footwear in disinfectant.

Such services as Tesco's `home deliveries' were not available in those
days. During one stage of the lockdown, community members would take the
wheelbarrow down to the main road and wait for a supporter who had rung
to say they were bringing out offerings.

The horror of the whole episode eventually passed and things returned to
usual again after about one year. While we were in the midst of it, it
felt terrible, but one of the unexpected consequences was that a few
years later a nearby farm came up for sale and we were able to acquire
Harnham Lake, where now there are several kutis and a thriving wildlife
sanctuary.

We only learn to ride the waves of experience by truly meeting and going
through experiences. If we allow ourselves to indulge in old patterns of
longing for the comfort of certainty, we fail to learn. It is in letting
go of our resistance to pain, and receiving it in the whole body-mind,
that we are more likely to learn the lesson, and let go. Even then, we
don't know how long it will take before we are freed from old habits of
resistance. We can't stop pain -- pleasure and pain are part of life --
but we are potentially capable of changing the way we relate to it. As
with waves, at times the pain might be a mere ripple, while at other
times it might be huge. What matters is: are we preparing ourselves to
mindfully receive the moments of pain that life gives us, and not
default to denying them? Denied pain, unreceived pain, accumulates in
our hearts, in our minds, in our bodies, and leads to chronic
insensitivity. Mindfully received pain leads to aliveness.

It is important to understand that, although letting go is the goal, we
can't `do' the letting go. Letting go is what happens when we see with
sufficient clarity, that we are creating the suffering by clinging to
something too tightly. If we have the good fortune to experience a
conscious moment of letting go, and we do manage to experience the
benefit for ourselves, then we might assume that that is what we are
supposed to keep doing -- keep letting go. However, if we are
discerning, we will see that we didn't make the letting go happen. By
paying attention to the way clinging causes the suffering, letting go
happens by itself. We make the suffering happen; being honest and
mindful about the process of causing suffering is what leads to the
letting go.

\emph{It is wisdom that leads to letting go\\
of a lesser happiness\\
in pursuit of a happiness\\
which is greater.}

In the Dhammapada verse 290 quoted above, the Buddha acknowledges that
there are different levels of happiness. And he spells out how it takes
wisdom to recognise the potential benefit to be found in releasing out
of a lesser happiness and discovering a happiness that is greater.

Earlier on in this compilation, I mentioned that there had been a
five-year period during which seven Western monks who were abbots of
branch monasteries, disrobed. (They weren't all abbots at the time of
disrobing.) The first of this group to give up the training was Ajahn
Pabakharo in early 1991. Then Ajahn Kittisaro left towards the end of
that year, followed by Ajahn Anando around the middle of 1992, and Ajahn
Puriso towards the end of 1992. By the time of the gathering at Wat Pah
Nanachat in January 1993, we were discussing possible causes for these
sad and unforeseen developments. It is probably safe to say that many of
us were still in somewhat of a state of shock at seeing four highly
respected and much loved Dhamma friends leave with such short notice. We
could only speculate about what had truly happened for these individuals
and, of course, nobody knew how the future was going to pan out for
them. What we did all agree on though, as far as I recall, was that
those of us who found ourselves in the role of teacher, needed to be
careful that we didn't allow ourselves to become too busy. Our community
was still in a pioneer phase of development and there was always more
work that needed to be done, but we must remember that the inner work
takes priority. We shouldn't wait until we become exhausted before
setting time aside for personal retreat. Also, I think it was at that
meeting that we started talking about abbots taking longer periods of
time out -- a sort of sabbatical leave.

Regardless of the causes that led so many senior monks to disrobe in
such close succession, we need to keep asking ourselves, what is it that
really matters? The temptation to place the perceived needs of the
community ahead of our own needs can be strong. It might be that such an
impulse is genuinely selfless and beneficial. Then again it might be a
sign of a lack of self-regard. If we hold too tightly to the role of
being a `teacher', we risk losing balance, which in turn, easily leads
to our feeling alienated from the community and from ourselves. This
isn't just true for teachers and leaders; it applies to all of us,
whatever role we might take up. In verse 157 of the Dhammapada, the
Buddha says,

\emph{If we hold ourselves dear,}

\emph{then we maintain careful self-regard}

\emph{both day and night.}

Whether we are wearing robes or jeans, maintaining careful self-regard
means putting the emphasis in the right place. Whatever understanding we
have regarding the Buddha's teachings, we will only integrate that
understanding into the rest of our lives if we manage to maintain faith
and confidence in the path and the practice.

When we start out on this journey we are energized by a sense of trust
that there is truth and it can be realized. This trust has as its
companion, interest: interest in finding out for ourselves how to let go
of that which limits us. And in every moment of letting go, faith,
confidence and trust deepen, becoming stronger, and our hearts open just
a little bit -- or maybe a lot. Now we need to be asking ourselves, what
is it that will nourish and sustain this process in the long term? When
our hearts open, we see more and feel more, and we are also faced with
more questions. In the beginning it was fine to run on the energy of
inspiration, but as we progress, we find such a source of energy
unsustainable; new ways of supporting ourselves on this journey need to
be found.

