\chapter{Journeying}

There is so much to thank Ram Dass for that I hardly know where to
start. The inspiration I received from that book was likely one of the
forces that motivated my leaving the Gordonton commune.

A few weeks before leaving, at an annual festival of devotion to
Dionysus -- AKA, Auckland University Arts Festival -- I had met an Arts
student from Wellington, John Vincent. We must have exchanged addresses
because when I was in Wellington a few weeks later I looked him up. This
was shortly before my 21st birthday. John was living in a typical (for
Wellington) wooden house in Tasman street, just across the road from
Wellington Polytechnic College where he was doing a course in graphic
design. Nearly all old buildings in Wellington are made of wood because
the city is located on a fault line and regularly receives earthquakes.
(New Zealand has approx. 20,000 quakes\cite{quakes} each year.)
There were five or six friends sharing the place and John kindly invited
me to stay. The house had two storeys (actually, it was two separate
apartments) and we lived downstairs.

Around that time, the government stopped providing me with sickness
benefit money and instead offered me a job, which I was more or less
obliged to take, working as a clerk in the Education Department of the
NZ Government. The only interesting aspect of the job was that it was
housed in the second largest wooden building\cite{building} in the world (at that time).
Architecturally it was glorious. Occupationally it was tedious.

Many interesting things did happen though at Tasman Street. The occasion
which shines brightest in my memory is meeting Jutta Passler. One
evening, during a get-together upstairs, one of the girls who lived
there returned home from an event with a German woman she had met. It
was Jutta. As soon as we were introduced we started talking, almost as
if we already knew each other. It turned out that we had both been at
the same encounter group weekend a few months earlier. It was a weekend
being run by two good friends of mine, Dr.~Jack Prichard and Helen
Merriman, and as I happened to be visiting them at the time, they had
taken me along. The conversation with Jutta that started that evening
lasted for decades. Often I would visit her in Palmerston North where
she was teaching French and German at the High School. Only some time
later did I discover how thoroughly out of character it was for her to
have me stay in her apartment. She was about 20 years older than I and
lived an intensely private life. Her neighbours probably thought her
eccentric.

When Jutta was 18 years old she had been living in Dresden. It was the
time of the horrendous firebombing\cite{firebombing}.
Her mother had died in 1939 because penicillin wasn't
available in those days to the public. Her father had little time for
her so, for a period during her teen years, she had been homeless. In
1945 she had found a place to stay at a girls' hostel. During the day
she was obliged to work at a factory and at night she would regularly
help out at the railway station where large numbers of refugees were
accumulating. They were fleeing from the Russians who were approaching
from the north. On the thirteenth February, 1945, Jutta was staying at
the hostel and had telephoned a friend to say she wouldn't be helping
out at the station that evening. It was that evening that the bombing
began, and along with the other girls, she hid in the basement. At one
stage she came up to look outside but then the second wave of bombings
began. In a record of her life that Jutta shared with me, she says that
as many as 120,000 people were killed. I don't know how that accords
with what was officially stated. The shock was indescribable.

She managed to survive the terror of those nights of bombing and then
she had to somehow survive the hell of the aftermath. It doesn't take a
lot to imagine how life might be for an 18 years old German girl when
she encountered the invading Russian soldiers. Many years later when she
shared her story with me I could see how skilfully she had turned her
struggles into great strength and great integrity.

After the war Jutta moved to live in Paris where she worked as a
housekeeper for an American family. Later on they helped her move to
live in California. Having obtained her Master's degree she went on to
live in Honolulu where she taught at the University, and then
eventually, to New Zealand where she settled. On three different
occasions over those years she met pilots who had flown in bomber planes
during those Dresden raids. Jutta didn't really mind if others thought
her eccentric; she was interested not just in surviving, but in truly
living. She was her own person. I am very, very grateful to have had
such a fine friend. It was Jutta who introduced me to Macrobiotics,
Martin Buber, Meister Eckhart, Constant Comment Tea, and Yellow Red
incense -- the latter still being burned today in the Dhamma Hall here
in our monastery in Northumberland.

When Jutta discussed the topic of authentic being she often used the
phrase, `the bigger the front, the bigger the back', an expression she
said she had learned from Macrobiotics, a discipline which focuses
primarily on food but that aims at finding balance in everything. The
expression refers to people who put up a big front in an attempt to hide
something they haven't yet learnt how to fully receive. It reminds me of
an expression I believe is attributed to Prof.~C.G. Jung, `The greater
the radiance, the bigger the shadow.' Jutta had suffered terribly, but
resolutely refused to play the victim. She wasn't into playing games.
She wasn't into putting up fronts. Several years later I was happy when
she visited me in Northeast Thailand and had an opportunity to meet Tan
Ajahn Chah. He called her Toyota.

Now back to Wellington.

There was one occasion when I spontaneously took off to visit an actor I
had met, Simon, who had a job manning a fire lookout tower on
Rainbow Mountain\cite{rainbow}, near Rotorua. The only thing they expected of him
was throughout the day to keep an eye out for fires in the surrounding
forests. If he saw anything suspicious he had to call it in using the
radio. Simon was an actor but was also writing plays. Being a
fire-watcher seemed to suit him. During my visit there I came around to
fancying myself perched up high in such a tower with nothing much to do
all day long, so Simon set up a meeting with the Overseer. On the day I
left to hitchhike back to Wellington, I met the Overseer at the base of
Rainbow Mountain and we discussed possibilities. I let him know I was
interested but he didn't appear convinced I would make a good fit. He
was right: I was attempting to put up another front. What I needed in
fact was community.

My job doing accounts in the Education Department didn't last long;
instead I took another job driving a delivery van for a pharmaceutical
company. Several of us from the house in Tasman street moved to a place
in Owen street, near the zoo. I managed to get three months free rent
since the previous occupants had left the place in such a filthy
condition. The owner said they planned to eventually knock the house
down anyway so we could be creative in renovating it. (That is how I
interpreted what he said.)

Again it was a wonderful wooden house and this time it had generously
high ceilings, at least by our standards. John lived out the back in
what might have been the old servants quarters, Sally baked excellent
cookies, Rob had an amazing sound system, Andrew was just a really sweet
guy, and Jude was the intelligent one. I had a lot of fun stripping back
some of the walls exposing the wooden panelling, painting other walls,
constructing a kitchen and designing a playroom in which there was an
open fire. It was almost too good to be true. We were a harmonious bunch
and appreciated each other's company. This was the time of \emph{The Yes
Album, Emerson Lake and Palmer, To Our Children's Children's Children,}
and \emph{Ziggy Stardust}.

John spent his time painting and drawing and sculpting. He is one of two
people from that period with whom I have stayed in touch. Our lives went
in different directions but there seems to be something we recognise in
each other. Somewhere around 1977 or `78 he turned up in Thailand. Ajahn
Pasanno and I both happened to be in Bangkok for some reason and were at
Wat Boworn when one day John, or Bodhi as he was to be known, suddenly
burst in. Clothed all in red, he had aligned himself with Rajneesh and
was on his way to Poona in India. I have no idea how he knew where to
find me though I am glad he did. These days I think he has an Advaita
Vedanta teacher. Such friendships are precious. Thank you, Bodhi.

The second person from that period with whom I am still in touch, is Rob
Green. In 1971 he was making money working on the Inter-Island ferry
service between Wellington and Picton in the South Island. Curiously, in
June 1980, almost as soon as I arrived at Cittaviveka Monastery in
England, I received an unexpected phone call from Rob. He was taking
part in a study program in a Tibetan Buddhist Institute in Britain. He
happened to be there when Tan Ajahn Chah visited in 1979. These days Rob
lives in Russia with his partner, and teaches English and we write from
time to time.

My life was taking on a very different shape due to the influence of
Jutta and Ram Dass. I suspect it was through Jutta that I was introduced
to books written by Hermann Hesse: \emph{Narcissus and Goldmund, Demian}
and \emph{Siddhartha}. The spirit of what he wrote was again congruent
with what Prof.~C.G. Jung and Marshall McLuhan had to say. Apparently in
real life Professor Jung and Hermann Hesse were close companions.
Already while living at Tasman street I had started attending hatha yoga
classes at a Centre in Aro Street. This helped get my damaged body back
into working order. Maybe it was through that Centre that I also found
out about a Mantra yoga group in which I began to participate. Having
been inspired by Jutta's dedication to Macrobiotics, my breakfast in
those days consisted of a bowl of plain boiled brown rice with some
super healthy (unsweetened) home-made marmalade stirred in. My
meditation practice, however, was sadly lacking in discipline; I was
very restless.

One day, on the spur of a moment, I caught the ferry from Wellington
down to Christchurch. I wanted to spend time with a fellow I had met at
a music festival in Whanganui, John Britten\cite{britten}.
The culture in New Zealand, at least back
then, allowed for spontaneously turning up at someone's house totally
unannounced. My visit coincided with his taking a trip down the coast to
deliver a car to a friend, so I joined him. John had a way with
vehicles; not only did what he had built run well, his work was
gorgeous. For instance he had converted an old truck into a stunningly
beautiful caravan/mobile home, lined with kauri wood. It was a
masterpiece. He was a character full of creative enthusiasm. Years
later, by which time I was already the abbot at Aruna Ratanagiri, I
reached out to reconnect, only to discover he had died shortly before my
letter arrived. It was then that I found out about how he had gone on to
create a world-famous handmade motorbike that achieved recognition at
Daytona\cite{daytona} and other racing competitions.

Back in Wellington I was beginning to have fantasies of setting up a
pancake shop. This was not about serving delicate dainties sprinkled
with icing sugar: what I had in mind would be made of wholemeal flour
and more likely to be served sprinkled with alfalfa sprouts. The idea
never got much traction. One night several of us went to a Ravi Shankar
concert and met up afterwards with friends and discussed the possibility
of such a shop, but the idea stopped there. What was great though about
that particular evening (besides Ravi Shankar) was that those friends
had just acquired a copy of the newly released Pink Floyd album,
\emph{Dark Side Of The Moon}.

