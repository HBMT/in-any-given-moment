\chapter{What Next?}

Unexpectedly, when the retreat ended, I discovered that not everyone
seemed as amazed as I was at what had just happened. Most participants
appeared keen to get back to what they were like before, doing what they
did before; talking a lot, hugging a lot. Was I missing something? That
initial taste of tranquillity did serve to heighten my sensitivities.
When a vehicle drove past on the road, the stench of the exhaust fumes
struck me as unbelievably offensive. On the way back to the commune,
when I stopped at a store to purchase some goods, I felt disturbed by
the evident lifelessness of the staff. My first real encounter with
practising the teachings of the Buddha had a profound effect on me;
however, I can't say I had a very good understanding of what it meant to
go for refuge to the Dhamma, to Reality. It would be a while before I
understood that heightened sensitivities alone were not enough.

Once I arrived back at Narada, at least some of the community members
there picked up on my enthusiasm and expressed an interest in learning
meditation. It wasn't long before a retreat had been organized to take
place at a dome not far down the valley. Ajahn Khantipalo had agreed
that, after the series of retreats at Nimbin had finished, he would lead
one for the Mullumbimby communities. The dome in which we would be
gathering belonged to a couple who ran a business bottling essential
oils. He was Australian and she was Maori from New Zealand. I liked that
they had made their place available.

As it happened, the joyous anticipation I felt at the thought of sharing
the amazing opportunity meditation offered, was naive. The whole
occasion felt like I had invited my friends to a party, only for them to
all turn up late and then leave obviously underwhelmed. I didn't get it:
why couldn't they see the significance of this? We all shared a sense of
disillusionment with the world as it was; we all wanted to make a
difference. Here was a way that could potentially make a massive
difference. It didn't involve taking unemployment money from the
government, didn't involve imbibing substances; all it required was
upgrading our level of integrity and learning to skilfully focus
attention.

The disappointment I experienced didn't deter me from keeping my own
meditation practice going. Up on the ridge, in my small canvas, bamboo
and plastic dwelling, I would spend hours sitting. There were periods of
exquisite delight that I would never have imagined possible. I
discovered an ability to focus attention on something such as the bark
of a tree, and it would trigger a rush of bliss up through my body. Who
would have thought that simply applying concentrated attention on the
natural rhythm of breathing in and out, or slowly walking up and down on
a track in the forest, could give rise to such joy. It was totally legal
and available to everyone. What a revelation! Around that time I read a
copy of Alan Watts' \emph{Nature, Man and Woman} which revealed more
new, inspiring perspectives. This was not about accumulating
information: this felt like recovering from an illness.

Ajahn Khantipalo had recommended that we read a book called, \emph{The
Life of the Buddha} by Nyanamoli. Somehow I managed to acquire a copy
(this was very many years before Amazon, internet and mobile phones) but
trying to read it reminded me of the difficulties I had at university.
It was a struggle to make my mind follow the text, and that struggle got
in the way of accessing the meaning. I have never been tested for
dyslexia so I don't know if there is something odd about how my brain is
wired, or if I am just mentally lazy. Or perhaps there are other
explanations. I do know my mind seems to operate somewhat faster than
some other people, and the struggle I have with reading seems to have to
do with the effort it takes to slow down. I gave up trying to read that
book, and emphasized instead the sitting and walking.

It wasn't long before it became obvious I wasn't going to fit in at
Narada; I no longer wanted to fit in there. The fantasy that was
currently occupying my imagination was travelling to Asia. I had very
little money and no concrete plan -- just felt drawn to travelling out
further into the world. Contrary to what most spiritual seekers were
advocating, I had no inclination to go to India. Yes, that is where the
Buddha was from, and it is where Ram Dass had found his teacher, Neem
Karoli Baba, but it didn't interest me. I felt drawn more in the
direction of Japan. Perhaps, I thought, I could get work teaching
English along the way and somehow just wing it -- start off hitchhiking
to Darwin in the Northern Territories of Australia, get a cheap flight
to Portuguese Timor, island-hop my way through the archipelago of
Indonesia, and then on up via Malaysia to Thailand. Ajahn Khantipalo
received his monastic training in a monastery in Bangkok so that could
be a good point to head for. Then perhaps on to Japan -- where if I was
lucky, I might be able to rub shoulders with some Zen monks and pick up
a few tips on gardening -- before taking the Trans-Siberian railway from
Vladivostok to Europe, ending up in England, or the Mother Country as we
had been brought up to think of her.

I wasn't so thoroughly naive as to think I could do all that without
some money so I went back down to Sydney to work for a few months. There
I met up again with some friends I had met earlier, and made new
friends. The enthusiasm that stemmed from my meditation practice gave
rise to a lot of energy and confidence. I expect I might have been
rather evangelical about the whole thing and could have even put
potential meditators off by my exuberance.

Work which paid well enough wasn't difficult to find even though it was
thoroughly uninspiring. The whole experience of living with and working
with people who had no interest in spiritual matters was draining. I was
learning the lesson that mindfulness and concentration were not the same
thing. At least for a while I was still able to focus concentrated
attention and temporarily access an inner space of ease. I recall
stopping on my way to work, and staring into a flower that was
overhanging a garden wall, until once again the bliss-rush through my
body was triggered. What I didn't understand was that, regardless of
whatever else was going on, I was blindly indulging in pleasurable
sensations.

Confusing the function of concentration with that of mindfulness is a
mistake regularly made by those starting out in meditation, especially
if through concentration they access a degree of happiness. Whether or
not that point had been made clear enough to me when I first received
teachings on meditation I can't recall. Similarly, I don't know if the
importance of \emph{sila} or integrity was sufficiently emphasized.
Perhaps those aspects of the training were explained but I wasn't
interested in them. Through applying the meditation technique I had
developed some increased sensitivity, however I was desperately short on
restraint. As time went by in Sydney, my mood deteriorated and gradually
I lost the brightness and confidence I had brought with me from Narada.

At one particularly painful point I found myself making a vow that, `if
ever I find myself in a position of teaching meditation, I will
emphasize precepts and restraint\emph{.}' Now, forty-eight years later,
I regularly speak about how utterly essential \emph{sila} (integrity)
and \emph{indriya samvara} (sense restraint) are in practice. Without
them we leave ourselves vulnerable. Too much intensity and sensitivity
without the protection and stability that comes with \emph{sila} can be
very dangerous. I heard it reported a few years later that Tan Ajahn
Chah visited the United States and was told about some of the teachings
being given there in the name of Buddhism, that he said it was like
putting people in a leaky boat and sending them out to sea. He was
referring to the lack of emphasis on \emph{sila.}

