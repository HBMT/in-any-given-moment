\chapter{The Land of the Free}

The name `Thailand' means the Land of the Free\cite{free}.
Soon after entering the country I had the feeling
that this country was distinctly different. Was it that the people felt
safe with each other? It turned out that over ninety percent of the
population of Thailand considered themselves Buddhists and they had a
great love for their King. Those two factors alone would be very unifying.
There was an atmosphere of strength, vitality and community; people were
so ready to smile.

It also became apparent that there were lots of rules of etiquette which
had to be followed, otherwise a less lovely side of the Thai character
could show itself. Rules of etiquette, at least in that situation,
didn't bother me. Mindfulness around their cultural conventions seemed
like a fine price to pay for the privilege of participating in such an
uplifting society. It wasn't difficult to remember to not touch
somebody's head or point your feet towards a Buddha image. I liked
clarity around form; it made sense to me.

The word on the street -- that is, the street that backpackers walked
along -- was that if you are in Bangkok, the Malaysia Hotel was a good
place to stay. So once I reached there I booked in.

It wasn't long before I had found my way to Wat Boworn (formally known as
Bowonniwet) where Ajahn Khantipalo had been living prior to
his moving to Australia. There I was introduced to a helpful American
monk, Tan Jotamalo. He mentioned he had met Bill and Randal who had
arrived a few months earlier during the Rainy Season Retreat period.
From what I gathered they didn't stay long in Bangkok but quickly moved
on up country to some monastery called Wat Nong Pah Pong where another
American monk was living along with a group of Westerners.

There were regular classes on the Abhidhamma being taught in English at
Wat Boworn, led by a well-known female teacher, Khun Suchin. It was the
rationality of the approach that first appealed to me. My radar was
probably on high alert for any sign that I might be about to be pulled
into some religious institution that was rife with meaningless rituals
and hidden agendas. It quickly became apparent though that there was no
need for concern. Instead of being put under pressure to subscribe to
dogmatic beliefs and obligations to conform, I was received with
warm-heartedness and generosity. The group didn't require anything at
all of me -- no conformity, no money, just interest in the teachings.
This religion was different. I remained at the Malaysia Hotel for a
while, helping to pay for it out of money I earned teaching
conversational English at a language school.

It might have been at those Abhidhamma classes where I met another
American fellow called Bruce. Bruce had been in correspondence with
Ajahn Khantipalo for quite some time and had come to Thailand
specifically to take up monastic training. He was staying at a newly
opened private hotel near Sukhumvit Road; it seemed it would be quieter
than the Malaysia, so I moved there. In conversation with Bruce the idea
gradually emerged that I too might do well to spend a period of time in
robes while in Thailand. Thank you very much, Bruce, for the
inspiration.

The option of taking on monastic precepts temporarily was something I
never knew was possible. Besides that, I imagine I was finding the
friendliness of the people and sense of community very welcome. Spending
a longer time in that environment was appealing; I could see the way the
monks, novices and laypeople were all cooperating in a harmonious
manner. At least from the outside it looked joyous -- none of the dreary
and dour atmosphere I was used to seeing in religious contexts.

Then there was the issue of having to travel to the border of Cambodia,
near Aranyaprathet, each month for a visa renewal. As a \emph{samanera}
(novice) I would be given a one-year visa. In theory, in 1974 the border
with Cambodia\cite{cambodia} was still open, but the situation there was becoming
increasingly unstable.

So one day towards the end of that year, I plucked up the courage and
asked what was involved in requesting samanera \emph{pabbajja} (novice
Precepts). I'm pretty sure that at that stage I was not planning on
spending several years in Thailand. I don't recall any hesitation in my
being given an appointment to meet the Abbot, Tan Somdet Nyanasamvara.
Wat Boworn was the Royal Temple where His Majesty the King of Thailand
had spent time as a monk. Tan Somdet was his Preceptor and teacher.
Thinking about it now I find it hard to imagine how accessible he was
and how willingly he gave time to this somewhat scruffy traveller.

When I went for my interview I was alone. That also strikes me as
strange, given the position Tan Somdet held. Part way through my
explaining how I wanted to be a samanera, I heard a voice behind
ordering me to squat down on the floor. Although I wanted to be
respectful, I had not learned how, in that culture, one should conduct
oneself in front of an Elder. The Australian monk who happened to be
passing by, witnessed me standing only a few feet in front of the Somdet
talking down to him. Thankfully, this unintended indiscretion didn't get
in the way, and it was agreed that I would be accepted as a samanera and
could stay at Wat Boworn.

Before long Bruce and I found ourselves in the grand main temple being given
samanera precepts by Phra Somdet Nyanasamvara. Phra Jotamalo was there
assisting. The name I was given was Samanera Dhammavicayo which means,
`one who investigates reality'.

