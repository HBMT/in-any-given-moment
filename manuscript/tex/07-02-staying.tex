\chapter{Staying in Touch}

Another structural change made early on after arriving at Harnham was at
the monastery's Trust meetings. I suggested that before the meetings
began, we might briefly check in with each other on a personal level;
also that we could start each meeting with a short period of meditation.
As spiritual director of the Trust I was invited to attend meetings but
didn't have any voting power. The trustees were obviously interested in
how I saw things and regularly consulted with me before any decisions
were made. For the most part it felt like a functional meeting. There
were a few teething issues but they were eventually ironed out. My
suggestions to set time aside to meet each other before turning to
business matters, and the short period of meditation, were aimed at
avoiding having the meeting lose the connection with the spirit of the
spiritual life. All of us were there because we shared a faith in the
validity of Dhamma practice, but if we were not careful, concerns about
funds and future planning had the power to overshadow that faith. During
the checking-in periods it was helpful to listen as the trustees spoke
about their lives and the issues they had to deal with, and I think they
were keen to hear how it was for me, dealing with the day-to-day demands
of being abbot. I believe this small modification to the procedure of
the meeting contributed to our having a good and supportive context
within which the business could be efficiently conducted. I am glad that
slight structural adjustment was made and continue to be impressed by
the competence of our trustees. Some of them travel all the way from
Glasgow, Edinburgh, even London, several times each year, to offer their
skills. Without them I doubt that the monastery would still be here. And
it is a credit to their commitment that most of them have been on the
Trust for many years, thereby contributing to a sense of stability in
the community.

\section{Living Together Harmoniously}

One of the most oft-quoted teachings from the Buddha that I have heard
within our group of Western monasteries is from the
Aparihaniya Sutta\cite{aparihaniya}, \emph{`As long as the monks meet often,
meet a great deal, their growth can be expected, not their decline. As
long as the monks meet in harmony, adjourn from their meetings in
harmony, and conduct Sangha business in harmony, their growth can be
expected, not their decline\ldots'} These two sentences appear at the
beginning of a discourse by the Buddha in which he lists the seven
causes for the growth (or decline) of the sangha. These words, and those
of Tan Ajahn Chah, on how to live together in harmony, have no doubt
informed the ways in which our communities outside of Thailand have
taken shape.

Throughout the 1980's and 1990's branch monasteries continued to be
established. By the year 2000, five monasteries\cite{monasteries}
had been started in Britain, two more on
continental Europe, one in the US, one in Australia and one in New
Zealand. The significance of the teachings quoted above was becoming
increasingly evident, not just because of that cluster of disrobings
which occurred towards the beginning of the 1990's, but also because the
greater the number of branch monasteries, the more complex our community
became. We needed each other to help with navigating the complexities.

To this end, the monasteries under Ajahn Sumedho's immediate auspices --
Cittaviveka, Amaravati, Aruna Ratanagiri, Hartridge, Dhammapala,
Santacittarama -- formed a body called the Elders' Council Meeting
(ECM). As the number of residents at this group of six monasteries
increased, so did the number of issues calling for attention. When this
body was first formed it was referred to as the Abbots' Meeting and, as
the name suggests, comprised the abbots of the six monasteries. This
initial Abbots' Meeting then morphed into the Theras' and Theris'
Meeting (to accommodate senior nuns) and eventually into the ECM. These
days, in the latter formation, if a community has four or more
\emph{samanas} resident, then a second representative of that community
(so long as they have eight or more vassa) can also attend meetings.
This structure introduced a helpful element of community representation
-- it wasn't just the abbots and perhaps their deputy -- the second
representative was to be someone whom the community had elected.

As I already explained when describing the 1993 gathering at Wat Pah
Nanachat, consensus is the underlying principle when it comes to
decision-making. The ECM has been meeting twice a year now, for nearly
fifteen years, and I can't recall any decision ever having been made on
the basis of a vote. I can recall our deciding to delay making a
decision and give each other more time to consider the matter being
discussed. Each meeting has an agenda which is supposed to be shared
with community members in advance of the meeting, and a synopsis of the
meeting is then to be reported back to each community afterwards. Any
committed community member is entitled to submit a topic for discussion
at an ECM.

This present structure of the ECM works well in terms of helping us all
stay in touch. Many of the issues we deal with might not appear to be of
world-shattering significance; however it can be over apparently very
small matters that community members fall out with each other. Even in
the time of the Buddha, communities of monks had major disputes over
matters that, from the outside, looked very minor. If a schism does
arise within the sangha it is considered extremely unfortunate; all
community members are encouraged to go to great lengths to try and avoid
it happening in the first place. Our ECM gatherings serve well the need
to maintain open channels of communication between our communities. So
far, this has meant that when difficult issues do appear, we already
have the means in place for handling them.

Regarding the larger world-wide community, as already mentioned, these
days there are fifteen Branch Monasteries\cite{monasteries},
including Wat Pah Nanachat, who maintain a
connection with each other and aim at offering mutual support. The
structure of the meetings and the degree of interaction is continually
evolving, not least because those leading the various communities keep
changing. The first generation of elders who trained under Tan Ajahn
Chah are now stepping back and taking a much less hands-on role in our
communities.

Also, as technological communication becomes more efficient, questions
are being asked about whether we really need to keep meeting quite so
often, with us all travelling to the same location. The recent v-BAM
(virtual Branch Abbots' Meeting) was very successful in terms of being
able to talk with each other and reach decisions on matters such as
ascribing official `Branch' status to a new monastery, but the
opportunities such virtual meetings provide are minimal compared to
those provided when we are actually together in the same place. Much of
the benefit of the `same location' meetings come not from those periods
spent sitting around going through agenda items, but in between, over
tea or when going for walks in twos and threes. These days there are
also questions being asked about our contribution to environmental
degradation through the use of air travel, maintaining standards of
modesty, and the physical strain involved with international travel.

The nexus within our world-wide communities is generated by our shared
interest in concord. Because of that interest we are motivated to stay
in touch, which, in turn, maintains the channels of communication
necessary for handling the questions that keep arising. Nobody knows how
human society is going to develop over the next few decades, and for the
monastic sangha to survive, in my opinion, cooperation is not just nice,
it is truly essential.

\section{Our Neighbours}

Returning now to the dynamics within our small community on Harnham
Hill, it is clearly important that we remain mindful of the
relationships between monastic residents, but it also matters that we
are alert to how we are perceived by our neighbours. Our neighbours are
not likely to feel drawn to understanding the teachings of the Buddha if
they feel threatened by us.

When first I arrived in this community, it quickly became apparent that
some of the locals were very happy to have us here, while others were
not quite so delighted. For example, we always felt welcome when, once a
week, community members went on alms-round to the nearby village of
Whalton. Gwenda Gofton, the wife of Canon Gofton, the vicar of Whalton,
was usually working at the Village Hall when the monks arrived, and
would regularly offer a tin of beans or some bread into the alms-bowls.
Most times, the caretaker of the hall, who lived next door, would then
invite the monks in for a cup of tea and a biscuit before also placing
offerings in the bowls. Once a year, around Christmas, our whole
community was invited to the Vicarage for a meal. Gwenda had gone out of
her way to learn how to properly pronounce the Thai words for addressing
members of the sangha and would warmly greet us at the door with her
hands raised in \emph{añjali}. She and her husband couldn't have been
more friendly. When he was younger, Canon Gofton had lived as a monk in
an Anglican monastery, and it was clear from our conversations that he
felt an affinity with what we were doing. Those Christmas meals and the
conversations were a joy.

Generally on Christmas Eve, several sangha members would walk to the
nearby Bolam church\cite{bolam} for the midnight service.
At that time of year people
seem to be a bit more relaxed and allow their sense of reserve to drop.
The majority of the local people appeared happy that we made the effort
to turn up for that traditional event.

Not everyone approved of the vicar of Whalton associating himself with
us. On one occasion (before I arrived here) when Canon Gofton had
generously offered to drive the monks back to the monastery, his car was
involved in a serious accident requiring an air ambulance. A car
approaching from the opposite direction had cut the corner causing a
near head-on collision. Some of the parishioners apparently interpreted
the accident as a sign confirming their beliefs that their vicar should
keep his distance from us. Fortunately nobody was too seriously injured,
but we did introduce a new agreement whereby monks travelling in cars
must store their bowls in the boot and not hold them on their laps. I
think the worst injury on that occasion was to the chest of one of the
monks from the impact of his bowl that he was holding. Thankfully the
incident did nothing to compromise the rewarding relationship between
the monastery and Canon and Gwenda Gofton.

Another of our near neighbours was David Robson who owned one of the
largest farms in the immediate vicinity. I understand his family is
known to have been in the area for several hundred years. David and his
wife Charlotte were always very welcoming. Whenever the monastery was
holding a festival we would ring them up to let them know, and they
would generally allow us to use their fields for parking. We made a
point of sending them a card afterwards by way of appreciation.

We also made a point of sending out a greeting card each New Year, to
them and our other neighbours, to let them know that we didn't take them
for granted. Even when it looks like everyone is getting along alright
together, it is useful to check that we are not taking each other for
granted. Regular reaffirmation of friendliness can be like watering a
houseplant: without it, there is a chance the plant will wither and die.

\section{Group Dynamics}

Over the years, there have been monks who have come to live here who
have suggested that we ought to introduce group meetings which give
sangha members an opportunity to talk openly about how they are feeling
about themselves, each other and the monastery. As I wrote much earlier
on in these notes, I am somewhat familiar with such group activities. I
am also familiar with what the Buddha had to say about the supportive
conditions that need to be in place before offering critical feedback to
others. In my experience, even when a meeting is ostensibly about
sharing where individuals might be at within themselves, it easily ends
up with things being said that cause more harm than good. I am not
suggesting this is the intention, but it can be the effect. The Buddha
spoke about this, saying that before we point out someone else's faults,
we should check to see if it is the right time, the right place, using
the right words, and that we speak with the right motivation. In my
view, it is unlikely that, without very skilful facilitation, a sizeable
group of monks are all going to be in that sort of space at the same
time. Unfortunately I have witnessed elsewhere how, when not handled
well, such meetings taking place within the sangha have caused
considerable emotional injury. Accordingly I tend to be cautious about
them.

That does not mean I hold to a fixed position on all such group
meetings. From what I have heard, several of our communities do hold
such meetings and it may well be that they are skilfully facilitated and
support community concord. If community members all share an interest in
being part of such a meeting, and they are equipped with a similar
understanding of the process, and they have sufficient skill in
exercising non-aggressive speech -- then yes, I can well imagine that
they could be constructive.

In the past I have personally benefited from participating in
psychotherapeutic style group meetings, but not with people with whom I
have been living seven days a week. Some of the group work that sangha
members have suggested is designed for people who want to put time
aside, step out of their usual living environment, and pay money with
the expectation they are going to be helped to solve a problem. That
might work well in those cases. However, the culture of the
contemplative community that we are living in here is predicated not on
goal-oriented ambitions, but rather on trusting that conflicts can
resolve themselves when everyone is cultivating the faculties of
embodied mindfulness, skilful restraint and wise reflection. Even within
a community where everyone does share such aspirations, it is still
likely that there will be times when we want to speak directly with each
other about things that trouble us. Hopefully it will happen without our
projecting our personal pain onto others. (I expect there will be more
to say on the subject of projection later on.)

\section{In Touch with Nature}

While considering this topic of `keeping in touch', I would like to
comment on keeping in touch with nature. Most of the meditation
monasteries in Thailand are to be found in forested areas, not in
cities. The majority of monks and nuns who are focused on developing
meditation are surrounded by trees and wildlife. In some monasteries in
the West, where the weather is not as mild as in Asia, we tend to spend
a lot more time indoors. I am not saying this is necessarily an
obstacle, just that we should be careful.

In recent years a lot has been written about the Japanese practice of `Forest
Bathing'\cite{bathing}. Here at Harnham we probably don't approach the
exercise of intentionally walking in nature with the same refined degree
of appreciation, however we are intent on cultivating a nature reserve
for the purposes of increased well-being. Penny, a long-time good friend
of all of our monasteries in Britain, trained in ecology and we have
benefited greatly over the years from her counsel. When Harnham Lake
became part of the monastery in 2010, the surrounding land was nearly
all covered in grass and the soil was saturated with fertilizer. Thanks
to Penny's skilful guidance, in only a few years that land has been
turned into a rich and diverse woodland with beautiful pockets of
wildflowers. It is an excellent location for nature walks and, perhaps
in a few more years as the trees get bigger, will even be suitable for
forest bathing. It is already a great location for three meditation
huts.

Whatever shifts in understanding we might experience while sitting
meditation, the process of integrating those new understandings calls
for mindful movement of the body. Our muscles, our nervous system, our
breath, have all been conditioned over the years by the activity of our
deluded personalities. As we hopefully grow out of our old ways of
compulsive self-centredness, and into more mindful and embodied
awareness, those conditioned pathways of our body's energy need to
adjust. Physical exercise and spending time in nature can help in that.

For many of us, when we first encounter meditation techniques we are
already suffering the condition of being misidentified with the thinking
mind -- in other words, we are disembodied. If we are fortunate, our
teachers will point out the risks of meditating in ways which can lead
to losing touch with our bodies even more. Possibly you might have read
the traditional Buddhist scriptures and think that the Buddha taught we
should contemplate the loathsomeness of our bodies so as to let go of
them; but you should also know that the Buddha laid considerable
emphasis on mindfulness of the body and bodily movement. If we are
already lost in our heads, then it is perhaps that aspect of the
teachings that we should be emphasizing. Without being well-grounded in
our bodies, we are at serious risk of meditation making us more
imbalanced.

Those who take up monastic training at Harnham are told that when it
comes to the way they use their personal time -- when not engaged in
community work or group practice -- they need to make effort in four
areas: formal meditation, formal study, learning chanting and physical
exercise. Just how they address the area of physical exercise is up to
each individual. Some might regularly perform a number of
circumambulations of the boardwalk around Harnham Lake. Others might
prefer to practise Tai Chi or Qigong or yoga in their rooms. Some go for
a longer walk to the nearby Bolam lake\cite{lake}.
Particularly in the early years of my being here, when
there seemed to be endless difficult issues to deal with, almost daily I
would walk the twenty minutes to Bolam Lake, do at least one
circumambulation, and then walk twenty minutes back again. Often I
noticed how different I felt on the return part of the walk. Whatever
aspects of community life had been bothering me on the way out, appeared
much more manageable on the way back.

This same welcome sense of groundedness and feeling refreshed is no
doubt what used to motivate Ajahn Puñño to go out on long walks. These
days, like me, he also has physical limitations which means he doesn't
walk quite so far, or at least doesn't carry a backpack as he used to.
Even in the middle of our Winter Retreat, it wasn't unusual for Ajahn
Puñño to head off, sometimes in thick snow, towards the Kielder Forest\cite{kielder}
where he would find a bothy and settle in for a few
days. Whatever direct benefits he might have received from his periods
of meditation, I am convinced it was the feelings of renewal resulting
from spending time in nature and being in touch with his body, that drew
him out of the confines of the monastery's buildings.

If, for whatever reason, we are obliged to spend a lot of time indoors,
it is still possible to develop a habit of regular physical activity.
For a period when I was a young monk living in Thailand, I was committed
to doing the yoga routine known as `Salutations to the Sun' (\emph{Surya
Namaskar}). I would do it fast so that not only was my body stretched,
but also my breathing. I recall thinking at the time that if ever I
found myself leading a spiritual community I would encourage absolutely
everyone to perform this routine daily. These days my knees don't permit
me to do this exercise, but thankfully I have my Qigong routine that I
am confident contributes to keeping me healthy.

Keeping our body healthy and free from inhibiting disabilities is one
reason why it is sensible for all meditators to establish a habit of
regular physical exercise. The second reason is because it helps keep us
grounded. Anyone who has been meditating for a while, or who has perhaps
gone through a minor or major mental breakdown, knows how dangerous it
is to allow ourselves to become lost in our inner worlds. The third
reason for maintaining a form of physical discipline is that it supports
the process of integrating insight. It is one thing to have inspiring
experiences while sitting on your cushion, but the process of learning
how to truly \emph{live} those insights can take time and a different
quality of effort.

All of us encounter challenges as we travel along this path of
purification. It is wise to equip ourselves in advance and not wait
until we find ourselves confronted by the demon of doubt -- the feeling
that we are sinking into the swamp of uncertainty. Regular physical
exercise and spending time in nature are ways of readying ourselves to
meet these challenges.

\section{Silence and Solitude}

Before leaving this topic of `staying in touch' we ought to consider the
place of silence and solitude in our practice. It might appear
counter-intuitive to raise the subjects of silence and solitude in the
context of contemplations on staying in touch with each other. However,
the truth is, as we might know from what the Buddha had to say in the
Sutta of the Acrobats\cite{acrobats}, if we are not truly in touch with
ourselves, we will feel obstructed in our efforts to relate with others.
Conversely, when we are capable of meeting ourselves, without being
driven by habitual patterns of grasping and rejecting, then we will be
better able to truly meet others.

There are many means of learning to meet ourselves in our experience of
limited being, some upon which I have already commented, such as
physical exercise and disciplined breathing practices. Placing ourselves
in solitude can be another very effective way of highlighting those
areas of our character which we have been avoiding. For some people,
solitude and silence will be a rewarding relief which energizes them.
For others it could feel intensely threatening. For all of us, to
periodically put ourselves into such an environment -- by way of
experiment, not to prove anything, but because we are interested in
learning about ourselves -- can be productive. I emphasize `by way of
experiment' because there can be a tendency to engage such practices
idealistically -- blindly clinging to an idea that they are good for us,
or because somebody else did it and had such and such a result. We are
all different. Surely what is important is finding out what works in our
case. Dhammapada verse 160 says,

\clearpage

\begin{quote}
  Truly it is ourselves that we depend upon;\\
  how could we really depend upon another?\\
  When we reach the state of self-reliance\\
  we find a rare refuge.
\end{quote}

All residents in our monastery are invited to make use of the meditation
huts down by the lake so as to experiment with solitude and silence. It
is essential, however, that nobody is ever required to do so.
Intimidating someone into spending time in solitude and silence can be
similar to pressuring someone to go into therapy. Similar to meditation
retreats, these \emph{upaya} can support deepening of our practice, or
they can create further obstructions. Nobody should ever feel obliged to
go on retreat. The Buddha instructed his monks that imposing silence on
the community is inappropriate. Here we do usually have seven or eight
weeks of group silence, spaced throughout the year, but they are
referred to as periods of `noble silence', and usually everyone
structures their own formal meditation routine. Since the encouragement
to be silent during those periods is nearly always honoured by everyone,
I feel satisfied community members are finding them beneficial.

One of the most basic principles of the monastic life is simplicity. I
think it was in an early publication of the \emph{Fragments of a Teaching}\cite{collected}
booklet by Jack Kornfield, that I read that
Tan Ajahn Chah defined Buddhism as: `simplify your life and watch your
mind'. We might be keen on watching our minds, but it takes skilful
effort to keep life simple.

