\chapter{Chithurst Emerging}

The extensive Hammer Wood and Hammer Pond property had been given as a
gift, but Chithurst House had been purchased by The English Sangha Trust
(EST). Presumably most of the purchase price was for the surrounding
fields, since the house itself was full of dry rot and in need of a
massive amount of repair. Personally, I would have been fine if there
had been an accidental fire and the whole place had burned down. That
was another aspect of the British about which I still had to learn:
their fondness for old buildings. In New Zealand there weren't really
many old buildings, and until relatively recently, there hadn't been
much of an appreciation for good architecture.

As it was, a huge amount of care and attention went into renovating
Chithurst House. The large internal wooden staircase was completely
dismantled, sanded, repaired, polished and reinstalled. The tall ornate
chimneys were taken down, cleaned brick by brick, and then put back up
in their original place. Ajahn Sumedho was part of the brick cleaning
team. One of the first jobs I was given was repointing the outer walls
of the house. I would never have imagined there could be so many
opinions about how one should point stone walls. The way the mortar was
mixed and the style in which that mortar was then shaped, were obviously
relevant, but they were not the most important thing. These were pioneer
days and thankfully, there was a spirit of cooperation and mutual
support. I think we all shared the view that it was good enough that we
were making an effort. Ultimately it didn't matter whether or not we
knew how to point properly; what mattered was that we had this amazing
opportunity to live as monks and nuns in a cooperative community, at
this time, in this country.

Those who were tasked with converting two of the downstairs rooms into
one big shrine room had a particularly challenging job. The old plaster
was stripped off the walls, which were then sprayed with an anti-dry rot
chemical. When it came to removing the wall that divided the two rooms,
care was taken to check to see whether or not it was structural. To that
end, a strip about thirty centimetres high was removed, the full width
of the room, at the bottom and also at the top. This way, it was thought
that if there were structural supporting beams they would be visible.
There was nothing; the space was clear, right through to the room next
door, from pillar to pillar. So it was decided that it was safe to knock
the wall down. I can't recall now exactly how it happened, perhaps the
job of taking out the wall was delegated to others, but a while later,
when one of the senior community members came back to see how the work
was progressing, he noticed a disturbing bow in the ceiling as it was
beginning to drop down, precisely where the wall had been. What hadn't
been understood was the way the Victorians had built that supporting
wall using cross beams: they weren't perpendicular, so there were no
supports between the floor and the ceiling. What was not seen were the
supports constructed diagonally, pillar to pillar. Fortunately, maybe
just in time, acrow props were put in place, preventing the building
from collapsing on top of everyone.

That was a good lesson, and one I have reflected on many times over the
years. Just because we can't see the point in a particular structure
doesn't mean there isn't one; and this applies to not just physical
structures, but social, relational, psychological ones also. When a
young Englishman once told me that he would really like to take up
monastic training but he couldn't abide the robes that we wore, he
wasn't able to see beyond the aesthetics. He suggested we should swap
our robes for saffron tracksuits. What he wasn't able to see was the
current of energy, metaphorically speaking, that runs through the
centuries, of millions of \emph{samanas} wearing the same form of
clothing: the robes are a symbol of what we are referring to when we
talk about lineage. Also, he wasn't aware of the risk of rupture in our
relationship with our brothers and sisters in Thailand if we had changed
our robes to suit our preferences.

The same principle applies to ideas one might have about altering our
adherence to the monastic rules. Plenty of people have been keen to tell
us that in this day and age, we need to be handling money. As it has
worked out, in truth, because we don't change the rules -- because we
don't handle money -- there are many people who have confidence in this
tradition and generously offer their support. It can be humbling to
realize how our attachment to outer forms often blinds us to the truth.
So long as we cling to the surface level of experience, our attention
readily falls short of what we really need to see. One of Ajahn
Sumedho's many valuable gifts to our community has been his hesitation
to change structures and conventions; thankfully, his approach has been
one of wait-and-see.

After the wall had been removed, I was invited to help with the
decorating. Part of that stage of the project meant a cluster of us (if
that is a suitable collective pronoun for a group of \emph{samanas})
were sent away for a day to the British Gypsum factory to don white
coats and learn how to work with their dry-wall lining system. Because
the house was riddled with dry rot we couldn't use wooden battens for
fixing the plaster boards to the walls; we had to use a system of gluing
metal furring strips to the very uneven stonework and then, with
self-tapping screws, attach the massive plaster boards. These weren't
just long and wide boards, they also had thick insulation on the back;
they were very heavy. Once the boards had been securely placed, the
joints were then taped and filled. Often the days were very long and the
work very tiring. After many weeks though, I like to think we ended up
with something quite suitable.

